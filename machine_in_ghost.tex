\part{The Machine in the Ghost}


\mysectiontwo{Minds: An Introduction}{Minds: An Introduction\newline by Rob Bensinger}

{
 You're a mind, and that puts you in a pretty
strange predicament.}

{
 Very few things get to be minds. You're that odd
bit of stuff in the universe that can form predictions and make plans,
weigh and revise beliefs, suffer, dream, notice ladybugs, or feel a
sudden craving for mango. You can even form, \textit{inside your mind},
a picture of your whole mind. You can reason about your own reasoning
process, and work to bring its operations more in line with your
goals.}

{
 You're a mind, implemented on a human brain. And
it turns out that a human brain, for all its marvelous flexibility, is
a lawful thing, a thing of pattern and routine. Your mind can follow a
routine for a lifetime, without ever once noticing that it is doing so.
And these routines can have great consequences.}

{
 When a mental pattern serves you well, we call that
``rationality.''}

{
 You exist as you are, hard-wired to exhibit certain species of
rationality and certain species of irrationality, because of your
ancestry. You, and all life on Earth, are descended from ancient
self-replicating molecules. This replication process was initially
clumsy and haphazard, and soon yielded replicable \textit{differences}
between the replicators.
``Evolution'' is our name for the
change in these differences over time.}

{
 Since some of these reproducible differences impact
reproducibility---a phenomenon called
``selection''---evolution has
resulted in organisms suited to reproduction in environments like the
ones their ancestors had. Everything about you is built on the echoes
of your ancestors' struggles and victories.}

{
 And so here you are: a mind, carved from weaker minds, seeking to
understand your own inner workings, that they can be improved
upon---improved upon relative to \textit{your} goals, and not those of
your designer, evolution. What useful policies and insights can we take
away from knowing that this is our basic situation?}


\subsection{Ghosts and Machines}

{
 Our brains, in their small-scale structure and dynamics, look like
many other mechanical systems. Yet we rarely think of our minds in the
same terms we think of objects in our environments or organs in our
bodies. Our basic mental categories---belief, decision, word, idea,
feeling, and so on---bear little resemblance to our physical
categories.}

{
 Past philosophers have taken this observation and run with it,
arguing that minds and brains are fundamentally distinct and separate
phenomena. This is the view the philosopher Gilbert Ryle called
``the dogma of the Ghost in the
Machine.''\footnote{Gilbert Ryle, \textit{The Concept of Mind} (University of
Chicago Press, 1949).\comment{1}} But modern scientists
and philosophers who have rejected dualism haven't
necessarily replaced it with a better predictive model of how the mind
works. \textit{Practically} speaking, our purposes and desires still
function like free-floating ghosts, like a magisterium cut off from the
rest of our scientific knowledge. We can talk about
``rationality'' and
``bias'' and ``how
to change our minds,'' but if those ideas are still
imprecise and unconstrained by any overarching theory, our
scientific-sounding language won't protect us from
making the same kinds of mistakes as those whose theoretical posits
include spirits and essences.}

{
 Interestingly, the mystery and mystification surrounding minds
doesn't just obscure our view of \textit{humans}. It
also accrues to systems that seem mind-like or purposeful in
evolutionary biology and artificial intelligence (AI). Perhaps, if we
cannot readily glean what we are from looking at ourselves, we can
learn more by using obviously \textit{in}human processes as a mirror.}

{
 There are many ghosts to learn from here---ghosts past, and
present, and yet to come. And these illusions are real cognitive
events, real phenomena that we can study and explain. If there
\textit{appears} to be a ghost in the machine, that appearance is
itself the hidden work of a machine.}

{
 The first sequence of \textit{The Machine in the Ghost},
``The Simple Math of Evolution,''
aims to communicate the dissonance and divergence between our
hereditary history, our present-day biology, and our ultimate
aspirations. This will require digging deeper than is common in
introductions to evolution for non-biologists, which often restrict
their attention to surface-level features of natural selection.}

{
 The third sequence, ``A Human's
Guide to Words,'' discusses the basic relationship
between cognition and concept formation. This is followed by a longer
essay introducing Bayesian inference.}

{
 Bridging the gap between these topics, ``Fragile
Purposes'' abstracts from human cognition and
evolution to the idea of minds and goal-directed systems at their most
general. These essays serve the secondary purpose of explaining the
author's general approach to philosophy and the science
of rationality, which is strongly informed by his work in AI.}

\subsection{Rebuilding Intelligence}

{
 Yudkowsky is a decision theorist and mathematician who works on
foundational issues in Artificial General Intelligence (AGI), the
theoretical study of domain-general problem-solving systems.
Yudkowsky's work in AI has been a major driving force
behind his exploration of the psychology of human rationality, as he
noted in his very first blog post on \textit{Overcoming Bias}, The
Martial Art of Rationality:}

\begin{quotation}
{
 Such understanding as I have of rationality, I acquired in the
course of wrestling with the challenge of Artificial General
Intelligence (an endeavor which, to actually succeed, would require
sufficient mastery of rationality to build a complete working
rationalist out of toothpicks and rubber bands). In most ways the AI
problem is enormously more demanding than the personal art of
rationality, but in some ways it is actually easier. In the martial art
of mind, we need to acquire the real-time procedural skill of pulling
the right levers at the right time on a large, pre-existing thinking
machine whose innards are not end-user-modifiable. Some of the
machinery is optimized for evolutionary selection pressures that run
directly counter to our declared goals in using it. Deliberately we
decide that we want to seek only the truth; but our brains have
hardwired support for rationalizing falsehoods. [ \ldots ]}

{
 Trying to synthesize a personal art of rationality, using the
science of rationality, may prove awkward: One imagines trying to
invent a martial art using an abstract theory of physics, game theory,
and human anatomy. But humans are not reflectively blind; we do have a
native instinct for introspection. The inner eye is not sightless; but
it sees blurrily, with systematic distortions. We need, then, to apply
the science to our intuitions, to use the abstract knowledge to correct
our mental movements and augment our metacognitive skills. We are not
writing a computer program to make a string puppet execute martial arts
forms; it is our own mental limbs that we must move. Therefore we must
connect theory to practice. We must come to see what the science means,
for ourselves, for our daily inner life.}
\end{quotation}

{
 From Yudkowsky's perspective, I gather, talking
about human rationality without saying anything interesting about AI is
about as difficult as talking about AI without saying anything
interesting about rationality.}

{
 In the long run, Yudkowsky predicts that AI will come to surpass
humans in an ``intelligence
explosion,'' a scenario in which self-modifying AI
improves its own ability to productively redesign itself, kicking off a
rapid succession of further self-improvements. The term
``technological singularity'' is
sometimes used in place of ``intelligence
explosion;'' until January 2013, MIRI was named
``the Singularity Institute for Artificial
Intelligence'' and hosted an annual Singularity
Summit. Since then, Yudkowsky has come to favor I.J.
Good's older term, ``intelligence
explosion,'' to help distinguish his views from other
futurist predictions, such as Ray Kurzweil's
exponential technological progress thesis.\footnote{Irving John Good, ``Speculations Concerning
the First Ultraintelligent Machine,'' in
\textit{Advances in Computers}, ed. Franz L. Alt and Morris Rubinoff,
vol. 6 (New York: Academic Press, 1965), 31--88,
doi:10.1016/S0065-2458(08)60418-0.\comment{2}}}

{
 Technologies like smarter-than-human AI seem likely to result in
large societal upheavals, for the better or for the worse. Yudkowsky
coined the term ``Friendly AI
theory'' to refer to research into techniques for
aligning an AGI's preferences with the preferences of
humans. At this point, very little is known about when generally
intelligent software might be invented, or what safety approaches would
work well in such cases. Present-day autonomous AI can already be quite
challenging to verify and validate with much confidence, and many
current techniques are not likely to generalize to more intelligent and
adaptive systems. ``Friendly AI'' is
therefore closer to a menagerie of basic mathematical and philosophical
questions than to a well-specified set of programming objectives.}

{
 As of 2015, Yudkowsky's views on the future of AI
continue to be debated by technology forecasters and AI researchers in
industry and academia, who have yet to converge on a consensus
position. Nick Bostrom's book
\textit{Superintelligence} provides a big-picture summary of the many
moral and strategic questions raised by smarter-than-human
AI.\footnote{Nick Bostrom, \textit{Superintelligence: Paths, Dangers,
Strategies} (Oxford University Press, 2014).\comment{3}}}

{
 For a general introduction to the field of AI, the most widely
used textbook is Russell and Norvig's
\textit{Artificial Intelligence: A Modern Approach}.\footnote{Stuart J. Russell and Peter Norvig, \textit{Artificial
Intelligence: A Modern Approach}, 3rd ed. (Upper Saddle River, NJ:
Prentice-Hall, 2010).\comment{4}}
In a chapter discussing the moral and philosophical questions raised by
AI, Russell and Norvig note the technical difficulty of specifying good
behavior in strongly adaptive AI:}

\begin{quotation}
{
 [Yudkowsky] asserts that friendliness (a desire not to harm
humans) should be designed in from the start, but that the designers
should recognize both that their own designs may be flawed, and that
the robot will learn and evolve over time. Thus the challenge is one of
mechanism design---to define a mechanism for evolving AI systems under
a system of checks and balances, and to give the systems utility
functions that will remain friendly in the face of such changes. We
can't just give a program a static utility function,
because circumstances, and our desired responses to circumstances,
change over time.}
\end{quotation}

{
 Disturbed by the possibility that future progress in AI,
nanotechnology, biotechnology, and other fields could endanger human
civilization, Bostrom and \'Cirkovi\'c compiled the first academic
anthology on the topic, \textit{Global Catastrophic
Risks}.\footnote{Bostrom and \'Cirkovi\'c, \textit{Global Catastrophic Risks}.\comment{5}} The most extreme of these are the
\textit{existential risks}, risks that could result in the permanent
stagnation or extinction of humanity.\footnote{An example of a possible existential risk is the
``grey goo'' scenario, in which
molecular robots designed to efficiently self-replicate do their job
too well, rapidly outcompeting living organisms as they consume the
Earth's available matter.\comment{6}}}

{
 People (experts included) tend to be \textit{extraordinarily bad}
at forecasting major future events (new technologies included). Part of
Yudkowsky's goal in discussing rationality is to figure
out which biases are interfering with our ability to predict and
prepare for big upheavals well in advance. Yudkowsky's
contributions to the \textit{Global Catastrophic Risks} volume,
``Cognitive biases potentially affecting judgement of
global risks'' and ``Artificial
intelligence as a positive and negative factor in global
risk,'' tie together his research in cognitive
science and AI. Yudkowsky and Bostrom summarize near-term concerns
along with long-term ones in a chapter of the \textit{Cambridge
Handbook of Artificial Intelligence}, ``The ethics of
artificial intelligence.''\footnote{Nick Bostrom and Eliezer Yudkowsky, ``The
Ethics of Artificial Intelligence,'' in \textit{The
Cambridge Handbook of Artificial Intelligence}, ed. Keith Frankish and
William Ramsey (New York: Cambridge University Press, 2014).\comment{7}}}

{
 Though this is a book about \textit{human} rationality, the topic
of AI has relevance as a source of simple illustrations of aspects of
human cognition. Long-term technology forecasting is also one of the
more important applications of Bayesian rationality, which can model
correct reasoning even in domains where the data is scarce or
equivocal.}

{
 Knowing the design can tell you much about the designer; and
knowing the designer can tell you much about the design.}

{
 We'll begin, then, by inquiring into what our own
designer can teach us about ourselves.}

{
 ~}

{\centering
 *
\par}


\bigskip

\mysectionnn{Interlude The Power of Intelligence}

{
 In our skulls we carry around three pounds of slimy, wet, grayish
tissue, corrugated like crumpled toilet paper.}

{
 You wouldn't think, to look at the unappetizing
lump, that it was some of the most powerful stuff in the known
universe. If you'd never seen an anatomy textbook, and
you saw a brain lying in the street, you'd say
``Yuck!'' and try not to get any of
it on your shoes. Aristotle thought the brain was an organ that cooled
the blood. It doesn't \textit{look} dangerous.}

{
 Five million years ago, the ancestors of lions ruled the day, the
ancestors of wolves roamed the night. The ruling predators were armed
with teeth and claws---sharp, hard cutting edges, backed up by powerful
muscles. Their prey, in self-defense, evolved armored shells, sharp
horns, toxic venoms, camouflage. The war had gone on through hundreds
of eons and countless arms races. Many a loser had been removed from
the game, but there was no sign of a winner. Where one species had
shells, another species would evolve to crack them; where one species
became poisonous, another would evolve to tolerate the poison. Each
species had its private niche---for who could live in the seas and the
skies and the land at once? There was no ultimate weapon and no
ultimate defense and no reason to believe any such thing was possible.}

{
 Then came the Day of the Squishy Things.}

{
 They had no armor. They had no claws. They had no venoms.}

{
 If you saw a movie of a nuclear explosion going off, and you were
told an Earthly life form had done it, you would never in your wildest
dreams imagine that the Squishy Things could be responsible. After all,
Squishy Things aren't radioactive.}

{
 In the beginning, the Squishy Things had no fighter jets, no
machine guns, no rifles, no swords. No bronze, no iron. No hammers, no
anvils, no tongs, no smithies, no mines. All the Squishy Things had
were squishy fingers---too weak to break a tree, let alone a mountain.
Clearly not dangerous. To cut stone you would need steel, and the
Squishy Things couldn't excrete steel. In the
environment there were no steel blades for Squishy fingers to pick up.
Their bodies could not generate temperatures anywhere near hot enough
to melt metal. The whole scenario was obviously absurd.}

{
 And as for the Squishy Things manipulating DNA---that would have
been beyond ridiculous. Squishy fingers are not that small. There is no
access to DNA from the Squishy level; it would be like trying to pick
up a hydrogen atom. Oh, \textit{technically} it's all
one universe, \textit{technically} the Squishy Things and DNA are part
of the same world, the same unified laws of physics, the same great web
of causality. But let's be realistic: you
can't get there from here.}

{
 Even if Squishy Things could \textit{someday} evolve to do any of
those feats, it would take thousands of millennia. We have watched the
ebb and flow of Life through the eons, and let us tell you, a year is
not even a single clock tick of evolutionary time. Oh, sure,
\textit{technically} a year is six hundred trillion trillion trillion
trillion Planck intervals. But nothing ever happens in less than six
hundred million trillion trillion trillion trillion Planck intervals,
so it's a moot point. The Squishy Things, as they run
across the savanna now, will not fly across continents for at least
another ten million years; \textit{no one} could have that much sex.}

{
 Now explain to me again why an Artificial Intelligence
can't do anything interesting over the Internet unless
a human programmer builds it a robot body.}

{
 I have observed that someone's flinch-reaction to
``intelligence''---the thought that
crosses their mind in the first half-second after they hear the word
``intelligence''---often determines
their flinch-reaction to the notion of an intelligence explosion. Often
they look up the keyword
``intelligence'' and retrieve the
concept \textit{booksmarts}{}---a mental image of the Grand Master
chess player who can't get a date, or a college
professor who can't survive outside academia.}

{
 ``It takes more than intelligence to succeed
professionally,'' people say, as if charisma resided
in the kidneys, rather than the brain. ``Intelligence
is no match for a gun,'' they say, as if guns had
grown on trees. ``Where will an Artificial
Intelligence get money?'' they ask, as if the first
\textit{Homo sapiens} had found dollar bills fluttering down from the
sky, and used them at convenience stores already in the forest. The
human species was not born into a market economy. Bees
won't sell you honey if you offer them an electronic
funds transfer. The human species \textit{imagined} money into
existence, and it exists---for \textit{us}, not mice or wasps---because
we go on believing in it.}

{
 I keep trying to explain to people that the archetype of
intelligence is not Dustin Hoffman in \textit{Rain Man}. It is a human
being, period. It is squishy things that explode in a vacuum, leaving
footprints on their moon. Within that gray wet lump is the power to
search paths through the great web of causality, and find a road to the
seemingly impossible---the power sometimes called creativity.}

{
 People---venture capitalists in particular---sometimes ask how, if
the Machine Intelligence Research Institute successfully builds a true
AI, the results will be \textit{commercialized}. This is what we call a
framing problem.}

{
 Or maybe it's something deeper than a simple clash
of assumptions. With a bit of creative thinking, people can imagine how
they would go about travelling to the Moon, or curing smallpox, or
manufacturing computers. To imagine a trick that could accomplish
\textit{all these things at once} seems downright impossible---even
though such a power resides only a few centimeters behind their own
eyes. The gray wet thing still seems mysterious to the gray wet thing.}

{
 And so, because people can't quite see how it
would all work, the power of intelligence seems less real; harder to
imagine than a tower of fire sending a ship to Mars. The prospect of
visiting Mars captures the imagination. But if one should promise a
Mars visit, and also a grand unified theory of physics, and a proof of
the Riemann Hypothesis, and a cure for obesity, and a cure for cancer,
and a cure for aging, and a cure for stupidity---well, it just sounds
wrong, that's all.}

{
 And well it should. It's a serious failure of
imagination to think that intelligence is good for so little. Who could
have imagined, ever so long ago, what minds would someday do? We may
not even \textit{know} what our real problems are.}

{
 But meanwhile, because it's hard to see how one
process could have such diverse powers, it's hard to
imagine that one fell swoop could solve even such prosaic problems as
obesity and cancer and aging.}

{
 Well, one trick cured smallpox and built airplanes and cultivated
wheat and tamed fire. Our current science may not agree yet on how
exactly the trick works, but it works anyway. If you are temporarily
ignorant about a phenomenon, that is a fact about your current state of
mind, not a fact about the phenomenon. A blank map does not correspond
to a blank territory. If one does not quite understand that power which
put footprints on the Moon, nonetheless, the footprints are still
there---real footprints, on a real Moon, put there by a real power. If
one were to understand deeply enough, one could create and shape that
power. Intelligence is as real as electricity. It's
merely far more powerful, far more dangerous, has far deeper
implications for the unfolding story of life in the universe---and
it's a tiny little bit harder to figure out how to
build a generator.}

\myendsectiontext

\chapter{The Simple Math of Evolution}

\mysection{An Alien God}

{
 ``A curious aspect of the theory of
evolution,'' said Jacques Monod,
``is that everybody thinks he understands
it.'' }

{
 A human being, looking at the natural world, sees a thousand times
\textit{purpose.} A rabbit's legs, built and
articulated for running; a fox's jaws, built and
articulated for tearing. But what you see is not exactly what is there
\ldots}

{
 In the days before Darwin, the cause of all this apparent
\textit{purposefulness} was a very great puzzle unto science. The
Goddists said ``God did it,''
because you get 50 bonus points each time you use the word
``God'' in a sentence. Yet perhaps
I'm being unfair. In the days before Darwin, it seemed
like a much more reasonable hypothesis. Find a watch in the desert,
said William Paley, and you can infer the existence of a watchmaker.}

{
 But when you look at \textit{all} the apparent purposefulness in
Nature, rather than picking and choosing your examples, you start to
notice things that don't fit the Judeo-Christian
concept of one benevolent God. Foxes seem well-designed to catch
rabbits. Rabbits seem well-designed to evade foxes. Was the Creator
having trouble making up Its mind?}

{
 When I design a toaster oven, I don't design one
part that tries to get electricity to the coils and a second part that
tries to prevent electricity from getting to the coils. It would be a
waste of effort. \textit{Who} designed the ecosystem, with its
predators and prey, viruses and bacteria? Even the cactus plant, which
you might think well-designed to provide water and fruit to desert
animals, is covered with inconvenient spines.}

{
 The ecosystem would make much more sense if it
wasn't designed by a unitary \textit{Who}, but, rather,
created by a horde of deities---say from the Hindu or Shinto religions.
This handily explains both the ubiquitous purposefulnesses, and the
ubiquitous conflicts: More than one deity acted, often at
cross-purposes. The fox and rabbit were both designed, but by distinct
competing deities. I wonder if anyone ever remarked on the seemingly
excellent evidence thus provided for Hinduism over Christianity.
Probably not.}

{
 Similarly, the Judeo-Christian God is alleged to be
benevolent---well, sort of. And yet much of nature's
\textit{purposefulness} seems downright cruel. Darwin suspected a
non-standard Creator for studying Ichneumon wasps, whose paralyzing
stings preserve its prey to be eaten alive by its larvae:
``I cannot persuade myself,'' wrote
Darwin, ``that a beneficent and omnipotent God would
have designedly created the Ichneumonidae with the express intention of
their feeding within the living bodies of Caterpillars, or that a cat
should play with mice.''\footnote{Francis Darwin, ed., \textit{The Life and Letters of Charles
Darwin}, vol. 2 (John Murray, 1887).\comment{1}} I wonder
if any earlier thinker remarked on the excellent evidence thus provided
for Manichaean religions over monotheistic ones.}

{
 By now we all know the punchline: you just say
``evolution.''}

{
 I worry that's how some people are absorbing the
``scientific'' explanation, as a
magical purposefulness factory in Nature. I've
previously discussed the case of Storm from the movie \textit{X-Men},
who in \textit{one mutation} gets the ability to throw lightning bolts.
Why? Well, there's this thing called
``evolution'' that somehow pumps a
lot of purposefulness into Nature, and the changes happen through
``mutations.'' So if Storm gets a
really \textit{large} mutation, she can be redesigned to throw
lightning bolts. Radioactivity is a popular super origin: radiation
causes mutations, so more powerful radiation causes more powerful
mutations. That's logic.}

{
 But evolution doesn't allow just \textit{any} kind
of purposefulness to leak into Nature. That's what
makes evolution a success as an empirical hypothesis. If evolutionary
biology could explain a toaster oven, not just a tree, it would be
worthless. There's a lot more to evolutionary theory
than pointing at Nature and saying, ``Now purpose is
allowed,'' or ``Evolution did
it!'' The strength of a theory is not what it allows,
but what it prohibits; if you can invent an equally persuasive
explanation for any outcome, you have zero knowledge.}

{
 ``Many non-biologists,''
observed George Williams, ``think that it is for their
benefit that rattles grow on rattlesnake
tails.''\footnote{George C. Williams, \textit{Adaptation and Natural Selection: A
Critique of Some Current Evolutionary Thought}, Princeton Science
Library (Princeton, NJ: Princeton University Press, 1966).\comment{2}} Bzzzt! \textit{This} kind
of purposefulness is not allowed. Evolution doesn't
work by letting flashes of purposefulness creep in at
random---reshaping one species for the benefit of a random recipient.}

{
 Evolution is powered by a systematic correlation between the
different ways that different genes construct organisms, and how many
copies of those genes make it into the next generation. For rattles to
grow on rattlesnake tails, rattle-growing genes must become more and
more frequent in each successive generation. (Actually genes for
incrementally more complex rattles, but if I start describing all the
fillips and caveats to evolutionary biology, we really \textit{will} be
here all day.)}

{
 There isn't an Evolution Fairy that looks over the
current state of Nature, decides what would be a
``good idea,'' and chooses to
increase the frequency of rattle-constructing genes.}

{
 I suspect this is where a lot of people get stuck, in evolutionary
biology. They understand that
``helpful'' genes become more
common, but ``helpful'' lets any
sort of purpose leak in. They don't think
there's an Evolution Fairy, yet they ask which genes
will be ``helpful'' as if a
rattlesnake gene could ``help''
non-rattlesnakes.}

{
 The key realization is that there \textit{is} no Evolution Fairy.
There's no outside force \textit{deciding} which genes
ought to be promoted. Whatever happens, happens \textit{because of} the
genes themselves.}

{
 Genes for constructing (incrementally better) rattles must have
somehow ended up more frequent in the rattlesnake gene pool,
\textit{because of} the rattle. In this case it's
\textit{probably} because rattlesnakes with better rattles survive more
often---rather than mating more successfully, or having brothers that
reproduce more successfully, etc.}

{
 Maybe predators are wary of rattles and don't step
on the snake. Or maybe the rattle diverts attention from the
snake's head. (As George Williams suggests,
``The outcome of a fight between a dog and a viper
would depend very much on whether the dog initially seized the reptile
by the head or by the tail.'')}

{
 But that's just a snake's rattle.
There are much more complicated ways that a gene can cause copies of
itself to become more frequent in the next generation. Your brother or
sister shares half your genes. A gene that sacrifices one unit of
resources to bestow three units of resource on a brother, may promote
some copies of itself by sacrificing one of its constructed organisms.
(If you really want to know all the details and caveats, buy a book on
evolutionary biology; there is no royal road.)}

{
 The main point is that the \textit{gene's effect}
must \textit{cause} copies of that gene to become more frequent in the
next generation. There's no Evolution Fairy that
reaches in from outside. There's nothing which
\textit{decides} that some genes are
``helpful'' and should, therefore,
increase in frequency. It's just cause and effect,
starting from the genes themselves.}

{
 This explains the strange conflicting purposefulness of Nature,
and its frequent cruelty. It explains even better than a horde of
Shinto deities.}

{
 Why is so much of Nature at war with other parts of Nature?
Because there isn't one Evolution directing the whole
process. There's as many different
``evolutions'' as reproducing
populations. Rabbit genes are becoming more or less frequent in rabbit
populations. Fox genes are becoming more or less frequent in fox
populations. Fox genes which construct foxes that catch rabbits, insert
more copies of themselves in the next generation. Rabbit genes which
construct rabbits that evade foxes are \textit{naturally} more common
in the next generation of rabbits. Hence the phrase
``natural selection.''}

{
 Why is Nature cruel? You, a human, can look at an Ichneumon wasp,
and decide that it's cruel to eat your prey alive. You
can decide that if you're going to eat your prey alive,
you can at least have the decency to stop it from hurting. It would
scarcely cost the wasp anything to anesthetize its prey as well as
paralyze it. Or what about old elephants, who die of starvation when
their last set of teeth fall out? These elephants
aren't going to reproduce anyway. What would it cost
evolution---the evolution of elephants, rather---to ensure that the
elephant dies right away, instead of slowly and in agony? What would it
cost evolution to anesthetize the elephant, or give it pleasant dreams
before it dies? Nothing; that elephant won't reproduce
more or less either way.}

{
 If you were talking to a fellow human, trying to resolve a
conflict of interest, you would be in a good negotiating
position---would have an easy job of persuasion. It would cost so
little to anesthetize the prey, to let the elephant die without agony!
Oh please, won't you do it, kindly \ldots um \ldots}

{
 There's no one to argue with.}

{
 Human beings fake their justifications, figure out what they want
using one method, and then justify it using another method.
There's no Evolution of Elephants Fairy
that's trying to (a) figure out what's
best for elephants, and then (b) figure out how to \textit{justify} it
to the Evolutionary Overseer, who (c) doesn't want to
see reproductive fitness decreased, but is (d) willing to go along with
the painless-death idea, so long as it doesn't actually
harm any genes.}

{
 There's no advocate for the elephants anywhere in
the system.}

{
 Humans, who are often deeply concerned for the well-being of
animals, can be very persuasive in arguing how various kindnesses
wouldn't harm reproductive fitness at all. Sadly, the
evolution of elephants doesn't use a similar algorithm;
it doesn't select nice genes \textit{that can plausibly
be argued} to help reproductive fitness. Simply: genes that replicate
more often become more frequent in the next generation. Like water
flowing downhill, and equally benevolent.}

{
 A human, looking over Nature, starts thinking of all the ways
\textit{we} would design organisms. And then we tend to start
rationalizing reasons why our design improvements would increase
reproductive fitness---a political instinct, trying to sell your own
preferred option as matching the boss's favored
justification.}

{
 And so, amateur evolutionary biologists end up making all sorts of
wonderful and \textit{completely mistaken} predictions. Because the
amateur biologists are drawing their bottom line---and more
importantly, locating their prediction in hypothesis-space---using a
\textit{different} algorithm than evolutions use to draw \textit{their}
bottom lines.}

{
 A human engineer would have designed human taste buds to measure
how much of each nutrient we had, and how much we needed. When fat was
scarce, almonds or cheeseburgers would taste delicious. But if you
started to become obese, or if vitamins were lacking, lettuce would
taste delicious. But there is no Evolution of Humans Fairy, which
intelligently planned ahead and designed a general system for every
contingency. It was a reliable invariant of humans'
ancestral environment that calories were scarce. So genes whose
organisms loved calories, became more frequent. Like water flowing
downhill.}

{
 We are simply the embodied history of which organisms \textit{did
in fact} survive and reproduce, not which organisms \textit{ought
prudentially to have} survived and reproduced.}

{
 The human retina is constructed backward: The light-sensitive
cells are at the back, and the nerves emerge from the front and go back
\textit{through} the retina into the brain. Hence the blind spot. To a
human engineer, this looks simply stupid---and other organisms have
independently evolved retinas the right way around. Why not redesign
the retina?}

{
 The problem is that no \textit{single} mutation will reroute the
\textit{whole} retina simultaneously. A human engineer can redesign
multiple parts simultaneously, or plan ahead for future changes. But if
a single mutation breaks some vital part of the organism, it
doesn't matter what wonderful things a Fairy could
build on top of it---the organism dies and the gene decreases in
frequency.}

{
 If you turn around the retina's cells without also
reprogramming the nerves and optic cable, the system as a whole
won't work. It doesn't matter that, to
a Fairy or a human engineer, this is one step forward in redesigning
the retina. The organism is blind. Evolution has no foresight, it is
simply the frozen history of which organisms \textit{did in fact}
reproduce. Evolution is as blind as a halfway-redesigned retina.}

{
 Find a watch in a desert, said William Paley, and you can infer
the watchmaker. There were once those who denied this, who thought that
life ``just happened'' without need
of an optimization process, mice being spontaneously generated from
straw and dirty shirts.}

{
 If we ask who was \textit{more} correct---the theologians who
argued for a Creator-God, or the intellectually unfulfilled atheists
who argued that mice spontaneously generated---then the theologians
must be declared the victors: evolution is not God, but it is closer to
God than it is to pure random entropy. Mutation is random, but
selection is non-random. This doesn't mean an
intelligent Fairy is reaching in and selecting. It means
there's a non-zero statistical correlation between the
gene and how often the organism reproduces. Over a few million years,
that non-zero statistical correlation adds up to something very
powerful. It's not a god, but it's more
closely akin to a god than it is to snow on a television screen.}

{
 In a lot of ways, evolution \textit{is} like unto theology.
``Gods are ontologically distinct from
creatures,'' said Damien Broderick,
``or they're not worth the paper
they're written on.'' And indeed, the
Shaper of Life is not itself a creature. Evolution is bodiless, like
the Judeo-Christian deity. Omnipresent in Nature, immanent in the fall
of every leaf. Vast as a planet's surface. Billions of
years old. Itself unmade, arising naturally from the structure of
physics. Doesn't that all sound like something that
might have been said about God?}

{
 And yet the Maker has no mind, as well as no body. In some ways,
its handiwork is incredibly poor design by human standards. It is
internally divided. Most of all, it isn't
\textit{nice.}}

{
 In a way, Darwin \textit{discovered} God---a God that failed to
match the preconceptions of theology, and so passed unheralded. If
Darwin had discovered that life was created by an intelligent agent---a
bodiless mind that loves us, and will smite us with lightning if we
dare say otherwise---people would have said ``My gosh!
That's God!''}

{
 But instead Darwin discovered a strange alien God---not
comfortably ``ineffable,'' but
\textit{really genuinely different from us}. Evolution is not a God,
but if it were, it wouldn't be Jehovah. It would be H.
P. Lovecraft's Azathoth, the blind idiot God burbling
chaotically at the center of everything, surrounded by the thin
monotonous piping of flutes.}

{
 Which you might have predicted, if you had really \textit{looked}
at Nature.}

{
 So much for the claim some religionists make, that they believe in
a vague deity with a correspondingly high probability. Anyone who
\textit{really} believed in a vague deity, would have recognized their
strange inhuman creator when Darwin said
``Aha!''}

{
 So much for the claim some religionists make, that they are
waiting innocently curious for Science to discover God. Science has
already discovered the sort-of-godlike maker of humans---but it
wasn't what the religionists wanted to hear. They were
waiting for the discovery of \textit{their} God, the highly specific
God \textit{they} want to be there. They shall wait forever, for the
great discovery has \textit{already taken place}, and the winner is
Azathoth.}

{
 Well, more power to us humans. I like having a Creator I can
outwit. Beats being a pet. I'm glad it was Azathoth and
not Odin.}

\myendsectiontext


\bigskip

\mysection{The Wonder of Evolution}

{
 The wonder of evolution is that it works at all. }

{
 I mean that literally: If you want to marvel at evolution,
that's what's marvel-worthy.}

{
 How does optimization first arise in the universe? If an
intelligent agent designed Nature, who designed the intelligent agent?
Where is the first design that has no designer? The puzzle is
\textit{not} how the first stage of the bootstrap can be super-clever
and super-efficient; the puzzle is how it can happen \textit{at all.}}

{
 Evolution resolves the infinite regression, not by being
super-clever and super-efficient, but by being stupid and inefficient
and working anyway. \textit{This} is the marvel.}

{
 For professional reasons, I often have to discuss the slowness,
randomness, and blindness of evolution. Afterward someone says:
``\textit{You} just said that evolution
can't plan simultaneous changes, and that evolution is
very inefficient because mutations are random. Isn't
that what the \textit{creationists} say? That you
couldn't assemble a watch by randomly shaking the parts
in a box?''}

{
 But the reply to creationists is not that you \textit{can}
assemble a watch by shaking the parts in a box. The reply is that this
is \textit{not} how evolution works. If you think that evolution
\textit{does} work by whirlwinds assembling 747s, then the creationists
have successfully misrepresented biology to you;
they've sold the strawman.}

{
 The real answer is that complex machinery evolves either
incrementally, or by adapting previous complex machinery used for a new
purpose. Squirrels jump from treetop to treetop using just their
muscles, but the length they can jump depends to some extent on the
aerodynamics of their bodies. So now there are flying squirrels, so
aerodynamic they can glide short distances. If birds were wiped out,
the descendants of flying squirrels might reoccupy that ecological
niche in ten million years, gliding membranes transformed into wings.
And the creationists would say, ``What good is half a
wing? You'd just fall down and splat. How could
squirrelbirds possibly have evolved incrementally?''}

{
 That's how one complex adaptation can jump-start a
new complex adaptation. Complexity can also accrete incrementally,
starting from a single mutation.}

{
 First comes some gene $A$ which is simple, but at least a
\textit{little} useful on its own, so that $A$ increases to universality
in the gene pool. Now along comes gene $B$, which is only useful in the
presence of $A$, but $A$ is reliably present in the gene pool, so
there's a reliable selection pressure in favor of $B$.
Now a modified version of $A^{*}$ arises, which depends on
$B$, but doesn't break $B$'s dependency on
$A/A^{*}$. Then along comes $C$, which depends on
$A^{*}$ and $B$, and $B^{*}$, which depends on
$A^{*}$ and $C$. Soon you've got
``irreducibly complex'' machinery
that breaks if you take out any single piece.}

{
 And yet you can still visualize the trail backward to that single
piece: you can, without breaking the whole machine, make one piece less
dependent on another piece, and do this a few times, until you can take
out one whole piece \textit{without} breaking the machine, and so on
until you've turned a ticking watch back into a crude
sundial.}

{
 Here's an example: DNA stores information very
nicely, in a durable format that allows for exact duplication. A
ribosome turns that stored information into a sequence of amino acids,
a protein, which folds up into a variety of chemically active shapes.
The combined system, DNA and ribosome, can build all sorts of protein
machinery. But what good is DNA, without a ribosome that turns DNA
information into proteins? What good is a ribosome, without DNA to tell
it which proteins to make?}

{
 Organisms don't always leave fossils, and
evolutionary biology can't \textit{always} figure out
the incremental pathway. But in this case we \textit{do} know how it
happened. RNA shares with DNA the property of being able to carry
information and replicate itself, although RNA is less durable and
copies less accurately. And RNA also shares the ability of proteins to
fold up into chemically active shapes, though it's not
as versatile as the amino acid chains of proteins. Almost certainly,
RNA is the single $A$ which predates the mutually dependent
$A^{*}$ and $B$.}

{
 It's just as important to note that RNA does the
combined job of DNA and proteins \textit{poorly}, as that it does the
combined job at all. It's amazing enough that a
\textit{single molecule} can both store information and manipulate
chemistry. For it to do the job \textit{well} would be a wholly
unnecessary miracle.}

{
 What was the very first replicator ever to exist? It may well have
been an RNA strand, because \textit{by some strange coincidence}, the
chemical ingredients of RNA are chemicals that would have arisen
naturally on the prebiotic Earth of 4 billion years ago. Please note:
evolution does \textit{not} explain the origin of life; evolutionary
biology is \textit{not supposed to} explain the first replicator,
because the \textit{first} replicator does not come from another
replicator. Evolution describes statistical trends in replication. The
first replicator wasn't a statistical trend, it was a
pure accident. The notion that evolution should explain the
\textit{origin} of life is a pure strawman---more creationist
misrepresentation.}

{
 If you'd been watching the primordial soup on the
day of the first replicator, the day that reshaped the Earth, you would
not have been impressed by \textit{how well} the first replicator
replicated. The first replicator probably copied itself like a drunken
monkey on LSD. It would have exhibited none of the signs of careful
fine-tuning embodied in modern replicators, because the first
replicator was an \textit{accident.} It was not \textit{needful} for
that single strand of RNA, or chemical hypercycle, or pattern in clay,
to replicate gracefully. It just had to happen \textit{at all.} Even
so, it was probably very improbable, considered in an isolated
event---but it only had to happen \textit{once}, and there were a lot
of tide pools. A few billions of years later, the replicators are
walking on the Moon.}

{
 The first accidental replicator was the most important molecule in
the history of time. But if you praised it too highly, attributing to
it all sorts of wonderful replication-aiding capabilities, you would be
\textit{missing the whole point.}}

{
 Don't think that, in the political battle between
evolutionists and creationists, whoever praises evolution must be on
the side of science. Science has a very exact idea of the capabilities
of evolution. If you praise evolution one millimeter higher than this,
you're not ``fighting on
evolution's side'' against
creationism. You're being scientifically inaccurate,
full stop. You're falling into a creationist trap by
insisting that, yes, a whirlwind \textit{does} have the power to
assemble a 747! Isn't that amazing! How wonderfully
intelligent is evolution, how praiseworthy! Look at me,
I'm pledging my allegiance to science! The more nice
things I say about evolution, the more I must be on
evolution's side against the creationists!}

{
 But to praise evolution too highly destroys the \textit{real}
wonder, which is not \textit{how well} evolution designs things, but
that a naturally occurring process manages to design anything
\textit{at all.}}

{
 So let us dispose of the idea that evolution is a wonderful
designer, or a wonderful conductor of species destinies, which we human
beings ought to imitate. For human intelligence to imitate evolution as
a designer, would be like a sophisticated modern bacterium trying to
imitate the first replicator as a biochemist. As T. H. Huxley,
``Darwin's
Bulldog,'' put it:\footnote{Thomas Henry Huxley, \textit{Evolution and Ethics and Other
Essays} (Macmillan, 1894).\comment{1}}}

{
 Let us understand, once and for all, that the ethical progress of
society depends, not on imitating the cosmic process, still less in
running away from it, but in combating it.}

{
 Huxley didn't say that because he disbelieved in
evolution, but because he understood it all too well.}

\myendsectiontext


\bigskip

\mysection{Evolutions Are Stupid (But Work Anyway)}

{
 In the previous essay, I wrote:}

\begin{quote}
{
 Science has a very exact idea of the capabilities of evolution. If
you praise evolution one millimeter higher than this,
you're not ``fighting on
evolution's side'' against
creationism. You're being scientifically inaccurate,
full stop.}
\end{quote}

{
 In this essay I describe some well-known inefficiencies and
limitations of evolutions. I say
``evolutions,'' plural, because fox
evolution works at cross-purposes to rabbit evolution, and neither can
talk to snake evolution to learn how to build venomous fangs.}

{
 So I am talking about limitations of evolution here, but this does
not mean I am trying to sneak in creationism. This is standard
Evolutionary Biology 201. (583 if you must derive the equations.)
Evolutions, thus limited, can still explain observed biology; in fact
the limitations are necessary to make sense of it. Remember that the
wonder of evolutions is not how well they work, but that they work at
all.}

{
 Human intelligence is so complicated that no one has any good way
to calculate how efficient it is. Natural selection, though not simple,
is \textit{simpler than} a human brain; and correspondingly slower and
less efficient, as befits the first optimization process ever to exist.
In fact, evolutions are simple enough that we can calculate
\textit{exactly how stupid} they are.}

{
 Evolutions are slow. How slow? Suppose there's a
beneficial mutation that conveys a fitness advantage of 3\%: on
average, bearers of this gene have 1.03 times as many children as
non-bearers. Assuming that the mutation spreads at all, how long will
it take to spread through the whole population? That depends on the
population size. A gene conveying a 3\% fitness advantage, spreading
through a population of 100,000, would require an average of 768
generations to reach universality in the gene pool. A population of
500,000 would require 875 generations. The general formula is}

\begin{equation*}
 \text{Generations to fixation} = 2 \ln(N) / s,
\end{equation*}


{
 where N is the population size and $(1 + s)$ is the fitness. (If
each bearer of the gene has 1.03 times as many children as a
non-bearer, $s = 0.03$.) }

{
 Thus, if the population size were 1,000,000---the estimated
population in hunter-gatherer times---then it would require 2,763
generations for a gene conveying a 1\% advantage to spread through the
gene pool.\footnote{Dan Graur and Wen-Hsiung Li, \textit{Fundamentals of Molecular
Evolution}, 2nd ed. (Sunderland, MA: Sinauer Associates, 2000).\comment{1}}}

{
 This should not be surprising; genes have to do all their own work
of spreading. There's no Evolution Fairy who can watch
the gene pool and say, ``Hm, that gene seems to be
spreading rapidly---I should distribute it to
everyone.'' In a human market economy, someone who is
legitimately getting 20\% returns on investment---especially if
there's an obvious, clear mechanism behind it---can
rapidly acquire more capital from other investors; and others will
start duplicate enterprises. Genes have to spread without stock markets
or banks or imitators---as if Henry Ford had to make one car, sell it,
buy the parts for 1.01 more cars (on average), sell those cars, and
keep doing this until he was up to a million cars.}

{
 All this assumes that the gene spreads in the first place. Here
the equation is simpler and ends up not depending at all on population
size:}

\begin{equation*}
\text{Probability of fixation} = 2s.
\end{equation*}


{
 A mutation conveying a 3\% advantage (which is pretty darned
large, as mutations go) has a 6\% chance of spreading, at least on that
occasion.\footnote{John B. S. Haldane, ``A Mathematical Theory of
Natural and Artificial Selection,''
\textit{Mathematical Proceedings of the Cambridge Philosophical
Society} 23 (5 1927): 607--615, doi:10.1017/S0305004100011750.\comment{2}} Mutations can happen more than once, but
in a population of a million with a copying fidelity of
$10^{-8}$ errors per base per generation, you may have
to wait a hundred generations for another chance, and then it still has
only a 6\% chance of fixating. }

{
 Still, \textit{in the long run}, an evolution has a good shot at
getting there eventually. (This is going to be a running theme.)}

{
 \textit{Complex} adaptations take a \textit{very} long time to
evolve. First comes allele $A$, which is advantageous of itself, and
requires a thousand generations to fixate in the gene pool.
\textit{Only} then can another allele $B$, which depends on $A$, begin
rising to fixation. A fur coat is not a strong advantage unless the
environment has a \textit{statistically reliable} tendency to throw
cold weather at you. Well, genes form part of the environment of other
genes, and if $B$ depends on $A$, then $B$ will not have a strong advantage
unless $A$ is \textit{reliably} present in the genetic environment.}

{
 Let's say that $B$ confers a 5\% advantage in the
presence of $A$, no advantage otherwise. Then while $A$ is still at 1\%
frequency in the population, $B$ only confers its advantage 1 out of 100
times, so the average fitness advantage of $B$ is 0.05\%, and
$B$'s probability of fixation is 0.1\%. With a complex
adaptation, \textit{first} $A$ has to evolve over a thousand generations,
\textit{then} $B$ has to evolve over another thousand generations,
\textit{then} $A^{*}$ evolves over another thousand
generations \ldots and several million years later,
you've got a new complex adaptation.}

{
 Then other evolutions don't imitate it. If snake
evolution develops an amazing new venom, it doesn't
help fox evolution or lion evolution.}

{
 Contrast all this to a human programmer, who can design a new
complex mechanism with a hundred interdependent parts over the course
of \textit{a single afternoon}. How is this even \textit{possible}? I
don't know all the answer, and my guess is that neither
does science; human brains are much more complicated than evolutions. I
could wave my hands and say something like
``goal-directed backward chaining using combinatorial
modular representations,'' but you would not thereby
be enabled to design your own human. Still: Humans can foresightfully
design new parts in anticipation of later designing other new parts;
produce coordinated simultaneous changes in interdependent machinery;
learn by observing single test cases; zero in on problem spots and
think abstractly about how to solve them; and prioritize which tweaks
are worth trying, rather than waiting for a cosmic ray strike to
produce a good one. By the standards of natural selection, this is
simply magic.}

{
 Humans can do things that evolutions probably
can't do \textit{period} over the expected lifetime of
the universe. As the eminent biologist Cynthia Kenyon once put it at a
dinner I had the honor of attending, ``One grad
student can do things in an hour that evolution could not do in a
billion years.'' According to
biologists' best current knowledge, evolutions have
invented a fully rotating wheel on a grand total of \textit{three}
occasions.}

{
 And don't forget the part where the programmer
posts the code snippet to the Internet.}

{
 Yes, some evolutionary handiwork is impressive even by comparison
to the best technology of \textit{Homo sapiens.} But our Cambrian
explosion only started, we only really began accumulating knowledge,
around \ldots what, four hundred years ago? In some ways, biology still
excels over the best human technology: we can't build a
self-replicating system the size of a butterfly. In other ways, human
technology leaves biology in the dust. We got wheels, we got steel, we
got guns, we got knives, we got pointy sticks; we got rockets, we got
transistors, we got nuclear power plants. With every passing decade,
that balance tips further.}

{
 So, once again: for a human to look to natural selection as
inspiration on the art of design is like a sophisticated modern
bacterium trying to imitate the first awkward
replicator's biochemistry. The first replicator would
be eaten instantly if it popped up in today's
competitive ecology. The same fate would accrue to any human planner
who tried making random point mutations to their strategies and waiting
768 iterations of testing to adopt a 3\% improvement.}

{
 Don't praise evolutions one millimeter more than
they deserve.}

{
 \textit{Coming up next: More exciting mathematical bounds on
evolution!}}

\myendsectiontext


\bigskip

\mysection{No Evolutions for Corporations or Nanodevices}

\begin{quotation}
{
 The laws of physics and the rules of math don't
cease to apply. That leads me to believe that evolution
doesn't stop. That further leads me to believe that
nature---bloody in tooth and claw, as some have termed it---will simply
be taken to the next level \ldots}

{
 [Getting rid of Darwinian evolution is] like trying to get rid of
gravitation. So long as there are limited resources and multiple
competing actors capable of passing on characteristics, you have
selection pressure.}

{\raggedleft
 {}---Perry Metzger, predicting that the reign of natural selection
would continue into the indefinite future
\par}
\end{quotation}


{
 In evolutionary biology, as in many other fields, it is important
to think quantitatively rather than qualitatively. Does a beneficial
mutation ``sometimes spread, but not
always''? Well, a psychic power would be a beneficial
mutation, so you'd expect it to spread, right? Yet this
is qualitative reasoning, not quantitative---if X is true, then Y is
true; if psychic powers are beneficial, they may spread. In Evolutions
Are Stupid, I described the equations for a beneficial
mutation's probability of fixation, roughly twice the
fitness advantage (6\% for a 3\% advantage). Only this kind of
numerical thinking is likely to make us realize that mutations which
are \textit{only rarely useful} are extremely unlikely to spread, and
that it is practically impossible for complex adaptations to arise
without constant use. If psychic powers really existed, we should
expect to see everyone using them all the time---not just because they
would be so amazingly useful, but because otherwise they
couldn't have evolved in the first place.}

{
 ``So long as there are limited resources and
multiple competing actors capable of passing on characteristics, you
have selection pressure.'' This is qualitative
reasoning. \textit{How much} selection pressure?}

{
 While there are several candidates for the most important equation
in evolutionary biology, I would pick Price's Equation,
which in its simplest formulation reads:}

\begin{equation*}
  \triangle \cov(v_{i}, z_{i})
\end{equation*}


{\centering
 change in average characteristic = covariance(relative fitness,
characteristic).
\par}


\bigskip

{
 This is a \textit{very} powerful and general formula. For example,
a particular gene for height can be the Z, the characteristic that
changes, in which case Price's Equation says that the
change in the probability of possessing this gene equals the covariance
of the gene with reproductive fitness. Or you can consider
\textit{height in general} as the characteristic Z, apart from any
particular genes, and Price's Equation says that the
change in height in the next generation will equal the covariance of
height with relative reproductive fitness. }

{
 (At least, this is true so long as height is straightforwardly
heritable. If nutrition improves, so that a fixed genotype becomes
taller, you have to add a correction term to Price's
Equation. If there are complex nonlinear interactions between many
genes, you have to either add a correction term, or calculate the
equation in such a complicated way that it ceases to enlighten.)}

{
 Many enlightenments may be attained by studying the different
forms and derivations of Price's Equation. For example,
the final equation says that the average characteristic changes
according to its \textit{covariance} with \textit{relative} fitness,
rather than its \textit{absolute} fitness. This means that if a Frodo
gene saves its whole species from extinction, the average Frodo
characteristic does not increase, since Frodo's act
benefited all genotypes equally and did not \textit{covary} with
\textit{relative} fitness.}

{
 It is said that Price became so disturbed with the implications of
his equation for altruism that he committed suicide, though he may have
had other issues. (\textit{Overcoming Bias} does not advocate
committing suicide after studying Price's Equation.)}

{
 One of the enlightenments which may be gained by meditating upon
Price's Equation is that ``limited
resources'' and ``multiple competing
actors capable of passing on characteristics'' are
\textit{not sufficient} to give rise to an evolution.
``Things that replicate themselves''
is not a sufficient condition. Even ``competition
between replicating things'' is not sufficient.}

{
 Do corporations evolve? They certainly compete. They occasionally
spin off children. Their resources are limited. They sometimes die.}

{
 But how much does the child of a corporation resemble its parents?
Much of the personality of a corporation derives from key officers, and
CEOs cannot divide themselves by fission. Price's
Equation only operates to the extent that characteristics are heritable
across generations. If great-great-grandchildren don't
much resemble their great-great-grandparents, you won't
get more than four generations' worth of cumulative
selection pressure---anything that happened more than four generations
ago will blur itself out. Yes, the personality of a corporation can
influence its spinoff---but that's nothing like the
heritability of DNA, which is digital rather than analog, and can
transmit itself with $10^{-8}$ errors per base per
generation.}

{
 With DNA you have heritability lasting for \textit{millions} of
generations. That's how complex adaptations can arise
by pure evolution---the digital DNA lasts long enough for a gene
conveying 3\% advantage to spread itself over 768 generations, and then
another gene dependent on it can arise. Even if corporations replicated
with digital fidelity, they would currently be at most ten generations
into the RNA World.}

{
 Now, corporations are certainly \textit{selected}, in the sense
that incompetent corporations go bust. This should logically make you
more likely to observe corporations with features contributing to
competence. And in the same sense, any star that goes nova shortly
after it forms, is less likely to be visible when you look up at the
night sky. But if an accident of stellar dynamics makes one star burn
longer than another star, that doesn't make it more
likely that future stars will also burn longer---the feature will not
be copied onto other stars. We should not expect future astrophysicists
to discover \textit{complex} internal features of stars which seem
designed to help them burn longer. That kind of mechanical adaptation
requires much larger \textit{cumulative} selection pressures than a
once-off winnowing.}

{
 Think of the principle introduced in Einstein's
Arrogance---that the vast majority of the evidence required to think of
General Relativity had to go into raising that one particular equation
to the level of Einstein's personal attention; the
amount of evidence required to raise it from a deliberately considered
possibility to 99.9\% certainty was trivial by comparison. In the same
sense, complex features of corporations that require hundreds of bits
to specify are produced primarily by human intelligence, not a handful
of generations of low-fidelity evolution. In biology, the mutations are
purely random and evolution supplies thousands of bits of cumulative
selection pressure. In corporations, humans offer up thousand-bit
intelligently designed complex
``mutations,'' and then the further
selection pressure of ``Did it go bankrupt or
not?'' accounts for a handful of additional bits in
explaining what you see.}

{
 Advanced molecular nanotechnology---the artificial sort, not
biology---should be able to copy itself with digital fidelity through
thousands of generations. Would Price's Equation
thereby gain a foothold?}

{
 Correlation is covariance divided by variance, so if A is highly
predictive of B, there can be a strong
``correlation'' between them even if
A is ranging from 0 to 9 and B is only ranging from 50.0001 and
50.0009. Price's Equation runs on \textit{covariance}
of characteristics with reproduction---not correlation! If you can
compress variance in characteristics into a tiny band, the covariance
goes way down, and so does the cumulative change in the
characteristic.}

{
 The Foresight Institute suggests, among other sensible proposals,
that the replication instructions for any nanodevice should be
encrypted. Moreover, encrypted such that flipping a single bit of the
encoded instructions will entirely scramble the decrypted output. If
all nanodevices produced are precise molecular copies, and moreover,
any mistakes on the assembly line are not heritable because the
offspring got a digital copy of the original encrypted instructions for
use in making grandchildren, then your nanodevices
ain't gonna be doin' much evolving.}

{
 You'd still have to worry about
prions---self-replicating assembly errors apart from the encrypted
instructions, where a robot arm fails to grab a carbon atom that is
used in assembling a homologue of itself, and this causes the
offspring's robot arm to likewise fail to grab a carbon
atom, etc., even with all the encrypted instructions remaining
constant. But how much \textit{correlation} is there likely to be,
between this sort of transmissible error, and a \textit{higher}
reproductive rate? Let's say that one nanodevice
produces a copy of itself every 1,000 seconds, and the new nanodevice
is magically more efficient (it not only has a prion, it has a
\textit{beneficial} prion) and copies itself every 999.99999 seconds.
It needs one less carbon atom attached, you see. That's
not a whole lot of variance in reproduction, so it's
not a whole lot of covariance either.}

{
 And how often will these nanodevices need to replicate? Unless
they've got more atoms available than exist in the
solar system, or for that matter, the visible Universe, only a small
number of generations will pass before they hit the resource wall.
``Limited resources'' are not a
sufficient condition for evolution; you need the frequently iterated
death of a substantial fraction of the population to free up resources.
Indeed, ``generations'' is not so
much an integer as an integral over the fraction of the population that
consists of newly created individuals.}

{
 This is, to me, the most frightening thing about gray goo or
nanotechnological weapons---that they could eat the whole Earth and
then that would be \textit{it}, nothing interesting would happen
afterward. Diamond is stabler than proteins held together by van der
Waals forces, so the goo would only need to reassemble some pieces of
itself when an asteroid hit. Even if prions were a powerful enough
idiom to support evolution at all---evolution is slow enough with
digital DNA!---fewer than 1.0 generations might pass between when the
goo ate the Earth and when the Sun died.}

{
 To sum up, if you have \textit{all} of the following properties:}

\begin{itemize}
\item  Entities that replicate;
\item  Substantial variation in their characteristics;
\item  Substantial variation in their reproduction;
\item Persistent correlation between the characteristics and
reproduction;
\item  High-fidelity long-range heritability in characteristics;
\item  Frequent birth of a significant fraction of the breeding
  population;
\item And \textit{all} this remains true through \textit{many}
  iterations \ldots
\end{itemize}

{
 \textit{Then} you will have significant \textit{cumulative}
selection pressures, enough to produce complex adaptations by the force
of evolution.}

\myendsectiontext

\mysection{Evolving to Extinction}

{
 It is a \textit{very} common misconception that an evolution works
for the good of its species. Can you remember hearing someone talk
about two rabbits breeding eight rabbits and thereby
``contributing to the survival of their
species''? A modern evolutionary biologist would
never say such a thing; they'd sooner breed with a
rabbit. }

{
 It's yet another case where you've
got to simultaneously consider multiple abstract concepts and keep them
distinct. Evolution doesn't \textit{operate} on
particular individuals; individuals keep whatever genes
they're born with. Evolution operates on a reproducing
population, a species, over time. There's a natural
tendency to think that if an Evolution Fairy is \textit{operating on}
the species, she must be \textit{optimizing for} the species. But what
really changes are the gene frequencies, and frequencies
don't increase or decrease according to how much the
gene helps the species as a whole. As we shall later see,
it's quite possible for a species to evolve to
extinction.}

{
 Why are boys and girls born in roughly equal numbers? (Leaving
aside crazy countries that use artificial gender selection
technologies.) To see why this is surprising, consider that 1 male can
impregnate 2, 10, or 100 females; it wouldn't seem that
you need the same number of males as females to ensure the survival of
the species. This is even more surprising in the vast majority of
animal species where the male contributes very little to raising the
children---humans are extraordinary, even among primates, for their
level of paternal investment. Balanced gender ratios are found even in
species where the male impregnates the female and vanishes into the
mist.}

{
 Consider two groups on different sides of a mountain; in group A,
each mother gives birth to 2 males and 2 females; in group B, each
mother gives birth to 3 females and 1 male. Group A and group B will
have the same number of children, but group B will have 50\% more
grandchildren and 125\% more great-grandchildren. You might think this
would be a significant evolutionary advantage.}

{
 But consider: The rarer males become, the more reproductively
valuable they become---not to the \textit{group}, but to the
\textit{individual} parent. Every child has one male and one female
parent. Then in every generation, the total genetic contribution from
all males equals the total genetic contribution from all females. The
fewer males, the greater the individual genetic contribution per male.
If all the females around you are doing what's good for
the group, what's good for the species, and birthing 1
male per 10 females, you can make a genetic \textit{killing} by
birthing all males, each of whom will have (on average) ten times as
many grandchildren as their female cousins.}

{
 So while group selection ought to favor more girls, individual
selection favors equal investment in male and female offspring. Looking
at the statistics of a maternity ward, you can see at a glance that the
quantitative balance between group selection forces and individual
selection forces is overwhelmingly tilted in favor of individual
selection in \textit{Homo sapiens.}}

{
 (Technically, this isn't quite a glance.
Individual selection favors equal \textit{parental investments} in male
and female offspring. If males cost half as much to birth and/or raise,
twice as many males as females will be born at the evolutionarily
stable equilibrium. If the same number of males and females were born
in the population at large, but males were twice as cheap to birth,
then you could again make a genetic killing by birthing more males. So
the maternity ward should reflect the balance of parental opportunity
costs, in a hunter-gatherer society, between raising boys and raising
girls; and you'd have to assess that somehow. But ya
know, it doesn't seem all that much \textit{more}
reproductive-opportunity-costly for a hunter-gatherer family to raise a
girl, so it's kinda suspicious that around the same
number of boys are born as girls.)}

{
 Natural selection isn't about groups, or species,
or even \textit{individuals.} In a sexual species, an individual
organism doesn't evolve; it keeps whatever genes
it's born with. An individual is a once-off collection
of genes that will never reappear; how can you select on that? When you
consider that nearly all of your ancestors are dead,
it's clear that ``survival of the
fittest'' is a tremendous misnomer.
``Replication of the fitter'' would
be more accurate, although technically fitness is \textit{defined} only
in terms of replication.}

{
 Natural selection is really about \textit{gene frequencies}. To
get a complex adaptation, a machine with multiple dependent parts, each
new gene as it evolves depends on the other genes being reliably
present in its genetic environment. They must have high frequencies.
The more complex the machine, the higher the frequencies must be. The
signature of natural selection occurring is a gene rising from
0.00001\% of the gene pool to 99\% of the gene pool. This is the
information, in an information-theoretic sense; and this is what must
happen for large complex adaptations to evolve.}

{
 The real struggle in natural selection is not the competition of
organisms for resources; this is an ephemeral thing when all the
participants will vanish in another generation. The real struggle is
the competition of alleles for frequency in the gene pool. This is the
lasting consequence that creates lasting information. The two rams
bellowing and locking horns are only passing shadows.}

{
 It's perfectly possible for an allele to spread to
fixation by outcompeting an alternative allele which was
``better for the species.'' If the
Flying Spaghetti Monster magically created a species whose gender mix
was perfectly optimized to ensure the survival \textit{of the
species}{}---the optimal gender mix to bounce back reliably from
near-extinction events, adapt to new niches, et cetera---then the
evolution would rapidly degrade this species optimum back into the
individual-selection optimum of equal parental investment in males and
females.}

{
 Imagine a ``Frodo gene'' that
sacrifices its vehicle \textit{to save its entire species} from an
extinction event. What happens to the allele frequency as a result? It
goes down. Kthxbye.}

{
 If species-level extinction threats occur regularly (call this a
``Buffy environment'') then the
Frodo gene will systematically decrease in frequency and vanish, and
soon thereafter, so will the species.}

{
 A hypothetical example? Maybe. If the human species was going to
stay biological for another century, it would be a good idea to start
cloning Gandhi.}

{
 In viruses, there's the tension between individual
viruses replicating as fast as possible, versus the benefit of leaving
the host alive long enough to transmit the illness. This is a good
real-world example of group selection, and if the virus evolves to a
point on the fitness landscape where the group selection pressures fail
to overcome individual pressures, the virus could vanish shortly
thereafter. I don't know if a disease has ever been
caught in the act of evolving to extinction, but it's
probably happened any number of times.}

{
 Segregation-distorters subvert the mechanisms that usually
guarantee fairness of sexual reproduction. For example, there is a
segregation-distorter on the male sex chromosome of some mice which
causes only male children to be born, all carrying the
segregation-distorter. Then these males impregnate females, who give
birth to only male children, and so on. You might cry
``This is cheating!'' but
that's a human perspective; the reproductive fitness of
this allele is extremely high, since it produces twice as many copies
of itself in the succeeding generation as its nonmutant alternative.
Even as females become rarer and rarer, males carrying this gene are no
less likely to mate than any other male, and so the
segregation-distorter remains twice as fit as its alternative allele.
It's speculated that real-world group selection may
have played a role in keeping the frequency of this gene as low as it
seems to be. In which case, if mice were to evolve the ability to fly
and migrate for the winter, they would probably form a single
reproductive population, and would evolve to extinction as the
segregation-distorter evolved to fixation.}

{
 Around 50\% of the total genome of maize consists of transposons,
DNA elements whose primary function is to copy themselves \textit{into
other locations of DNA.} A class of transposons called
``P elements'' seem to have first
appeared in Drosophila only in the middle of the twentieth century, and
spread to every population of the species within 50 years. The
``Alu sequence'' in humans, a
300-base transposon, is repeated between 300,000 and a million times in
the human genome. This may not extinguish a species, but it
doesn't help it; transposons cause more mutations which
are as always mostly harmful, decrease the effective copying fidelity
of DNA. Yet such cheaters are extremely fit.}

{
 Suppose that in some sexually reproducing species, a
\textit{perfect} DNA-copying mechanism is invented. Since most
mutations are detrimental, this gene complex is an advantage to its
holders. Now you might wonder about beneficial mutations---they do
happen occasionally, so wouldn't the unmutable be at a
disadvantage? But in a sexual species, a beneficial mutation that began
in a mutable can spread to the descendants of unmutables as well. The
mutables suffer from degenerate mutations in each generation; and the
unmutables can sexually acquire, and thereby benefit from, any
beneficial mutations that occur in the mutables. Thus the mutables have
a pure disadvantage. The perfect DNA-copying mechanism rises in
frequency to fixation. Ten thousand years later there's
an ice age and the species goes out of business. It evolved to
extinction.}

{
 The ``bystander effect'' is
that, when someone is in trouble, solitary individuals are more likely
to intervene than groups. A college student apparently having an
epileptic seizure was helped 85\% of the time by a single bystander,
and 31\% of the time by five bystanders. I speculate that even if the
kinship relation in a hunter-gatherer tribe was strong enough to create
a selection pressure for helping individuals not directly related, when
\textit{several} potential helpers were present, a genetic arms race
might occur to be the \textit{last} one to step forward. Everyone
delays, hoping that someone else will do it. Humanity is facing
multiple species-level extinction threats right now, and I gotta tell
ya, there ain't a lot of people
steppin' forward. If we lose this fight because
virtually no one showed up on the battlefield, then---like a
probably-large number of species which we don't see
around today---we will have evolved to extinction.}

{
 Cancerous cells do pretty well in the body, prospering and
amassing more resources, far outcompeting their more obedient
counterparts. For a while.}

{
 Multicellular organisms can only exist because
they've evolved powerful internal mechanisms to
\textit{outlaw evolution}. If the cells start evolving, they rapidly
evolve to extinction: the organism dies.}

{
 So praise not evolution for the solicitous concern it shows for
the individual; nearly all of your ancestors are dead. Praise not
evolution for the solicitous concern it shows for a species; no one has
ever found a complex adaptation which can only be interpreted as
operating to preserve a species, and the mathematics would seem to
indicate that this is virtually impossible. Indeed,
it's perfectly possible for a species to evolve to
extinction. Humanity may be finishing up the process right now. You
can't even praise evolution for the solicitous concern
it shows for genes; the battle between two alternative alleles at the
same location is a zero-sum game for frequency.}

{
 Fitness is not always your friend.}

\myendsectiontext

\mysection{The Tragedy of Group Selectionism}

{
 Before 1966, it was not unusual to see serious biologists
advocating evolutionary hypotheses that we would now regard as magical
thinking. These muddled notions played an important historical role in
the development of later evolutionary theory, error calling forth
correction; like the folly of English kings provoking into existence
the Magna Carta and constitutional democracy. }

{
 As an example of romance, Vero Wynne-Edwards, Warder Allee, and J.
L. Brereton, among others, believed that predators would voluntarily
restrain their breeding to avoid overpopulating their habitat and
exhausting the prey population.}

{
 But evolution does not open the floodgates to arbitrary purposes.
You cannot explain a rattlesnake's rattle by saying
that it exists to benefit other animals who would otherwise be bitten.
No outside Evolution Fairy decides when a gene \textit{ought} to be
promoted; the gene's effect must somehow
\textit{directly cause} the gene to be more prevalent in the next
generation. It's clear why our human sense of
aesthetics, witnessing a population crash of foxes
who've eaten all the rabbits, cries
``Something should've been
done!'' But how would a gene complex for
\textit{restraining reproduction}{}---of all things!---cause itself to
become more frequent in the next generation?}

{
 A human being designing a neat little toy ecology---for
entertainment purposes, like a model railroad---might be annoyed if
their painstakingly constructed fox and rabbit populations
self-destructed by the foxes eating all the rabbits and then dying of
starvation themselves. So the human would tinker with the toy
ecology---a fox-breeding-restrainer is the obvious solution that leaps
to our human minds---until the ecology looked nice and neat. Nature has
no human, of course, but \textit{that} needn't stop
us---now that we know what \textit{we} want on \textit{aesthetic}
grounds, we just have to come up with a plausible argument that
persuades Nature to want the \textit{same} thing on
\textit{evolutionary} grounds.}

{
 Obviously, selection on the level of the individual
won't produce individual restraint in breeding.
Individuals who reproduce unrestrainedly will, naturally, produce more
offspring than individuals who restrain themselves.}

{
 (Individual selection will not produce \textit{individual
sacrifice of breeding opportunities}. Individual selection can
certainly produce individuals who, after acquiring all available
resources, use those resources to produce four big eggs instead of
eight small eggs---\textit{not} to conserve social resources, but
because that is the \textit{individual} sweet spot for (number of
eggs){\texttimes}(egg survival probability). This does not get rid of
the commons problem.)}

{
 But suppose that the species population was broken up into
subpopulations, which were mostly isolated, and only occasionally
interbred. Then, surely, subpopulations that restrained their breeding
would be less likely to go extinct, and would send out more messengers,
and create new colonies to reinhabit the territories of crashed
populations.}

{
 The problem with this scenario wasn't that it was
mathematically \textit{impossible.} The problem was that it was
\textit{possible but very difficult.}}

{
 The fundamental problem is that it's not only
restrained breeders who reap the benefits of restrained breeding. If
some foxes refrain from spawning cubs who eat rabbits, then the uneaten
rabbits don't go to \textit{only} cubs who carry the
restrained-breeding adaptation. The unrestrained foxes, and their many
more cubs, will happily eat any rabbits left unhunted. The only way the
restraining gene can survive against this pressure, is if the benefits
of restraint preferentially go to restrainers.}

{
 Specifically, the requirement is $C/B <
F_{ST}$ where $C$ is the cost of altruism to the donor, $B$ is
the benefit of altruism to the recipient, and $F_{ST}$ is
the spatial structure of the population: the average
\textit{relatedness} between a randomly selected organism and its
randomly selected neighbor, where a
``neighbor'' is any other fox who
benefits from an altruistic fox's
restraint.\footnote{David Sloan Wilson, ``A Theory of Group
Selection,'' \textit{Proceedings of the National
Academy of Sciences of the United States of America} 72, no. 1 (1975):
143--146.\comment{1}}}

{
 So is the cost of restrained breeding sufficiently small, and the
empirical benefit of less famine sufficiently large, compared to the
empirical spatial structure of fox populations and rabbit populations,
that the group selection argument can work?}

{
 The math suggests this is pretty unlikely. In this simulation, for
example, the cost to altruists is 3\% of fitness, pure altruist groups
have a fitness twice as great as pure selfish groups, the subpopulation
size is 25, and 20\% of all deaths are replaced with messengers from
another group: the result is polymorphic for selfishness and altruism.
If the subpopulation size is doubled to 50, selfishness is fixed; if
the cost to altruists is increased to 6\%, selfishness is fixed; if the
altruistic benefit is decreased by half, selfishness is fixed or in
large majority. Neighborhood-groups must be very small, with only
around 5 members, for group selection to operate when the cost of
altruism exceeds 10\%. This doesn't seem plausibly true
of foxes restraining their breeding.}

{
 You can guess by now, I think, that the group selectionists
ultimately lost the scientific argument. The kicker was not the
mathematical argument, but empirical observation: foxes
\textit{didn't} restrain their breeding (I forget the
exact species of dispute; it wasn't foxes and rabbits),
and indeed, predator-prey systems crash all the time. Group
selectionism would later revive, somewhat, in drastically different
form---mathematically speaking, there \textit{is} neighborhood
structure, which implies nonzero group selection pressure \textit{not}
necessarily capable of overcoming countervailing individual selection
pressure, and if you don't take it into account your
math will be wrong, full stop. And evolved enforcement mechanisms (not
originally postulated) change the game entirely. So why is this
now-historical scientific dispute worthy material for
\textit{Overcoming Bias}?}

{
 A decade after the controversy, a biologist had a fascinating
idea. The mathematical conditions for group selection overcoming
individual selection were too extreme to be found in Nature. Why not
create them artificially, in the laboratory? Michael J. Wade proceeded
to do just that, repeatedly selecting populations of insects for low
numbers of adults per subpopulation.\footnote{Michael J. Wade, ``Group selections among
laboratory populations of Tribolium,''
\textit{Proceedings of the National Academy of Sciences of the United
States of America} 73, no. 12 (1976): 4604--4607,
doi:10.1073/pnas.73.12.4604.\comment{2}} And what was
the result? Did the insects restrain their breeding and live in quiet
peace with enough food for all?}

{
 No; the adults adapted to cannibalize eggs and larvae, especially
female larvae.}

{
 \textit{Of course} selecting for small subpopulation sizes would
not select for individuals who restrained their \textit{own} breeding;
it would select for individuals who ate \textit{other}
individuals' children. Especially the girls.}

{
 Once you have that experimental result in hand---and
it's massively obvious in retrospect---then it suddenly
becomes clear how the original group selectionists allowed romanticism,
a human sense of aesthetics, to cloud their predictions of Nature.}

{
 This is an archetypal example of a missed Third Alternative,
resulting from a rationalization of a predetermined bottom line that
produced a fake justification and then motivatedly stopped. The group
selectionists didn't start with clear, fresh minds,
happen upon the idea of group selection, and \textit{neutrally}
extrapolate forward the probable outcome. They started out with the
beautiful idea of fox populations voluntarily restraining their
reproduction to what the rabbit population would bear, Nature in
perfect harmony; then they searched for a reason why this would happen,
and came up with the idea of group selection; then, since they knew
what they \textit{wanted} the outcome of group selection to be, they
didn't look for any \textit{less} beautiful and
aesthetic adaptations that group selection would be more likely to
promote instead. If they'd \textit{really} been trying
to calmly and neutrally predict the result of selecting for small
subpopulation sizes resistant to famine, they would have thought of
cannibalizing other organisms' children or some
similarly ``ugly''
outcome---\textit{long before} they imagined anything so evolutionarily
outré as \textit{individual restraint in breeding!}}

{
 This also illustrates the point I was trying to make in
Einstein's Arrogance: With large answer spaces, nearly
all of the real work goes into promoting one possible answer to the
point of being singled out for attention. If a hypothesis is improperly
promoted to your attention---your sense of aesthetics suggests a
beautiful way for Nature to be, and yet natural selection
doesn't involve an Evolution Fairy who shares your
appreciation---then this alone may seal your doom, unless you can
manage to clear your mind entirely and start over.}

{
 In principle, the world's stupidest person may say
the Sun is shining, but that doesn't make it dark out.
Even if an answer is suggested by a lunatic on LSD, you should be able
to neutrally calculate the evidence for and against, and if necessary,
un-believe.}

{
 In practice, the group selectionists were doomed because their
bottom line was originally suggested by their sense of aesthetics, and
Nature's bottom line was produced by natural selection.
These two processes had no principled reason for their outputs to
correlate, and indeed they didn't. All the furious
argument afterward didn't change that.}

{
 If you start with your own desires for what Nature should do,
consider Nature's own observed reasons for doing
things, and then rationalize an extremely persuasive argument for why
Nature should produce your preferred outcome for
Nature's own reasons, then Nature, alas, \textit{still}
won't listen. The universe has no mind and is not
subject to clever political persuasion. You can argue all day why
gravity should really make water flow \textit{up}hill, and the water
just ends up in the same place regardless. It's like
the universe plain isn't listening. J. R. Molloy said:
``Nature is the ultimate bigot, because it is
obstinately and intolerantly devoted to its own prejudices and
absolutely refuses to yield to the most persuasive rationalizations of
humans.''}

{
 I often recommend evolutionary biology to friends just because the
modern field tries to train its students against rationalization, error
calling forth correction. Physicists and electrical engineers
don't have to be carefully trained to avoid
anthropomorphizing electrons, because electrons don't
exhibit mindish behaviors. Natural selection creates purposefulnesses
which are alien to humans, and students of evolutionary theory are
warned accordingly. It's good training for any thinker,
but it is \textit{especially} important if you want to think clearly
about other weird mindish processes that do not work like you do.}

\myendsectiontext


\bigskip

\mysection{Fake Optimization Criteria}

{
 I've previously dwelt in considerable length upon
forms of rationalization whereby our beliefs appear to match the
evidence much more strongly than they actually do. And
I'm not overemphasizing the point, either. If we could
beat this fundamental metabias and see what every hypothesis
\textit{really} predicted, we would be able to recover from almost any
other error of fact. }

{
 The mirror challenge for decision theory is seeing which option a
choice criterion \textit{really} endorses. If your stated moral
principles call for you to provide laptops to everyone, does that
\textit{really} endorse buying a \$1 million gem-studded laptop for
yourself, or spending the same money on shipping 5,000 OLPCs?}

{
 We seem to have evolved a knack for arguing that practically any
goal implies practically any action. A phlogiston theorist explaining
why magnesium gains weight when burned has nothing on an Inquisitor
explaining why God's infinite love for all His children
requires burning some of them at the stake.}

{
 There's no mystery about this. Politics was a
feature of the ancestral environment. We are descended from those who
argued most persuasively that the good of the tribe meant executing
their hated rival Uglak. (We sure ain't descended from
Uglak.)}

{
 And yet \ldots is it possible to \textit{prove} that if Robert
Mugabe cared \textit{only} for the good of Zimbabwe, he would resign
from its presidency? You can \textit{argue} that the policy follows
from the goal, but haven't we just seen that humans can
match up any goal to any policy? How do you know that
you're right and Mugabe is wrong? (There are a number
of reasons this is a good guess, but bear with me here.)}

{
 Human motives are manifold and obscure, our decision processes as
vastly complicated as our brains. And the world itself is vastly
complicated, on every choice of real-world policy. Can we even
\textit{prove} that human beings are rationalizing---that
we're systematically distorting the link from
principles to policy---when we lack a single firm place on which to
stand? When there's no way to find out \textit{exactly}
what even a single optimization criterion implies? (Actually, you can
just observe that people \textit{disagree} about office politics in
ways that strangely correlate to their own interests, while
simultaneously denying that any such interests are at work. But again,
bear with me here.)}

{
 Where is the standardized, open-source, generally intelligent,
consequentialist optimization process into which we can feed a complete
morality as an XML file, to find out what that morality \textit{really}
recommends when applied to our world? Is there even a single real-world
case where we can know \textit{exactly} what a choice criterion
recommends? Where is the \textit{pure} moral reasoner---of known
utility function, purged of all other stray desires that might distort
its optimization---whose trustworthy output we can contrast to human
rationalizations of the same utility function?}

{
 Why, it's our old friend the alien god, of course!
Natural selection is guaranteed free of all mercy, all love, all
compassion, all aesthetic sensibilities, all political factionalism,
all ideological allegiances, all academic ambitions, all
libertarianism, all socialism, all Blue and all Green. Natural
selection doesn't \textit{maximize} its criterion of
inclusive genetic fitness---it's not that smart. But
when you look at the output of natural selection, you are guaranteed to
be looking at an output that was optimized \textit{only} for inclusive
genetic fitness, and not the interests of the US agricultural
industry.}

{
 In the case histories of evolutionary science---in, for example,
The Tragedy of Group Selectionism---we can directly compare human
rationalizations to the result of \textit{pure} optimization for a
known criterion. What did Wynne-Edwards think would be the result of
group selection for small subpopulation sizes? Voluntary individual
restraint in breeding, and enough food for everyone. What was the
actual laboratory result? Cannibalism.}

{
 Now you might ask: Are these case histories of evolutionary
science really relevant to human morality, which
doesn't give two figs for inclusive genetic fitness
when it gets in the way of love, compassion, aesthetics, healing,
freedom, fairness, et cetera? Human societies didn't
even have a concept of ``inclusive genetic
fitness'' until the twentieth century.}

{
 But I ask in return: If we can't see clearly the
result of a single monotone optimization criterion---if we
can't even train ourselves to hear a single pure
note---then how will we listen to an orchestra? How will we see that
``Always be selfish'' or
``Always obey the government'' are
poor guiding principles for human beings to adopt---if we think that
even \textit{optimizing genes for inclusive fitness} will yield
organisms that sacrifice reproductive opportunities in the name of
social resource conservation?}

{
 To train ourselves to see clearly, we need simple practice cases.}

\myendsectiontext

\mysection{Adaptation{}-Executers, Not Fitness{}-Maximizers}

\begin{quote}
{
 Individual organisms are best thought of as adaptation-executers
rather than as fitness-maximizers.}

{\raggedleft
 {}---John Tooby and Leda Cosmides,\newline
 ``The Psychological Foundations of
Culture''\footnote{John Tooby and Leda Cosmides, ``The
Psychological Foundations of Culture,'' in
\textit{The Adapted Mind: Evolutionary Psychology and the Generation of
Culture}, ed. Jerome H. Barkow, Leda Cosmides, and John Tooby (New
York: Oxford University Press, 1992), 19--136.\comment{1}}
\par}
\end{quote}


{
 Fifty thousand years ago, the taste buds of \textit{Homo sapiens}
directed their bearers to the scarcest, most critical food
resources---sugar and fat. Calories, in a word. Today, the context of a
taste bud's function has changed, but the taste buds
themselves have not. Calories, far from being scarce (in First World
countries), are actively harmful. Micronutrients that were reliably
abundant in leaves and nuts are absent from bread, but our taste buds
don't complain. A scoop of ice cream is a
superstimulus, containing more sugar, fat, and salt than anything in
the ancestral environment.}

{
 No human being with the \textit{deliberate} goal of maximizing
their alleles' inclusive genetic fitness would ever eat
a cookie unless they were starving. But individual organisms are best
thought of as adaptation-executers, not fitness-maximizers.}

{
 A Phillips-head screwdriver, though its designer intended it to
turn screws, won't reconform itself to a flat-head
screw to fulfill its function. We created these tools, but they exist
independently of us, and they continue independently of us.}

{
 The atoms of a screwdriver don't have tiny little
XML tags inside describing their
``objective'' purpose. The designer
had something in mind, yes, but that's not the same as
what \textit{happens} in the real world. If you forgot that the
designer is a separate entity from the designed thing, you might think,
``The \textit{purpose} of the screwdriver is to drive
screws''---as though this were an explicit property
of the screwdriver itself, rather than a property of the
designer's state of mind. You might be surprised that
the screwdriver didn't reconfigure itself to the
flat-head screw, since, after all, the screwdriver's
\textit{purpose} is to turn screws.}

{
 The \textit{cause} of the screwdriver's existence
is the designer's mind, which imagined an imaginary
screw, and imagined an imaginary handle turning. The \textit{actual}
operation of the screwdriver, its \textit{actual} fit to an actual
screw head, \textit{cannot} be the objective cause of the
screwdriver's existence: The future cannot cause the
past. But the designer's brain, as an actually existent
thing within the past, can indeed be the cause of the screwdriver.}

{
 The \textit{consequence} of the screwdriver's
existence may not correspond to the imaginary consequences in the
designer's mind. The screwdriver blade could slip and
cut the user's hand.}

{
 And the \textit{meaning} of the screwdriver---why,
that's something that exists in the mind of a user, not
in tiny little labels on screwdriver atoms. The designer may intend it
to turn screws. A murderer may buy it to use as a weapon. And then
accidentally drop it, to be picked up by a child, who uses it as a
chisel.}

{
 So the screwdriver's \textit{cause}, and its
\textit{shape}, and its \textit{consequence}, and its various
\textit{meanings}, are all different things; and only \textit{one} of
these things is found within the screwdriver itself.}

{
 Where do taste buds come from? Not from an intelligent designer
visualizing their consequences, but from a frozen history of ancestry:
Adam liked sugar and ate an apple and reproduced, Barbara liked sugar
and ate an apple and reproduced, Charlie liked sugar and ate an apple
and reproduced, and 2763 generations later, the allele became fixed in
the population. For convenience of thought, we sometimes compress this
giant history and say: ``Evolution did
it.'' But it's not a quick, local
event like a human designer visualizing a screwdriver. This is the
\textit{objective cause} of a taste bud.}

{
 What is the \textit{objective shape} of a taste bud? Technically,
it's a molecular sensor connected to reinforcement
circuitry. This adds another level of indirection, because the taste
bud isn't directly acquiring food. It's
influencing the organism's mind, making the organism
want to eat foods that are similar to the food just eaten.}

{
 What is the \textit{objective consequence} of a taste bud? In a
modern First World human, it plays out in multiple chains of causality:
from the desire to eat more chocolate, to the plan to eat more
chocolate, to eating chocolate, to getting fat, to getting fewer dates,
to reproducing less successfully. This consequence is directly
\textit{opposite} the key regularity in the long chain of ancestral
successes that caused the taste bud's shape. But, since
overeating has only recently become a problem, no significant evolution
(compressed regularity of ancestry) has further influenced the taste
bud's shape.}

{
 What is the \textit{meaning} of eating chocolate?
That's between you and your moral philosophy.
Personally, I think chocolate tastes good, but I wish it were less
harmful; acceptable solutions would include redesigning the chocolate
or redesigning my biochemistry.}

{
 Smushing several of the concepts together, you could sort-of-say,
``Modern humans do today what would have propagated
our genes in a hunter-gatherer society, whether or not it helps our
genes in a modern society.'' But this still
isn't quite right, because we're not
\textit{actually} asking ourselves which behaviors would maximize our
ancestors' inclusive fitness. And many of our
activities today have no ancestral analogue. In the hunter-gatherer
society there wasn't any such thing as chocolate.}

{
 So it's better to view our taste buds as an
\textit{adaptation} fitted to ancestral conditions that included
near-starvation and apples and roast rabbit, which modern humans
\textit{execute} in a new context that includes cheap chocolate and
constant bombardment by advertisements.}

{
 Therefore it is said: Individual organisms are best thought of as
adaptation-executers, not fitness-maximizers.}

\myendsectiontext


\bigskip

\mysection{Evolutionary Psychology}

{
 Like ``IRC chat'' or
``TCP/IP protocol,'' the phrase
``reproductive organ'' is redundant.
\textit{All} organs are reproductive organs. Where do a
bird's wings come from? An Evolution-of-Birds Fairy who
thinks that flying is really neat? The bird's wings are
there because they contributed to the bird's
ancestors' reproduction. Likewise the
bird's heart, lungs, and genitals. At most we might
find it worthwhile to distinguish between \textit{directly}
reproductive organs and \textit{indirectly} reproductive organs. }

{
 This observation holds true also of the brain, the most complex
organ system known to biology. Some brain organs are directly
reproductive, like lust; others are indirectly reproductive, like
anger.}

{
 Where does the human emotion of anger come from? An
Evolution-of-Humans Fairy who thought that anger was a worthwhile
feature? The neural circuitry of anger is a reproductive organ as
surely as your liver. Anger exists in \textit{Homo sapiens} because
angry ancestors had more kids. \textit{There's no other
way it could have gotten there.}}

{
 This \textit{historical} fact about the origin of anger confuses
all too many people. They say, ``Wait, are you saying
that when I'm angry, I'm subconsciously
trying to have children? That's not what
I'm thinking after someone punches me in the
nose.''}

{
 No. \textit{No.} \textbf{\textit{No.}} \textbf{\textit{NO!}}}

{
 Individual organisms are best thought of as adaptation-executers,
not fitness-maximizers. The \textit{cause} of an adaptation, the
\textit{shape} of an adaptation, and the \textit{consequence} of an
adaptation are all separate things. If you built a toaster, you
wouldn't expect the toaster to reshape itself when you
tried to cram in a whole loaf of bread; yes, you intended it to make
toast, but that intention is a fact about you, not a fact about the
toaster. The toaster has no sense of its own purpose.}

{
 But a toaster is not an intention-bearing object. It is not a mind
at all, so we are not tempted to attribute goals to it. If \textit{we}
see the toaster as purposed, we don't think the toaster
knows it, because we don't think the toaster knows
\textit{anything.}}

{
 It's like the old test of being asked to say the
color of the letters in ``blue.'' It
takes longer for subjects to name this color, because of the need to
untangle the meaning of the letters and the color of the letters. You
wouldn't have similar trouble naming the color of the
letters in ``wind.''}

{
 But a human brain, in addition to being an artifact historically
produced by evolution, is also a mind capable of bearing its own
intentions, purposes, desires, goals, and plans. Both a bee and a human
are designs, but only a human is a designer. The bee is
``wind;'' the human is
``blue.''}

{
 Cognitive causes are \textit{ontologically distinct} from
evolutionary causes. They are made out of a different kind of stuff.
Cognitive causes are made of neurons. Evolutionary causes are made of
ancestors.}

{
 The most obvious kind of cognitive cause is deliberate, like an
intention to go to the supermarket, or a plan for toasting toast. But
an emotion also exists physically in the brain, as a train of neural
impulses or a cloud of spreading hormones. Likewise an instinct, or a
flash of visualization, or a fleetingly suppressed thought; if you
could scan the brain in three dimensions and you understood the code,
you would be able to \textit{see} them.}

{
 Even subconscious cognitions exist physically in the brain.
``Power tends to corrupt,'' observed
Lord Acton. Stalin may or may not have believed himself an altruist,
working toward the greatest good for the greatest number. But it seems
likely that, somewhere in Stalin's brain, there were
neural circuits that reinforced pleasurably the exercise of power, and
neural circuits that detected anticipations of increases and decreases
in power. If there were nothing in Stalin's brain that
correlated to power---no little light that went on for political
command, and off for political weakness---then how could
Stalin's brain have known to be corrupted by power?}

{
 Evolutionary selection pressures are \textit{ontologically
distinct} from the biological artifacts they create. The evolutionary
cause of a bird's wings is millions of ancestor-birds
who reproduced more often than other ancestor-birds, with statistical
regularity owing to their possession of incrementally improved wings
compared to their competitors. We compress this gargantuan
historical-statistical macrofact by saying ``evolution
did it.''}

{
 Natural selection is ontologically distinct from creatures;
evolution is not a little furry thing lurking in an undiscovered
forest. Evolution is a causal, statistical regularity in the
reproductive history of ancestors.}

{
 And this logic applies also to the brain. Evolution has made wings
that flap, but do not understand flappiness. It has made legs that
walk, but do not understand walkyness. Evolution has carved bones of
calcium ions, but the bones themselves have no explicit concept of
strength, let alone inclusive genetic fitness. And evolution designed
brains themselves capable of designing; yet these brains had no more
concept of evolution than a bird has of aerodynamics. Until the
twentieth century, not a single human brain explicitly represented the
complex abstract concept of \textit{inclusive genetic fitness}.}

{
 When we're told that ``The
evolutionary purpose of anger is to increase inclusive genetic
fitness,'' there's a tendency to
slide to ``\textit{The purpose of} anger is
reproduction'' to ``The cognitive
purpose of anger is reproduction.'' No! The
statistical regularity of ancestral history isn't
\textit{in} the brain, even subconsciously, any more than the
designer's intentions of toast are in a toaster!}

{
 Thinking that your built-in anger-circuitry embodies an explicit
desire to reproduce is like thinking your hand is an embodied mental
desire to pick things up.}

{
 Your hand is not wholly cut off from your mental desires. In
particular circumstances, you can control the flexing of your fingers
by an act of will. If you bend down and pick up a penny, then this may
represent an act of will; but it is not an act of will that made your
hand grow in the first place.}

{
 One must distinguish a one-time event of particular anger
(anger-1, anger-2, anger-3) from the underlying neural circuitry for
anger. An anger-event is a cognitive cause, and an anger-event may have
cognitive causes, but you didn't will the
anger-circuitry to be wired into the brain.}

{
 So you have to distinguish the event of anger, from the circuitry
of anger, from the gene complex that laid down the neural template,
from the ancestral macrofact that explains the gene
complex's presence.}

{
 If there were ever a discipline that genuinely \textit{demanded}
X-Treme Nitpicking, it is evolutionary psychology.}

{
 Consider, O my readers, this sordid and joyful tale: A man and a
woman meet in a bar. The man is attracted to her clear complexion and
firm breasts, which would have been fertility cues in the ancestral
environment, but which in this case result from makeup and a bra. This
does not bother the man; he just likes the way she looks. His
clear-complexion-detecting neural circuitry does not know that its
purpose is to detect fertility, any more than the atoms in his hand
contain tiny little XML tags reading
``{\textless}purpose{\textgreater}pick things
up{\textless}/purpose{\textgreater}.'' The woman is
attracted to his confident smile and firm manner, cues to high status,
which in the ancestral environment would have signified the ability to
provide resources for children. She plans to use birth control, but her
confident-smile-detectors don't know this any more than
a toaster knows its designer intended it to make toast.
She's not concerned philosophically with the meaning of
this rebellion, because her brain is a creationist and denies
vehemently that evolution exists. He's not concerned
philosophically with the meaning of this rebellion, because he just
wants to get laid. They go to a hotel, and undress. He puts on a
condom, because he doesn't want kids, just the
dopamine-noradrenaline rush of sex, which reliably produced offspring
50,000 years ago when it was an invariant feature of the ancestral
environment that condoms did not exist. They have sex, and shower, and
go their separate ways. The main objective consequence is to keep the
bar and the hotel and the condom-manufacturer in business; which was
not the cognitive purpose in their minds, and has virtually nothing to
do with the key statistical regularities of reproduction 50,000 years
ago which explain how they got the genes that built their brains that
executed all this behavior.}

{
 To reason correctly about evolutionary psychology you must
simultaneously consider many complicated abstract facts that are
strongly related yet importantly distinct, without a single mixup or
conflation.}

\myendsectiontext

\mysection{An Especially Elegant Evolutionary Psychology Experiment}

\begin{quote}
{
 In a 1989 Canadian study, adults were asked to imagine the death
of children of various ages and estimate which deaths would create the
greatest sense of loss in a parent. The results, plotted on a graph,
show grief growing until just before adolescence and then beginning to
drop. When this curve was compared with a curve showing changes in
reproductive potential over the life cycle (a pattern calculated from
Canadian demographic data), the correlation was fairly strong. But much
stronger---nearly perfect, in fact---was the correlation between the
grief curves of these modern Canadians and the reproductive-potential
curve of a hunter-gatherer people, the !Kung of Africa. In other words,
the pattern of changing grief was almost exactly what a Darwinian would
predict, given demographic realities in the ancestral environment.}

{\raggedleft
 {}---Robert Wright, \textit{The Moral Animal},\newline
 summarizing Crawford et al.\footnote{Robert Wright, \textit{The Moral Animal: Why We Are the Way We
Are: The New Science of Evolutionary Psychology} (Pantheon Books,
1994); Charles B. Crawford, Brenda E. Salter, and Kerry L. Jang,
``Human Grief: Is Its Intensity Related to the
Reproductive Value of the Deceased?,''
\textit{Ethology and Sociobiology} 10, no. 4 (1989): 297--307.\comment{1}}
\par}
\end{quote}


{
 The first correlation was 0.64, the second an extremely high 0.92
(N = 221).}

{
 The most obvious \textit{in}elegance of this study, as described,
is that it was conducted by asking human adults to imagine parental
grief, rather than asking real parents with children of particular
ages. (Presumably that would have cost more / allowed fewer subjects.)
However, my understanding is that the results here squared well with
the data from closer studies of parental grief that were looking for
other correlations (i.e., a raw correlation between parental grief and
child age).}

{
 That said, consider some of this experiment's
elegant aspects:}

\begin{enumerate}
\item {
 A correlation of 0.92(!) This may sound suspiciously high---could
evolution really do such exact fine-tuning?---until you realize that
this selection pressure was not only great enough to \textit{fine-tune}
parental grief, but, in fact, \textit{carve it out of existence from
scratch in the first place.}}

\item {
 People who say that evolutionary psychology hasn't
made any advance predictions are (ironically) mere victims of
``no one knows what science doesn't
know'' syndrome. You wouldn't even
\textit{think of this as an experiment to be performed} if not for
evolutionary psychology.}

\item {
 The experiment illustrates, as beautifully and as cleanly as any I
have ever seen, the distinction between a \textit{conscious or
subconscious ulterior motive} and an \textit{executing adaptation with
no realtime sensitivity to the original selection pressure that created
it.}}

\end{enumerate}

{
 The parental grief is \textit{not even subconsciously} about
reproductive value---otherwise it would update for Canadian
reproductive value instead of !Kung reproductive value. Grief is an
adaptation that now simply exists, real in the mind and continuing
under its own inertia.}

{
 Parents do \textit{not} care about children for the sake of their
reproductive contribution. Parents care about children for their own
sake; and the \textit{non-cognitive, evolutionary-historical} reason
why such minds exist in the universe in the first place is that
children carry their parents' genes.}

{
 Indeed, evolution is the reason why there are any minds in the
universe at all. So you can see why I'd want to draw a
sharp line through my cynicism about ulterior motives at the
evolutionary-cognitive boundary; otherwise, I might as well stand up in
a supermarket checkout line and say, ``Hey!
You're only correctly processing visual information
while bagging my groceries in order to maximize your inclusive genetic
fitness!''}

{
 1. I think 0.92 is the highest correlation I've
ever seen in any evolutionary psychology experiment, and indeed, one of
the highest correlations I've seen in any psychology
experiment. (Although I've seen e.g. a correlation of
0.98 reported for asking one group of subjects ``How
similar is A to B?'' and another group
``What is the probability of A given
B?'' on questions like ``How likely
are you to draw 60 red balls and 40 white balls from this barrel of 800
red balls and 200 white balls?''---in other words,
these are simply processed as the same question.)}

{
 Since we are all Bayesians here, we may take our priors into
account and ask if at least \textit{some} of this unexpectedly high
correlation is due to luck. The evolutionary fine-tuning we can
probably take for granted; this is a huge selection pressure
we're talking about. The remaining sources of
suspiciously low variance are (a) whether a large group of adults could
correctly envision, on average, relative degrees of parental grief
(apparently they can), and (b) whether the surviving !Kung are
\textit{typical} ancestral hunter-gatherers in this dimension, or
whether variance \textit{between} hunter-gatherer tribal types should
have been too high to allow a correlation of 0.92.}

{
 But even after taking into account any skeptical priors,
correlation 0.92 and N = 221 is pretty strong evidence, and our
posteriors should be less skeptical on all these counts.}

{
 2. You might think it an inelegance of the experiment that it was
performed \textit{prospectively} on imagined grief, rather than
retrospectively on real grief. But it is \textit{prospectively}
imagined grief that will actually operate to steer parental behavior
\textit{away} from losing the child! From an evolutionary standpoint,
an actual dead child is a sunk cost; evolution
``wants'' the parent to learn from
the pain, not do it again, adjust back to their hedonic set point, and
go on raising other children.}

{
 3. Similarly, the graph that correlates to parental grief is for
\textit{the future reproductive potential of a child that has survived
to a given age}, and not \textit{the sunk cost of raising the child
which has survived to that age}. (Might we get an even \textit{higher}
correlation if we tried to take into account the reproductive
opportunity cost of raising a child of age X to independent maturity,
while discarding all sunk costs to raise a child to age X?)}

{
 Humans usually do notice sunk costs---this is presumably either an
adaptation to prevent us from switching strategies too often
(compensating for an overeager opportunity-noticer?) or an unfortunate
spandrel of pain felt on wasting resources.}

{
 Evolution, on the other hand---it's not that
evolution ``doesn't care about sunk
costs,'' but that evolution doesn't
even remotely ``think'' that way;
``evolution'' is just a macrofact
about the real historical reproductive consequences.}

{
 So---of course---the parental grief adaptation is fine-tuned in a
way that has nothing to do with past investment in a child, and
everything to do with the future reproductive consequences of losing
that child. Natural selection isn't crazy about sunk
costs the way we are.}

{
 But---of course---the parental grief adaptation goes on
functioning as if the parent were living in a !Kung tribe rather than
Canada. Most humans would notice the difference.}

{
 Humans and natural selection are insane in \textit{different
stable complicated ways}.}

\myendsectiontext


\bigskip

\mysection{Superstimuli and the Collapse of Western Civilization}

{
 At least three people have died playing online games for days
without rest. People have lost their spouses, jobs, and children to
World of Warcraft. If people have the right to play video games---and
it's hard to imagine a more fundamental right---then
the market is going to respond by supplying the most \textit{engaging}
video games that can be sold, to the point that exceptionally engaged
consumers are removed from the gene pool. }

{
 How does a consumer product become so \textit{involving} that,
after 57 hours of using the product, the consumer would rather use the
product for one more hour than eat or sleep? (I suppose one could argue
that the consumer makes a rational decision that they'd
rather play Starcraft for the next hour than live out the rest of their
life, but let's just not go there. Please.)}

{
 A candy bar is a \textit{superstimulus}: it contains more
concentrated sugar, salt, and fat than anything that exists in the
ancestral environment. A candy bar matches taste buds that evolved in a
hunter-gatherer environment, but it matches those taste buds much more
strongly than anything that actually existed \textit{in} the
hunter-gatherer environment. The signal that once reliably correlated
to healthy food has been hijacked, blotted out with a point in
tastespace that wasn't in the training dataset---an
impossibly distant outlier on the old ancestral graphs. Tastiness,
formerly representing the evolutionarily identified correlates of
healthiness, has been reverse-engineered and perfectly matched with an
artificial substance. Unfortunately there's no equally
powerful market incentive to make the resulting food item as healthy as
it is tasty. We can't taste healthfulness, after all.}

{
 The now-famous Dove Evolution\footnote{\url{http://www.youtube.com/watch?v=iYhCn0jf46U}} video shows the painstaking
construction of another superstimulus: an ordinary woman transformed by
makeup, careful photography, and finally extensive Photoshopping, into
a billboard model---a beauty impossible, unmatchable by human women in
the unretouched real world. Actual women are killing themselves (e.g.,
supermodels using cocaine to keep their weight down) to keep up with
competitors that literally don't exist.}

{
 And likewise, a video game can be so much more \textit{engaging}
than mere reality, even through a simple computer monitor, that someone
will play it without food or sleep until they literally die. I
don't know all the tricks used in video games, but I
can guess some of them---challenges poised at the critical point
between ease and impossibility, intermittent reinforcement, feedback
showing an ever-increasing score, social involvement in massively
multiplayer games.}

{
 Is there a limit to the market incentive to make video games more
engaging? You might hope there'd be no incentive past
the point where the players lose their jobs; after all, they must be
able to pay their subscription fee. This would imply a
``sweet spot'' for the addictiveness
of games, where the mode of the bell curve is having fun, and only a
few unfortunate souls on the tail become addicted to the point of
losing their jobs. As of 2007, playing World of Warcraft for 58 hours
straight until you literally die \textit{is} still the exception rather
than the rule. But video game manufacturers compete against each other,
and if you can make your game 5\% more addictive, you may be able to
steal 50\% of your competitor's customers. You can see
how this problem could get a \textit{lot} worse.}

{
 If people have the right to be tempted---and
that's what free will is all about---the market is
going to respond by supplying as much temptation as can be sold. The
incentive is to make your stimuli 5\% more tempting than those of your
current leading competitors. This continues well beyond the point where
the stimuli become ancestrally anomalous superstimuli. Consider how our
standards of product-selling feminine beauty have changed since the
advertisements of the 1950s. And as candy bars demonstrate, the market
incentive also continues well beyond the point where the superstimulus
begins wreaking collateral damage on the consumer.}

{
 So why don't we just say no? A key assumption of
free-market economics is that, in the absence of force and fraud,
people can always refuse to engage in a harmful transaction. (To the
extent this is true, a free market would be, not merely the
\textit{best} policy on the whole, but a policy with few or no
downsides.)}

{
 An organism that regularly passes up food will die, as some video
game players found out the hard way. But, on some occasions in the
ancestral environment, a typically beneficial (and therefore tempting)
act may in fact be harmful. Humans, as organisms, have an unusually
strong ability to perceive these special cases using abstract thought.
On the other hand we also tend to imagine lots of special-case
consequences that don't exist, like ancestor spirits
commanding us not to eat perfectly good rabbits.}

{
 Evolution seems to have struck a compromise, or perhaps just
aggregated new systems on top of old. \textit{Homo sapiens} are still
tempted by food, but our oversized prefrontal cortices give us a
\textit{limited} ability to resist temptation. Not unlimited
ability---our ancestors with too much willpower probably starved
themselves to sacrifice to the gods, or failed to commit adultery one
too many times. The video game players who died must have exercised
willpower (in some sense) to keep playing for so long without food or
sleep; the evolutionary hazard of self-control.}

{
 Resisting any temptation takes conscious expenditure of an
exhaustible supply of mental energy. It is not in fact \textit{true}
that we can ``just say no''---not
\textit{just} say no, without cost to ourselves. Even humans who won
the birth lottery for willpower or foresightfulness still pay a price
to resist temptation. The price is just more easily paid.}

{
 Our limited willpower evolved to deal with ancestral temptations;
it may not operate well against enticements beyond anything known to
hunter-gatherers. Even where we successfully resist a superstimulus, it
seems plausible that the effort required would deplete willpower much
faster than resisting ancestral temptations.}

{
 Is public display of superstimuli a negative externality, even to
the people who say no? Should we ban chocolate cookie ads, or
storefronts that openly say ``Ice
Cream''?}

{
 Just because a problem exists doesn't show
(without further justification and a substantial burden of proof) that
the government can fix it. The regulator's career
incentive does not focus on products that combine low-grade consumer
harm with addictive superstimuli; it focuses on products with failure
modes spectacular enough to get into the newspaper. Conversely, just
because the government may \textit{not} be able to fix something,
doesn't mean it \textit{isn't} going
wrong.}

{
 I leave you with a final argument from fictional evidence: Simon
Funk's online novel After Life depicts (among other
plot points) the planned extermination of biological \textit{Homo
sapiens}{}---not by marching robot armies, but by artificial children
that are much cuter and sweeter and more fun to raise than real
children. Perhaps the demographic collapse of advanced societies
happens because the market supplies ever-more-tempting alternatives to
having children, while the attractiveness of changing diapers remains
constant over time. Where are the advertising billboards that say
``BREED''? Who will pay professional
image consultants to make arguing with sullen teenagers seem more
alluring than a vacation in Tahiti?}

{
 ``In the end,'' Simon Funk
wrote, ``the human species was simply marketed out of
existence.''}

\myendsectiontext

\mysection{Thou Art Godshatter}

{
 Before the twentieth century, not a single human being had an
explicit concept of ``inclusive genetic
fitness,'' the sole and absolute obsession of the
blind idiot god. We have no instinctive revulsion of condoms or oral
sex. Our brains, those supreme reproductive organs,
don't perform a check for reproductive efficacy before
granting us sexual pleasure. }

{
 Why not? Why \textit{aren't} we consciously
obsessed with inclusive genetic fitness? Why did the
Evolution-of-Humans Fairy create brains that would invent condoms?
``It would have been so
\textit{easy},'' thinks the human, who can design new
complex systems in an afternoon.}

{
 The Evolution Fairy, as we all know, is obsessed with inclusive
genetic fitness. When she decides which genes to promote to
universality, she doesn't seem to take into account
\textit{anything} except the number of copies a gene produces. (How
strange!)}

{
 But since the maker of intelligence is thus obsessed, why not
create intelligent agents---you can't call them
humans---who would likewise care purely about inclusive genetic
fitness? Such agents would have sex only as a means of reproduction,
and wouldn't bother with sex that involved birth
control. They could eat food out of an explicitly reasoned belief that
food was necessary to reproduce, not because they liked the taste, and
so they wouldn't eat candy if it became detrimental to
survival or reproduction. Post-menopausal women would babysit
grandchildren until they became sick enough to be a net drain on
resources, and would then commit suicide.}

{
 It seems like such an obvious design improvement---from the
Evolution Fairy's perspective.}

{
 Now it's clear that it's hard to
build a powerful enough consequentialist. Natural selection sort-of
reasons consequentially, but only by depending on the \textit{actual
consequences.} Human evolutionary theorists have to do really
high-falutin' abstract reasoning in order to
\textit{imagine} the links between adaptations and reproductive
success.}

{
 But human brains clearly \textit{can} imagine these links in
protein. So when the Evolution Fairy made humans, why did It bother
with \textit{any} motivation except inclusive genetic fitness?}

{
 It's been less than two centuries since a protein
brain first represented the concept of natural selection. The modern
notion of ``inclusive genetic
fitness'' is even more subtle, a highly abstract
concept. What matters is not the number of shared genes. Chimpanzees
share 95\% of your genes. What matters is shared genetic
\textit{variance}, \textit{within} a reproducing population---your
sister is one-half related to you, because any variations in your
genome, within the human species, are 50\% likely to be shared by your
sister.}

{
 Only in the last century---arguably only in the last fifty
years---have evolutionary biologists really begun to understand the
full range of causes of reproductive success, things like reciprocal
altruism and costly signaling. Without all this highly detailed
knowledge, an intelligent agent that set out to
``maximize inclusive fitness'' would
fall flat on its face.}

{
 So why not preprogram protein brains with the knowledge? Why
wasn't a concept of ``inclusive
genetic fitness'' \textit{programmed} into us, along
with a library of explicit strategies? Then you could dispense with all
the reinforcers. The organism would be born knowing that, with high
probability, fatty foods would lead to fitness. If the organism later
learned that this was no longer the case, it would stop eating fatty
foods. You could refactor the whole system. And it
wouldn't invent condoms or cookies.}

{
 This looks like it should be quite possible in principle. I
occasionally run into people who don't quite understand
consequentialism, who say, ``But if the organism
doesn't have a separate drive to eat, it will starve,
and so fail to reproduce.'' So long as the organism
knows \textit{this very fact}, and has a utility function that values
reproduction, it will automatically eat. In fact, this is
\textit{exactly} the consequentialist reasoning that natural selection
itself used to build automatic eaters.}

{
 What about curiosity? Wouldn't a consequentialist
only be curious when it saw some specific reason to be curious? And
wouldn't this cause it to miss out on lots of important
knowledge that came with no specific reason for investigation attached?
Again, a consequentialist will investigate given only the knowledge of
this very same fact. If you consider the curiosity drive of a
human---which is not undiscriminating, but responds to particular
features of problems---then this complex adaptation is purely the
result of consequentialist reasoning by DNA, an \textit{implicit}
representation of knowledge: Ancestors who engaged in this kind of
inquiry left more descendants.}

{
 So in principle, the pure reproductive consequentialist is
possible. In principle, all the ancestral history \textit{implicitly}
represented in cognitive adaptations can be converted to
\textit{explicitly} represented knowledge, running on a core
consequentialist.}

{
 But the blind idiot god isn't that smart.
Evolution is not a human programmer who can simultaneously refactor
whole code architectures. Evolution is not a human programmer who can
sit down and type out instructions at sixty words per minute.}

{
 For millions of years before hominid consequentialism, there was
reinforcement learning. The reward signals were events that correlated
reliably to reproduction. You can't ask a nonhominid
brain to foresee that a child eating fatty foods now will live through
the winter. So the DNA builds a protein brain that generates a reward
signal for eating fatty food. Then it's up to the
organism to learn which prey animals are tastiest.}

{
 DNA constructs protein brains with reward signals that have a
\textit{long-distance} correlation to reproductive fitness, but a
\textit{short-distance} correlation to organism behavior. You
don't have to figure out that eating sugary food in the
fall will lead to digesting calories that can be stored fat to help you
survive the winter so that you mate in spring to produce offspring in
summer. An apple simply tastes good, and your brain just has to plot
out how to get more apples off the tree.}

{
 And so organisms evolve rewards for eating, and building nests,
and scaring off competitors, and helping siblings, and discovering
important truths, and forming strong alliances, and arguing
persuasively, and of course having sex \ldots}

{
 When hominid brains capable of cross-domain consequential
reasoning began to show up, they reasoned consequentially about how to
get the \textit{existing} reinforcers. It was a relatively simple hack,
vastly simpler than rebuilding an ``inclusive fitness
maximizer'' from scratch. The protein brains plotted
how to acquire calories and sex, without any explicit cognitive
representation of ``inclusive
fitness.''}

{
 A human engineer would have said, ``Whoa,
I've just invented a consequentialist! Now I can take
all my previous hard-won knowledge about which behaviors improve
fitness, and declare it explicitly! I can convert all this complicated
reinforcement learning machinery into a simple declarative knowledge
statement that `fatty foods and sex usually improve your
inclusive fitness.' Consequential reasoning will
automatically take care of the rest. Plus, it won't
have the obvious failure mode where it invents
condoms!''}

{
 But then a human engineer wouldn't have built the
retina backward, either.}

{
 The blind idiot god is not a unitary purpose, but a
many-splintered attention. Foxes evolve to catch rabbits, rabbits
evolve to evade foxes; there are as many evolutions as species. But
\textit{within} each species, the blind idiot god is purely obsessed
with inclusive genetic fitness. No quality is valued, not even
survival, except insofar as it increases reproductive fitness.
There's no point in an organism with steel skin if it
ends up having 1\% less reproductive capacity.}

{
 Yet when the blind idiot god created protein computers, its
monomaniacal focus on inclusive genetic fitness was not faithfully
transmitted. Its optimization criterion did not successfully quine. We,
the handiwork of evolution, are as alien to evolution as our Maker is
alien to us. One pure utility function splintered into a thousand
shards of desire.}

{
 Why? Above all, because evolution is stupid in an absolute sense.
But also because the \textit{first} protein computers
weren't anywhere near as \textit{general} as the blind
idiot god, and could only utilize short-term desires.}

{
 In the final analysis, asking why evolution didn't
build humans to maximize inclusive genetic fitness is like asking why
evolution didn't hand humans a ribosome and tell them
to design their own biochemistry. Because evolution
can't refactor code that fast, that's
why. But maybe in a billion years of continued natural selection
that's exactly what \textit{would} happen, if
intelligence were foolish enough to allow the idiot god continued
reign.}

{
 \textit{The Mote in God's Eye} by Niven and
Pournelle depicts an intelligent species that stayed biological a
little too long, slowly becoming truly enslaved by evolution, gradually
turning into true fitness maximizers obsessed with outreproducing each
other. But thankfully that's not what happened. Not
here on Earth. At least not yet.}

{
 So humans love the taste of sugar and fat, and we love our sons
and daughters. We seek social status, and sex. We sing and dance and
play. We learn for the love of learning.}

{
 A thousand delicious tastes, matched to ancient reinforcers that
once correlated with reproductive fitness---now sought whether or not
they enhance reproduction. Sex with birth control, chocolate, the music
of long-dead Bach on a CD.}

{
 And when we finally learn about evolution, we think to ourselves:
``Obsess all day about inclusive genetic fitness?
Where's the fun in \textit{that}?''}

{
 The blind idiot god's single monomaniacal goal
splintered into a thousand shards of desire. And this is well, I think,
though I'm a human who says so. Or else what would we
do with the future? What would we do with the billion galaxies in the
night sky? Fill them with maximally efficient replicators? Should our
descendants deliberately obsess about maximizing their inclusive
genetic fitness, regarding all else only as a means to that end?}

{
 Being a thousand shards of desire isn't always
fun, but at least it's not \textit{boring}. Somewhere
along the line, we evolved tastes for novelty, complexity, elegance,
and challenge---tastes that judge the blind idiot god's
monomaniacal focus, and find it aesthetically unsatisfying.}

{
 And yes, we got those very same tastes from the blind
idiot's godshatter.}

{
 So what?}

\myendsectiontext

\chapter{Fragile Purposes}

\mysection{Belief in Intelligence}

{
 I don't know what moves Garry Kasparov would make
in a chess game. What, then, is the empirical content of my belief that
``Kasparov is a highly intelligent chess
player''? What real-world experience does my belief
tell me to anticipate? Is it a cleverly masked form of total ignorance?
}

{
 To sharpen the dilemma, suppose Kasparov plays against some mere
chess grandmaster Mr. G, who's not in the running for
world champion. My own ability is far too low to distinguish between
these levels of chess skill. When I try to guess
Kasparov's move, or Mr. G's next move,
all I can do is try to guess ``the best chess
move'' using my own meager knowledge of chess. Then I
would produce exactly the same prediction for
Kasparov's move or Mr. G's move in any
particular chess position. So what is the empirical content of my
belief that ``Kasparov is a \textit{better} chess
player than Mr. G''?}

{
 The empirical content of my belief is the testable, falsifiable
prediction that the \textit{final} chess position will occupy the class
of chess positions that are wins for Kasparov, rather than drawn games
or wins for Mr. G. (Counting resignation as a legal move that leads to
a chess position classified as a loss.) The degree to which I think
Kasparov is a ``better player'' is
reflected in the amount of probability mass I concentrate into the
``Kasparov wins'' class of outcomes,
versus the ``drawn game'' and
``Mr. G wins'' class of outcomes.
These classes are extremely vague in the sense that they refer to vast
spaces of possible chess positions---but ``Kasparov
wins'' \textit{is} more specific than maximum
entropy, because it can be definitely falsified by a vast set of chess
positions.}

{
 The \textit{outcome} of Kasparov's game is
predictable because I know, and understand, Kasparov's
goals. Within the confines of the chess board, I know
Kasparov's motivations---I know his success criterion,
his utility function, his target as an optimization process. I know
where Kasparov is \textit{ultimately} trying to steer the future and I
anticipate he is powerful enough to get there, although I
don't anticipate much about \textit{how} Kasparov is
going to do it.}

{
 Imagine that I'm visiting a distant city, and a
local friend volunteers to drive me to the airport. I
don't know the neighborhood. Each time my friend
approaches a street intersection, I don't know whether
my friend will turn left, turn right, or continue straight ahead. I
can't predict my friend's move even as
we approach each individual intersection---let alone predict the whole
sequence of moves in advance.}

{
 Yet I can predict the \textit{result} of my
friend's unpredictable actions: we will arrive at the
airport. Even if my friend's house were located
elsewhere in the city, so that my friend made a completely different
sequence of turns, I would just as confidently predict our arrival at
the airport. I can predict this long in advance, before I even get into
the car. My flight departs soon, and there's no time to
waste; I wouldn't get into the car in the first place,
if I couldn't confidently predict that the car would
travel to the airport along an unpredictable pathway.}

{
 Isn't this a remarkable situation to be in, from a
scientific perspective? I can predict the \textit{outcome} of a
process, without being able to predict any of the \textit{intermediate
steps} of the process.}

{
 How is this even possible? Ordinarily one predicts by imagining
the present and then running the visualization forward in time. If you
want a \textit{precise} model of the Solar System, one that takes into
account planetary perturbations, you must start with a model of all
major objects and run that model forward in time, step by step.}

{
 Sometimes simpler problems have a closed-form solution, where
calculating the future at time T takes the same amount of work
regardless of T. A coin rests on a table, and after each minute, the
coin turns over. The coin starts out showing heads. What face will it
show a hundred minutes later? Obviously you did not answer this
question by visualizing a hundred intervening steps. You used a
closed-form solution that worked to predict the outcome, and would
\textit{also} work to predict any of the intervening steps.}

{
 But when my friend drives me to the airport, I can predict the
outcome successfully using a strange model that won't
work to predict \textit{any} of the intermediate steps. My model
doesn't even require me to input the initial
conditions---I don't need to know where we start out in
the city!}

{
 I do need to know something about my friend. I must know that my
friend wants me to make my flight. I must credit that my friend is a
good enough planner to successfully drive me to the airport (if he
wants to). These are properties of my \textit{friend's}
initial state---properties which let me predict the final destination,
though not any intermediate turns.}

{
 I must also credit that my friend knows enough about the city to
drive successfully. This may be regarded as a relation between my
friend and the city; hence, a property of both. But an extremely
\textit{abstract} property, which does not require any
\textit{specific} knowledge about either the city, or about my
friend's knowledge about the city.}

{
 This is one way of viewing the subject matter to which
I've devoted my life---these \textit{remarkable
situations} which place us in such odd epistemic positions. And my
work, in a sense, can be viewed as unraveling the exact form of that
strange abstract knowledge we can possess; whereby, not knowing the
actions, we can justifiably know the consequence.}

{
 ``Intelligence'' is too narrow
a term to describe these remarkable situations in full generality. I
would say rather ``optimization
process.'' A similar situation accompanies the study
of biological natural selection, for example; we can't
predict the exact form of the next organism observed.}

{
 But my own specialty is the kind of optimization process called
``intelligence''; and even narrower,
a particular kind of intelligence called ``Friendly
Artificial Intelligence''---of which, I hope, I will
be able to obtain especially precise abstract knowledge.}

\myendsectiontext

\mysection{Humans in Funny Suits}

{
 Many times the human species has travelled into space, only to
find the stars inhabited by aliens who look remarkably like humans in
funny suits---or even humans with a touch of makeup and latex---or just
beige Caucasians in fee simple.}

{
 ~}

{\centering
\mygraphicss{images/img179.jpg}{0.5}
 \newline
 \textit{Star Trek: The Original Series},
``Arena,'' © CBS Corporation
\par}


\bigskip

{
 ~}

{
 It's remarkable how the human form is the natural
baseline of the universe, from which all other alien species are
derived via a few modifications.}

{
 What could possibly explain this fascinating phenomenon?
Convergent evolution, of course! Even though these alien life-forms
evolved on a thousand alien planets, completely independently from
Earthly life, they all turned out the same.}

{
 Don't be fooled by the fact that a kangaroo (a
mammal) resembles us rather less than does a chimp (a primate), nor by
the fact that a frog (amphibians, like us, are tetrapods) resembles us
less than the kangaroo. Don't be fooled by the
bewildering variety of the insects, who split off from us even longer
ago than the frogs; don't be fooled that insects have
six legs, and their skeletons on the outside, and a different system of
optics, and rather different sexual practices.}

{
 You might think that a truly alien species would be more different
from us than we are from insects. As I said, don't be
fooled. For an alien species to evolve \textit{intelligence}, it must
have two legs with one knee each attached to an upright torso, and must
walk in a way similar to us. You see, any \textit{intelligence} needs
hands, so you've got to repurpose a pair of legs for
that---and if you don't start with a four-legged being,
it can't develop a running gait and walk upright,
freeing the hands.}

{
 \ldots Or perhaps we should consider, as an alternative theory,
that it's the \textit{easy way out} to use humans in
funny suits.}

{
 But the real problem is not shape; it is mind.
``Humans in funny suits'' is a
well-known term in literary science-fiction fandom, and it does
\textit{not} refer to something with four limbs that walks upright. An
angular creature of pure crystal is a ``human in a
funny suit'' if she \textit{thinks} remarkably like a
human---especially a human of an English-speaking culture of the
late-twentieth/early-twenty-first century.}

{
 I don't watch a lot of ancient movies. When I was
watching the movie \textit{Psycho} (1960) a few years back, I was taken
aback by the cultural gap between the Americans on the screen and my
America. The buttoned-shirted characters of \textit{Psycho} are
considerably more alien than the vast majority of so-called
``aliens'' I encounter on TV or the
silver screen.}

{
 To write a culture that isn't just like your own
culture, you have to be able to see your own culture as a
\textit{special case}{}---not as a norm which all other cultures must
take as their point of departure. Studying history may help---but then
it is only little black letters on little white pages, not a living
experience. I suspect that it would help more to live for a year in
China or Dubai or among the !Kung \ldots this I have never done, being
busy. Occasionally I wonder what things I might not be seeing (not
there, but here).}

{
 Seeing your \textit{humanity} as a special case is very much
harder than this.}

{
 In every known culture, humans seem to experience joy, sadness,
fear, disgust, anger, and surprise. In every known culture, these
emotions are indicated by the same facial expressions. Next time you
see an ``alien''---or an
``AI,'' for that matter---I bet that
when it gets angry (and it will get angry), it will show the
human-universal facial expression for anger.}

{
 We humans are very much alike under our skulls---that goes with
being a sexually reproducing species; you can't have
everyone using different \textit{complex} adaptations, they
wouldn't assemble. (Do the aliens reproduce sexually,
like humans and many insects? Do they share small bits of genetic
material, like bacteria? Do they form colonies, like fungi? Does the
rule of psychological unity apply among them?)}

{
 The only intelligences your ancestors had to
\textit{manipulate}{}---complexly so, and not just tame or catch in
nets---the only minds your ancestors had to model \textit{in
detail}{}---were minds that worked more or less like their own. And so
we evolved to predict Other Minds by putting \textit{ourselves} in
their shoes, asking what we would do in their situations; for that
which was to be predicted, was similar to the predictor.}

{
 ``What?'' you say.
``I don't assume other people are just
like me! Maybe I'm sad, and they happen to be angry!
They believe other things than I do; their personalities are different
from mine!'' Look at it this way: a human brain is an
\textit{extremely} complicated physical system. You are not modeling it
neuron-by-neuron or atom-by-atom. If you came across a physical system
as complex as the human brain which was \textit{not} like you, it would
take scientific lifetimes to unravel it. You do \textit{not} understand
how human brains work in an abstract, general sense; you
can't build one, and you can't even
build a computer model that predicts other brains as well as
\textit{you} predict them.}

{
 The only reason you can try at all to grasp anything as physically
complex and poorly understood as the brain of another human being is
that you configure your own brain to imitate it. You empathize (though
perhaps not sympathize). You impose on your own brain the shadow of the
other mind's anger and the shadow of its beliefs. You
may never think the words, ``What would I do in this
situation?,'' but that little shadow of the other
mind that you hold within yourself is something animated within your
own brain, invoking the same complex machinery that exists in the other
person, synchronizing gears you don't understand. You
may not be angry yourself, but you know that if \textit{you} were angry
at you, and \textit{you} believed that you were godless scum,
\textit{you} would try to hurt you \ldots}

{
 This ``empathic inference'' (as
I shall call it) works for humans, more or less.}

{
 But minds with \textit{different} emotions---minds that feel
emotions you've never felt yourself, or that fail to
feel emotions you would feel? That's something you
can't grasp by putting your brain into the other
brain's shoes. I can tell you to imagine an alien that
grew up in a universe with four spatial dimensions, instead of three
spatial dimensions, but you won't be able to
reconfigure your visual cortex to see like that alien would see. I can
try to write a story about aliens with different emotions, but you
won't be able to feel those emotions, and neither will
I.}

{
 Imagine an alien watching a video of the Marx Brothers and having
absolutely no idea what was going on, or why you would actively seek
out such a sensory experience, because the alien has never conceived of
anything remotely like a sense of humor. Don't pity
them for missing out; \textit{you've} never
\textit{antled.}}

{
 You might ask: Maybe the aliens do have a sense of humor, but
you're not telling funny enough jokes? This is roughly
the equivalent of trying to speak English very loudly, and very slowly,
in a foreign country, on the theory that those foreigners must have an
inner ghost that can hear the \textit{meaning} dripping from your
words, inherent in your words, if only you can speak them loud enough
to overcome whatever strange barrier stands in the way of your
perfectly sensible English.}

{
 It is important to appreciate that laughter can be a beautiful and
valuable thing, even if it is not universalizable, even if it is not
possessed by all possible minds. It would be our own \textit{special}
part of the gift we give to tomorrow. That can count for something
too.}

{
 It had better, because universalizability is one metaethical
notion that I can't salvage for you. Universalizability
among humans, maybe; but not among all possible minds.}

{
 And what about minds that don't run on emotional
architectures like your own---that don't have things
analogous to \textit{emotions}? No, don't bother
explaining why any intelligent mind powerful enough to build complex
machines must inevitably have states analogous to emotions. Natural
selection builds complex machines without itself having emotions. Now
\textit{there's} a Real Alien for you---an optimization
process that \textit{really} Does Not Work Like You Do.}

{
 Much of the progress in biology since the 1960s has consisted of
trying to enforce a moratorium on anthropomorphizing evolution. That
was a major academic slap-fight, and I'm not sure that
sanity would have won the day if not for the availability of crushing
experimental evidence backed up by clear math. Getting people to stop
putting themselves in alien shoes is a long, hard, uphill slog.
I've been fighting that battle on AI for years.}

{
 Our anthropomorphism runs very deep in us; it cannot be excised by
a simple act of will, a determination to say, ``Now I
shall stop thinking like a human!'' Humanity is the
air we breathe; it is our \textit{generic}, the white paper on which we
begin our sketches. And we do not think of ourselves as being human
when we are being human.}

{
 It is proverbial in literary science fiction that the true test of
an author is their ability to write Real Aliens. (And not just
conveniently incomprehensible aliens who, for their own mysterious
reasons, do whatever the plot happens to require.) Jack Vance was one
of the great masters of this art. Vance's
\textit{humans}, if they come from a different culture, are more alien
than most ``aliens.'' (Never read
any Vance? I would recommend starting with \textit{City of the
Chasch}.) Niven and Pournelle's \textit{The Mote in
God's Eye} also gets a standard mention here.}

{
 And conversely---well, I once read a science fiction author (I
think Orson Scott Card) say that the all-time low point of television
science fiction was the \textit{Star Trek} episode where parallel
evolution has proceeded to the extent of producing aliens who not only
look just like humans, who not only speak English, but have also
independently rewritten, word for word, the preamble to the US
Constitution.}

{
 This is the Great Failure of Imagination. Don't
think that it's just about science fiction, or even
just about AI. The inability to imagine the alien is the inability to
see \textit{yourself}{}---the inability to understand your own
specialness. Who can see a human camouflaged against a human
background?}

\myendsectiontext

\mysection{Optimization and the Intelligence Explosion}

{
 Among the topics I haven't delved into here is the
notion of an optimization process. Roughly, this is the idea that your
power as a mind is your ability to hit small targets in a large search
space---this can be either the space of possible futures (planning) or
the space of possible designs (invention).}

{
 Suppose you have a car, and suppose we already know that your
preferences involve travel. Now suppose that you take all the parts in
the car, or all the atoms, and jumble them up at random.
It's very unlikely that you'll end up
with a travel-artifact at all, even so much as a wheeled cart; let
alone a travel-artifact that ranks as high in your preferences as the
original car. So, relative to your preference ordering, the car is an
extremely \textit{improbable} artifact. The power of an optimization
process is that it can produce this kind of improbability.}

{
 You can view both intelligence and natural selection as special
cases of \textit{optimization}: processes that hit, in a large search
space, very small targets defined by implicit preferences. Natural
selection prefers more efficient replicators. Human intelligences have
more complex preferences. Neither evolution nor humans have consistent
utility functions, so viewing them as ``optimization
processes'' is understood to be an approximation.
You're trying to get at the \textit{sort of work being
done}, not claim that humans or evolution do this work
\textit{perfectly}.}

{
 This is how I see the story of life and intelligence---as a story
of improbably good designs being produced by optimization processes.
The ``improbability'' here is
improbability relative to a random selection from the design space, not
improbability in an absolute sense---if you have an optimization
process around, then ``improbably''
good designs become probable.}

{
 Looking over the history of optimization on Earth up until now,
the first step is to conceptually separate the meta level from the
object level---separate the \textit{structure of optimization} from
\textit{that which is optimized}.}

{
 If you consider biology in the absence of hominids, then on the
object level we have things like dinosaurs and butterflies and cats. On
the meta level we have things like sexual recombination and natural
selection of asexual populations. The object level, you will observe,
is rather more complicated than the meta level. Natural selection is
not an \textit{easy} subject and it involves math. But if you look at
the anatomy of a whole cat, the cat has dynamics immensely more
complicated than ``mutate, recombine,
reproduce.''}

{
 This is not surprising. Natural selection is an
\textit{accidental} optimization process, that basically just started
happening one day in a tidal pool somewhere. A cat is the
\textit{subject} of millions of years and billions of years of
evolution.}

{
 Cats have brains, of course, which operate to learn over a
lifetime; but at the end of the cat's lifetime, that
information is thrown away, so it does not accumulate. The cumulative
effects of cat-brains upon the world as optimizers, therefore, are
relatively small.}

{
 Or consider a bee brain, or a beaver brain. A bee builds hives,
and a beaver builds dams; but they didn't figure out
how to build them from scratch. A beaver can't figure
out how to build a hive, a bee can't figure out how to
build a dam.}

{
 So animal brains---up until recently---were not major players in
the planetary game of optimization; they were \textit{pieces} but not
\textit{players}. Compared to evolution, brains lacked both generality
of optimization power (they could not produce the amazing range of
artifacts produced by evolution) and cumulative optimization power
(their products did not accumulate complexity over time). For more on
this theme see Protein Reinforcement and DNA Consequentialism.}

{
 \textit{Very recently}, certain animal brains have begun to
exhibit both generality of optimization power (producing an amazingly
wide range of artifacts, in time scales too short for natural selection
to play any significant role) and cumulative optimization power
(artifacts of increasing complexity, as a result of skills passed on
through language and writing).}

{
 Natural selection takes hundreds of generations to do anything and
millions of years for de novo complex designs. Human programmers can
design a complex machine with a hundred interdependent elements in a
single afternoon. This is not surprising, since natural selection is an
\textit{accidental} optimization process that basically just started
happening one day, whereas humans are \textit{optimized} optimizers
handcrafted by natural selection over millions of years.}

{
 The wonder of evolution is not how well it works, but that it
works \textit{at all} without being optimized. This is how optimization
bootstrapped itself into the universe---starting, as one would expect,
from an extremely inefficient accidental optimization process. Which is
not the accidental first replicator, mind you, but the accidental first
process of natural selection. Distinguish the object level and the meta
level!}

{
 Since the dawn of optimization in the universe, a certain
structural commonality has held across both natural selection and human
intelligence \ldots}

{
 Natural selection \textit{selects on genes}, but generally
speaking, the genes do not turn around and optimize natural selection.
The invention of sexual recombination is an exception to this rule, and
so is the invention of cells and DNA. And you can see both the power
and the \textit{rarity} of such events, by the fact that evolutionary
biologists structure entire histories of life on Earth around them.}

{
 But if you step back and take a human standpoint---if you think
like a programmer---then you can see that natural selection is
\textit{still} not all that complicated. We'll try
bundling different genes together? We'll try separating
information storage from moving machinery? We'll try
randomly recombining groups of genes? On an absolute scale, these are
the sort of bright ideas that any smart hacker comes up with during the
first ten minutes of thinking about system architectures.}

{
 Because natural selection started out \textit{so} inefficient (as
a completely accidental process), this tiny handful of meta-level
improvements feeding back in from the replicators---nowhere near as
complicated as the structure of a cat---structure the evolutionary
epochs of life on Earth.}

{
 And \textit{after} all that, natural selection is \textit{still} a
blind idiot of a god. Gene pools can evolve to extinction, despite all
cells and sex.}

{
 Now natural selection does feed on itself in the sense that each
new adaptation opens up new avenues of further adaptation; but that
takes place on the object level. The gene pool feeds on its own
complexity---but only thanks to the protected interpreter of natural
selection that runs in the background, and that is not itself rewritten
or altered by the evolution of species.}

{
 Likewise, human beings invent sciences and technologies, but we
have not \textit{yet} begun to rewrite the protected structure of the
human brain itself. We have a prefrontal cortex and a temporal cortex
and a cerebellum, just like the first inventors of agriculture. We
haven't started to genetically engineer ourselves. On
the object level, science feeds on science, and each new discovery
paves the way for new discoveries---but all that takes place with a
protected interpreter, the human brain, running untouched in the
background.}

{
 We have meta-level inventions like science, that try to instruct
humans in how to think. But the first person to invent
Bayes's Theorem did not become a Bayesian; they could
not rewrite themselves, lacking both that knowledge and that power. Our
significant innovations in the art of thinking, like writing and
science, are so powerful that they structure the course of human
history; but they do not rival the brain itself in complexity, and
their effect upon the brain is comparatively shallow.}

{
 The present state of the art in rationality training is not
sufficient to turn an arbitrarily selected mortal into Albert Einstein,
which shows the power of a few minor genetic quirks of brain design
compared to all the self-help books ever written in the twentieth
century.}

{
 Because the brain hums away invisibly in the background, people
tend to overlook its contribution and take it for granted; and talk as
if the simple instruction to ``Test ideas by
experiment,'' or the p {\textless} 0.05 significance
rule, were the same order of contribution as an entire human brain. Try
telling chimpanzees to test their ideas by experiment and see how far
you get.}

{
 Now \ldots some of us \textit{want} to intelligently design an
intelligence that would be capable of intelligently redesigning itself,
right down to the level of machine code.}

{
 The machine code at first, and the laws of physics later, would be
a protected level of a sort. But that ``protected
level'' would not contain the \textit{dynamic of
optimization}; the protected levels would not structure the work. The
human brain does quite a bit of optimization on its own, and screws up
on its own, no matter what you try to tell it in school. But this
\textit{fully wraparound recursive optimizer} would have no protected
level that was \textit{optimizing}. All the structure of optimization
would be subject to optimization itself.}

{
 And that is a sea change which breaks with the entire past since
the first replicator, because it breaks the idiom of a protected meta
level.}

{
 The history of Earth up until now has been a history of optimizers
spinning their wheels at a constant rate, generating a constant
optimization pressure. And creating optimized products, \textit{not} at
a constant rate, but at an accelerating rate, because of how
object-level innovations open up the pathway to other object-level
innovations. But that acceleration is taking place with a protected
meta level doing the actual optimizing. Like a search that leaps from
island to island in the search space, and good islands tend to be
adjacent to even better islands, but the jumper doesn't
change its legs. \textit{Occasionally}, a few tiny little changes
manage to hit back to the meta level, like sex or science, and then the
history of optimization enters a new epoch and everything proceeds
faster from there.}

{
 Imagine an economy without investment, or a university without
language, a technology without tools to make tools. Once in a hundred
million years, or once in a few centuries, someone invents a hammer.}

{
 That is what optimization has been like on Earth up until now.}

{
 When I look at the history of Earth, I don't see a
history of optimization \textit{over time}. I see a history of
\textit{optimization power} in, and \textit{optimized products} out. Up
until now, thanks to the existence of almost entirely protected
meta-levels, it's been possible to split up the history
of optimization into epochs, and, \textit{within} each epoch, graph the
cumulative \textit{object-level optimization} over time, because the
protected level is running in the background and is not itself changing
within an epoch.}

{
 What happens when you build a fully wraparound, recursively
self-improving AI? Then you take the graph of
``optimization in, optimized out,''
and fold the graph in on itself. Metaphorically speaking.}

{
 If the AI is weak, it does nothing, because it is not powerful
enough to significantly improve itself---like telling a chimpanzee to
rewrite its own brain.}

{
 If the AI is powerful enough to rewrite itself in a way that
increases its ability to make further improvements, and this reaches
all the way down to the AI's full understanding of its
own source code and its own design as an optimizer \ldots then even if
the graph of ``optimization power
in'' and ``optimized product
out'' looks essentially the same, the graph of
optimization \textit{over time} is going to look completely different
from Earth's history so far.}

{
 People often say something like, ``But what if it
requires exponentially greater amounts of self-rewriting for only a
linear improvement?'' To this the obvious answer is,
``Natural selection exerted roughly constant
optimization power on the hominid line in the course of coughing up
humans; and this doesn't seem to have required
exponentially more time for each linear increment of
improvement.''}

{
 All of this is still mere analogic reasoning. A full Artificial
General Intelligence thinking about the nature of optimization and
doing its own AI research and rewriting its own source code, is not
\textit{really} like a graph of Earth's history folded
in on itself. It is a different sort of beast. These analogies are
\textit{at best} good for qualitative predictions, and even then, I
have a large amount of other beliefs I haven't yet
explained, which are telling me \textit{which} analogies to make, et
cetera.}

{
 But if you want to know why I might be reluctant to extend the
graph of biological and economic growth \textit{over time}, into the
future and over the horizon of an AI that thinks at transistor speeds
and invents self-replicating molecular nanofactories and
\textit{improves its own source code}, then there is my reason: you are
drawing the wrong graph, and it should be optimization power in versus
optimized product out, not optimized product versus time.}

\myendsectiontext

\mysection{Ghosts in the Machine}

{
 People hear about Friendly AI and say---this is one of the top
three initial reactions: }

{
 ``Oh, you can try to tell the AI to be Friendly,
but if the AI can modify its own source code, it'll
just remove any constraints you try to place on
it.''}

{
 And where does \textit{that} decision come from?}

{
 Does it enter from outside causality, rather than being an effect
of a lawful chain of causes that started with the source code as
originally written? Is the AI the ultimate source of its own free
will?}

{
 A Friendly AI is not a selfish AI constrained by a special extra
conscience module that overrides the AI's natural
impulses and tells it what to do. You just build the conscience, and
that \textit{is} the AI. If you have a program that computes which
decision the AI should make, you're \textit{done}. The
buck stops immediately.}

{
 At this point, I shall take a moment to quote some case studies
from the Computer Stupidities site and Programming subtopic. (I am not
linking to this, because it is a fearsome time-trap; you can Google if
you dare.)}

\begin{quote}
{
 I tutored college students who were taking a computer programming
course. A few of them didn't understand that computers
are not sentient. More than one person used comments in their Pascal
programs to put detailed explanations such as, ``Now I
need you to put these letters on the screen.'' I
asked one of them what the deal was with those comments. The reply:
``How else is the computer going to understand what I
want it to do?'' Apparently they would assume that
since they couldn't make sense of Pascal, neither could
the computer.}

{
 While in college, I used to tutor in the school's
math lab. A student came in because his BASIC program would not run. He
was taking a beginner course, and his assignment was to write a program
that would calculate the recipe for oatmeal cookies, depending upon the
number of people you're baking for. I looked at his
program, and it went something like this:}

\whencolumns{
\begin{flushleft}
\texttt{10 Preheat oven to 350\\
20 Combine all ingredients in a large mixing bowl\\
30 Mix until smooth}
\end{flushleft}
}{
\begin{flushleft}
\texttt{10 Preheat oven to 350\\
20 Combine all ingredients in a\\
\ \  large mixing bowl\\
30 Mix until smooth}
\end{flushleft}
}

{
 An introductory programming student once asked me to look at his
program and figure out why it was always churning out zeroes as the
result of a simple computation. I looked at the program, and it was
pretty obvious:}

\begin{verbatim}
begin
   read("Number of Apples", apples)
   read("Number of Carrots", carrots)
   read("Price for 1 Apple", a_price)
   read("Price for 1 Carrot", c_price)
   write("Total for Apples", a_total)
   write("Total for Carrots", c_total)
   write("Total", total)
   total = a_total + c_total
   a_total = apples * a_price
   c_total = carrots * c_price
end
\end{verbatim}

{
 Me: ``Well, your program can't
print correct results before they're
computed.''}

{
 Him: ``Huh? It's logical what the
right solution is, and the computer should reorder the instructions the
right way.''}
\end{quote}

{
 There's an instinctive way of imagining the
scenario of ``programming an AI.''
It maps onto a similar-seeming human endeavor: Telling a human being
what to do. Like the ``program'' is
giving instructions to a little ghost that sits inside the machine,
which will look over your instructions and decide whether it likes them
or not.}

{
 There is no ghost who looks over the instructions and decides how
to follow them. The program \textit{is} the AI.}

{
 That doesn't mean the ghost does anything you wish
for, like a genie. It doesn't mean the ghost does
everything you want the way you want it, like a slave of exceeding
docility. It means your instruction \textit{is} the only ghost
that's there, at least at boot time.}

{
 AI is much harder than people instinctively imagined, exactly
because you can't just \textit{tell} the ghost what to
do. You have to build the ghost from scratch, and everything that seems
obvious to you, the ghost will not see unless you know how to make the
ghost see it. You can't just \textit{tell} the ghost to
see it. You have to create that-which-sees from scratch.}

{
 If you don't know how to build something that
seems to have some strange ineffable elements like, say,
``decision-making,'' then you
can't just shrug your shoulders and let the
ghost's free will do the job. You're
left forlorn and ghostless.}

{
 There's more to building a chess-playing program
than building a really fast processor---so the AI will be really
smart---and then typing at the command prompt ``Make
whatever chess moves \textit{you} think are best.''
You might think that, since the programmers themselves are not very
good chess players, any advice they tried to give the electronic
superbrain would just slow the ghost down. But there is no ghost. You
see the problem.}

{
 And there isn't a simple spell you can perform
to---poof!---summon a complete ghost into the machine. You
can't say, ``I summoned the ghost, and
it appeared; that's cause and effect for
you.'' (It doesn't work if you use
the notion of ``emergence'' or
``complexity'' as a substitute for
``summon,'' either.) You
can't give an instruction to the CPU,
``Be a good chess player!'' You have
to see inside the mystery of chess-playing thoughts, and structure the
whole ghost from scratch.}

{
 No matter how common-sensical, no matter how logical, no matter
how ``obvious'' or
``right'' or
``self-evident'' or
``intelligent'' something seems to
you, it will not happen inside the ghost. \textit{Unless} it happens at
the end of a chain of cause and effect that began with the instructions
that \textit{you} had to decide on, plus any causal dependencies on
sensory data that you built into the starting instructions.}

{
 This doesn't mean you program in every decision
explicitly. Deep Blue was a chess player far superior to its
programmers. Deep Blue made better chess moves than anything its makers
could have explicitly programmed---but not because the programmers
shrugged and left it up to the ghost. Deep Blue moved better than its
programmers \ldots at the end of a chain of cause and effect that began
in the programmers' code and proceeded lawfully from
there. Nothing happened \textit{just} because it was so obviously a
good move that Deep Blue's ghostly free will took over,
without the code and its lawful consequences being involved.}

{
 If you try to wash your hands of constraining the AI, you
aren't left with a free ghost like an emancipated
slave. You are left with a heap of sand that no one has purified into
silicon, shaped into a CPU and programmed to think.}

{
 Go ahead, try telling a computer chip ``Do
whatever you want!'' See what happens? Nothing.
Because you haven't constrained it to understand
freedom.}

{
 All it takes is one single step that is \textit{so obvious, so
logical, so self-evident} that your mind just skips right over it, and
you've left the path of the AI programmer. It takes an
effort like the one I illustrate in Grasping Slippery Things to prevent
your mind from doing this.}

\myendsectiontext

\mysection{Artificial Addition}

{
 Suppose that human beings had absolutely \textit{no idea} how they
performed arithmetic. Imagine that human beings had \textit{evolved},
rather than having \textit{learned}, the ability to count sheep and add
sheep. People using this built-in ability have no idea how it worked,
the way Aristotle had no idea how his visual cortex supported his
ability to see things. Peano Arithmetic as we know it has not been
invented. There are philosophers working to formalize numerical
intuitions, but they employ notations such as}

\begin{center}
\texttt{Plus-Of(Seven,Six) = Thirteen}
\end{center}

{
 to formalize the intuitively obvious fact that when you add
``seven'' plus
``six,'' of course you get
``thirteen.'' }

{
 In this world, pocket calculators work by storing a giant lookup
table of arithmetical facts, entered manually by a team of expert
Artificial Arithmeticians, for starting values that range between zero
and one hundred. While these calculators may be helpful in a pragmatic
sense, many philosophers argue that they're only
\textit{simulating} addition, rather than really \textit{adding.} No
machine can really \textit{count}{}---that's why humans
have to count thirteen sheep before typing
``thirteen'' into the calculator.
Calculators can recite back stored facts, but they can never know what
the statements mean---if you type in ``two hundred
plus two hundred'' the calculator says
``Error: Outrange,'' when
it's intuitively \textit{obvious}, if you \textit{know}
what the words \textit{mean}, that the answer is
``four hundred.''}

{
 Some philosophers, of course, are not so naive as to be taken in
by these intuitions. Numbers are really a purely formal system---the
label ``thirty-seven'' is
meaningful, not because of any inherent property of the words
themselves, but because the label \textit{refers to} thirty-seven sheep
in the external world. A number is given this referential property by
its \textit{semantic network} of relations to other numbers.
That's why, in computer programs, the LISP token for
``thirty-seven''
doesn't need any \textit{internal}
structure---it's only meaningful because of reference
and relation, not some computational property of
``thirty-seven'' itself.}

{
 No one has ever developed an Artificial General Arithmetician,
though of course there are plenty of domain-specific, narrow Artificial
Arithmeticians that work on numbers between
``twenty'' and
``thirty,'' and so on. And if you
look at how slow progress has been on numbers in the range of
``two hundred,'' then it becomes
clear that we're not going to get Artificial General
Arithmetic any time soon. The best experts in the field estimate it
will be at least a hundred years before calculators can add as well as
a human twelve-year-old.}

{
 But not everyone agrees with this estimate, or with merely
conventional beliefs about Artificial Arithmetic. It's
common to hear statements such as the following:}

\begin{itemize}
\item {
 ``It's a framing problem---what
`twenty-one plus' equals depends on
whether it's `plus
three' or `plus four.'
If we can just get enough arithmetical facts stored to cover the
common-sense truths that everyone knows, we'll start to
see real addition in the network.''}

\item {
 ``But you'll never be able to
program in that many arithmetical facts by hiring experts to enter them
manually. What we need is an Artificial Arithmetician that can
\textit{learn} the vast network of relations between numbers that
humans acquire during their childhood by observing sets of
apples.''}

\item {
 ``No, what we really need is an Artificial
Arithmetician that can understand natural language, so that instead of
having to be explicitly told that twenty-one plus sixteen equals
thirty-seven, it can get the knowledge by exploring the
Web.''}

\item {
 ``Frankly, it seems to me that
you're just trying to convince yourselves that you can
solve the problem. None of you really know what arithmetic is, so
you're floundering around with these generic sorts of
arguments. `We need an AA that can learn
X,' `We need an AA that can extract X
from the Internet.' I mean, it sounds good, it sounds
like you're making progress, and it's
even good for public relations, because everyone thinks they understand
the proposed solution---but it doesn't really get you
any closer to \textit{general} addition, as opposed to domain-specific
addition. Probably we will never know the fundamental nature of
arithmetic. The problem is just too hard for humans to
solve.''}

\item {
 ``That's why we need to develop a
general arithmetician the same way Nature
did---evolution.''}

\item {
 ``Top-down approaches have clearly failed to
produce arithmetic. We need a bottom-up approach, some way to make
arithmetic \textit{emerge.} We have to acknowledge the basic
unpredictability of complex systems.''}

\item {
 ``You're all wrong. Past efforts
to create machine arithmetic were futile from the start, because they
just didn't have enough computing power. If you look at
how many trillions of synapses there are in the human brain,
it's clear that calculators don't have
lookup tables anywhere near that large. We need calculators as powerful
as a human brain. According to Moore's Law, this will
occur in the year 2031 on April 27 between 4:00 and 4:30 in the
morning.''}

\item {
 ``I believe that machine arithmetic will be
developed when researchers scan each neuron of a complete human brain
into a computer, so that we can simulate the biological circuitry that
performs addition in humans.''}

\item {
 ``I don't think we have to wait
to scan a whole brain. Neural networks are just like the human brain,
and you can train them to do things without knowing how they do them.
We'll create programs that will do arithmetic without
we, our creators, ever understanding how they do
arithmetic.''}

\item {
 ``But Gödel's Theorem shows that
no formal system can ever capture the basic properties of arithmetic.
Classical physics is formalizable, so to add two and two, the brain
must take advantage of quantum physics.''}

\item {
 ``Hey, if human arithmetic were simple enough
that we could reproduce it in a computer, we wouldn't
be able to count high enough to build computers.''}

\item {
 ``Haven't you heard of John
Searle's Chinese Calculator Experiment? Even if you did
have a huge set of rules that would let you add
`twenty-one' and
`sixteen,' just imagine translating all
the words into Chinese, and you can see that there's no
genuine addition going on. There are no real \textit{numbers} anywhere
in the system, just labels that humans use for numbers
\ldots''}
\end{itemize}

{
 There is more than one moral to this parable, and I have told it
with different morals in different contexts. It illustrates the idea of
levels of organization, for example---a CPU can add two large numbers
because the numbers aren't black-box opaque objects,
they're ordered structures of 32 bits.}

{
 But for purposes of overcoming bias, let us draw two morals:}

\begin{itemize}
\item {
 First, the danger of believing assertions you
can't regenerate from your own knowledge.}

\item {
  Second, the danger of trying to dance around basic confusions.}
\end{itemize}

{
 Lest anyone accuse me of generalizing from fictional evidence,
both lessons may be drawn from the real history of Artificial
Intelligence as well.}

{
 The first danger is the object-level problem that the AA devices
ran into: they functioned as tape recorders playing back
``knowledge'' generated from outside
the system, using a process they couldn't capture
internally. A human could tell the AA device that
``twenty-one plus sixteen equals
thirty-seven,'' and the AA devices could record this
sentence and play it back, or even pattern-match
``twenty-one plus sixteen'' to
output ``thirty-seven!''---but the
AA devices couldn't generate such knowledge for
themselves.}

{
 Which is strongly reminiscent of believing a physicist who tells
you ``Light is waves,'' recording
the fascinating words and playing them back when someone asks
``What is light made of?,'' without
being able to generate the knowledge for yourself.}

{
 The second moral is the meta-level danger that consumed the
Artificial Arithmetic researchers and opinionated bystanders---the
danger of dancing around confusing gaps in your knowledge. The tendency
to do just about anything \textit{except} grit your teeth and buckle
down and fill in the damn gap.}

{
 Whether you say, ``It is
emergent!,'' or whether you say,
``It is unknowable!,'' in neither
case are you acknowledging that there is a basic insight required which
is possessable, but unpossessed by you.}

{
 How can you know when you'll have a new basic
insight? And there's no way to get one except by
banging your head against the problem, learning everything you can
about it, studying it from as many angles as possible, perhaps for
years. It's not a pursuit that academia is set up to
permit, when you need to publish at least one paper per month.
It's certainly not something that venture capitalists
will fund. You want to either go ahead and build the system
\textit{now}, or give up and do something else instead.}

{
 Look at the comments above: none are aimed at setting out on a
quest for the missing insight which would \textit{make numbers no
longer mysterious}, make
``twenty-seven'' more than a black
box. None of the commenters realized that their difficulties arose from
ignorance or confusion in their own minds, rather than an inherent
property of arithmetic. They were not trying to achieve a state where
the confusing thing ceased to be confusing.}

{
 If you read Judea Pearl's \textit{Probabilistic
Reasoning in Intelligent Systems: Networks of Plausible
Inference},\footnote{Pearl, \textit{Probabilistic Reasoning in Intelligent
Systems}.\comment{1}} then you will see that the basic insight
behind graphical models is \textit{indispensable} to problems that
require it. (It's not something that fits on a T-shirt,
I'm afraid, so you'll have to go and
read the book yourself. I haven't seen any online
popularizations of Bayesian networks that adequately convey the reasons
behind the principles, or the importance of the math being exactly the
way it is, but Pearl's book is wonderful.) There were
once dozens of ``non-monotonic
logics'' awkwardly trying to capture intuitions such
as ``If my burglar alarm goes off, there was probably
a burglar, but if I then learn that there was a small earthquake near
my home, there was probably not a burglar.'' With the
graphical-model insight in hand, you can give a mathematical
explanation of exactly why first-order logic has the wrong properties
for the job, and express the correct solution in a compact way that
captures all the common-sense details in one elegant swoop. Until you
have that insight, you'll go on patching the logic
here, patching it there, adding more and more hacks to force it into
correspondence with everything that seems ``obviously
true.''}

{
 You won't \textit{know} the Artificial Arithmetic
problem is unsolvable without its key. If you don't
know the rules, you don't know the rule that says you
need to know the rules to do anything. And so there will be all sorts
of clever ideas that seem like they might work, like building an
Artificial Arithmetician that can read natural language and download
millions of arithmetical assertions from the Internet.}

{
 And yet \textit{somehow} the clever ideas never work. Somehow it
always turns out that you ``couldn't
see any reason it wouldn't work''
because you were ignorant of the obstacles, not because no obstacles
existed. Like shooting blindfolded at a distant target---you can fire
blind shot after blind shot, crying, ``You
can't prove to me that I won't hit the
center!'' But until you take off the blindfold,
you're not even in the aiming game. When
``no one can prove to you'' that
your precious idea \textit{isn't} right, it means you
don't have enough information to strike a small target
in a vast answer space. \textit{Until you know your idea will work, it
won't.}}

{
 From the history of previous key insights in Artificial
Intelligence, and the grand messes that were proposed prior to those
insights, I derive an important real-life lesson: \textit{When the
basic problem is your ignorance, clever strategies for bypassing your
ignorance lead to shooting yourself in the foot.}}

\myendsectiontext


\bigskip

\mysection{Terminal Values and Instrumental Values}

{
 On a purely instinctive level, any human planner behaves as if
they distinguish between means and ends. Want chocolate?
There's chocolate at the Publix supermarket. You can
get to the supermarket if you drive one mile south on Washington Ave.
You can drive if you get into the car. You can get into the car if you
open the door. You can open the door if you have your car keys. So you
put your car keys into your pocket, and get ready to leave the house
\ldots }

{
 \ldots when suddenly the word comes on the radio that an earthquake
has destroyed all the chocolate at the local Publix. Well,
there's no point in driving to the Publix if
there's no chocolate there, and no point in getting
into the car if you're not driving anywhere, and no
point in having car keys in your pocket if you're not
driving. So you take the car keys out of your pocket, and call the
local pizza service and have them deliver a chocolate pizza. Mm,
delicious.}

{
 I rarely notice people losing track of plans they devised
themselves. People usually don't drive to the
supermarket if they know the chocolate is gone. But
I've also noticed that when people begin
\textit{explicitly} talking about goal systems instead of just
\textit{wanting} things, \textit{mentioning}
``goals'' instead of \textit{using}
them, they oft become confused. Humans are experts at planning, not
experts on planning, or there'd be a lot more AI
developers in the world.}

{
 In particular, I've noticed people get confused
when---in abstract philosophical discussions rather than everyday
life---they consider the distinction between means and ends; more
formally, between ``instrumental
values'' and ``terminal
values.''}

{
 Part of the problem, it seems to me, is that the human mind uses a
rather ad-hoc system to keep track of its goals---it works, but not
cleanly. English doesn't embody a sharp distinction
between means and ends: ``I want to save my
sister's life'' and
``I want to administer penicillin to my
sister'' use the same word
``want.''}

{
 Can we describe, in mere English, the distinction that is getting
lost?}

{
 As a first stab:}

{
 ``Instrumental values'' are
desirable strictly conditional on their anticipated consequences.
``I want to administer penicillin to my
sister,'' not because a penicillin-filled sister is
an intrinsic good, but in anticipation of penicillin curing her
flesh-eating pneumonia. If instead you anticipated that injecting
penicillin would melt your sister into a puddle like the Wicked Witch
of the West, you'd fight just as hard to keep her
penicillin-free.}

{
 ``Terminal values'' are
desirable without conditioning on other consequences:
``I want to save my sister's
life'' has nothing to do with your anticipating
whether she'll get injected with penicillin after
that.}

{
 This first attempt suffers from obvious flaws. If saving my
sister's life would cause the Earth to be swallowed up
by a black hole, then I would go off and cry for a while, but I
wouldn't administer penicillin. Does this mean that
saving my sister's life was not a
``terminal'' or
``intrinsic'' value, because
it's theoretically conditional on its consequences? Am
I \textit{only} trying to save her life because of my belief that a
black hole won't consume the Earth afterward? Common
sense should say that's not what's
happening.}

{
 So forget English. We can set up a mathematical description of a
decision system in which terminal values and instrumental values are
separate and incompatible types---like integers and floating-point
numbers, in a programming language with no automatic conversion between
them.}

{
 An ideal Bayesian decision system can be set up using only four
elements:}

\begin{itemize}
\item  \texttt{Outcomes : type Outcome[]}
  \begin{itemize}
    \item list of possible outcomes
    \item \{sister lives, sister dies\}
  \end{itemize}

\item \texttt{Actions : type Action[]}
  \begin{itemize}
    \item list of possible actions
    \item \{administer penicillin, don't administer penicillin\}
  \end{itemize}

\item \texttt{Utility\_function : type Outcome -{\textgreater} Utility}
  \begin{itemize}
    \item utility
      function that maps each outcome onto a utility
    \item (a utility being
      representable as a real number between negative and positive infinity)
    \item \{sister lives $\rightarrow $ 1,
      sister dies $\rightarrow $ 0\}
  \end{itemize}

\item \texttt{Conditional\_probability\_function :\newline
  type Action -{\textgreater} (Outcome -{\textgreater} Probability)}
  \begin{itemize}
    \item conditional probability function that maps each action onto a
      probability distribution over outcomes
    \item (a probability being
      representable as a real number between 0 and 1)
    \item \{administer penicillin $\rightarrow
$ (sister lives $\rightarrow $ 0.9, sister dies $\rightarrow $ 0.1),
don't administer penicillin $\rightarrow $ (sister
lives $\rightarrow $ 0.3, sister dies $\rightarrow $
0.7)\}
  \end{itemize}
\end{itemize}

{
 If you can't read the type system directly,
don't worry, I'll always translate into
English. For programmers, seeing it described in distinct statements
helps to set up distinct mental objects.}

{
 And the decision system itself?}

\begin{itemize}
\item \texttt{ Expected\_Utility : Action A -{\textgreater}\newline
  (Sum O in Outcomes: Utility(O) * Probability(O{\textbar}A))}
  \begin{itemize}
    \item  The ``expected utility'' of an action
equals the sum, over all outcomes, of the utility of that outcome times
the conditional probability of that outcome given that action.
\item \{EU(administer penicillin) = 0.9,
EU(don't administer penicillin) =
0.3\}
  \end{itemize}

\item \texttt{ Choose :\newline
  -{\textgreater} (Argmax A in Actions: Expected\_Utility(A))}
  \begin{itemize}
  \item  Pick an action whose ``expected utility'' is maximal.
  \item \{return: administer penicillin\}
  \end{itemize}
\end{itemize}

{
 For every action, calculate the conditional probability of all the
consequences that might follow, then add up the utilities of those
consequences times their conditional probability. Then pick the best
action.}

{
 This is a mathematically simple sketch of a decision system. It is
not an efficient way to compute decisions in the real world.}

{
 What if, for example, you need a \textit{sequence} of acts to
carry out a plan? The formalism can easily represent this by letting
each Action stand for a whole sequence. But this creates an
exponentially large space, like the space of all sentences you can type
in 100 letters. As a simple example, if one of the possible acts on the
first turn is ``Shoot my own foot
off,'' a human planner will decide this is a bad idea
generally---eliminate \textit{all} sequences beginning with this
action. But we've flattened this structure out of our
representation. We don't have sequences of acts, just
flat ``actions.''}

{
 So, yes, there are \textit{a few minor complications.} Obviously
so, or we'd just run out and build a real AI this way.
In that sense, it's much the same as Bayesian
probability theory itself.}

{
 But this is one of those times when it's a
\textit{surprisingly good idea} to consider the absurdly simple version
before adding in any high-falutin' complications.}

{
 Consider the philosopher who asserts, ``All of us
are ultimately selfish; we care only about our own states of mind. The
mother who claims to care about her son's welfare,
really wants to \textit{believe} that her son is doing well---this
belief is what makes the mother happy. She helps him for the sake of
her own happiness, not his.'' You say,
``Well, suppose the mother sacrifices her life to push
her son out of the path of an oncoming truck. That's
not going to make her happy, just dead.'' The
philosopher stammers for a few moments, then replies,
``But she still did it because \textit{she valued}
that choice above others---because of the \textit{feeling of
importance} she attached to that decision.''}

{
 So you say,}

\texttt{
 TYPE ERROR: No constructor found for Expected\_Utility
-{\textgreater} Utility.}

{
 Allow me to explain that reply.}

{
 Even our simple formalism illustrates a sharp distinction between
\textit{expected utility}, which is something that \textit{actions}
have; and \textit{utility}, which is something that \textit{outcomes}
have. Sure, you can map both utilities and expected utilities onto real
numbers. But that's like observing that you can map
wind speed and temperature onto real numbers. It
doesn't make them the same thing.}

{
 The philosopher begins by arguing that all your Utilities must be
over Outcomes consisting of your state of mind. If this were true, your
intelligence would operate \textit{as an engine to steer the future}
into regions where you were happy. Future states would be distinguished
only by your state of mind; you would be indifferent between any two
futures in which you had the same state of mind.}

{
 And you would, indeed, be rather unlikely to sacrifice your own
life to save another.}

{
 When we object that people sometimes \textit{do} sacrifice their
lives, the philosopher's reply shifts to discussing
Expected Utilities over Actions: ``The feeling of
\textit{importance} she attached to that
\textit{decision}.'' This is a drastic jump that
\textit{should} make us leap out of our chairs in indignation. Trying
to convert an \texttt{Expected\_Utility} into a \texttt{Utility} would cause an outright
error in our programming language. But in English it all sounds the
same.}

{
 The choices of our simple decision system are those with highest
\texttt{Expected\_Utility}, but this doesn't say anything
whatsoever about \textit{where it steers the future.} It
doesn't say anything about the utilities the decider
assigns, or which real-world outcomes are likely to happen as a result.
It doesn't say anything about the
mind's function as an engine.}

{
 The physical cause of a physical action is a cognitive state, in
our ideal decider an \texttt{Expected\_Utility}, and this expected utility is
calculated by evaluating a utility function over imagined consequences.
To save your son's life, you must imagine the event of
your son's life being saved, and this imagination is
not the event itself. It's a \textit{quotation}, like
the difference between ``snow'' and
snow. But that doesn't mean that what's
\textit{inside the quote marks} must itself be a cognitive state. If
you choose the action that leads to the future that you represent with
``my son is still alive,'' then you
have functioned as an engine to steer the future into a region where
your son is still alive. Not an engine that steers the future into a
region where \textit{you represent the sentence} ``my
son is still alive.'' To steer the future
\textit{there}, your utility function would have to return a high
utility when fed ````my son is still
alive'''', the quotation of the
quotation, your imagination of yourself imagining. Recipes make poor
cake when you grind them up and toss them in the batter.}

{
 And that's why it's helpful to
consider the simple decision systems first. Mix enough complications
into the system, and formerly clear distinctions become harder to see.}

{
 So now let's look at some complications. Clearly
the \texttt{Utility} function (mapping \texttt{Outcomes} onto \texttt{Utilities}) is meant to
formalize what I earlier referred to as ``terminal
values,'' values not contingent upon their
consequences. What about the case where saving your
sister's life leads to Earth's
destruction by a black hole? In our formalism, we've
flattened out this possibility. \texttt{Outcomes} don't lead to
\texttt{Outcomes}, only \texttt{Actions} lead to \texttt{Outcomes}. Your sister recovering from
pneumonia \textit{followed by} the Earth being devoured by a black hole
would be flattened into a single ``possible
outcome.''}

{
 And where are the ``instrumental
values'' in this simple formalism? Actually,
they've vanished entirely! You see, in this formalism,
actions lead directly to outcomes with no intervening events.
There's no notion of throwing a rock that flies through
the air and knocks an apple off a branch so that it falls to the
ground. Throwing the rock is the \texttt{Action}, and it leads straight to the
\texttt{Outcome} of the apple lying on the ground---according to the conditional
probability function that turns an \texttt{Action} directly into a \texttt{Probability}
distribution over \texttt{Outcomes}.}

{
 In order to \textit{actually compute} the conditional probability
function, and in order to separately consider the utility of a
sister's pneumonia and a black hole swallowing Earth,
we would have to represent the network structure of causality---the way
that events lead to other events.}

{
 And then the instrumental values would start coming back. If the
causal network was sufficiently regular, you could find a state $B$ that
tended to lead to $C$ regardless of how you achieved $B$. Then if you
wanted to achieve $C$ for some reason, you could plan efficiently by
first working out a $B$ that led to $C$, and then an $A$ that led to $B$. This
would be the phenomenon of ``instrumental
value''---$B$ would have
``instrumental value'' because it
led to $C$. The state $C$ itself might be terminally valued---a term in the
utility function over the total outcome. Or $C$ might just be an
instrumental value, a node that was not directly valued by the utility
function.}

{
 Instrumental value, in this formalism, is purely an aid to the
efficient computation of plans. It can and should be discarded wherever
this kind of regularity does not exist.}

{
 Suppose, for example, that there's some particular
value of $B$ that doesn't lead to $C$. Would you choose an
$A$ which led to that $B$? Or never mind the abstract philosophy: If you
wanted to go to the supermarket to get chocolate, and you wanted to
drive to the supermarket, and you needed to get into your car, would
you gain entry by ripping off the car door with a steam shovel? (No.)
Instrumental value is a ``leaky
abstraction,'' as we programmers say; you sometimes
have to toss away the cached value and compute out the actual expected
utility. Part of being \textit{efficient} without being
\textit{suicidal} is noticing when convenient shortcuts break down.
Though this formalism does give rise to instrumental values, it does so
only where the requisite regularity exists, and strictly as a
convenient shortcut in computation.}

{
 But if you complicate the formalism before you understand the
simple version, then you may start thinking that instrumental values
have some strange life of their own, even in a normative sense. That,
once you say $B$ is usually good because it leads to $C$,
you've committed yourself to always try for $B$ even in
the absence of $C$. People make this kind of mistake in abstract
philosophy, even though they would never, in real life, rip open their
car door with a steam shovel. You may start thinking that
there's no way to develop a consequentialist that
maximizes only inclusive genetic fitness, because it will starve unless
you include an explicit terminal value for ``eating
food.'' People make this mistake even though they
would never stand around opening car doors all day long, for fear of
being stuck outside their cars if they didn't have a
terminal value for opening car doors.}

{
 Instrumental values live in (the network structure of) the
conditional probability function. This makes instrumental value
strictly dependent on beliefs-of-fact given a fixed utility function.
If I believe that penicillin causes pneumonia, and that the absence of
penicillin cures pneumonia, then my perceived instrumental value of
penicillin will go from high to low. Change the beliefs of
fact---change the conditional probability function that associates
actions to believed consequences---and the instrumental values will
change in unison.}

{
 In moral arguments, some disputes are about instrumental
consequences, and some disputes are about terminal values. If your
debating opponent says that banning guns will lead to lower crime, and
you say that banning guns will lead to higher crime, then you agree
about a superior instrumental value (crime is bad), but you disagree
about which intermediate events lead to which consequences. But I do
not think an argument about female circumcision is really a factual
argument about how to best achieve a shared value of treating women
fairly or making them happy.}

{
 This important distinction often gets \textit{flushed down the
toilet} in angry arguments. People with factual disagreements and
shared values each decide that their debating opponents must be
sociopaths. As if your hated enemy, gun control/rights advocates,
\textit{really wanted to kill people}, which should be implausible as
realistic psychology.}

{
 I fear the human brain does not strongly type the distinction
between terminal moral beliefs and instrumental moral beliefs.
``We should ban guns'' and
``We should save lives''
don't \textit{feel different}, as moral beliefs, the
way that sight feels different from sound. Despite all the other ways
that the human goal system complicates everything in sight, this
\textit{one distinction} it manages to collapse into a mishmash of
things-with-conditional-value.}

{
 To extract out the terminal values we have to inspect this
mishmash of valuable things, trying to figure out which ones are
getting their value from somewhere else. It's a
difficult project! If you say that you want to ban guns in order to
reduce crime, it may take a moment to realize that
``reducing crime''
isn't a terminal value, it's a superior
instrumental value with links to terminal values for human lives and
human happinesses. And then the one who advocates gun rights may have
links to the superior instrumental value of ``reducing
crime'' plus a link to a value for
``freedom,'' which might be a
terminal value unto them, or another instrumental value \ldots}

{
 We can't print out our complete network of values
derived from other values. We probably don't even store
the whole history of how values got there. By considering the right
moral dilemmas, ``Would you do $X$ if
$Y$,'' we can often figure out where our values came
from. But even this project itself is full of pitfalls; misleading
dilemmas and gappy philosophical arguments. We don't
know what our own values are, or where they came from, and
can't find out except by undertaking error-prone
projects of cognitive archaeology. Just forming a conscious distinction
between ``terminal value'' and
``instrumental value,'' and keeping
track of what it means, and using it correctly, is hard work. Only by
inspecting the simple formalism can we see how easy it ought to be, in
principle.}

{
 And that's to say nothing of all the other
complications of the human reward system---the whole use of
reinforcement architecture, and the way that eating chocolate is
pleasurable, and anticipating eating chocolate is pleasurable, but
they're different kinds of pleasures \ldots}

{
 But I don't complain too much about the mess.}

{
 Being ignorant of your own values may not always be fun, but at
least it's not boring.}

\myendsectiontext

\mysection{Leaky Generalizations}

{
 Are apples good to eat? Usually, but some apples are rotten. }

{
 Do humans have ten fingers? Most of us do, but plenty of people
have lost a finger and nonetheless qualify as
``human.''}

{
 Unless you descend to a level of description far below any
macroscopic object---below societies, below people, below fingers,
below tendon and bone, below cells, all the way down to particles and
fields where the laws are truly universal---practically every
generalization you use in the real world will be leaky.}

{
 (Though there may, of course, be some exceptions to the above rule
\ldots)}

{
 Mostly, the way you deal with leaky generalizations is that, well,
you just have to deal. If the cookie market almost always closes at 10
p.m., except on Thanksgiving it closes at 6 p.m., and today happens to
be National Native American Genocide Day, you'd better
show up before 6 p.m. or you won't get a cookie.}

{
 Our ability to manipulate leaky generalizations is opposed by
\textit{need for closure}, the degree to which we want to say once and
for all that humans have ten fingers, and get frustrated when we have
to tolerate continued ambiguity. Raising the value of the stakes can
increase need for closure---which shuts down complexity tolerance when
complexity tolerance is most needed.}

{
 Life would be complicated even if the things we wanted were simple
(they aren't). The leakyness of leaky generalizations
about what-to-do-next would leak in from the leaky structure of the
real world. Or to put it another way:}

{
 Instrumental values often have no specification that is both
compact and local.}

{
 Suppose there's a box containing a million
dollars. The box is locked, not with an ordinary combination lock, but
with a dozen keys controlling a machine that can open the box. If you
know how the machine works, you can deduce which sequences of
key-presses will open the box. There's more than one
key sequence that can trigger the machine to open the box. But if you
press a sufficiently wrong sequence, the machine incinerates the money.
And if you \textit{don't know} about the machine,
there's no simple rules like
``Pressing any key three times opens the
box'' or ``Pressing five different
keys with no repetitions incinerates the money.''}

{
 There's a \textit{compact nonlocal} specification
of which keys you want to press: You want to press keys such that they
open the box. You can write a compact computer program that computes
which key sequences are good, bad or neutral, but the computer program
will need to describe the machine, not just the keys themselves.}

{
 There's likewise a \textit{local noncompact}
specification of which keys to press: a giant lookup table of the
results for each possible key sequence. It's a very
large computer program, but it makes no mention of anything except the
keys.}

{
 But there's no way to describe which key sequences
are good, bad, or neutral, which is both \textit{simple} and phrased
\textit{only in terms of the keys themselves.}}

{
 It may be even worse if there are tempting local generalizations
which turn out to be \textit{leaky}. Pressing \textit{most} keys three
times in a row will open the box, but there's a
particular key that incinerates the money if you press it just once.
You might think you had found a perfect generalization---a locally
describable class of sequences that \textit{always} opened the
box---when you had merely failed to visualize all the possible paths of
the machine, or failed to value all the side effects.}

{
 The machine represents the complexity of the real world. The
openness of the box (which is good) and the incinerator (which is bad)
represent the thousand shards of desire that make up our terminal
values. The keys represent the actions and policies and strategies
available to us.}

{
 When you consider how many different ways we value outcomes, and
how complicated are the paths we take to get there,
it's a wonder that there exists any such thing as
helpful ethical \textit{advice}. (Of which the strangest of all
advices, \textit{and yet still helpful}, is that ``the
end does not justify the means.'')}

{
 But conversely, the complicatedness of action need not say
anything about the complexity of goals. You often find people who smile
wisely, and say, ``Well, morality is complicated, you
know, female circumcision is right in one culture and wrong in another,
it's not always a bad thing to torture people. How
naive you are, how full of need for closure, that you think there are
any simple rules.''}

{
 You can say, unconditionally and flatly, that killing
\textit{anyone} is a huge dose of negative terminal utility. Yes, even
Hitler. That doesn't mean you shouldn't
shoot Hitler. It means that the net instrumental utility of shooting
Hitler carries a giant dose of negative utility from
Hitler's death, and a hugely larger dose of positive
utility from all the other lives that would be saved as a consequence.}

{
 Many commit the type error that I warned against in Terminal
Values and Instrumental Values, and think that if the net consequential
expected utility of Hitler's death is conceded to be
positive, then the immediate local terminal utility must also be
positive, meaning that the moral principle ``Death is
always a bad thing'' is itself a leaky
generalization. But this is double counting, with utilities instead of
probabilities; you're setting up a resonance between
the expected utility and the utility, instead of a one-way flow from
utility to expected utility.}

{
 Or maybe it's just the urge toward a one-sided
policy debate: the best policy must have \textit{no} drawbacks.}

{
 In my moral philosophy, the \textit{local} negative utility of
Hitler's death is stable, no matter what happens to the
external consequences and hence to the \textit{expected} utility.}

{
 Of course, you can set up a moral argument that
it's an \textit{inherently} good thing to punish evil
people, even with capital punishment for sufficiently evil people. But
you can't carry this moral argument by pointing out
that the \textit{consequence} of shooting a man holding a leveled gun
may be to save other lives. This is appealing to the value of life, not
appealing to the value of death. If expected utilities are leaky and
complicated, it doesn't mean that utilities must be
leaky and complicated as well. They might be! But it would be a
separate argument.}

\myendsectiontext

\mysection{The Hidden Complexity of Wishes}

\begin{quote}
{
 I wish to live in the locations of my choice, in a physically
healthy, uninjured, and apparently normal version of my current body
containing my current mental state, a body which will heal from all
injuries at a rate three sigmas faster than the average given the
medical technology available to me, and which will be protected from
any diseases, injuries or illnesses causing disability, pain, or
degraded functionality or any sense, organ, or bodily function for more
than ten days consecutively or fifteen days in any year \ldots}

{\raggedleft
 {}---The Open-Source Wish Project, Wish For Immortality 1.1
 \par}
\end{quote}



{
 There are three kinds of genies: Genies to whom you can safely
say, ``I wish for you to do what I should wish
for''; genies for which \textit{no} wish is safe; and
genies that aren't very powerful or intelligent.}

{
 Suppose your aged mother is trapped in a burning building, and it
so happens that you're in a wheelchair; you
can't rush in yourself. You could cry,
``Get my mother out of that
building!'' but there would be no one to hear.}

{
 Luckily you have, in your pocket, an Outcome Pump. This handy
device squeezes the flow of time, pouring probability into some
outcomes, draining it from others.}

{
 The Outcome Pump is not sentient. It contains a tiny time machine,
which resets time \textit{unless} a specified outcome occurs. For
example, if you hooked up the Outcome Pump's sensors to
a coin, and specified that the time machine should keep resetting until
it sees the coin come up heads, and then you actually flipped the coin,
\textit{you} would see the coin come up heads. (The physicists say that
any future in which a ``reset''
occurs is inconsistent, and therefore never happens in the first
place---so you aren't actually killing any versions of
yourself.)}

{
 Whatever proposition you can manage to input into the Outcome Pump
\textit{somehow happens}, though not in a way that violates the laws of
physics. If you try to input a proposition that's
\textit{too} unlikely, the time machine will suffer a spontaneous
mechanical failure before that outcome ever occurs.}

{
 You can also redirect probability flow in more quantitative ways,
using the ``future function'' to
scale the temporal reset probability for different outcomes. If the
temporal reset probability is 99\% when the coin comes up heads, and
1\% when the coin comes up tails, the odds will go from 1:1 to 99:1 in
favor of tails. If you had a mysterious machine that spit out money,
and you wanted to maximize the amount of money spit out, you would use
reset probabilities that diminished as the amount of money increased.
For example, spitting out \$10 might have a 99.999999\% reset
probability, and spitting out \$100 might have a 99.99999\% reset
probability. This way you can get an outcome that tends to be as high
as possible in the future function, even when you don't
know the best attainable maximum.}

{
 So you desperately yank the Outcome Pump from your pocket---your
mother is still trapped in the burning building, remember?---and try to
describe your goal: \textit{get your mother out of the building!}}

{
 The user interface doesn't take English inputs.
The Outcome Pump isn't sentient, remember? But it does
have 3D scanners for the near vicinity, and built-in utilities for
pattern matching. So you hold up a photo of your
mother's head and shoulders; match on the photo; use
object contiguity to select your mother's whole body
(not just her head and shoulders); and define the \textit{future
function} using your mother's distance from the
building's center. The further she gets from the
building's center, the less the time
machine's reset probability.}

{
 You cry ``Get my mother out of the
building!,'' for luck, and press Enter.}

{
 For a moment it seems like nothing happens. You look around,
waiting for the fire truck to pull up, and rescuers to arrive---or even
just a strong, fast runner to haul your mother out of the building---}

{
 \textit{BOOM!} With a thundering roar, the gas main under the
building explodes. As the structure comes apart, in what seems like
slow motion, you glimpse your mother's shattered body
being hurled high into the air, traveling fast, rapidly increasing its
distance from the former center of the building.}

{
 On the side of the Outcome Pump is an Emergency Regret Button. All
future functions are automatically defined with a huge negative value
for the Regret Button being pressed---a temporal reset probability of
nearly 1---so that the Outcome Pump is extremely unlikely to do
anything which upsets the user enough to make them press the Regret
Button. You can't ever remember pressing it. But
you've barely started to reach for the Regret Button
(and what good will it do now?) when a flaming wooden beam drops out of
the sky and smashes you flat.}

{
 Which wasn't really what you wanted, but scores
very high in the defined future function \ldots}

{
 The Outcome Pump is a genie of the second class. \textit{No} wish
is safe.}

{
 If someone asked you to get their poor aged mother out of a
burning building, you might help, or you might pretend not to hear. But
it wouldn't even \textit{occur} to you to explode the
building. ``Get my mother out of the
building'' \textit{sounds} like a much safer wish
than it really is, because you don't even
\textit{consider} the plans that you assign extreme negative values.}

{
 Consider again the Tragedy of Group Selectionism: Some early
biologists asserted that group selection for low subpopulation sizes
would produce individual restraint in breeding; and yet actually
enforcing group selection in the laboratory produced cannibalism,
especially of immature females. It's obvious in
hindsight that, given strong selection for small subpopulation sizes,
cannibals will outreproduce individuals who voluntarily forego
reproductive opportunities. But eating little girls is such an
\textit{un-aesthetic} solution that Wynne-Edwards, Allee, Brereton, and
the other group-selectionists simply didn't think of
it. They only saw the solutions they would have used themselves.}

{
 Suppose you try to patch the future function by specifying that
the Outcome Pump should not explode the building: outcomes in which the
building materials are distributed over too much volume will have \~{}1
temporal reset probabilities.}

{
 So your mother falls out of a second-story window and breaks her
neck. The Outcome Pump took a different path through time that still
ended up with your mother outside the building, and it still
wasn't what you wanted, and it still
wasn't a solution that would occur to a human rescuer.}

{
 If only the Open-Source Wish Project had developed a Wish To Get
Your Mother Out Of A Burning Building:}

\begin{quote}
{
 I wish to move my mother (defined as the woman who shares half my
genes and gave birth to me) to outside the boundaries of the building
currently closest to me which is on fire; but not by exploding the
building; nor by causing the walls to crumble so that the building no
longer has boundaries; nor by waiting until after the building finishes
burning down for a rescue worker to take out the body \ldots}
\end{quote}

{
 All these special cases, the seemingly unlimited number of
required patches, should remind you of the parable of Artificial
Addition---programming an Arithmetic Expert Systems by explicitly
adding ever more assertions like ``fifteen plus
fifteen equals thirty, but fifteen plus sixteen equals thirty-one
instead.''}

{
 How do \textit{you} exclude the outcome where the building
explodes and flings your mother into the sky? You look ahead, and you
foresee that your mother would end up dead, and you
don't want that consequence, so you try to forbid the
event leading up to it.}

{
 Your brain isn't hardwired with a specific,
prerecorded statement that ``Blowing up a burning
building containing my mother is a bad idea.'' And
yet you're trying to prerecord that exact specific
statement in the Outcome Pump's future function. So the
wish is exploding, turning into a giant lookup table that records your
judgment of every possible path through time.}

{
 You failed to ask for what you really wanted. You \textit{wanted}
your mother to go on living, but you \textit{wished} for her to become
more distant from the center of the building.}

{
 Except that's not all you wanted. If your mother
was rescued from the building but was horribly burned, that outcome
would rank lower in your preference ordering than an outcome where she
was rescued safe and sound. So you not only value your
mother's life, but also her health.}

{
 And you value not just her bodily health, but her state of mind.
Being rescued in a fashion that traumatizes her---for example, a giant
purple monster roaring up out of nowhere and seizing her---is inferior
to a fireman showing up and escorting her out through a non-burning
route. (Yes, we're supposed to stick with physics, but
maybe a powerful enough Outcome Pump has aliens coincidentally showing
up in the neighborhood at exactly that moment.) You would certainly
prefer her being rescued by the monster to her being roasted alive,
however.}

{
 How about a wormhole spontaneously opening and swallowing her to a
desert island? Better than her being dead; but worse than her being
alive, well, healthy, untraumatized, and in continual contact with you
and the other members of her social network.}

{
 Would it be okay to save your mother's life at the
cost of the family dog's life, if it ran to alert a
fireman but then got run over by a car? Clearly yes, but it would be
better \textit{ceteris paribus} to avoid killing the dog. You
wouldn't want to swap a human life for hers, but what
about the life of a convicted murderer? Does it matter if the murderer
dies trying to save her, from the goodness of his heart? How about two
murderers? If the cost of your mother's life was the
destruction of every extant copy, including the memories, of
Bach's \textit{Little Fugue in G Minor}, would that be
worth it? How about if she had a terminal illness and would die anyway
in eighteen months?}

{
 If your mother's foot is crushed by a burning
beam, is it worthwhile to extract the rest of her? What if her head is
crushed, leaving her body? What if her body is crushed, leaving only
her head? What if there's a cryonics team waiting
outside, ready to suspend the head? Is a frozen head a person? Is Terry
Schiavo a person? How much is a chimpanzee worth?}

{
 Your brain is not infinitely complicated; there is only a finite
Kolmogorov complexity / message length which suffices to describe all
the judgments you would make. But just because this complexity is
finite does not make it small. We value many things, and no they are
\textit{not} reducible to valuing happiness or valuing reproductive
fitness.}

{
 There is no safe wish smaller than an entire human morality. There
are too many possible paths through Time. You can't
visualize all the roads that lead to the destination you give the
genie. ``Maximizing the distance between your mother
and the center of the building'' can be done even
more effectively by detonating a nuclear weapon. Or, at higher levels
of genie power, flinging her body out of the Solar System. Or, at
higher levels of genie intelligence, doing something that neither you
nor I would think of, just like a chimpanzee wouldn't
think of detonating a nuclear weapon. You can't
visualize all the paths through time, any more than you can program a
chess-playing machine by hardcoding a move for every possible board
position.}

{
 And real life is far more complicated than chess. You cannot
predict, in advance, which of your values will be needed to judge the
path through time that the genie takes. Especially if you wish for
something longer-term or wider-range than rescuing your mother from a
burning building.}

{
 I fear the Open-Source Wish Project is futile, except as an
illustration of how \textit{not} to think about genie problems. The
only safe genie is a genie that shares all your judgment criteria, and
at that point, you can just say ``I wish for you to do
what I should wish for.'' Which simply runs the
genie's \textit{should} function.}

{
 Indeed, it shouldn't be necessary to say
\textit{anything.} To be a safe fulfiller of a wish, a genie must share
the same values that led you to make the wish. Otherwise the genie may
not choose a path through time that leads to the destination you had in
mind, or it may fail to exclude horrible side effects that would lead
you to not even consider a plan in the first place. Wishes are leaky
generalizations, derived from the huge but finite structure that is
your entire morality; only by including this entire structure can you
plug all the leaks.}

{
 With a safe genie, wishing is superfluous. Just run the genie.}

\myendsectiontext

\mysection{Anthropomorphic Optimism}

{
 The core fallacy of anthropomorphism is expecting something to be
predicted by the black box of your brain, when its causal structure is
so different from that of a human brain as to give you no license to
expect any such thing. }

{
 The early (pre-1966) biologists in The Tragedy of Group
Selectionism believed that predators would voluntarily restrain their
breeding to avoid overpopulating their habitat and exhausting the prey
population. Later on, when Michael J. Wade actually went out and
created in the laboratory the nigh-impossible conditions for group
selection, the adults adapted to cannibalize eggs and larvae,
especially female larvae.\footnote{Wade, ``Group selections among laboratory
populations of Tribolium.''\comment{1}}}

{
 Now, why might the group selectionists have \textit{not} thought
of that possibility?}

{
 Suppose you were a member of a tribe, and you knew that, in the
near future, your tribe would be subjected to a resource squeeze. You
might propose, as a solution, that no couple have more than one
child---after the first child, the couple goes on birth control.
Saying, ``Let's all individually have
as many children as we can, but then hunt down and cannibalize each
other's children, especially the
girls,'' would not even \textit{occur to you as a
possibility}.}

{
 Think of a preference ordering over solutions, relative to your
goals. You want a solution as high in this preference ordering as
possible. How do you find one? With a brain, of course! Think of your
brain as a \textit{high-ranking-solution-generator}{}---a search
process that produces solutions that rank high in your innate
preference ordering.}

{
 The solution space on all real-world problems is generally fairly
large, which is why you need an \textit{efficient} brain that
doesn't even \textit{bother to formulate} the vast
majority of low-ranking solutions.}

{
 If your tribe is faced with a resource squeeze, you could try
hopping everywhere on one leg, or chewing off your own toes. These
``solutions'' obviously
wouldn't work and would incur large costs, as you can
see upon examination---but in fact your brain is too efficient to waste
time considering such poor solutions; it doesn't
generate them in the first place. Your brain, in its search for
high-ranking solutions, flies directly to parts of the solution space
like ``Everyone in the tribe gets together, and agrees
to have no more than one child per couple until the resource squeeze is
past.''}

{
 Such a \textit{low-ranking} solution as
``Everyone have as many kids as possible, then
cannibalize the girls'' would not be
\textit{generated in your search process}.}

{
 But the ranking of an option as
``low'' or
``high'' is not an inherent property
of the option. It is a property of the optimization process that does
the preferring. And different optimization processes will search in
different orders.}

{
 So far as \textit{evolution} is concerned, individuals reproducing
to the fullest and then cannibalizing others' daughters
is a no-brainer; whereas individuals voluntarily restraining their own
breeding for the good of the group is absolutely ludicrous. Or to say
it less anthropomorphically, the first set of alleles would rapidly
replace the second in a population. (And natural selection has no
obvious search order here---these two alternatives seem around equally
simple as mutations.)}

{
 Suppose that one of the biologists had said, ``If
a predator population has only finite resources, evolution will craft
them to voluntarily restrain their breeding---that's
how \textit{I'd} do it if \textit{I} were in charge of
building predators.'' This would be anthropomorphism
outright, the lines of reasoning naked and exposed: \textit{I} would do
it this way, therefore I infer that \textit{evolution} will do it this
way.}

{
 One does occasionally encounter the fallacy outright, in my line
of work. But suppose you say to the one, ``An AI will
not necessarily work like you do.'' Suppose you say
to this hypothetical biologist, ``Evolution
doesn't work like you do.'' What will
the one say in response? I can tell you a reply you will \textit{not}
hear: ``Oh my! I didn't realize that!
One of the steps of my inference was invalid; I will throw away the
conclusion and start over from scratch.''}

{
 No: what you'll hear \textit{instead} is a reason
why any AI has to reason the same way as the speaker. Or a reason why
natural selection, following entirely different criteria of
optimization and using entirely different methods of optimization,
ought to do \textit{the same thing} that would occur to a human as a
good idea.}

{
 Hence the elaborate idea that group selection would favor predator
groups where the individuals voluntarily forsook reproductive
opportunities.}

{
 The group selectionists went just as far astray, in their
predictions, as someone committing the fallacy outright. Their final
conclusions were the same as if they were assuming outright that
evolution necessarily thought like themselves. But they erased what had
been written \textit{above} the bottom line of their argument,
\textit{without} erasing the actual bottom line, and wrote in new
rationalizations. Now the fallacious reasoning is disguised; the
\textit{obviously} flawed step in the inference has been hidden---even
though the conclusion remains exactly the same; and hence, in the real
world, exactly as wrong.}

{
 But why would any scientist do this? In the end, the data came out
against the group selectionists and they were embarrassed.}

{
 As I remarked in Fake Optimization Criteria, we humans seem to
have evolved an instinct for arguing that \textit{our} preferred policy
arises from practically \textit{any} criterion of optimization.
Politics was a feature of the ancestral environment; we are descended
from those who argued most persuasively that the
\textit{tribe's} interest---not just their own
interest---required that their hated rival Uglak be executed. We
certainly aren't descended from Uglak, who failed to
argue that his tribe's moral code{}---not just his own
obvious self-interest---required his survival.}

{
 And because we can more persuasively argue for what we honestly
believe, we have evolved an instinct to honestly believe that other
people's goals, and our tribe's moral
code, truly do imply that they should do things \textit{our} way for
\textit{their} benefit.}

{
 So the group selectionists, imagining this beautiful picture of
predators restraining their breeding, instinctively rationalized why
natural selection ought to do things \textit{their} way, even according
to natural selection's own purposes. The foxes will be
fitter if they restrain their breeding! No, really!
They'll even outbreed other foxes who
don't restrain their breeding! Honestly!}

{
 The problem with trying to argue natural selection into doing
things your way is that evolution does not contain that which could be
moved by your arguments. Evolution does not work like you do---not even
to the extent of having any element that could listen to or
\textit{care about} your painstaking explanation of why evolution ought
to do things your way. Human arguments are not even
\textit{commensurate} with the internal structure of natural selection
as an optimization process---human arguments aren't
used in promoting alleles, as human arguments would play a causal role
in human politics.}

{
 So instead of \textit{successfully} persuading natural selection
to do things their way, the group selectionists were simply embarrassed
when reality came out differently.}

{
 There's a fairly heavy subtext here about
Unfriendly AI.}

{
 But the point generalizes: this is the problem with optimistic
reasoning \textit{in general.} What is optimism? It is ranking the
possibilities by your own preference ordering, and selecting an outcome
high in that preference ordering, and somehow that outcome ends up as
your prediction. What kind of elaborate rationalizations were generated
along the way is probably not so relevant as one might fondly believe;
look at the cognitive history and it's optimism in,
optimism out. But Nature, or whatever other process is under
discussion, is not \textit{actually, causally} choosing between
outcomes by ranking them in your preference ordering and picking a high
one. So the brain fails to synchronize with the environment, and the
prediction fails to match reality.}

\myendsectiontext


\bigskip

\mysection{Lost Purposes}

{
 It was in either kindergarten or first grade that I was first
asked to pray, given a transliteration of a Hebrew prayer. I asked what
the words meant. I was told that so long as I prayed in Hebrew, I
didn't need to know what the words meant, it would work
anyway. }

{
 That was the beginning of my break with Judaism.}

{
 As you read this, some young man or woman is sitting at a desk in
a university, earnestly studying material they have no intention of
ever using, and no interest in knowing for its own sake. They want a
high-paying job, and the high-paying job requires a piece of paper, and
the piece of paper requires a previous master's degree,
and the master's degree requires a
bachelor's degree, and the university that grants the
bachelor's degree requires you to take a class in
twelfth-century knitting patterns to graduate. So they diligently
study, intending to forget it all the moment the final exam is
administered, but still seriously working away, because they
\textit{want} that piece of paper.}

{
 Maybe \textit{you} realized it was all madness, but I bet you did
it anyway. You didn't have a choice, right? A recent
study here in the Bay Area showed that 80\% of teachers in K-5 reported
spending less than one hour per week on science, and 16\% said they
spend no time on science. Why? I'm given to understand
the proximate cause is the No Child Left Behind Act and similar
legislation. Virtually all classroom time is now spent on preparing for
tests mandated at the state or federal level. I seem to recall (though
I can't find the source) that just \textit{taking}
mandatory tests was 40\% of classroom time in one school.}

{
 The old Soviet bureaucracy was famous for being more interested in
appearances than reality. One shoe factory overfulfilled its quota by
producing lots of tiny shoes. Another shoe factory reported cut but
unassembled leather as a ``shoe.''
The superior bureaucrats weren't interested in looking
too hard, because they also wanted to report quota overfulfillments.
All this was a great help to the comrades freezing their feet off.}

{
 It is now being suggested in several sources that an actual
\textit{majority} of published findings in medicine, though
``statistically significant with p {\textless}
0.05,'' are untrue. But so long as p {\textless} 0.05
remains the threshold for publication, why should anyone hold
themselves to higher standards, when that requires bigger research
grants for larger experimental groups, and decreases the likelihood of
getting a publication? Everyone knows that \textit{the whole point of
science} is to publish lots of papers, just as \textit{the whole point
of a university} is to print certain pieces of parchment, and
\textit{the whole point of a school} is to pass the mandatory tests
that guarantee the annual budget. You don't get to set
the rules of the game, and if you try to play by different rules,
you'll just lose.}

{
 (Though for some reason, physics journals require a threshold of p
{\textless} 0.0001. It's as if they conceive of some
other purpose to their existence than publishing physics papers.)}

{
 There's chocolate at the supermarket, and you can
get to the supermarket by driving, and driving requires that you be in
the car, which means opening your car door, which needs keys. If you
find there's no chocolate at the supermarket, you
won't stand around opening and slamming your car door
because the car door still needs opening. I rarely notice people losing
track of plans they devised themselves.}

{
 It's another matter when incentives must flow
through large organizations---or worse, many different organizations
and interest groups, some of them governmental. Then you see behaviors
that would mark literal insanity, if they were born from a single mind.
Someone gets paid every time they open a car door, because
that's what's measurable; and this
person doesn't care whether the driver ever gets paid
for arriving at the supermarket, let alone whether the buyer purchases
the chocolate, or whether the eater is happy or starving.}

{
 From a Bayesian perspective, subgoals are epiphenomena of
conditional probability functions. There is no expected utility without
utility. How silly would it be to think that instrumental value could
take on a \textit{mathematical} life of its own, leaving terminal value
in the dust? It's not sane by decision-theoretical
criteria of sanity.}

{
 But consider the No Child Left Behind Act. The politicians want to
look like they're doing something about educational
difficulties; the politicians have to look busy to voters \textit{this}
year, not fifteen years later when the kids are looking for jobs. The
politicians are not the consumers of education. The bureaucrats have to
show progress, which means that they're only interested
in progress that can be measured this year. They aren't
the ones who'll end up ignorant of science. The
publishers who commission textbooks, and the committees that purchase
textbooks, don't sit in the classrooms bored out of
their skulls.}

{
 The actual consumers of knowledge are the children---who
can't pay, can't vote,
can't sit on the committees. Their parents care for
them, but don't sit in the classes themselves; they can
only hold politicians responsible according to surface images of
``tough on education.'' Politicians
are too busy being re-elected to study all the data themselves; they
have to rely on surface images of bureaucrats being busy and
commissioning studies---it may not work to help any children, but it
works to let politicians appear caring. Bureaucrats
don't expect to use textbooks themselves, so they
don't care if the textbooks are hideous to read, so
long as the process by which they are purchased looks good on the
surface. The textbook publishers have no motive to produce \textit{bad}
textbooks, but they know that the textbook purchasing committee will be
comparing textbooks based on how many different subjects they cover,
and that the fourth-grade purchasing committee isn't
coordinated with the third-grade purchasing committee, so they cram as
many subjects into one textbook as possible. Teachers
won't get through a fourth of the textbook before the
end of the year, and then the next year's teacher will
start over. Teachers might complain, but they aren't
the decision-makers, and ultimately, it's not their
future on the line, which puts sharp bounds on how much effort
they'll spend on unpaid altruism \ldots}

{
 It's amazing, when you look at it that
way---consider all the lost information and lost incentives---that
anything at all remains of the original purpose, gaining knowledge.
Though many educational systems seem to be currently in the process of
collapsing into a state not much better than nothing.}

{
 Want to see the problem \textit{really} solved? Make the
politicians go to school.}

{
 A single human mind can track a probabilistic expectation of
utility as it flows through the conditional chances of a dozen
intermediate events---including nonlocal dependencies, places where the
expected utility of opening the car door depends on whether
there's chocolate in the supermarket. But organizations
can only reward today what is measurable today, what can be written
into legal contract today, and this means measuring intermediate events
rather than their distant consequences. These intermediate measures, in
turn, are leaky generalizations---often \textit{very} leaky.
Bureaucrats are untrustworthy genies, for they do not share the values
of the wisher.}

{
 Miyamoto Musashi said:\footnote{Miyamoto Musashi, \textit{Book of Five Rings} (New Line
Publishing, 2003).\comment{1}}}

\begin{quote}
{
 The primary thing when you take a sword in your hands is your
intention to cut the enemy, whatever the means. Whenever you parry,
hit, spring, strike or touch the enemy's cutting sword,
you must cut the enemy in the same movement. It is essential to attain
this. If you think only of hitting, springing, striking or touching the
enemy, you will not be able actually to cut him. More than anything,
you must be thinking of carrying your movement through to cutting him.
You must thoroughly research this.}
\end{quote}

{
 (I wish \textit{I} lived in an era where I could just tell my
readers they have to thoroughly research something, without giving
insult.)}

{
 Why would any individual lose track of their purposes in a
swordfight? If someone else had taught them to fight, if they had not
generated the \textit{entire} art from within themselves, they might
not understand the reason for parrying at one moment, or springing at
another moment; they might not realize when the rules had exceptions,
fail to see the times when the usual method won't cut
through.}

{
 The essential thing in the art of epistemic rationality is to
understand how every rule is cutting through to the truth in the same
movement. The corresponding essential of pragmatic
rationality---decision theory, versus probability theory---is to always
see how every expected utility cuts through to utility. You must
thoroughly research this.}

{
 C. J. Cherryh said:\footnote{Carolyn J. Cherryh, \textit{The Paladin} (Baen, 2002).\comment{2}}}

\begin{quote}
{
 Your sword has no blade. It has only your intention. When that
 goes astray you have no weapon.}
\end{quote}

{
 I have seen many people go astray when they wish to the genie of
an imagined AI, dreaming up wish after wish that seems good to them,
sometimes with many patches and sometimes without even that pretense of
caution. And they don't jump to the meta-level. They
don't instinctively look-to-purpose, the instinct that
started me down the track to atheism at the age of five. They do not
ask, as I reflexively ask, ``Why do \textit{I} think
this wish is a good idea? Will the genie judge
likewise?'' They don't \textit{see}
the source of their judgment, hovering behind the judgment as its
generator. They lose track of the ball; they know the ball bounced, but
they don't instinctively look back to see where it
bounced from---the criterion that generated their judgments.}

{
 Likewise with people not automatically noticing when supposedly
selfish people give altruistic arguments in favor of selfishness, or
when supposedly altruistic people give selfish arguments in favor of
altruism.}

{
 People can handle goal-tracking for driving to the supermarket
just fine, when it's \textit{all} inside their own
heads, and no genies or bureaucracies or philosophies are involved. The
trouble is that real civilization is immensely more complicated than
this. Dozens of organizations, and dozens of years, intervene between
the child suffering in the classroom, and the new-minted college
graduate not being very good at their job. (But will the interviewer or
manager notice, if the college graduate is good at looking busy?) With
every new link that intervenes between the action and its consequence,
intention has one more chance to go astray. With every intervening
link, information is lost, incentive is lost. And this bothers most
people a lot less than it bothers me, or why were all my classmates
willing to say prayers without knowing what they meant? They
didn't feel the same instinct to
look-to-the-generator.}

{
 Can people learn to keep their eye on the ball? To keep their
intention from going astray? To never spring or strike or touch,
without knowing the higher goal they will complete in the same
movement? People \textit{do} often want to do their jobs, all else
being equal. Can there be such a thing as a sane corporation? A sane
civilization, even? That's only a distant dream, but
it's what I've been getting at with all
of these essays on the flow of intentions (a.k.a. expected utility,
a.k.a. instrumental value) without losing purpose (a.k.a. utility,
a.k.a. terminal value). Can people learn to feel the flow of parent
goals and child goals? To know consciously, as well as implicitly, the
distinction between expected utility and utility?}

{
 Do you care about threats to your civilization? The worst
metathreat to complex civilization is its own complexity, for that
complication leads to the loss of many purposes.}

{
 I look back, and I see that more than anything, my life has been
driven by an exceptionally strong abhorrence to lost purposes. I hope
it can be transformed to a learnable skill.}

\myendsectiontext


\bigskip

\chapter{A Human's Guide to Words}

\mysection{The Parable of the Dagger}

{
 \textit{(Adapted from Raymond
Smullyan.}\textit{\footnote{Raymond M. Smullyan, \textit{What Is the Name of This Book?:
The Riddle of Dracula and Other Logical Puzzles} (Penguin Books,
1990).\comment{1}}}\textit{)} }

{
 Once upon a time, there was a court jester who dabbled in logic.}

{
 The jester presented the king with two boxes. Upon the first box
was inscribed:}

\begin{quote}
{
 Either this box contains an angry frog, or the box with a false
 inscription contains an angry frog, but not both.}
\end{quote}

{
 On the second box was inscribed:}

\begin{quote}
{
 Either this box contains gold and the box with a false inscription
contains an angry frog, or this box contains an angry frog and the box
with a true inscription contains gold.}
\end{quote}

{
 And the jester said to the king: ``One box
contains an angry frog, the other box gold; and one, and only one, of
the inscriptions is true.''}

{
 The king opened the wrong box, and was savaged by an angry frog.}

{
 ``You see,'' the jester said,
``let us hypothesize that the first inscription is the
true one. Then suppose the first box contains gold. Then the other box
would have an angry frog, while the box with a true inscription would
contain gold, which would make the second statement true as well. Now
hypothesize that the first inscription is false, and that the first box
contains gold. Then the second inscription would
be---''}

{
 The king ordered the jester thrown in the dungeons.}

{
 A day later, the jester was brought before the king in chains and
shown two boxes.}

{
 ``One box contains a key,''
said the king, ``to unlock your chains; and if you
find the key you are free. But the other box contains a dagger for your
heart if you fail.''}

{
 And the first box was inscribed:}

\begin{quote}
{
 Either both inscriptions are true, or both inscriptions are
 false.}
\end{quote}

{
 And the second box was inscribed:}

\begin{quote}
{
  This box contains the key.}
\end{quote}

{
 The jester reasoned thusly: ``Suppose the first
inscription is true. Then the second inscription must also be true. Now
suppose the first inscription is false. Then again the second
inscription must be true. So the second box must contain the key, if
the first inscription is true, and also if the first inscription is
false. Therefore, the second box must logically contain the
key.''}

{
 The jester opened the second box, and found a dagger.}

{
 ``How?!'' cried the jester in
horror, as he was dragged away. ``It's
logically impossible!''}

{
 ``It is entirely possible,''
replied the king. ``I merely wrote those inscriptions
on two boxes, and then I put the dagger in the second
one.''}

\myendsectiontext


\bigskip

\mysection{The Parable of Hemlock}

\begin{quote}
{
 All men are mortal. Socrates is a man. Therefore Socrates is
mortal.}

{\raggedleft
 {}---Standard Medieval syllogism
\par}
\end{quote}


{
 \textit{Socrates raised the glass of hemlock to his lips \ldots}}

{
 ``Do you suppose,'' asked one
of the onlookers, ``that even hemlock will not be
enough to kill so wise and good a man?''}

{
 ``No,'' replied another
bystander, a student of philosophy; ``all men are
mortal, and Socrates is a man; and if a mortal drinks hemlock, surely
he dies.''}

{
 ``Well,'' said the onlooker,
``what if it happens that Socrates
\textit{isn't} mortal?''}

{
 ``Nonsense,'' replied the
student, a little sharply; ``all men are mortal
\textit{by definition}; it is part of what we mean by the word
`man.' All men are mortal, Socrates is a
man, therefore Socrates is mortal. It is not merely a guess, but a
\textit{logical certainty.}''}

{
 ``I suppose that's right
\ldots'' said the onlooker. ``Oh,
look, Socrates already drank the hemlock while we were
talking.''}

{
 ``Yes, he should be keeling over any minute
now,'' said the student.}

{
 \textit{And they waited, and they waited, and they waited \ldots}}

{
 ``Socrates appears not to be
mortal,'' said the onlooker.}

{
 ``Then Socrates must not be a
man,'' replied the student. ``All
men are mortal, Socrates is not mortal, therefore Socrates is not a
man. And that is not merely a guess, but a \textit{logical
certainty.}''}

{
 The fundamental problem with arguing that things are true
``by definition'' is that you
can't make reality go a different way by choosing a
different definition.}

{
 You could reason, perhaps, as follows: ``All
things I have observed which wear clothing, speak language, and use
tools, have also shared certain other properties as well, such as
breathing air and pumping red blood. The last thirty
`humans' belonging to this cluster whom
I observed to drink hemlock soon fell over and stopped moving. Socrates
wears a toga, speaks fluent ancient Greek, and drank hemlock from a
cup. So I predict that Socrates will keel over in the next five
minutes.''}

{
 But that would be mere \textit{guessing.} It
wouldn't be, y'know, absolutely and
eternally certain. The Greek philosophers---like most prescientific
philosophers---were rather fond of certainty.}

{
 Luckily the Greek philosophers have a crushing rejoinder to your
questioning. You have misunderstood the meaning of
``All humans are mortal,'' they say.
It is not a mere \textit{observation.} It is part of the
\textit{definition} of the word
``human.'' Mortality is one of
several properties that are individually necessary, and together
sufficient, to determine membership in the class
``human.'' The statement
``All humans are mortal'' is a
logically valid truth, absolutely unquestionable. And if Socrates is
human, he \textit{must} be mortal: it is a logical deduction, as
certain as certain can be.}

{
 But then we can never know for certain that Socrates is a
``human'' until after Socrates has
been observed to be mortal. It does no good to observe that Socrates
speaks fluent Greek, or that Socrates has red blood, or even that
Socrates has human DNA. None of these characteristics are
\textit{logically equivalent} to mortality. You have to \textit{see him
die} before you can conclude that he was human.}

{
 (And even then it's not infinitely certain. What
if Socrates rises from the grave a night after you see him die? Or more
realistically, what if Socrates is signed up for cryonics? If mortality
is defined to mean finite lifespan, then you can never really
\textit{know} if someone was human, until you've
observed to the end of eternity---just to make sure they
don't come back. Or you could \textit{think} you saw
Socrates keel over, but it could be an illusion projected onto your
eyes with a retinal scanner. Or maybe you just hallucinated the whole
thing \ldots)}

{
 The problem with syllogisms is that they're
\textit{always} valid. ``All humans are mortal;
Socrates is human; therefore Socrates is mortal''
is---if you treat it as a logical syllogism---logically valid within
our own universe. It's also logically valid within
neighboring Everett branches in which, due to a slightly different
evolved biochemistry, hemlock is a delicious treat rather than a
poison. And it's logically valid even in universes
where Socrates never existed, or for that matter, where humans never
existed.}

{
 The Bayesian definition of evidence favoring a hypothesis is
evidence which we are more likely to see if the hypothesis is true than
if it is false. Observing that a syllogism is logically valid can never
be evidence favoring any empirical proposition, because the syllogism
will be logically valid whether that proposition is true or false.}

{
 Syllogisms are valid in all possible worlds, and therefore,
observing their validity never tells us anything about \textit{which}
possible world we actually live in.}

{
 This doesn't mean that logic is useless---just
that logic can only tell us that which, \textit{in some sense}, we
already know. But we do not always believe what we know. Is the number
29,384,209 prime? By virtue of how I define my decimal system and my
axioms of arithmetic, I have already determined my answer to this
question---but I do not know what my answer is yet, and I must do some
logic to find out.}

{
 Similarly, if I form the uncertain empirical generalization
``Humans are vulnerable to
hemlock,'' and the uncertain empirical guess
``Socrates is human,'' logic can
tell me that my previous guesses are predicting that Socrates will be
vulnerable to hemlock.}

{
 It's been suggested that we can view logical
reasoning as resolving our uncertainty about impossible possible
worlds---eliminating probability mass in logically impossible worlds
which we did not know to be logically impossible. In this sense,
logical argument can be treated as observation.}

{
 But when you talk about an empirical prediction like
``Socrates is going to keel over and stop
breathing'' or ``Socrates is going
to do fifty jumping jacks and then compete in the Olympics next
year,'' that is a matter of possible worlds, not
impossible possible worlds.}

{
 Logic can tell us which hypotheses match up to which observations,
and it can tell us what these hypotheses predict for the future---it
can bring old observations and previous guesses to bear on a new
problem. But logic never flatly says, ``Socrates
\textit{will} stop breathing now.'' Logic never
dictates any empirical question; it never settles any real-world query
which could, by any stretch of the imagination, go either way.}

{
 Just remember the Litany Against Logic:}

\begin{quote}
{
 Logic stays true, wherever you may go,}

{
  So logic never tells you where you live.}
\end{quote}

\myendsectiontext

\mysection{Words as Hidden Inferences}

{
 Suppose I find a barrel, sealed at the top, but with a hole large
enough for a hand. I reach in and feel a small, curved object. I pull
the object out, and it's blue---a bluish egg. Next I
reach in and feel something hard and flat, with edges---which, when I
extract it, proves to be a red cube. I pull out 11 eggs and 8 cubes,
and every egg is blue, and every cube is red. }

{
 Now I reach in and I feel another egg-shaped object. Before I pull
it out and look, I have to guess: What will it look like?}

{
 The evidence doesn't prove that every egg in the
barrel is blue and every cube is red. The evidence
doesn't even argue this all that strongly: 19 is not a
large sample size. Nonetheless, I'll guess that this
egg-shaped object is blue---or as a runner-up guess, red. If I guess
anything else, there's as many possibilities as
distinguishable colors---and for that matter, who says the egg has to
be a single shade? Maybe it has a picture of a horse painted on.}

{
 So I say ``blue,'' with a
dutiful patina of humility. For I am a sophisticated rationalist-type
person, and I keep track of my assumptions and dependencies---I guess,
but I'm aware that I'm guessing \ldots
right?}

{
 But when a large yellow striped feline-shaped object leaps out at
me from the shadows, I think, ``Yikes! A
tiger!'' Not, ``Hm \ldots objects
with the properties of largeness, yellowness, stripedness, and feline
shape, have previously often possessed the properties
`hungry' and
`dangerous,' and thus, although it is
not logically necessary, it may be an empirically good guess that
\textit{aaauuughhhh} CRUNCH CRUNCH GULP.''}

{
 The human brain, for some odd reason, seems to have been adapted
to make this inference quickly, automatically, and without keeping
explicit track of its assumptions.}

{
 And if I name the egg-shaped objects
``bleggs'' (for blue eggs) and the
red cubes ``rubes,'' then, when I
reach in and feel another egg-shaped object, I may think, \textit{Oh,
it's a blegg,} rather than considering all that
problem-of-induction stuff.}

{
 It is a common misconception that you can define a word any way
you like.}

{
 This would be true \textit{if} the brain treated words as purely
logical constructs, Aristotelian classes, and you never took out any
more information than you put in.}

{
 Yet the brain goes on about its work of categorization, whether or
not we consciously approve. ``All humans are mortal;
Socrates is a human; therefore Socrates is
mortal''---thus spake the ancient Greek philosophers.
Well, if mortality is part of your logical definition of
``human,'' you can't
logically classify Socrates as human until you observe him to be
mortal. But---this is the problem---Aristotle knew perfectly well that
Socrates was a human. Aristotle's brain placed Socrates
in the ``human'' category as
efficiently as your own brain categorizes tigers, apples, and
everything else in its environment: Swiftly, silently, and without
conscious approval.}

{
 Aristotle laid down rules under which no one could conclude
Socrates was ``human'' until after
he died. Nonetheless, Aristotle and his students went on concluding
that living people were humans and therefore mortal; they saw
distinguishing properties such as human faces and human bodies, and
their brains made the leap to inferred properties such as mortality.}

{
 Misunderstanding the working of your own mind does \textit{not},
thankfully, prevent the mind from doing its work. Otherwise
Aristotelians would have starved, unable to conclude that an object was
edible merely because it looked and felt like a banana.}

{
 So the Aristotelians went on classifying environmental objects on
the basis of partial information, the way people had always done.
Students of Aristotelian logic went on thinking exactly the same way,
but they had acquired an erroneous picture of \textit{what} they were
doing.}

{
 If you asked an Aristotelian philosopher whether Carol the grocer
was mortal, they would say ``Yes.''
If you asked them how they knew, they would say ``All
humans are mortal; Carol is human; therefore Carol is
mortal.'' Ask them whether it was a guess or a
certainty, and they would say it was a certainty (if you asked before
the sixteenth century, at least). Ask them how they knew that humans
were mortal, and they would say it was established by definition.}

{
 The Aristotelians were still the same people, they retained their
original natures, but they had acquired incorrect beliefs about their
own functioning. They looked into the mirror of self-awareness, and saw
something unlike their true selves: they reflected incorrectly.}

{
 Your brain doesn't treat words as logical
definitions with no empirical consequences, and so neither should you.
The mere act of creating a word can cause your mind to allocate a
category, and thereby trigger unconscious inferences of similarity. Or
block inferences of similarity; if I create two labels I can get your
mind to allocate two categories. Notice how I said
``you'' and ``your
brain'' as if they were different things?}

{
 Making errors about the inside of your head
doesn't change what's there; otherwise
Aristotle would have died when he concluded that the brain was an organ
for cooling the blood. Philosophical mistakes usually
don't interfere with blink-of-an-eye perceptual
inferences.}

{
 But philosophical mistakes can severely mess up the deliberate
thinking processes that we use to try to correct our first impressions.
If you believe that you can ``define a word any way
you like,'' without realizing that your brain goes on
categorizing without your conscious oversight, then you
won't make the effort to choose your definitions
wisely.}

\myendsectiontext

\mysection{Extensions and Intensions}

\begin{quote}
{
 ``What is red?''}

{
 ``Red is a color.''}

{
 ``What's a
color?''}

{
 ``A color is a property of a
  thing.''}
\end{quote}

{
 But what is a thing? And what's a property? Soon
the two are lost in a maze of words defined in other words, the problem
that Steven Harnad once described as trying to learn Chinese from a
Chinese/Chinese dictionary.}

{
 Alternatively, if you asked me ``What is
red?'' I could point to a stop sign, then to someone
wearing a red shirt, and a traffic light that happens to be red, and
blood from where I accidentally cut myself, and a red business card,
and then I could call up a color wheel on my computer and move the
cursor to the red area. This would probably be sufficient, though if
you know what the word ``No'' means,
the truly strict would insist that I point to the sky and say
``No.''}

{
 I think I stole this example from S. I. Hayakawa---though
I'm really not sure, because I heard this way back in
the indistinct blur of my childhood. (When I was twelve, my father
accidentally deleted all my computer files. I have no memory of
anything before that.)}

{
 But that's how I remember first learning about the
difference between intensional and extensional definition. To give an
``intensional definition'' is to
define a word or phrase in terms of other words, as a dictionary does.
To give an ``extensional
definition'' is to point to examples, as adults do
when teaching children. The preceding sentence gives an intensional
definition of ``extensional
definition,'' which makes it an extensional example
of ``intensional definition.''}

{
 In Hollywood Rationality and popular culture generally,
``rationalists'' are depicted as
word-obsessed, floating in endless verbal space disconnected from
reality.}

{
 But the actual Traditional Rationalists have long insisted on
maintaining a tight connection to experience:}

\begin{quote}
{
 If you look into a textbook of chemistry for a definition of
lithium, you may be told that it is that element whose atomic weight is
7 very nearly. But if the author has a more logical mind he will tell
you that if you search among minerals that are vitreous, translucent,
grey or white, very hard, brittle, and insoluble, for one which imparts
a crimson tinge to an unluminous flame, this mineral being triturated
with lime or witherite rats-bane, and then fused, can be partly
dissolved in muriatic acid; and if this solution be evaporated, and the
residue be extracted with sulphuric acid, and duly purified, it can be
converted by ordinary methods into a chloride, which being obtained in
the solid state, fused, and electrolyzed with half a dozen powerful
cells, will yield a globule of a pinkish silvery metal that will float
on gasolene; and the material of that is a specimen of lithium.}

{\raggedleft
 {}---Charles Sanders Peirce\footnote{Charles Sanders Peirce, \textit{Collected Papers} (Harvard
University Press, 1931).\comment{1}}
\par}
\end{quote}


{
 That's an example of ``logical
mind'' as described by a genuine Traditional
Rationalist, rather than a Hollywood scriptwriter.}

{
 But note: Peirce isn't \textit{actually} showing
you a piece of lithium. He didn't have pieces of
lithium stapled to his book. Rather he's giving you a
treasure map---an intensionally defined procedure which, when executed,
will lead you to an extensional example of lithium. This is not the
same as just tossing you a hunk of lithium, but it's
not the same as saying ``atomic weight
7'' either. (Though if you had \textit{sufficiently
sharp} eyes, saying ``3 protons''
might let you pick out lithium at a glance \ldots)}

{
 So that is intensional and extensional \textit{definition}, which
is a way of telling someone else what you mean by a concept. When I
talked about ``definitions'' above,
I talked about a way of \textit{communicating}
concepts---\textit{telling someone else} what you mean by
``red,''
``tiger,''
``human,'' or
``lithium.'' Now
let's talk about the actual concepts themselves.}

{
 The actual intension of my
``tiger'' concept would be the
neural pattern (in my temporal cortex) that inspects an incoming signal
from the visual cortex to determine whether or not it is a tiger.}

{
 The actual extension of my
``tiger'' concept is everything I
call a tiger.}

{
 Intensional definitions don't capture entire
intensions; extensional definitions don't capture
entire extensions. If I point to just one tiger and say the word
``tiger,'' the communication may
fail if they think I mean ``dangerous
animal'' or ``male
tiger'' or ``yellow
thing.'' Similarly, if I say
``dangerous yellow-black striped
animal,'' without pointing to anything, the listener
may visualize giant hornets.}

{
 You can't capture in words all the details of the
cognitive concept---as it exists in your mind---that lets you recognize
things as tigers or nontigers. It's too large. And you
can't point to all the tigers you've
ever seen, let alone everything you \textit{would} call a tiger.}

{
 The strongest definitions use a crossfire of intensional and
extensional communication to nail down a concept. Even so, you only
communicate \textit{maps to} concepts, or instructions for building
concepts---you don't communicate the \textit{actual}
categories as they exist in your mind or in the world.}

{
 (Yes, with enough creativity you can construct exceptions to this
rule, like ``Sentences Eliezer Yudkowsky has published
containing the term `huragaloni' as of
Feb 4, 2008.'' I've just shown you
this concept's entire extension. But except in
mathematics, definitions are usually treasure maps, not treasure.)}

{
 So that's another reason you can't
``define a word any way you like'':
You can't directly program concepts into someone
else's brain.}

{
 Even within the Aristotelian paradigm, where we pretend that the
definitions are the actual concepts, you don't have
\textit{simultaneous} freedom of intension and extension. Suppose I
define Mars as ``A huge red rocky sphere, around a
tenth of Earth's mass and 50\% further away from the
Sun.'' It's then a separate matter to
show that this intensional definition matches some particular
extensional thing in my experience, or indeed, that it matches any real
thing whatsoever. If instead I say
``That's Mars'' and
point to a red light in the night sky, it becomes a separate matter to
show that this extensional light matches any particular intensional
definition I may propose---or any intensional beliefs I may have---such
as ``Mars is the God of War.''}

{
 But most of the brain's work of applying
intensions happens sub-deliberately. We aren't
consciously aware that our identification of a red light as
``Mars'' is a separate matter from
our verbal definition ``Mars is the God of
War.'' No matter what kind of intensional definition
I make up to describe Mars, my mind believes that
``Mars'' refers to this thingy, and
that it is the fourth planet in the Solar System.}

{
 When you take into account the way the human mind actually,
pragmatically works, the notion ``I can define a word
any way I like'' soon becomes ``I
can believe anything I want about a fixed set of
objects'' or ``I can move any object
I want in or out of a fixed membership test.'' Just
as you can't usually convey a concept's
whole intension in words because it's a big complicated
neural membership test, you can't \textit{control} the
concept's entire intension because it's
applied sub-deliberately. This is why arguing that XYZ is true
``by definition'' is so popular. If
definition changes behaved like the empirical null-ops
they're supposed to be, no one would bother arguing
them. But abuse definitions just a little, and they turn into magic
wands---in arguments, of course; not in reality.}

\myendsectiontext


\bigskip

\mysection{Similarity Clusters}

{
 Once upon a time, the philosophers of Plato's
Academy claimed that the best definition of human was a
``featherless biped.'' Diogenes of
Sinope, also called Diogenes the Cynic, is said to have promptly
exhibited a plucked chicken and declared ``Here is
Plato's man.'' The Platonists
promptly changed their definition to ``a featherless
biped with broad nails.'' }

{
 No dictionary, no encyclopedia, has ever listed all the things
that humans have in common. We have red blood, five fingers on each of
two hands, bony skulls, 23 pairs of chromosomes---but the same might be
said of other animal species. We make complex tools to make complex
tools, we use syntactical combinatorial language, we harness critical
fission reactions as a source of energy: these things may serve out to
single out only humans, but not all humans---many of us have never
built a fission reactor. With the right set of necessary-and-sufficient
gene sequences you could single out all humans, and only humans---at
least for now---but it would still be far from \textit{all} that humans
have in common.}

{
 But so long as you don't happen to be near a
plucked chicken, saying ``Look for featherless
bipeds'' may serve to pick out a few dozen of the
particular things that are humans, as opposed to houses, vases,
sandwiches, cats, colors, or mathematical theorems.}

{
 Once the definition ``featherless
biped'' has been bound to some \textit{particular}
featherless bipeds, you can look over the group, and begin harvesting
some of the \textit{other} characteristics---beyond mere featherfree
twolegginess---that the ``featherless
bipeds'' seem to share in common. The particular
featherless bipeds that you see seem to also use language, build
complex tools, speak combinatorial language with syntax, bleed red
blood if poked, die when they drink hemlock.}

{
 Thus the category ``human''
grows richer, and adds more and more characteristics; and when Diogenes
finally presents his plucked chicken, we are not fooled: This plucked
chicken is obviously not similar to the other
``featherless bipeds.''}

{
 (If Aristotelian logic were a good model of human psychology, the
Platonists would have looked at the plucked chicken and said,
``Yes, that's a human;
what's your point?'')}

{
 If the \textit{first} featherless biped you see is a plucked
chicken, then you may end up thinking that the verbal label
``human'' denotes a plucked chicken;
so I can modify my treasure map to point to
``featherless bipeds with broad
nails,'' and if I am wise, go on to say,
``See Diogenes over there? That's a
human, and I'm a human, and you're a
human; and that chimpanzee is not a human, though fairly
close.''}

{
 The initial clue only has to lead the user to the similarity
cluster---the group of things that have many characteristics in common.
After that, the initial clue has served its purpose, and I can go on to
convey the new information ``humans are currently
mortal,'' or whatever else I want to say about us
featherless bipeds.}

{
 A dictionary is best thought of, not as a book of Aristotelian
class definitions, but a book of hints for matching verbal labels to
similarity clusters, or matching labels to properties that are useful
in distinguishing similarity clusters.}

\myendsectiontext

\mysection{Typicality and Asymmetrical Similarity}

{
 Birds fly. Well, except ostriches don't. But which
is a more typical bird---a robin, or an ostrich? }

{
 Which is a more typical chair: a desk chair, a rocking chair, or a
beanbag chair?}

{
 Most people would say that a robin is a more typical bird, and a
desk chair is a more typical chair. The cognitive psychologists who
study this sort of thing experimentally, do so under the heading of
``typicality effects'' or
``prototype
effects.''\footnote{Eleanor Rosch, ``Principles of
Categorization,'' in \textit{Cognition and
Categorization}, ed. Eleanor Rosch and Barbara B. Lloyd (Hillsdale, NJ:
Lawrence Erlbaum, 1978).\comment{1}} For example, if you ask
subjects to press a button to indicate
``true'' or
``false'' in response to statements
like ``A robin is a bird'' or
``A penguin is a bird,'' reaction
times are faster for more central examples.\footnote{George Lakoff, \textit{Women, Fire, and Dangerous Things: What
Categories Reveal about the Mind} (Chicago: Chicago University Press,
1987).\comment{2}}
Typicality measures correlate well using different investigative
methods---reaction times are one example; you can also ask people to
directly rate, on a scale of 1 to 10, how well an example (like a
specific robin) fits a category (like
``bird'').}

{
 So we have a mental measure of typicality---which might, perhaps,
function as a heuristic---but is there a corresponding bias we can use
to pin it down?}

{
 Well, which of these statements strikes you as more natural:
``98 is approximately 100,'' or
``100 is approximately 98''? If
you're like most people, the first statement seems to
make more sense.\footnote{Jerrold Sadock, ``Truth and
Approximations,'' \textit{Papers from the Third
Annual Meeting of the Berkeley Linguistics Society} (1977): 430--439.\comment{3}} For similar reasons, people asked
to rate how similar Mexico is to the United States, gave consistently
higher ratings than people asked to rate how similar the United States
is to Mexico.\footnote{Amos Tversky and Itamar Gati, ``Studies of
Similarity,'' in \textit{Cognition and
Categorization}, ed. Eleanor Rosch and Barbara Lloyd (Hillsdale, NJ:
Lawrence Erlbaum Associates, Inc., 1978), 79--98.\comment{4}}}

{
 And if that still seems harmless, a study by Rips showed that
people were more likely to expect a disease would spread from robins to
ducks on an island, than from ducks to robins.\footnote{Lance J. Rips, ``Inductive Judgments about
Natural Categories,'' \textit{Journal of Verbal
Learning and Verbal Behavior} 14 (1975): 665--681.\comment{5}} Now
this is not a \textit{logical} impossibility, but in a pragmatic sense,
whatever difference separates a duck from a robin and would make a
disease less likely to spread from a duck to a robin, must also be a
difference between a robin and a duck, and would make a disease less
likely to spread from a robin to a duck.}

{
 Yes, you can come up with rationalizations, like
``Well, there could be more neighboring species of the
robins, which would make the disease more likely to spread initially,
etc.,'' but be careful not to try too hard to
rationalize the probability ratings of subjects who
didn't even realize there was a comparison going on.
And don't forget that Mexico is more similar to the
United States than the United States is to Mexico, and that 98 is
closer to 100 than 100 is to 98. A simpler interpretation is that
people are using the (demonstrated) similarity heuristic as a proxy for
the probability that a disease spreads, and this heuristic is
(demonstrably) asymmetrical.}

{
 Kansas is unusually close to the center of the United States, and
Alaska is unusually far from the center of the United States; so Kansas
is probably closer to most places in the US and Alaska is probably
farther. It does not follow, however, that Kansas is closer to Alaska
than is Alaska to Kansas. But people seem to reason (metaphorically
speaking) as if closeness is an inherent property of Kansas and
distance is an inherent property of Alaska; so that Kansas is still
close, even to Alaska; and Alaska is still distant, even from Kansas.}

{
 So once again we see that Aristotle's notion of
categories---logical classes with membership determined by a collection
of properties that are individually strictly necessary, and together
strictly sufficient---is not a good model of human cognitive
psychology. (Science's view has changed somewhat over
the last 2350 years? Who would've thought?) We
don't even reason as if set membership is a
true-or-false property: statements of set membership can be more or
less true. (Note: This is \textit{not} the same thing as being more or
less probable.)}

{
 One more reason not to pretend that you, or anyone else, is
\textit{really} going to treat words as Aristotelian logical classes.}

\myendsectiontext


\bigskip

\mysection{The Cluster Structure of Thingspace}

{
 The notion of a ``configuration
space'' is a way of translating object
\textit{descriptions} into object \textit{positions.} It may seem like
blue is ``closer'' to blue-green
than to red, but how much closer? It's hard to answer
that question by just staring at the colors. But it helps to know that
the (proportional) color coordinates in RGB are 0:0:5, 0:3:2, and
5:0:0. It would be even clearer if plotted on a 3D graph.}

{
 In the same way, you can see a robin as a robin---brown tail, red
breast, standard robin shape, maximum flying speed when unladen, its
species-typical DNA and individual alleles. Or you could see a robin as
a single point in a configuration space whose dimensions described
everything we knew, or could know, about the robin.}

{
 A robin is bigger than a virus, and smaller than an aircraft
carrier---that might be the
``volume'' dimension. Likewise a
robin weighs more than a hydrogen atom, and less than a galaxy; that
might be the ``mass'' dimension.
Different robins will have strong correlations between
``volume'' and
``mass,'' so the robin-points will
be lined up in a fairly linear string, in those two dimensions---but
the correlation won't be exact, so we do need two
separate dimensions.}

{
 This is the benefit of viewing robins as points in space: You
couldn't see the linear lineup as easily if you were
just imagining the robins as cute little wing-flapping creatures.}

{
 A robin's DNA is a highly multidimensional
variable, but you can still think of it as part of a
robin's location in thingspace---millions of quaternary
coordinates, one coordinate for each DNA base---or maybe a more
sophisticated view than that. The shape of the robin, and its color
(surface reflectance), you can likewise think of as part of the
robin's position in thingspace, even though they
aren't \textit{single} dimensions.}

{
 Just like the coordinate point 0:0:5 contains the same information
as the actual HTML color blue, we shouldn't actually
lose information when we see robins as points in space. We believe the
same statement about the robin's mass whether we
visualize a robin balancing the scales opposite a 0.07-kilogram weight,
or a robin-point with a mass-coordinate of +70.}

{
 We can even imagine a configuration space with one or more
dimensions for every distinct characteristic of an object, so that the
\textit{position} of an object's point in this space
corresponds to \textit{all} the information in the real object itself.
Rather redundantly represented, too---dimensions would include the
mass, the volume, and the density.}

{
 If you think that's extravagant, quantum
physicists use an \textit{infinite-dimensional} configuration space,
and a single point in that space describes the location of every
particle in the universe. So we're actually being
comparatively conservative in our visualization of
\textit{thingspace}{}---a point in thingspace describes just one
object, not the entire universe.}

{
 If we're not sure of the robin's
exact mass and volume, then we can think of a little cloud in
thingspace, a \textit{volume of uncertainty}, within which the robin
might be. The density of the cloud is the density of our belief that
the robin has that particular mass and volume. If
you're more sure of the robin's density
than of its mass and volume, your probability-cloud will be highly
concentrated in the density dimension, and concentrated around a
slanting line in the subspace of mass/volume. (Indeed, the cloud here
is actually a surface, because of the relation V D = M.)}

{
 ``Radial categories'' are how
cognitive psychologists describe the non-Aristotelian boundaries of
words. The central ``mother''
conceives her child, gives birth to it, and supports it. Is an egg
donor who never sees her child a mother? She is the
``genetic mother.'' What about a
woman who is implanted with a foreign embryo and bears it to term? She
is a ``surrogate mother.'' And the
woman who raises a child that isn't hers genetically?
Why, she's an ``adoptive
mother.'' The Aristotelian syllogism would run,
``Humans have ten fingers, Fred has nine fingers,
therefore Fred is not a human,'' but the way we
actually think is ``Humans have ten fingers, Fred is a
human, therefore Fred is a `nine-fingered
human.'''}

{
 We can think about the radial-ness of categories in intensional
terms, as described above---properties that are usually present, but
optionally absent. If we thought about the intension of the word
``mother,'' it might be like a
distributed glow in thingspace, a glow whose intensity matches the
degree to which that volume of thingspace matches the category
``mother.'' The glow is concentrated
in the center of genetics and birth and child-raising; the volume of
egg donors would also glow, but less brightly.}

{
 Or we can think about the radial-ness of categories extensionally.
Suppose we mapped all the birds in the world into thingspace, using a
distance metric that corresponds as well as possible to perceived
similarity in humans: A robin is more similar to another robin, than
either is similar to a pigeon, but robins and pigeons are all more
similar to each other than either is to a penguin, et cetera.}

{
 Then the center of all birdness would be densely populated by many
neighboring tight clusters, robins and sparrows and canaries and
pigeons and many other species. Eagles and falcons and other large
predatory birds would occupy a nearby cluster. Penguins would be in a
more distant cluster, and likewise chickens and ostriches.}

{
 The result might look, indeed, something like an astronomical
cluster: many galaxies orbiting the center, and a few outliers.}

{
 Or we could think simultaneously about both the intension of the
cognitive category ``bird,'' and its
extension in real-world birds: The central clusters of robins and
sparrows glowing brightly with highly typical birdness; satellite
clusters of ostriches and penguins glowing more dimly with atypical
birdness, and Abraham Lincoln a few megaparsecs away and glowing not at
all.}

{
 I prefer that last visualization---the glowing points---because as
I see it, the structure of the cognitive intension followed from the
extensional cluster structure. First came the structure-in-the-world,
the empirical distribution of birds over thingspace; then, by observing
it, we formed a category whose intensional glow roughly overlays this
structure.}

{
 This gives us yet another view of why words are not Aristotelian
classes: the empirical clustered structure of the real universe is not
so crystalline. A natural cluster, a group of things highly similar to
each other, may have \textit{no} set of necessary and sufficient
properties---\textit{no} set of characteristics that all group members
have, and no non-members have.}

{
 But even if a category is irrecoverably blurry and bumpy,
there's no need to panic. I would not object if someone
said that birds are ``feathered flying
things.'' \textit{But penguins don't
fly!}{}---well, fine. The usual rule has an exception;
it's not the end of the world. Definitions
can't be expected to exactly match the empirical
structure of thingspace in any event, because the map is smaller and
much less complicated than the territory. The point of the definition
``feathered flying things'' is to
lead the listener to the bird cluster, not to give a total description
of every existing bird down to the molecular level.}

{
 When you draw a boundary around a group of extensional points
\textit{empirically} clustered in thingspace, you may find at least one
exception to every simple intensional rule you can invent.}

{
 But if a definition works well enough in practice to point out the
intended empirical cluster, objecting to it may justly be called
``nitpicking.''}

\myendsectiontext

\mysection{Disguised Queries}

{
 Imagine that you have a peculiar job in a peculiar factory: Your
task is to take objects from a mysterious conveyor belt, and sort the
objects into two bins. When you first arrive, Susan the Senior Sorter
explains to you that blue egg-shaped objects are called
``bleggs'' and go in the
``blegg bin,'' while red cubes are
called ``rubes'' and go in the
``rube bin.'' }

{
 Once you start working, you notice that bleggs and rubes differ in
ways besides color and shape. Bleggs have fur on their surface, while
rubes are smooth. Bleggs flex slightly to the touch; rubes are hard.
Bleggs are opaque, the rube's surface slightly
translucent.}

{
 Soon after you begin working, you encounter a blegg shaded an
unusually dark blue---in fact, on closer examination, the color proves
to be purple, halfway between red and blue.}

{
 Yet wait! Why are you calling this object a
``blegg''? A
``blegg'' was originally defined as
blue and egg-shaped---the qualification of blueness appears in the very
name ``blegg,'' in fact. This object
is not blue. One of the necessary qualifications is missing; you should
call this a ``purple egg-shaped
object,'' not a
``blegg.''}

{
 But it so happens that, in addition to being purple and
egg-shaped, the object is also furred, flexible, and opaque. So when
you saw the object, you thought, ``Oh, a strangely
colored blegg.'' It certainly isn't a
rube \ldots right?}

{
 Still, you aren't quite sure what to do next. So
you call over Susan the Senior Sorter.}

\begin{quotation}
{
 ``Oh, yes, it's a
blegg,'' Susan says, ``you can put
it in the blegg bin.''}

{
 You start to toss the purple blegg into the blegg bin, but pause
for a moment. ``Susan,'' you say,
``how do you \textit{know} this is a
blegg?''}

{
 Susan looks at you oddly. ``Isn't
it obvious? This object may be purple, but it's still
egg-shaped, furred, flexible, and opaque, like all the other bleggs.
You've got to expect a few color defects. Or is this
one of those philosophical conundrums, like `How do you
know the world wasn't created five minutes ago complete
with false memories?' In a philosophical sense
I'm not \textit{absolutely certain} that this is a
blegg, but it seems like a good guess.''}

{
 ``No, I mean \ldots'' You pause,
searching for words. ``\textit{Why} is there a blegg
bin and a rube bin? What's the \textit{difference}
between bleggs and rubes?''}

{
 ``Bleggs are blue and egg-shaped, rubes are red
and cube-shaped,'' Susan says patiently.
``You got the standard orientation lecture,
right?''}

{
 ``Why do bleggs and rubes \textit{need} to be
sorted?''}

{
 ``Er \ldots because otherwise
they'd be all mixed up?'' says Susan.
``Because nobody will pay us to sit around all day and
\textit{not} sort bleggs and rubes?''}

{
 ``Who originally determined that the first blue
egg-shaped object was a `blegg,' and how
did they determine that?''}

{
 Susan shrugs. ``I suppose you could just as
easily call the red cube-shaped objects
`bleggs' and the blue egg-shaped objects
`rubes,' but it seems easier to remember
this way.''}

{
 You think for a moment. ``Suppose a completely
mixed-up object came off the conveyor. Like, an orange sphere-shaped
furred translucent object with writhing green tentacles. How could I
tell whether it was a blegg or a rube?''}

{
 ``Wow, no one's ever found an
object \textit{that} mixed up,'' says Susan,
``but I guess we'd take it to the
sorting scanner.''}

{
 ``How does the sorting scanner
work?'' you inquire. ``X-rays?
Magnetic resonance imaging? Fast neutron transmission
spectroscopy?''}

{
 ``I'm told it works by
Bayes's Rule, but I don't quite
understand how,'' says Susan. ``I
like to say it, though. Bayes Bayes Bayes Bayes
Bayes.''}

{
 ``What does the sorting scanner \textit{tell}
you?''}

{
 ``It tells you whether to put the object into the
blegg bin or the rube bin. That's why
it's called a sorting scanner.''}

{
 At this point you fall silent.}

{
 ``Incidentally,'' Susan says
casually, ``it may interest you to know that bleggs
contain small nuggets of vanadium ore, and rubes contain shreds of
palladium, both of which are useful industrially.''}

{
 ``Susan, you are pure evil.''}

{
  ``Thank you.''}
\end{quotation}

{
 So now it seems we've discovered the heart and
essence of bleggness: a blegg is an object that contains a nugget of
vanadium ore. Surface characteristics, like blue color and furredness,
do not \textit{determine} whether an object is a blegg; surface
characteristics only matter because they help you \textit{infer}
whether an object is a blegg, that is, whether the object contains
vanadium.}

{
 Containing vanadium is a necessary and sufficient definition: all
bleggs contain vanadium and everything that contains vanadium is a
blegg: ``blegg'' is just a shorthand
way of saying ``vanadium-containing
object.'' Right?}

{
 Not so fast, says Susan: Around 98\% of bleggs contain vanadium,
but 2\% contain palladium instead. To be precise (Susan continues)
around 98\% of blue egg-shaped furred flexible opaque objects contain
vanadium. For unusual bleggs, it may be a different percentage: 95\% of
purple bleggs contain vanadium, 92\% of hard bleggs contain vanadium,
etc.}

{
 Now suppose you find a blue egg-shaped furred flexible opaque
object, an ordinary blegg in every visible way, and just for kicks you
take it to the sorting scanner, and the scanner says
``palladium''---this is one of the
rare 2\%. Is it a blegg?}

{
 At first you might answer that, since you intend to throw this
object in the rube bin, you might as well call it a
``rube.'' However, it turns out that
almost all bleggs, if you switch off the lights, glow faintly in the
dark, while almost all rubes do not glow in the dark. And the
percentage of bleggs that glow in the dark is not significantly
different for blue egg-shaped furred flexible opaque objects that
contain palladium, instead of vanadium. Thus, if you want to guess
whether the object glows like a blegg, or remains dark like a rube, you
should guess that it glows like a blegg.}

{
 So is the object \textit{really} a blegg or a rube?}

{
 On one hand, you'll throw the object in the rube
bin no matter what else you learn. On the other hand, if there are any
unknown characteristics of the object you need to infer,
you'll infer them as if the object were a blegg, not a
rube---group it into the similarity cluster of blue egg-shaped furred
flexible opaque things, and not the similarity cluster of red
cube-shaped smooth hard translucent things.}

{
 The question ``Is this object a
blegg?'' may stand in for different queries on
different occasions.}

{
 If it weren't standing in for \textit{some} query,
you'd have no reason to care.}

{
 Is atheism a ``religion''? Is
transhumanism a ``cult''? People who
argue that atheism is a religion ``because it states
beliefs about God'' are really trying to argue (I
think) that the reasoning methods used in atheism are on a par with the
reasoning methods used in religion, or that atheism is no safer than
religion in terms of the probability of causally engendering violence,
etc\ldots~. What's really at stake is an
atheist's claim of substantial difference and
superiority relative to religion, which the religious person is trying
to reject by denying the difference rather than the superiority(!).}

{
 But that's not the a priori irrational part: The a
priori irrational part is where, in the course of the argument, someone
pulls out a dictionary and looks up the definition of
``atheism'' or
``religion.'' (And yes,
it's just as silly whether an atheist or religionist
does it.) How could a dictionary \textit{possibly} decide whether an
empirical cluster of atheists is really substantially different from an
empirical cluster of theologians? How can reality vary with the meaning
of a word? The points in thingspace don't move around
when we redraw a boundary.}

{
 But people often don't \textit{realize} that their
argument about where to draw a definitional boundary, is really a
dispute over whether to infer a characteristic shared by most things
inside an empirical cluster \ldots}

{
 Hence the phrase, ``disguised
query.''}

\myendsectiontext

\mysection{Neural Categories}

{
 In Disguised Queries, I talked about a classification task of
``bleggs'' and
``rubes.'' The typical blegg is
blue, egg-shaped, furred, flexible, opaque, glows in the dark, and
contains vanadium. The typical rube is red, cube-shaped, smooth, hard,
translucent, unglowing, and contains palladium. For the sake of
simplicity, let us forget the characteristics of flexibility/hardness
and opaqueness/translucency. This leaves five dimensions in thingspace:
color, shape, texture, luminance, and interior. }

{
 Suppose I want to create an Artificial Neural Network (ANN) to
predict unobserved blegg characteristics from observed blegg
characteristics. And suppose I'm fairly naive about
ANNs: I've read excited popular science books about how
neural networks are distributed, emergent, and parallel \textit{just
like the human brain!!} but I can't derive the
differential equations for gradient descent in a non-recurrent
multilayer network with sigmoid units (which is actually a lot easier
than it sounds).}

{
 Then I might design a neural network that looks something like
Figure \ref{nc_network_1}.}

%161.1
\myfigurec{images/img198.jpg}{nc_network_1}{Network 1}

{
 Network 1 is for classifying bleggs and rubes. But since
``blegg'' is an unfamiliar and
synthetic concept, I've also included a similar Network
1b in Figure \ref{nc_network_1b} for distinguishing humans from Space Monsters, with
input from Aristotle (``All men are
mortal'') and Plato's Academy
(``A featherless biped with broad
nails'').}

%161.2
\myfigurec{images/img199.jpg}{nc_network_1b}{Network 1b}

{
 A neural network needs a learning rule. The obvious idea is that
when two nodes are often active at the same time, we should strengthen
the connection between them---this is one of the first rules ever
proposed for training a neural network, known as Hebb's
Rule.}

{
 Thus, if you often saw things that were both blue and
furred---thus simultaneously activating the
``color'' node in the + state and
the ``texture'' node in the +
state---the connection would strengthen between color and texture, so
that + colors activated + textures, and vice versa. If you saw things
that were blue and egg-shaped and vanadium-containing, that would
strengthen positive mutual connections between color and shape and
interior.}

{
 Let's say you've already seen
plenty of bleggs and rubes come off the conveyor belt. But now you see
something that's furred, egg-shaped,
and---gasp!---reddish purple (which we'll model as a
``color'' activation level of -2/3).
You haven't yet tested the luminance, or the interior.
What to predict, what to predict?}

{
 What happens then is that the activation levels in Network 1
bounce around a bit. Positive activation flows to luminance from shape,
negative activation flows to interior from color, negative activation
flows from interior to luminance \ldots Of course all these messages are
passed in \textit{parallel!!} and \textit{asynchronously!!} just like
the human brain \ldots}

{
 Finally Network 1 settles into a stable state, which has high
positive activation for
``luminance'' and
``interior.'' The network may be
said to ``expect'' (though it has
not yet seen) that the object will glow in the dark, and that it
contains vanadium.}

{
 And lo, Network 1 exhibits this behavior even though
there's no explicit node that says whether the object
is a blegg or not. The judgment is \textit{implicit in the whole
network!!} Bleggness is an \textit{attractor!!} which arises as the
result of \textit{emergent behavior!!} from the \textit{distributed!!}
learning rule.}

{
 Now in real life, this kind of network design---however faddish it
may sound---runs into \textit{all sorts} of problems. Recurrent
networks don't always settle right away: They can
oscillate, or exhibit chaotic behavior, or just take a very long time
to settle down. This is a Bad Thing when you see something big and
yellow and striped, and you have to wait five minutes for your
distributed neural network to settle into the
``tiger'' attractor. Asynchronous
and parallel it may be, but it's not real-time.}

{
 And there are other problems, like double-counting the evidence
when messages bounce back and forth: If you suspect that an object
glows in the dark, your suspicion will activate belief that the object
contains vanadium, which in turn will activate belief that the object
glows in the dark.}

{
 Plus if you try to scale up the Network 1 design, it requires
$O(N^{2})$ connections, where N is the total number of
observables.}

{
 So what might be a more realistic neural network design?}

%161.3
\myfigurec{images/img200.jpg}{nc_network_2}{Network 2}

{
 In Network 2 of Figure \ref{nc_network_2}, a wave of activation converges on
the central node from any clamped (observed) nodes, and then surges
back out again to any unclamped (unobserved) nodes. Which means we can
compute the answer in one step, rather than waiting for the network to
settle---an important requirement in biology when the neurons only run
at 20Hz. And the network architecture scales as O(N), rather than
$O(N^{2})$.}

{
 Admittedly, there are some things you can notice more easily with
the first network architecture than the second. Network 1 has a direct
connection between every two nodes. So if red objects \textit{never}
glow in the dark, but red furred objects usually have the other blegg
characteristics like egg-shape and vanadium, Network 1 can easily
represent this: it just takes a very strong direct negative connection
from color to luminance, but more powerful positive connections from
texture to all other nodes except luminance.}

{
 Nor is this a ``special
exception'' to the general rule that bleggs
glow---remember, in Network 1, there is no unit that represents
blegg-ness; blegg-ness emerges as an attractor in the distributed
network.}

{
 So yes, those $O(N^{2})$ connections were buying us
something. But not very much. Network 1 is not \textit{more} useful on
most real-world problems, where you rarely find an animal stuck halfway
between being a cat and a dog.}

{
 (There are also facts that you can't easily
represent in Network 1 \textit{or} Network 2. Let's say
sea-blue color and spheroid shape, when found together, always indicate
the presence of palladium; but when found individually, without the
other, they are each very strong evidence for vanadium. This is hard to
represent, in either architecture, without extra nodes. Both Network 1
and Network 2 embody implicit assumptions about what kind of
environmental structure is likely to exist; the ability to read this
off is what separates the adults from the babes, in machine learning.)}

{
 Make no mistake: Neither Network 1 nor Network 2 is biologically
realistic. \textit{But} it still seems like a fair guess that however
the brain really works, it is in some sense closer to Network 2 than
Network 1. Fast, cheap, scalable, works well to distinguish dogs and
cats: natural selection goes for that sort of thing like water running
down a fitness landscape.}

{
 It seems like an ordinary enough task to classify objects as
either bleggs or rubes, tossing them into the appropriate bin. But
would you notice if sea-blue objects never glowed in the dark?}

{
 Maybe, if someone presented you with twenty objects that were
alike only in being sea-blue, and then switched off the light, and none
of the objects glowed. If you got hit over the head with it, in other
words. Perhaps by presenting you with all these sea-blue objects in a
group, your brain forms a new subcategory, and can detect the
``doesn't glow''
characteristic within that subcategory. But you probably
wouldn't notice if the sea-blue objects were scattered
among a hundred other bleggs and rubes. It wouldn't be
\textit{easy} or \textit{intuitive} to notice, the way that
distinguishing cats and dogs is easy and intuitive.}

{
 Or: ``Socrates is human, all humans are mortal,
therefore Socrates is mortal.'' How did Aristotle
know that Socrates was human? Well, Socrates had no feathers, and broad
nails, and walked upright, and spoke Greek, and, well, was generally
shaped like a human and acted like one. So the brain decides, once and
for all, that Socrates is human; and from there, infers that Socrates
is mortal like all other humans thus yet observed. It
doesn't seem easy or intuitive to ask how much wearing
clothes, as opposed to using language, is associated with mortality.
Just, ``things that wear clothes and use language are
human'' and ``humans are
mortal.''}

{
 Are there biases associated with trying to classify things into
categories once and for all? Of course there are. See e.g. Cultish
Countercultishness.}

\myendsectiontext

\mysection{How An Algorithm Feels From Inside}

{
 ``If a tree falls in the forest, and no one hears
it, does it make a sound?'' I remember seeing an
actual argument get started on this subject---a fully naive argument
that went nowhere near Berkeleian subjectivism. Just:}

\begin{quote}
{
 ``It makes a sound, just like any other falling
tree!''}

{
 ``But how can there be a sound that no one
  hears?''}
\end{quote}

{
 The standard rationalist view would be that the first person is
speaking as if ``sound'' means
acoustic vibrations in the air; the second person is speaking as if
``sound'' means an auditory
experience in a brain. If you ask ``Are there acoustic
vibrations?'' or ``Are there
auditory experiences?,'' the answer is at once
obvious. And so the argument is really about the definition of the word
``sound.''}

{
 I think the standard analysis is essentially correct. So
let's accept that as a premise, and ask: Why do people
get into such arguments? What's the underlying
psychology?}

{
 A key idea of the heuristics and biases program is that mistakes
are often more revealing of cognition than correct answers. Getting
into a heated dispute about whether, if a tree falls in a deserted
forest, it makes a sound, is traditionally considered a mistake.}

{
 So what kind of mind design corresponds to that error?}

{
 In Disguised Queries I introduced the blegg/rube classification
task, in which Susan the Senior Sorter explains that your job is to
sort objects coming off a conveyor belt, putting the blue eggs or
``bleggs'' into one bin, and the red
cubes or ``rubes'' into the rube
bin. This, it turns out, is because bleggs contain small nuggets of
vanadium ore, and rubes contain small shreds of palladium, both of
which are useful industrially.}

{
 Except that around 2\% of blue egg-shaped objects contain
palladium instead. So if you find a blue egg-shaped thing that contains
palladium, should you call it a
``rube'' instead?
You're going to put it in the rube bin---why not call
it a ``rube''?}

{
 But when you switch off the light, nearly all bleggs glow faintly
in the dark. And blue egg-shaped objects that contain palladium are
just as likely to glow in the dark as any other blue egg-shaped
object.}

{
 So if you find a blue egg-shaped object that contains palladium
and you ask ``Is it a blegg?,'' the
answer depends on what you have to do with the answer. If you ask
``Which bin does the object go
in?,'' then you choose as if the object is a rube.
But if you ask ``If I turn off the light, will it
glow?,'' you predict as if the object is a blegg. In
one case, the question ``Is it a
blegg?'' stands in for the disguised query,
``Which bin does it go in?'' In the
other case, the question ``Is it a
blegg?'' stands in for the disguised query,
``Will it glow in the dark?''}

{
 Now suppose that you have an object that is blue and egg-shaped
and contains palladium; and you have already observed that it is
furred, flexible, opaque, and glows in the dark.}

{
 This answers \textit{every} query, observes every observable
introduced. There's nothing left for a disguised query
to stand \textit{for}.}

{
 So why might someone feel an impulse to go on arguing whether the
object is \textit{really} a blegg?}

%162.1
\myfigurec{images/img198.jpg}{afi_network_1}{Network 1}

%162.2
\myfigurec{images/img200.jpg}{afi_network_2}{Network 2}


{
 These diagrams from Neural Categories show two different neural
networks that might be used to answer questions about bleggs and rubes.
Network 1 (Figure \ref{afi_network_1}) has a number of disadvantages---such as
potentially oscillating/chaotic behavior, or requiring
$O(N^{2})$ connections---but Network 1's
structure does have one major advantage over Network 2: every unit in
the network corresponds to a testable query. If you observe every
observable, clamping every value, there are no units in the network
left over.}

{
 Network 2 (Figure \ref{afi_network_2}), however, is a far better candidate for
being something vaguely like how the human brain works:
It's fast, cheap, scalable---and has an extra dangling
unit in the center, whose activation can still vary, even after
we've observed every single one of the surrounding
nodes.}

{
 Which is to say that even after you know whether an object is blue
or red, egg or cube, furred or smooth, bright or dark, and whether it
contains vanadium or palladium, it \textit{feels} like
there's a leftover, unanswered question: \textit{But is
it really a blegg?}}

{
 Usually, in our daily experience, acoustic vibrations and auditory
experience go together. But a tree falling in a deserted forest
unbundles this common association. And even after you know that the
falling tree creates acoustic vibrations but not auditory experience,
it \textit{feels} like there's a leftover question:
\textit{Did it make a sound?}}

{
 We know where Pluto is, and where it's going; we
know Pluto's shape, and Pluto's
mass---but is it a planet?}

{
 Now remember: When you look at Network 2, as I've
laid it out here, you're seeing the algorithm from the
outside. People don't think to themselves,
``Should the central unit fire, or
not?'' any more than you think
``Should neuron \#12,234,320,242 in my visual cortex
fire, or not?''}

{
 It takes a deliberate effort to visualize your brain from the
outside---and then you still don't see your actual
brain; you imagine what you \textit{think} is there. Hopefully based on
science, but regardless, you don't have any direct
access to neural network structures from introspection.
That's why the ancient Greeks didn't
invent computational neuroscience.}

{
 When you look at Network 2, you are seeing from the
\textit{outside}; but the way that neural network structure feels from
the \textit{inside}, if you yourself \textit{are} a brain running that
algorithm, is that even after you know every characteristic of the
object, you still find yourself wondering: ``But is it
a blegg, or not?''}

{
 This is a great gap to cross, and I've seen it
stop people in their tracks. Because we don't
instinctively see our intuitions as
``intuitions,'' we just see them as
the world. When you look at a green cup, you don't
think of yourself as seeing a picture reconstructed in your visual
cortex---although that \textit{is} what you are seeing---you just see a
green cup. You think, ``Why, look, this cup is
green,'' not, ``The picture in my
visual cortex of this cup is green.''}

{
 And in the same way, when people argue over whether the falling
tree makes a sound, or whether Pluto is a planet, they
don't see themselves as arguing over whether a
categorization should be active in their neural networks. It seems like
either the tree makes a sound, or not.}

{
 We know where Pluto is, and where it's going; we
know Pluto's shape, and Pluto's
mass---but is it a planet? And yes, there were people who said this was
a fight over definitions---but even that is a Network 2 sort of
perspective, because you're arguing about how the
central unit ought to be wired up. If you were a mind constructed along
the lines of Network 1, you wouldn't say
``It depends on how you define
`planet,''' you would
just say, ``Given that we know Pluto's
orbit and shape and mass, there is no question left to
ask.'' Or, rather, that's how it
would \textit{feel}{}---it would \textit{feel} like there was no
question left---if you were a mind constructed along the lines of
Network 1.}

{
 Before you can question your intuitions, you have to realize that
what your mind's eye is looking at \textit{is} an
intuition---some cognitive algorithm, as seen from the inside---rather
than a direct perception of the Way Things Really Are.}

{
 People cling to their intuitions, I think, not so much because
they believe their cognitive algorithms are perfectly reliable, but
because they can't see their intuitions \textit{as the
way their cognitive algorithms happen to look from the inside}.}

{
 And so everything you try to say about how the native cognitive
algorithm goes astray, ends up being contrasted to their direct
perception of the Way Things Really Are---and discarded as obviously
wrong.}

\myendsectiontext

\mysection{Disputing Definitions}

{
 I have watched more than one conversation---even conversations
supposedly about cognitive science---go the route of disputing over
definitions. Taking the classic example to be ``If a
tree falls in a forest, and no one hears it, does it make a
sound?,'' the dispute often follows a course like
this:}

\begin{quote}
{
 \textit{If a tree falls in the forest, and no one hears it, does
it make a sound?}}
\end{quote}
\begin{quotation}
{
 ALBERT: ``Of course it does. What kind of silly
question is that? Every time I've listened to a tree
fall, it made a sound, so I'll guess that other trees
falling also make sounds. I don't believe the world
changes around when I'm not
looking.''}

{
 BARRY: ``Wait a minute. If no one hears it, how
 can it be a sound?''}
\end{quotation}

{
 In this example, Barry is arguing with Albert because of a
genuinely different intuition about what constitutes a sound. But
there's more than one way the Standard Dispute can
start. Barry could have a motive for rejecting Albert's
conclusion. Or Barry could be a skeptic who, upon hearing
Albert's argument, reflexively scrutinized it for
possible logical flaws; and then, on finding a counterargument,
automatically accepted it without applying a second layer of search for
a counter-counterargument; thereby arguing himself into the opposite
position. This doesn't require that
Barry's \textit{prior} intuition---the intuition Barry
would have had, if we'd asked him before Albert
spoke---differs from Albert's.}

{
 Well, if Barry didn't have a differing intuition
before, he sure has one now.}

\begin{quotation}
{
 ALBERT: ``What do you mean,
there's no sound? The tree's roots
snap, the trunk comes crashing down and hits the ground. This generates
vibrations that travel through the ground and the air.
That's where the energy of the fall goes, into heat and
sound. Are you saying that if people leave the forest, the tree
violates Conservation of Energy?''}

{
 BARRY: ``But no one hears anything. If there are
no humans in the forest, or, for the sake of argument, anything else
with a complex nervous system capable of
`hearing,' then no one hears a
sound.''}
\end{quotation}

{
 Albert and Barry recruit arguments that \textit{feel} like support
for their respective positions, describing in more detail the thoughts
that caused their
``sound''-detectors to fire or stay
silent. But so far the conversation has still focused on the forest,
rather than definitions. And note that they don't
actually disagree on anything that happens in the forest.}

\begin{quotation}
{
 ALBERT: ``This is the dumbest argument
I've ever been in. You're a
niddlewicking fallumphing pickleplumber.''}

{
 BARRY: ``Yeah? Well, you look like your face
caught on fire and someone put it out with a
shovel.''}
\end{quotation}

{
 Insult has been proffered and accepted; now neither party can back
down without losing face. Technically, this isn't part
of the \textit{argument}, as rationalists account such things; but
it's such an important part of the Standard Dispute
that I'm including it anyway.}

\begin{quotation}
{
 ALBERT: ``The tree produces acoustic vibrations.
By definition, that is a sound.''}

{
 BARRY: ``No one hears anything. By definition,
 that is not a sound.''}
\end{quotation}

{
 The argument starts shifting to focus on definitions. Whenever you
feel tempted to say the words ``by
definition'' in an argument that is not literally
about pure mathematics, remember that anything which is true
``by definition'' is true in all
possible worlds, and so observing its truth can never constrain
\textit{which} world you live in.}

\begin{quotation}
{
 ALBERT: ``My computer's
microphone can record a sound without anyone being around to hear it,
store it as a file, and it's called a
`sound file.' And what's
stored in the file is the pattern of vibrations in air, not the pattern
of neural firings in anyone's brain.
`Sound' means a pattern of
vibrations.''}
\end{quotation}

{
 Albert deploys an argument that \textit{feels} like support for
the word ``sound'' \textit{having a
particular meaning.} This is a different kind of question from whether
acoustic vibrations take place in a forest---but the shift usually
passes unnoticed.}

\begin{quotation}
{
 BARRY: ``Oh, yeah? Let's just see
 if the dictionary agrees with you.''}
\end{quotation}

{
 There's a lot of things I could be curious about
in the falling-tree scenario. I could go into the forest and look at
trees, or learn how to derive the wave equation for changes of air
pressure, or examine the anatomy of an ear, or study the neuroanatomy
of the auditory cortex. Instead of doing any of these things, I am to
consult a dictionary, apparently. Why? Are the editors of the
dictionary expert botanists, expert physicists, expert neuroscientists?
Looking in an encyclopedia might make sense, but why a
\textit{dictionary}?}

\begin{quotation}
{
 ALBERT: ``Hah! Definition 2c in Merriam-Webster:
`Sound: Mechanical radiant energy that is transmitted by
longitudinal pressure waves in a material medium (as
air).'''}

{
 BARRY: ``Hah! Definition 2b in Merriam-Webster:
`Sound: The sensation perceived by the sense of
hearing.'''}

{
 ALBERT AND BARRY, CHORUS: ``Consarned dictionary!
 This doesn't help at all!''}
\end{quotation}

{
 Dictionary editors are historians of usage, not legislators of
language. Dictionary editors find words in current usage, then write
down the words next to (a small part of) what people seem to mean by
them. If there's more than one usage, the editors write
down more than one definition.}

\begin{quotation}
{
 ALBERT: ``Look, suppose that I left a microphone
in the forest and recorded the pattern of the acoustic vibrations of
the tree falling. If I played that back to someone,
they'd call it a
`sound'! That's the
common usage! Don't go around making up your own wacky
definitions!''}

{
 BARRY: ``One, I can define a word any way I like
so long as I use it consistently. Two, the meaning I gave was in the
dictionary. Three, who gave you the right to decide what is or
isn't common usage?''}
\end{quotation}

{
 There's quite a lot of rationality errors in the
Standard Dispute. Some of them I've already covered,
and some of them I've yet to cover; likewise the
remedies.}

{
 But for now, I would just like to point out---in a mournful sort
of way---that Albert and Barry seem to agree on virtually every
question of what is \textit{actually} going on inside the forest, and
yet it doesn't seem to generate any feeling of
agreement.}

{
 Arguing about definitions is a garden path; people
wouldn't go down the path if they saw at the outset
where it led. If you asked Albert (Barry) why he's
still arguing, he'd probably say something like:
``Barry (Albert) is trying to sneak in his own
definition of `sound,' the scurvey
scoundrel, to support his ridiculous point; and I'm
here to defend the standard definition.''}

{
 But suppose I went back in time to before the start of the
argument:}

\begin{quotation}
{
 \textit{(Eliezer appears from nowhere in a peculiar conveyance
that looks just like the time machine from the original The Time
Machine movie.)}}

{
 BARRY: ``Gosh! A time
traveler!''}

{
 ELIEZER: ``I am a traveler from the future! Hear
my words! I have traveled far into the past---around fifteen
minutes---''}

{
 ALBERT: ``Fifteen
\textit{minutes}?''}

{
 ELIEZER: ``---to bring you this
message!''}

{
 \textit{(There is a pause of mixed confusion and expectancy.)}}

{
 ELIEZER: ``Do you think that
`sound' should be defined to require
both acoustic vibrations (pressure waves in air) and also auditory
experiences (someone to listen to the sound), or should
`sound' be defined as meaning only
acoustic vibrations, or only auditory experience?''}

{
 BARRY: ``You went back in time to ask us
\textit{that}?''}

{
 ELIEZER: ``My purposes are my own!
Answer!''}

{
 ALBERT: ``Well \ldots I don't see
why it would matter. You can pick any definition so long as you use it
consistently.''}

{
 BARRY: ``Flip a coin. Er, flip a coin
twice.''}

{
 ELIEZER: ``Personally I'd say
that if the issue arises, both sides should switch to describing the
event in unambiguous lower-level constituents, like acoustic vibrations
or auditory experiences. Or each side could designate a new word, like
`alberzle' and
`bargulum,' to use for what they
respectively used to call `sound'; and
then both sides could use the new words consistently. That way neither
side has to back down or lose face, but they can still communicate. And
of course you should try to keep track, at all times, of some testable
proposition that the argument is actually about. Does that sound right
to you?''}

{
 ALBERT: ``I guess \ldots''}

{
 BARRY: ``Why are we talking about
this?''}

{
 ELIEZER: ``To preserve your friendship against a
contingency you will, now, never know. For the future has already
changed!''}

{
 \textit{(Eliezer and the machine vanish in a puff of smoke.)}}

{
 BARRY: ``Where were we
again?''}

{
 ALBERT: ``Oh, yeah: If a tree falls in the
forest, and no one hears it, does it make a sound?''}

{
 BARRY: ``It makes an alberzle but not a bargulum.
What's the next question?''}
\end{quotation}

{
 This remedy doesn't destroy \textit{every} dispute
over categorizations. But it destroys a substantial fraction.}

\myendsectiontext

\mysection{Feel the Meaning}

{
 When I hear someone say, ``Oh, look, a
butterfly,'' the spoken phonemes
``butterfly'' enter my ear and
vibrate on my ear drum, being transmitted to the cochlea, tickling
auditory nerves that transmit activation spikes to the auditory cortex,
where phoneme processing begins, along with recognition of words, and
reconstruction of syntax (a by no means serial process), and all manner
of other complications. }

{
 But at the end of the day, or rather, at the end of the second, I
am primed to look where my friend is pointing and see a visual pattern
that I will recognize as a butterfly; and I would be quite surprised to
see a wolf instead.}

{
 My friend looks at a butterfly, his throat vibrates and lips move,
the pressure waves travel invisibly through the air, my ear hears and
my nerves transduce and my brain reconstructs, and lo and behold, I
know what my friend is looking at. Isn't that
marvelous? If we didn't know about the pressure waves
in the air, it would be a tremendous discovery in all the newspapers:
Humans are telepathic! Human brains can transfer thoughts to each
other!}

{
 Well, we \textit{are} telepathic, in fact; but magic
isn't exciting when it's merely
\textit{real}, and all your friends can do it too.}

{
 Think telepathy is simple? Try building a computer that will be
telepathic with you. Telepathy, or
``language,'' or whatever you want
to call our partial thought transfer ability, is more complicated than
it looks.}

{
 But it would be quite inconvenient to go around thinking,
``Now I shall partially transduce some features of my
thoughts into a linear sequence of phonemes which will invoke similar
thoughts in my conversational partner \ldots''}

{
 So the brain hides the complexity---or rather, never represents it
in the first place---which leads people to think some peculiar thoughts
about words.}

{
 As I remarked earlier, when a large yellow striped object leaps at
me, I think ``Yikes! A tiger!'' not
``Hm \ldots objects with the properties of largeness,
yellowness, and stripedness have previously often possessed the
properties `hungry' and
`dangerous,' and therefore, although it
is not logically necessary, \textit{auughhhh} CRUNCH CRUNCH
GULP.''}

{
 Similarly, when someone shouts ``Yikes! A
tiger!,'' natural selection would not favor an
organism that thought, ``Hm \ldots I have just heard
the syllables `Tie' and
`Grr' which my fellow tribe members
associate with their internal analogues of my own \textit{tiger}
concept, and which they are more likely to utter if they see an object
they categorize as \textit{aiiieeee} CRUNCH CRUNCH \textit{help
it's got my arm} CRUNCH GULP.''}

%164.1
\myfigurec{images/img206.jpg}{fm_network_3}{Network 3}

{
 Considering this as a design constraint on the human cognitive
architecture, you wouldn't want \textit{any} extra
steps between when your auditory cortex recognizes the syllables
``tiger,'' and when the tiger
concept gets activated.}

{
 Going back to the parable of bleggs and rubes, and the centralized
network that categorizes quickly and cheaply, you might visualize a
direct connection running from the unit that recognizes the syllable
``blegg'' to the unit at the center
of the blegg network. The central unit, the blegg concept, gets
activated almost as soon as you hear Susan the Senior Sorter say,
``Blegg!''}

{
 Or, for purposes of talking---which also shouldn't
take eons---as soon as you see a blue egg-shaped thing and the central
blegg unit fires, you holler
``Blegg!'' to Susan.}

{
 And what that algorithm feels like from inside is that the label,
and the concept, are very nearly \textit{identified}; the meaning
\textit{feels like} an intrinsic property of the word itself.}

{
 The cognoscenti will recognize this as a case of E. T.
Jaynes's ``Mind Projection
Fallacy.'' It feels like a word \textit{has a}
meaning, as a property of the word itself; just like how redness is a
property of a red apple, or mysteriousness is a property of a
mysterious phenomenon.}

{
 Indeed, on most occasions, the brain will not distinguish at all
between the word and the meaning---only bothering to separate the two
while learning a new language, perhaps. And even then,
you'll see Susan pointing to a blue egg-shaped thing
and saying ``Blegg!,'' and
you'll think, \textit{I wonder what
``blegg'' means,} and not, \textit{I
wonder what mental category Susan associates to the auditory label
``blegg.''}}

{
 Consider, in this light, the part of the Standard Dispute of
Definitions where the two parties argue about what the word
``sound'' \textit{really}
means---the same way they might argue whether a particular apple is
\textit{really} red or green:}

\begin{quotation}
{
 ALBERT: ``My computer's
microphone can record a sound without anyone being around to hear it,
store it as a file, and it's called a
`sound file.' And what's
stored in the file is the pattern of vibrations in air, not the pattern
of neural firings in anyone's brain.
`Sound' means a pattern of
vibrations.''}

{
 BARRY: ``Oh, yeah? Let's just see
 if the dictionary agrees with you.''}
\end{quotation}

{
 Albert feels intuitively that the word
``sound'' \textit{has a meaning} and
that the meaning \textit{is} acoustic vibrations. Just as Albert feels
that a tree falling in the forest \textit{makes a sound} (rather than
causing an event that \textit{matches the sound category}).}

{
 Barry likewise \textit{feels} that:}

\begin{verbatim}
  sound.meaning == auditory experiences

  forest.sound == false.
\end{verbatim}

{
 Rather than:}

\whencolumns{
\begin{flushleft}
\texttt{\ \  myBrain.FindConcept("sound") ==\\
\ \ \  concept\_AuditoryExperience\\
\ \\
\ \ concept\_AuditoryExperience.match(forest) == false.}
\end{flushleft}
}{
\begin{flushleft}
\texttt{\ \  myBrain.FindConcept("sound") ==\\
\ \ \  concept\_AuditoryExperience\\
\ \\
\ \ concept\_AuditoryExperience.match(forest)\\
\ \ \ == false.}
\end{flushleft}
}

{
 Which is closer to what's \textit{really} going
on; but humans have not evolved to know this, anymore than humans
instinctively know the brain is made of neurons.}

{
 Albert and Barry's conflicting intuitions provide
the fuel for continuing the argument in the phase of arguing over what
the word ``sound'' means---which
\textit{feels} like arguing over a fact like any other fact, like
arguing over whether the sky is blue or green.}

{
 You may not even notice that anything has gone astray, until you
try to perform the rationalist ritual of stating a testable experiment
whose result depends on the facts you're so heatedly
disputing \ldots}

\myendsectiontext

\mysection{The Argument from Common Usage}

{
 Part of the Standard Definitional Dispute runs as follows:}

\begin{quotation}
{
 ALBERT: ``Look, suppose that I left a microphone
in the forest and recorded the pattern of the acoustic vibrations of
the tree falling. If I played that back to someone,
they'd call it a
`sound'! That's the
common usage! Don't go around making up your own wacky
definitions!''}

{
 BARRY: ``One, I can define a word any way I like
so long as I use it consistently. Two, the meaning I gave was in the
dictionary. Three, who gave you the right to decide what is or
isn't common usage?''}
\end{quotation}

{
 Not all definitional disputes progress as far as recognizing the
notion of common usage. More often, I think, someone picks up a
dictionary because they believe that words have meanings, and the
dictionary faithfully records what this meaning is. Some people even
seem to believe that the dictionary \textit{determines} the
meaning---that the dictionary editors are the Legislators of Language.
Maybe because back in elementary school, their authority-teacher said
that they had to obey the dictionary, that it was a mandatory rule
rather than an optional one?}

{
 Dictionary editors read what other people write, and record what
the words seem to mean; they are historians. The Oxford English
Dictionary may be \textit{comprehensive}, but never
\textit{authoritative.}}

{
 But surely there is a social imperative to use words in a commonly
understood way? Does not our human telepathy, our valuable power of
language, rely on mutual coordination to work? Perhaps we should
voluntarily treat dictionary editors as supreme arbiters---even if
\textit{they} prefer to think of themselves as historians---in order to
maintain the quiet cooperation on which all speech depends.}

{
 The phrase ``authoritative
dictionary'' is almost never used correctly, an
example of proper usage being \textit{The Authoritative Dictionary of
IEEE Standards Terms}. The IEEE is a body of voting members who have a
professional need for exact agreement on terms and definitions, and so
\textit{The Authoritative Dictionary of IEEE Standards Terms} is
actual, negotiated legislation, which exerts whatever authority one
regards as residing in the IEEE.}

{
 In everyday life, shared language usually does not arise from a
deliberate agreement, as of the IEEE. It's more a
matter of infection, as words are invented and diffuse through the
culture. (A ``meme,'' one might say,
following Richard Dawkins forty years ago---but you already know what I
mean, and if not, you can look it up on Google, and then you too will
have been infected.)}

{
 Yet as the example of the IEEE shows, agreement on language can
also be a cooperatively established public good. If you and I wish to
undergo an exchange of thoughts via language, the human telepathy, then
it is in our mutual interest that we use the \textit{same} word for
similar concepts---preferably, concepts similar to the limit of
resolution in our brain's representation thereof---even
though we have no obvious mutual interest in using any
\textit{particular} word for a concept.}

{
 We have no obvious mutual interest in using the word
``oto'' to mean sound, or
``sound'' to mean oto; but we have a
mutual interest in using the \textit{same} word, whichever word it
happens to be. (Preferably, words we use frequently should be short,
but let's not get into information theory just yet.)}

{
 But, while we have a mutual interest, it is not strictly
\textit{necessary} that you and I use the similar labels
\textit{internally}; it is only convenient. If I know that, to you,
``oto'' means sound---that is, you
associate ``oto'' to a concept very
similar to the one I associate to
``sound''---then I can say
``Paper crumpling makes a crackling
oto.'' It requires extra thought, but I can do it if
I want.}

{
 Similarly, if you say ``What is the walking-stick
of a bowling ball dropping on the floor?'' and I know
which concept \textit{you} associate with the syllables
``walking-stick,'' then I can figure
out what you mean. It may require some thought, and give me pause,
because I ordinarily associate
``walking-stick'' with a different
concept. But I can do it just fine.}

{
 When humans really \textit{want} to communicate with each other,
we're hard to stop! If we're stuck on a
deserted island with no common language, we'll take up
sticks and draw pictures in sand.}

{
 Albert's appeal to the Argument from Common Usage
assumes that agreement on language is a cooperatively established
public good. Yet Albert assumes this for the sole purpose of
rhetorically accusing Barry of breaking the agreement, and endangering
the public good. Now the falling-tree argument has gone all the way
from botany to semantics to politics; and so Barry responds by
challenging Albert for the authority to define the word.}

{
 A rationalist, with the discipline of hugging the query active,
would notice that the conversation had gone rather far astray.}

{
 Oh, dear reader, is it all really necessary? Albert knows what
Barry means by ``sound.'' Barry
knows what Albert means by
``sound.'' Both Albert and Barry
have access to words, such as ``acoustic
vibrations'' or ``auditory
experience,'' which they already associate to the
same concepts, and which can describe events in the forest without
ambiguity. If they were stuck on a deserted island, trying to
communicate with each other, their work would be \textit{done}.}

{
 When both sides \textit{know} what the other side \textit{wants}
to say, and both sides accuse the other side of defecting from
``common usage,'' then whatever it
is they are about, it is clearly not \textit{working out a way to
communicate with each other.} But this is the whole benefit that common
usage provides in the first place.}

{
 Why would you argue about the meaning of a word, two sides trying
to wrest it back and forth? If it's just a namespace
conflict that has gotten blown out of proportion, and nothing more is
at stake, then the two sides need merely generate two new words and use
them consistently.}

{
 Yet often categorizations function as hidden inferences and
disguised queries. Is atheism a
``religion''? If someone is arguing
that the reasoning methods used in atheism are on a par with the
reasoning methods used in Judaism, or that atheism is on a par with
Islam in terms of causally engendering violence, then they have a clear
argumentative stake in lumping it all together into an indistinct gray
blur of ``faith.''}

{
 Or consider the fight to blend together blacks and whites as
``people.'' This would not be a time
to generate two words---what's at stake is exactly the
idea that you shouldn't draw a moral distinction.}

{
 But once any empirical proposition is at stake, \textit{or} any
moral proposition, you can no longer appeal to common usage.}

{
 If the question is how to cluster together similar things for
purposes of inference, empirical predictions will depend on the answer;
which means that definitions can be \textit{wrong}. A conflict of
predictions cannot be settled by an opinion poll.}

{
 If you want to know whether atheism should be clustered with
supernaturalist religions for purposes of some particular empirical
inference, the dictionary can't answer you.}

{
 If you want to know whether blacks are people, the dictionary
can't answer you.}

{
 If everyone believes that the red light in the sky is Mars the God
of War, the dictionary will define
``Mars'' as the God of War. If
everyone believes that fire is the release of phlogiston, the
dictionary will define ``fire'' as
the release of phlogiston.}

{
 There is an art to using words; even when definitions are not
literally true or false, they are often wiser or more foolish.
Dictionaries are mere histories of past usage; if you treat them as
supreme arbiters of meaning, it binds you to the wisdom of the past,
forbidding you to do better.}

{
 Though do take care to ensure (if you must depart from the wisdom
of the past) that people can figure out what you're
trying to swim.}

\myendsectiontext

\mysection{Empty Labels}

{
 Consider (yet again) the Aristotelian idea of categories.
Let's say that there's some object with
properties $A$, $B$, $C$, $D$, and $E$, or at least it looks $E$-ish.}

\begin{quotation}
{
 FRED: ``You mean that thing over there is blue,
round, fuzzy, and---''}

{
 ME: ``In Aristotelian logic, it's
not supposed to make a difference what the properties are, or what I
call them. That's why I'm just using
the letters.''}
\end{quotation}

{
 Next, I invent the Aristotelian category
``zawa,'' which describes those
objects, all those objects, and only those objects, that have
properties $A$, $C$, and $D$.}

\begin{quotation}
{
 ME: ``Object 1 is zawa, $B$, and
$E$.''}

{
 FRED: ``And it's blue---I mean,
$A$---too, right?''}

{
 ME: ``That's implied when I say
it's zawa.''}

{
 FRED: ``Still, I'd like you to
say it explicitly.''}

{
 ME: ``Okay. Object 1 is $A$, $B$, zawa, and
$E$.''}
\end{quotation}

{
 Then I add another word,
``yokie,'' which describes all and
only objects that are $B$ and $E$; and the word
``xippo,'' which describes all and
only objects which are $E$ but not $D$.}

\begin{quotation}
{
 ME: ``Object 1 is zawa and yokie, but not
xippo.''}

{
 FRED: ``Wait, is it luminescent? I mean, is it
$E$?''}

{
 ME: ``Yes. That is the only possibility on the
information given.''}

{
 FRED: ``I'd rather you spelled it
out.''}

{
 ME: ``Fine: Object 1 is $A$, zawa, $B$, yokie, $C$, $D$,
$E$, and not xippo.''}

{
 FRED: ``Amazing! You can tell all that just by
 looking?''}
\end{quotation}

{
 Impressive, isn't it? Let's invent
even more new words: ``Bolo'' is $A$,
$C$, and yokie; ``mun'' is $A$, $C$, and
xippo; and ``merlacdonian'' is bolo
and mun.}

{
 Pointlessly confusing? I think so too. Let's
replace the labels with the definitions:}

\begin{quotation}
{
 ``Zawa, $B$, and $E$'' becomes $[A,
C, D], B, E$}

{
 ``Bolo and $A$'' becomes $[A, C,
[B, E]], A$}

{
 ``Merlacdonian'' becomes $[A, C,
    [B, E]], [A, C, [E, \lnot D]]$.}
\end{quotation}

{
 And the thing to remember about the Aristotelian idea of
categories is that $[A, C, D]$ is the \textit{entire} information of
``zawa.'' It's not
just that I can vary the label, but that I can get along just fine
without any label at all---the rules for Aristotelian classes work
purely on structures like $[A, C, D]$. To call one of these structures
``zawa,'' or attach any other label
to it, is a human convenience (or inconvenience) which makes not the
slightest difference to the Aristotelian rules.}

{
 Let's say that
``human'' is to be defined as a
mortal featherless biped. Then the classic syllogism would have the
form:}

\begin{quotation}
{
 All [mortal, {\textlnot}feathers, bipedal] are mortal.}

{
 Socrates is a [mortal, {\textlnot}feathers, bipedal].}

{
  Therefore, Socrates is mortal.}
\end{quotation}

{
 The feat of reasoning looks a lot less impressive now,
doesn't it?}

{
 Here the \textit{illusion of inference} comes from the labels,
which conceal the premises, and pretend to novelty in the conclusion.
Replacing labels with definitions reveals the illusion, making visible
the tautology's empirical unhelpfulness. You can never
say that Socrates is a [mortal, {\textlnot}feathers, biped] until you
have observed him to be mortal.}

{
 There's an idea, which you may have noticed I
hate, that ``you can define a word any way you
like.'' This idea came from the Aristotelian notion
of categories; since, if you follow the Aristotelian rules
\textit{exactly} and \textit{without flaw}{}---which humans never do;
Aristotle knew perfectly well that Socrates was human, even though that
wasn't justified under his rules---but, \textit{if}
some imaginary nonhuman entity were to follow the rules exactly, they
would never arrive at a contradiction. They wouldn't
arrive at much of anything: they couldn't say that
Socrates is a [mortal, {\textlnot}feathers, biped] until they observed
him to be mortal.}

{
 But it's not so much that labels are
\textit{arbitrary} in the Aristotelian system, as that the Aristotelian
system works fine without \textit{any labels at all}{}---it cranks out
exactly the same stream of tautologies, they just look a lot less
impressive. The labels are only there to create the \textit{illusion}
of inference.}

{
 So if you're going to have an Aristotelian proverb
at all, the proverb should be, not ``I can define a
word any way I like,'' nor even,
``Defining a word never has any
consequences,'' but rather,
``Definitions don't need
words.''}

\myendsectiontext

\mysection{Taboo Your Words}

{
 In the game Taboo (by Hasbro), the objective is for a player to
have their partner guess a word written on a card, without using that
word or five additional words listed on the card. For example, you
might have to get your partner to say
``baseball'' without using the words
``sport,''
``bat,''
``hit,''
``pitch,''
``base'' or of course
``baseball.'' }

{
 As soon as I see a problem like that, I at once think,
``An artificial group conflict in which you use a long
wooden cylinder to whack a thrown spheroid, and then run between four
safe positions.'' It might not be the most efficient
strategy to convey the word
``baseball'' under the stated
rules---that might be, ``It's what the
Yankees play''---but the general skill of
\textit{blanking a word out of my mind} was one I'd
practiced for years, albeit with a different purpose.}

{
 In the previous essay we saw how replacing terms with definitions
could reveal the empirical unproductivity of the classical Aristotelian
syllogism. All humans are mortal (and also, apparently, featherless
bipeds); Socrates is human; therefore Socrates is mortal. When we
replace the word ``human'' by its
apparent definition, the following underlying reasoning is revealed:}

\begin{quotation}
{
 All [mortal, {\textlnot}feathers, biped] are mortal;}

{
 Socrates is a [mortal, {\textlnot}feathers, biped];}

{
  Therefore Socrates is mortal.}
\end{quotation}

{
 But the principle of replacing words by definitions applies much
more broadly:}

\begin{quotation}
{
 ALBERT: ``A tree falling in a deserted forest
makes a sound.''}

{
 BARRY: ``A tree falling in a deserted forest does
 not make a sound.''}
\end{quotation}

{
 Clearly, since one says
``sound'' and one says
``not sound,'' we must have a
contradiction, right? But suppose that they both dereference their
pointers before speaking:}

\begin{quotation}
{
 ALBERT: ``A tree falling in a deserted forest
matches [membership test: this event generates acoustic
vibrations].''}

{
 BARRY: ``A tree falling in a deserted forest does
not match [membership test: this event generates auditory
  experiences].''}
\end{quotation}

{
 Now there is no longer an apparent collision---all they had to do
was prohibit themselves from using the word \textit{sound}. If
``acoustic vibrations'' came into
dispute, we would just play Taboo again and say
``pressure waves in a material
medium''; if necessary we would play Taboo again on
the word ``wave'' and replace it
with the wave equation. (Play Taboo on ``auditory
experience'' and you get ``That form
of sensory processing, within the human brain, that takes as input a
linear time series of frequency mixes \ldots'')}

{
 But suppose, on the other hand, that Albert and Barry were to have
the argument:}

\begin{quotation}
{
 ALBERT: ``Socrates matches the concept
[membership test: this person will die after drinking
hemlock].''}

{
 BARRY: ``Socrates matches the concept [membership
test: this person will not die after drinking
hemlock].''}
\end{quotation}

{
 Now Albert and Barry have a substantive clash of expectations; a
difference in what they anticipate seeing after Socrates drinks
hemlock. But they might not notice this, if they happened to use the
same word ``human'' for their
different concepts.}

{
 You get a very different picture of what people agree or disagree
about, depending on whether you take a label's-eye-view
(Albert says ``sound'' and Barry
says ``not sound,'' so they must
disagree) or taking the test's-eye-view
(Albert's membership test is acoustic vibrations,
Barry's is auditory experience).}

{
 Get together a pack of \textit{soi-disant} futurists and ask them
if they believe we'll have Artificial Intelligence in
thirty years, and I would guess that at least half of them will say
yes. If you leave it at that, they'll shake hands and
congratulate themselves on their consensus. But make the term
``Artificial Intelligence'' taboo,
and ask them to describe \textit{what} they expect to see, without ever
using words like ``computers'' or
``think,'' and you might find quite
a conflict of expectations hiding under that featureless standard word.
See also Shane Legg's compilation of 71 definitions of
``intelligence.''}

{
 The illusion of unity across religions can be dispelled by making
the term ``God'' taboo, and asking
them to say what it is they believe in; or making the word
``faith'' taboo, and asking them why
they believe it. Though mostly they won't be able to
answer at all, because it is mostly profession in the first place, and
you cannot cognitively zoom in on an audio recording.}

{
 When you find yourself in philosophical difficulties, \textit{the
first line of defense is not to define your problematic terms, but to
see whether you can think without using those terms at all.} Or any of
their short synonyms. And be careful not to let yourself invent a new
word to use instead. Describe outward observables and interior
mechanisms; don't use a single handle, whatever that
handle may be.}

{
 Albert says that people have ``free
will.'' Barry says that people don't
have ``free will.'' Well, that will
certainly generate an apparent conflict. Most philosophers would advise
Albert and Barry to try to define exactly what they mean by
``free will,'' on which topic they
will certainly be able to discourse at great length. I would advise
Albert and Barry to describe what it is that they think people do, or
do not have, without using the phrase ``free
will'' at all. (If you want to try this at home, you
should also avoid the words
``choose,''
``act,''
``decide,''
``determined,''
``responsible,'' or any of their
synonyms.)}

{
 This is one of the nonstandard tools in my toolbox, and in my
humble opinion, it works \textit{way way} better than the standard one.
It also requires more effort to use; you get what you pay for.}

\myendsectiontext

\mysection{Replace the Symbol with the Substance}

{
 What does it take to---as in the previous essay's
example---see a ``baseball game'' as
``An artificial group conflict in which you use a long
wooden cylinder to whack a thrown spheroid, and then run between four
safe positions''? What does it take to play the
rationalist version of Taboo, in which the goal is not to find a
synonym that isn't on the card, but to find a way of
describing without the standard concept-handle? }

{
 You have to visualize. You have to make your
mind's eye see the details, as though looking for the
first time. You have to perform an Original Seeing.}

{
 Is that a ``bat''? No,
it's a long, round, tapering, wooden rod, narrowing at
one end so that a human can grasp and swing it.}

{
 Is that a ``ball''? No,
it's a leather-covered spheroid with a symmetrical
stitching pattern, hard but not metal-hard, which someone can grasp and
throw, or strike with the wooden rod, or catch.}

{
 Are those ``bases''? No,
they're fixed positions on a game field, that players
try to run to as quickly as possible because of their safety within the
game's artificial rules.}

{
 The chief obstacle to performing an original seeing is that your
mind already has a nice neat summary, a nice little easy-to-use concept
handle. Like the word ``baseball,''
or ``bat,'' or
``base.'' It takes an effort to stop
your mind from sliding down the familiar path, the easy path, the path
of least resistance, where the small featureless word rushes in and
obliterates the details you're trying to see. A word
itself can have the destructive force of cliché; a word itself can
carry the poison of a cached thought.}

{
 Playing the game of Taboo---being able to describe without using
the standard pointer/label/handle---is one of the \textit{fundamental}
rationalist capacities. It occupies the same primordial level as the
habit of constantly asking ``Why?''
or ``What does this belief make me
anticipate?''}

{
 The art is closely related to:}

\begin{itemize}
\item {
 Pragmatism, because seeing in this way often gives you a much
closer connection to anticipated experience, rather than propositional
belief;}

\item {
 Reductionism, because seeing in this way often forces you to drop
down to a lower level of organization, look at the parts instead of
your eye skipping over the whole;}

\item {
 Hugging the query, because words often distract you from the
question you really want to ask;}

\item {
 Avoiding cached thoughts, which will rush in using standard words,
so you can block them by tabooing standard words;}

\item {
 The writer's rule of ``Show,
don't tell!,'' which has power among
rationalists;}

\item {
  And not losing sight of your original purpose.}
\end{itemize}

{
 How could tabooing a word help you keep your purpose?}

{
 From Lost Purposes:}

\begin{quote}
{
 As you read this, some young man or woman is sitting at a desk in
a university, earnestly studying material they have no intention of
ever using, and no interest in knowing for its own sake. They want a
high-paying job, and the high-paying job requires a piece of paper, and
the piece of paper requires a previous master's degree,
and the master's degree requires a
bachelor's degree, and the university that grants the
bachelor's degree requires you to take a class in
twelfth-century knitting patterns to graduate. So they diligently
study, intending to forget it all the moment the final exam is
administered, but still seriously working away, because they
\textit{want} that piece of paper.}
\end{quote}

{
 Why are you going to
``school''? To get an
``education'' ending in a
``degree.'' Blank out the forbidden
words and all their obvious synonyms, visualize the actual details, and
you're much more likely to notice that
``school'' currently seems to
consist of sitting next to bored teenagers listening to material you
already know, that a ``degree'' is a
piece of paper with some writing on it, and that
``education'' is forgetting the
material as soon as you're tested on it.}

{
 Leaky generalizations often manifest through categorizations:
People who actually learn in classrooms are categorized as
``getting an education,'' so
``getting an education'' must be
good; but then anyone who actually shows up at a college will also
match against the concept ``getting an
education,'' whether or not they learn.}

{
 Students who understand math will do well on tests, but if you
require schools to produce good test scores, they'll
spend all their time teaching to the test. A \textit{mental category},
that imperfectly matches your goal, can produce the same kind of
incentive failure \textit{internally.} You want to learn, so you need
an ``education''; and then as long
as you're getting anything that matches against the
category ``education,'' you may not
notice whether you're learning or not. Or
you'll notice, but you won't realize
you've lost sight of your original purpose, because
you're ``getting an
education'' and that's how you
mentally described your goal.}

{
 To categorize is to throw away information. If
you're told that a falling tree makes a
``sound,'' you don't
know what the actual sound is; you haven't actually
heard the tree falling. If a coin lands
``heads,'' you don't
know its radial orientation. A blue egg-shaped thing may be a
``blegg,'' but what if the exact egg
shape varies, or the exact shade of blue? You want to use categories to
throw away irrelevant information, to sift gold from dust, but often
the standard categorization ends up throwing out relevant information
too. And when you end up in that sort of mental trouble, the first and
most obvious solution is to play Taboo.}

{
 For example: ``Play Taboo'' is
itself a leaky generalization. Hasbro's version is not
the rationalist version; they only list five additional banned words on
the card, and that's not nearly enough coverage to
exclude thinking in familiar old words. What rationalists do would
count as playing Taboo---it would match against the
``play Taboo'' concept---but not
everything that counts as playing Taboo works to force original seeing.
If you just think ``play Taboo to force original
seeing,'' you'll start thinking that
anything that counts as playing Taboo must count as original seeing.}

{
 The rationalist version isn't a game, which means
that you can't win by trying to be clever and
stretching the rules. You have to play Taboo with a voluntary handicap:
Stop yourself from using synonyms that aren't on the
card. You also have to stop yourself from inventing a new simple word
or phrase that functions as an equivalent mental handle to the old one.
You are trying to zoom in on your map, not rename the cities;
dereference the pointer, not allocate a new pointer; see the events as
they happen, not rewrite the cliché in a different wording.}

{
 By visualizing the problem in more detail, you can see the lost
purpose: Exactly what do you do when you ``play
Taboo''? What purpose does each and every part
serve?}

{
 If you see your activities and situation originally, you will be
able to originally see your goals as well. If you can look with fresh
eyes, as though for the first time, you will see yourself doing things
that you would never dream of doing if they were not habits.}

{
 Purpose is lost whenever the substance (learning, knowledge,
health) is displaced by the symbol (a degree, a test score, medical
care). To heal a lost purpose, or a lossy categorization, you must do
the reverse:}

{
 Replace the symbol with the substance; replace the signifier with
the signified; replace the property with the membership test; replace
the word with the meaning; replace the label with the concept; replace
the summary with the details; replace the proxy question with the real
question; dereference the pointer; drop into a lower level of
organization; mentally simulate the process instead of naming it; zoom
in on your map.}

{
 The Simple Truth was generated by an exercise of this discipline
to describe ``truth'' on a lower
level of organization, without invoking terms like
``accurate,''
``correct,''
``represent,''
``reflect,''
``semantic,''
``believe,''
``knowledge,''
``map,'' or
``real.'' (And remember that the
goal is not \textit{really} to play Taboo---the word
``true'' appears in the text, but
\textit{not} to define truth. It would get a buzzer in
Hasbro's game, but we're not
\textit{actually} playing that game. Ask yourself whether the document
fulfilled its purpose, not whether it followed the rules.)}

{
 Bayes's Rule itself describes
``evidence'' in pure math, without
using words like ``implies,''
``means,''
``supports,''
``proves,'' or
``justifies.'' Set out to
\textit{define} such philosophical terms, and you'll
just go in circles.}

{
 And then there's the most important word of all to
Taboo. I've often warned that you should be careful not
to overuse it, or even avoid the concept in certain cases. Now you know
the real reason why. It's not a bad subject to think
about. But your true understanding is measured by your ability to
describe what you're doing and why, \textit{without}
using that word or any of its synonyms.}

\myendsectiontext

\mysection{Fallacies of Compression}

{
 ``The map is not the
territory,'' as the saying goes. The only life-size,
atomically detailed, 100\% accurate map of California is California.
But California has important regularities, such as the shape of its
highways, that can be described using vastly less information---not to
mention vastly less \textit{physical material}{}---than it would take
to describe every atom within the state borders. Hence the
\textit{other} saying: ``The map is not the territory,
but you can't fold up the territory and put it in your
glove compartment.'' }

{
 A paper map of California, at a scale of 10 kilometers to 1
centimeter (a million to one), doesn't have room to
show the distinct position of two fallen leaves lying a centimeter
apart on the sidewalk. Even if the map tried to show the leaves, the
leaves would appear as the same point on the map; or rather the map
would need a feature size of 10 nanometers, which is a finer resolution
than most book printers handle, not to mention human eyes.}

{
 Reality is very large---just the part we can see is billions of
lightyears across. But your map of reality is written on a few pounds
of neurons, folded up to fit inside your skull. I don't
mean to be insulting, but your skull is tiny. Comparatively speaking.}

{
 Inevitably, then, certain things that are distinct in reality,
will be compressed into the same point on your map.}

{
 But what this feels like from inside is not that you say,
``Oh, look, I'm compressing two things
into one point on my map.'' What it \textit{feels}
like from inside is that there is just \textit{one} thing, and you are
seeing it.}

{
 A sufficiently young child, or a sufficiently ancient Greek
philosopher, would not know that there were such things as
``acoustic vibrations'' or
``auditory experiences.'' There
would just be a single thing that happened when a tree fell; a single
event called ``sound.''}

{
 To realize that there are \textit{two} distinct events, underlying
\textit{one} point on your map, is an essentially \textit{scientific}
challenge---a big, difficult scientific challenge.}

{
 Sometimes fallacies of compression result from confusing two known
things under the same label---you know about acoustic vibrations, and
you know about auditory processing in brains, but you call them both
``sound'' and so confuse yourself.
But the more dangerous fallacy of compression arises from having
\textit{no idea whatsoever} that two distinct entities even
\textit{exist}. There is just one mental folder in the filing system,
labeled ``sound,'' and everything
thought about ``sound'' drops into
that one folder. It's not that there are two folders
with the same label; there's just a single folder. By
default, the map is compressed; why would the brain create two mental
buckets where one would serve?}

{
 Or think of a mystery novel in which the
detective's critical insight is that one of the
suspects has an identical twin. In the course of the
detective's ordinary work, their job is just to observe
that Carol is wearing red, that she has black hair, that her sandals
are leather---but all these are \textit{facts about} Carol.
It's easy enough to question an individual fact, like
WearsRed(Carol) or BlackHair(Carol). Maybe BlackHair(Carol) is false.
Maybe Carol dyes her hair. Maybe BrownHair(Carol). But it takes a
subtler detective to wonder if the Carol in WearsRed(Carol) and
BlackHair(Carol)---the Carol file into which their observations
drop---should be split into \textit{two} files. Maybe there are two
Carols, so that the Carol who wore red is not the same woman as the
Carol who had black hair.}

{
 Here it is the very act of \textit{creating} two different buckets
that is the stroke of genius insight. 'Tis easier to
question one's facts than one's
ontology.}

{
 The map of reality contained in a human brain, unlike a paper map
of California, can expand dynamically when we write down more detailed
descriptions. But what this feels like from inside is not so much
zooming in on a map, as fissioning an indivisible atom---taking
\textit{one thing} (it felt like one thing) and splitting it into two
or more things.}

{
 Often this manifests in the creation of new words, like
``acoustic vibrations'' and
``auditory experiences'' instead of
just ``sound.'' Something about
creating the new name seems to allocate the new bucket. The detective
is liable to start calling one of their suspects
``Carol-2'' or
``the Other Carol'' almost as soon
as they realize that there are two Carols.}

{
 But expanding the map isn't always as simple as
generating new city names. It is a stroke of scientific insight to
realize that such things as acoustic vibrations, or auditory
experiences, even \textit{exist.}}

{
 The obvious modern-day illustration would be words like
``intelligence'' or
``consciousness.'' Every now and
then one sees a press release claiming that a research study has
``explained consciousness'' because
a team of neurologists investigated a 40Hz electrical rhythm that might
have something to do with cross-modality binding of sensory
information, or because they investigated the reticular activating
system that keeps humans awake. That's an extreme
example, and the usual failures are more subtle, but they are of the
same kind. The part of
``consciousness'' that people find
most interesting is reflectivity, self-awareness, realizing that the
person I see in the mirror is
``me''; that and the hard problem of
subjective experience as distinguished by David Chalmers. We also label
``conscious'' the state of being
awake, rather than asleep, in our daily cycle. But they are all
different concepts going under the same name, and the underlying
phenomena are different scientific puzzles. You can explain being awake
without explaining reflectivity or subjectivity.}

{
 Fallacies of compression also underlie the bait-and-switch
technique in philosophy---you argue about
``consciousness'' under one
definition (like the ability to think about thinking) and then apply
the conclusions to ``consciousness''
under a different definition (like subjectivity). Of course it may be
that the two are the same thing, but if so, genuinely
\textit{understanding} this fact would require \textit{first} a
conceptual split and \textit{then} a genius stroke of reunification.}

{
 Expanding your map is (I say again) a \textit{scientific}
challenge: part of the art of science, the skill of inquiring into the
world. (And of course you cannot solve a scientific challenge by
appealing to dictionaries, nor master a complex skill of inquiry by
saying ``I can define a word any way I
like.'') Where you see a single confusing thing, with
protean and self-contradictory attributes, it is a good guess that your
map is cramming too much into one point---you need to pry it apart and
allocate some new buckets. This is not like \textit{defining} the
single thing you see, but it \textit{does} often follow from figuring
out how to talk about the thing without using a single mental handle.}

{
 So the skill of prying apart the map is linked to the rationalist
version of Taboo, and to the wise use of words; because words often
represent the points on our map, the labels under which we file our
propositions and the buckets into which we drop our information.
Avoiding a single word, or allocating new ones, is often part of the
skill of expanding the map.}

\myendsectiontext

\mysection{Categorizing Has Consequences}

{
 Among the many genetic variations and mutations you carry in your
genome, there are a very few alleles you probably know---including
those determining your blood type: the presence or absence of the A, B,
and + antigens. If you receive a blood transfusion containing an
antigen you don't have, it will trigger an allergic
reaction. It was Karl Landsteiner's discovery of this
fact, and how to test for compatible blood types, that made it possible
to transfuse blood without killing the patient. (1930 Nobel Prize in
Medicine.) Also, if a mother with blood type A (for example) bears a
child with blood type A+, the mother may acquire an allergic reaction
to the + antigen; if she has another child with blood type A+, the
child will be in danger, unless the mother takes an allergic
suppressant during pregnancy. Thus people learn their blood types
before they marry. }

{
 Oh, and \textit{also}: people with blood type A are earnest and
creative, while people with blood type B are wild and cheerful. People
with type O are agreeable and sociable, while people with type AB are
cool and controlled. (You would think that O would be the absence of A
and B, while AB would just be A plus B, but no \ldots) All this,
according to the Japanese blood type theory of personality. It would
seem that blood type plays the role in Japan that astrological signs
play in the West, right down to blood type horoscopes in the daily
newspaper.}

{
 This fad is especially odd because blood types have \textit{never
been} mysterious, not in Japan and not anywhere. We only know blood
types even \textit{exist} thanks to Karl Landsteiner. No mystic witch
doctor, no venerable sorcerer, ever said a word about blood types;
there are no ancient, dusty scrolls to shroud the error in the aura of
antiquity. If the medical profession claimed tomorrow that it had all
been a colossal hoax, we layfolk would not have one scrap of evidence
from our unaided senses to contradict them.}

{
 There's never been a war between blood types.
There's never even been a political conflict between
blood types. The stereotypes must have arisen \textit{strictly} from
the \textit{mere existence} of the labels.}

{
 Now, someone is bound to point out that this is a story of
categorizing humans. Does the same thing happen if you categorize
plants, or rocks, or office furniture? I can't recall
reading about such an experiment, but of course, that
doesn't mean one hasn't been done.
(I'd expect the chief difficulty of doing such an
experiment would be finding a protocol that didn't
mislead the subjects into thinking that, since the label was given you,
it must be significant somehow.) So while I don't mean
to update on imaginary evidence, I would predict a positive result for
the experiment: I would expect them to find that mere labeling had
power over all things, at least in the human imagination.}

{
 You can see this in terms of similarity clusters: once you draw a
boundary around a group, the mind starts trying to harvest similarities
from the group. And unfortunately the human pattern-detectors seem to
operate in such overdrive that we see patterns whether
they're there or not; a weakly negative correlation can
be mistaken for a strong positive one with a bit of selective memory.}

{
 You can see this in terms of neural algorithms: creating a name
for a set of things is like allocating a subnetwork to find patterns in
them.}

{
 You can see this in terms of a compression fallacy: things given
the same name end up dumped into the same mental bucket, blurring them
together into the same point on the map.}

{
 Or you can see this in terms of the boundless human ability to
make stuff up out of thin air and believe it because no one can prove
it's wrong. As soon as you name the category, you can
start making up stuff about it. The named thing doesn't
have to be perceptible; it doesn't have to exist; it
doesn't even have to be coherent.}

{
 And no, it's not just Japan: Here in the West, a
blood-type-based diet book called Eat Right 4 Your Type was a
bestseller.}

{
 Any way you look at it, drawing a boundary in thingspace is not a
neutral act. Maybe a more cleanly designed, more purely Bayesian AI
could ponder an arbitrary class and not be influenced by it. But you, a
human, do not have that option. Categories are not static things in the
context of a human brain; as soon as you actually think of them, they
exert force on your mind. One more reason not to believe you can define
a word any way you like.}

\myendsectiontext

\mysection{Sneaking in Connotations}

{
 In the previous essay, we saw that in Japan, blood types have
taken the place of astrology---if your blood type is AB, for example,
you're supposed to be ``cool and
controlled.'' }

{
 So suppose we decided to invent a new word,
``wiggin,'' and \textit{defined}
this word to mean people with green eyes and black hair---}

\begin{quotation}
{
 A green-eyed man with black hair walked into a restaurant.}

{
 ``Ha,'' said Danny, watching
from a nearby table, ``did you see that? A wiggin just
walked into the room. Bloody wiggins. Commit all sorts of crimes, they
do.''}

{
 His sister Erda sighed. ``You
haven't \textit{seen} him commit any crimes, have you,
Danny?''}

{
 ``Don't need
to,'' Danny said, producing a dictionary.
``See, it says right here in the Oxford English
Dictionary. `Wiggin. (1) A person with green eyes and
black hair.' He's got green eyes and
black hair, he's a wiggin. You're not
going to argue with the Oxford English Dictionary, are you? \textit{By
definition,} a green-eyed black-haired person is a
wiggin.''}

{
 ``But you called him a
wiggin,'' said Erda.
``That's a nasty thing to say about
someone you don't even know. You've got
no evidence that he puts too much ketchup on his burgers, or that as a
kid he used his slingshot to launch baby
squirrels.''}

{
 ``But he \textit{is} a
wiggin,'' Danny said patiently.
``He's got green eyes and black hair,
right? Just you watch, as soon as his burger arrives,
he's reaching for the ketchup.''}
\end{quotation}

{
 The human mind passes from observed characteristics to inferred
characteristics via the medium of words. In ``All
humans are mortal, Socrates is a human, therefore Socrates is
mortal,'' the observed characteristics are
Socrates's clothes, speech, tool use, and generally
human shape; the categorization is
``human''; the inferred
characteristic is poisonability by hemlock.}

{
 Of course there's no hard distinction between
``observed characteristics'' and
``inferred characteristics.'' If you
hear someone speak, they're probably shaped like a
human, all else being equal. If you see a human figure in the shadows,
then \textit{ceteris paribus} it can probably speak.}

{
 And yet some properties do tend to be more inferred than observed.
You're more likely to decide that someone is human, and
will therefore burn if exposed to open flame, than carry through the
inference the other way around.}

{
 If you look in a dictionary for the definition of
``human,'' you're
more likely to find characteristics like
``intelligence'' and
``featherless
biped''---characteristics that are useful for quickly
eyeballing what is and isn't a human---rather than the
ten thousand connotations, from vulnerability to hemlock, to
overconfidence, that we can infer from someone's being
human. Why? Perhaps dictionaries are intended to let you match up
labels to similarity groups, and so are designed to quickly isolate
clusters in thingspace. Or perhaps the big, distinguishing
characteristics are the most salient, and therefore first to pop into a
dictionary editor's mind. (I'm not sure
how aware dictionary editors are of what they \textit{really} do.)}

{
 But the upshot is that when Danny pulls out his OED to look up
``wiggin,'' he sees listed only the
first-glance characteristics that distinguish a wiggin: Green eyes and
black hair. The OED doesn't list the many minor
\textit{connotations} that have come to attach to this term, such as
criminal proclivities, culinary peculiarities, and some unfortunate
childhood activities.}

{
 How did those connotations get there in the first place? Maybe
there was once a famous wiggin with those properties. Or maybe someone
made stuff up at random, and wrote a series of bestselling books about
it (\textit{The Wiggin}, \textit{Talking to Wiggins}, \textit{Raising
Your Little Wiggin}, \textit{Wiggins in the Bedroom}). Maybe even the
wiggins believe it now, and act accordingly. As soon as you call some
people ``wiggins,'' the word will
begin acquiring connotations.}

{
 But remember the Parable of Hemlock: If we go by the logical class
definitions, we can never class Socrates as a
``human'' until after we observe him
to be mortal. Whenever someone pulls a dictionary,
they're generally trying to sneak in a
\textit{connotation}, not the actual definition written down in the
dictionary.}

{
 After all, if the \textit{only} meaning of the word
``wiggin'' is
``green-eyed black-haired person,''
then why not just call those people ``green-eyed
black-haired people''? And if you're
wondering whether someone is a ketchup-reacher, why not ask directly,
``Is he a ketchup-reacher?'' rather
than ``Is he a wiggin?'' (Note
substitution of substance for symbol.)}

{
 Oh, but arguing the \textit{real} question would require
\textit{work.} You'd have to actually watch the wiggin
to see if he reached for the ketchup. Or maybe see if you can find
statistics on how many green-eyed black-haired people actually like
ketchup. At any rate, you wouldn't be able to do it
sitting in your living room with your eyes closed. And people are lazy.
They'd rather argue ``by
definition,'' especially since they think
``you can define a word any way you
like.''}

{
 But of course the \textit{real} reason they care whether someone
is a ``wiggin'' is a connotation---a
feeling that comes along with the word---that isn't in
the definition they \textit{claim} to use.}

{
 Imagine Danny saying, ``Look,
he's got green eyes and black hair.
He's a wiggin! It says so right there in the
dictionary!---\textit{therefore,} he's got black hair.
Argue with that, if you can!''}

{
 Doesn't have much of a triumphant ring to it, does
it? If the real point of the argument actually \textit{was} contained
in the dictionary definition---if the argument genuinely \textit{was}
logically valid---then the argument would \textit{feel} empty; it would
either say nothing new, or beg the question.}

{
 It's only the attempt to smuggle in connotations
\textit{not} explicitly listed in the definition, that makes anyone
feel they can \textit{score a point} that way.}

\myendsectiontext

\mysection{Arguing ``By Definition''}

{
 ``This plucked chicken has two legs and no
feathers---therefore, \textit{by definition,} it is a
human!'' }

{
 When people argue definitions, they usually start with some
visible, known, or at least widely believed set of characteristics;
then pull out a dictionary, and point out that these characteristics
fit the dictionary definition; and so conclude,
``Therefore, \textit{by definition,} atheism is a
religion!''}

{
 But visible, known, widely believed characteristics are rarely the
real point of a dispute. Just the fact that someone thinks
Socrates's two legs are evident enough to make a good
premise for the argument, ``Therefore, \textit{by
definition,} Socrates is human!'' indicates that
bipedalism probably isn't \textit{really}
what's at stake---or the listener would reply,
``Whaddaya mean Socrates is bipedal?
That's what we're arguing about in the
first place!''}

{
 Now there is an important sense in which we can legitimately move
from evident characteristics to not-so-evident ones. You can,
legitimately, see that Socrates is human-shaped, and predict his
vulnerability to hemlock. But this \textit{probabilistic} inference
does not rely on dictionary definitions or common usage; it relies on
the universe containing empirical clusters of similar things.}

{
 This cluster structure is not going to change depending on how you
define your words. Even if you look up the dictionary definition of
``human'' and it says
``all featherless bipeds except
Socrates,'' that isn't going to
change the \textit{actual} degree to which Socrates is similar to the
rest of us featherless bipeds.}

{
 When you are arguing \textit{correctly} from cluster structure,
you'll say something like, ``Socrates
has two arms, two feet, a nose and tongue, speaks fluent Greek, uses
tools, and in every aspect I've been able to observe
him, seems to have every major and minor property that characterizes
\textit{Homo sapiens}; so I'm going to guess that he
has human DNA, human biochemistry, and is vulnerable to hemlock just
like all other \textit{Homo sapiens} in whom hemlock has been
clinically tested for lethality.''}

{
 And suppose I reply, ``But I saw Socrates out in
the fields with some herbologists; I think they were trying to prepare
an antidote. Therefore I \textit{don't} expect Socrates
to keel over after he drinks the hemlock---he will be an exception to
the general behavior of objects in his cluster: they did not take an
antidote, and he did.''}

{
 Now there's not much point in arguing over whether
Socrates is ``human'' or not. The
conversation has to move to a more detailed level, poke around
\textit{inside} the details that make up the
``human'' category---talk about
human biochemistry, and specifically, the neurotoxic effects of
coniine.}

{
 If you go on insisting, ``But Socrates is a human
and humans, \textit{by definition,} are mortal!''
then what you're really trying to do is blur out
everything you know about Socrates \textit{except} the fact of his
humanity---insist that the only correct prediction is the one you would
make if you knew nothing about Socrates \textit{except} that he was
human.}

{
 Which is like insisting that a coin is 50\% likely to be showing
heads or tails, because it is a ``fair
coin,'' after you've \textit{actually
looked at the coin} and it's showing heads.
It's like insisting that Frodo has ten fingers, because
most hobbits have ten fingers, after you've
\textit{already looked at his hands} and seen nine fingers. Naturally
this is illegal under Bayesian probability theory: You
can't just refuse to condition on new evidence.}

{
 And you can't just keep one categorization and
make estimates based on that, while deliberately throwing out
everything else you know.}

{
 Not every piece of new evidence makes a significant difference, of
course. If I see that Socrates has nine fingers, this
isn't going to noticeably change my estimate of his
vulnerability to hemlock, because I'll expect that the
way Socrates lost his finger didn't change the rest of
his biochemistry. And this is true, \textit{whether or not} the
dictionary's definition says that human beings have ten
fingers. The legal inference is based on the cluster structure of the
environment, and the causal structure of biology; \textit{not} what the
dictionary editor writes down, nor even ``common
usage.''}

{
 Now ordinarily, when you're doing this
\textit{right}{}---in a \textit{legitimate} way---you just say,
``The coniine alkaloid found in hemlock produces
muscular paralysis in humans, resulting in death by
asphyxiation.'' Or more simply,
``Humans are vulnerable to
hemlock.'' That's how
it's usually said in a \textit{legitimate} argument.}

{
 When would someone feel the need to \textit{strengthen} the
argument with the emphatic phrase ``by
definition''? (E.g. ``Humans are
vulnerable to hemlock \textit{by definition!}'') Why,
when the inferred characteristic has been called into doubt---Socrates
has been seen consulting herbologists---and so the speaker feels the
need to tighten the vise of logic.}

{
 So when you see ``by
definition'' used like this, it usually means:
``Forget what you've heard about
Socrates consulting herbologists---humans, \textit{by definition,} are
mortal!''}

{
 People feel the need to squeeze the argument onto a single course
by saying ``Any P, by definition, has property
Q!,'' on exactly those occasions when they see, and
prefer to dismiss out of hand, \textit{additional arguments} that call
into doubt the default inference based on clustering.}

{
 So too with the argument ``X, \textit{by
definition,} is a Y!'' E.g.,
``Atheists believe that God doesn't
exist; therefore atheists have beliefs about God, because a negative
belief is still a belief; therefore atheism asserts answers to
theological questions; therefore atheism is, \textit{by definition,} a
religion.''}

{
 You wouldn't feel the need to say,
``Hinduism, \textit{by definition,} is a
religion!'' because, well, of course Hinduism is a
religion. It's not just a religion
``by definition,''
it's, like, an \textit{actual} religion.}

{
 Atheism does not resemble the central members of the
``religion'' cluster, so if it
wasn't for the fact that atheism is a religion
\textit{by definition,} you might go around thinking that atheism
\textit{wasn't} a religion. That's why
you've got to crush all opposition by pointing out that
``Atheism is a religion'' is true
\textit{by definition,} because it isn't true any other
way.}

{
 Which is to say: People insist that ``X,
\textit{by definition,} is a Y!'' on those occasions
when they're trying to sneak in a connotation of Y that
isn't directly in the definition, and X
doesn't look all that much like other members of the Y
cluster.}

{
 Over the last thirteen years I've been keeping
track of how often this phrase is used correctly versus
incorrectly---though not with literal statistics, I fear. But
eyeballing suggests that using the phrase \textit{by definition,}
anywhere outside of math, is among the most alarming signals of flawed
argument I've ever found. It's right up
there with ``Hitler,''
``God,''
``absolutely certain,'' and
``can't prove
that.''}

{
 This heuristic of failure is not perfect---the first time I ever
spotted a correct usage outside of math, it was by Richard Feynman; and
since then I've spotted more. But
you're probably better off just deleting the phrase
``by definition'' from your
vocabulary---and \textit{always} on any occasion where you might be
tempted to say it in italics or followed with an exclamation mark.
That's a bad idea \textit{by definition!}}

\myendsectiontext

\mysection{Where to Draw the Boundary?}

{
 The one comes to you and says:}

\begin{quote}
{
 Long have I pondered the meaning of the word
``Art,'' and at last
I've found what seems to me a satisfactory definition:
``Art is that which is designed for the purpose of
creating a reaction in an audience.''}
\end{quote}

{
 \textit{Just because there's a word
``art'' doesn't mean
that it }\textbf{\textit{has a meaning}}\textit{, floating out there in
the void, which you can }\textbf{\textit{discover}}\textit{ by finding
the right definition.}}

{
 It feels that way, but it is not so.}

{
 Wondering how to \textit{define a word} means
you're looking at the problem the wrong way---searching
for the mysterious essence of what is, in fact, a communication
signal.}

{
 Now, there \textit{is} a real challenge which a rationalist may
legitimately attack, but the challenge is not to find a satisfactory
definition of a word. The real challenge can be played as a
single-player game, without speaking aloud. The challenge is figuring
out which things are similar to each other---which things are clustered
together---and sometimes, which things have a common cause.}

{
 If you define
``eluctromugnetism'' to include
lightning, include compasses, exclude light, and include
Mesmer's ``animal
magnetism'' (what we now call hypnosis), then you
will have some trouble asking ``How does
eluctromugnetism work?'' You have lumped together
things which do not belong together, and excluded others that would be
needed to complete a set. (This example is historically plausible;
Mesmer came before Faraday.)}

{
 We could say that eluctromugnetism is a \textit{wrong word}, a
boundary in thingspace that loops around and swerves through the
clusters, a cut that fails to carve reality along its natural joints.}

{
 Figuring where to cut reality in order to carve along the
joints---\textit{this} is the problem worthy of a rationalist. It is
what people \textit{should} be trying to do, when they set out in
search of the floating essence of a word.}

{
 And make no mistake: it is a \textit{scientific} challenge to
realize that you need a single word to describe breathing and fire. So
do not think to consult the dictionary editors, for that is not their
job.}

{
 What is ``art''? But there is
no essence of the word, floating in the void.}

{
 Perhaps you come to me with a long list of the things that you
call ``art'' and
``not art'':}

\begin{quote}
{
 The \textit{Little Fugue in G Minor}: Art.}

{
 A punch in the nose: Not art.}

{
 Escher's \textit{Relativity}: Art.}

{
 A flower: Not art.}

{
 The Python programming language: Art.}

{
 A cross floating in urine: Not art.}

{
 Jack Vance's \textit{Tschai} novels: Art.}

{
  Modern Art: Not art.}
\end{quote}

{
 And you say to me: ``It feels intuitive to me to
draw this boundary, but I don't know why---can you find
me an intension that matches this extension? Can you give me a
\textit{simple} description of this boundary?''}

{
 So I reply: ``I think it has to do with
admiration of craftsmanship: work going in and wonder coming out. What
the included items have in common is the similar aesthetic emotions
that they inspire, and the deliberate human effort that went into them
with the intent of producing such an emotion.''}

{
 Is this helpful, or is it just cheating at Taboo? I would argue
that the list of which human emotions are or are not \textit{aesthetic}
is far more compact than the list of everything that is or
isn't art. You might be able to see those emotions
lighting up an fMRI scan---I say this by way of emphasizing that
emotions are not ethereal.}

{
 But of course my definition of art is not the real point. The real
point is that you could well dispute either the intension or the
extension of my definition.}

{
 You could say, ``Aesthetic emotion is
\textit{not} what these things have in common; what they have in common
is an intent to inspire \textit{any} complex emotion for the sake of
inspiring it.'' That would be disputing my intension,
my attempt to draw a curve through the data points. You would say,
``Your equation may roughly fit those points, but it
is not the true generating distribution.''}

{
 Or you could dispute my extension by saying,
``Some of these things do belong together---I can see
what you're getting at---but the Python language
shouldn't be on the list, and Modern Art should
be.'' (This would mark you as a philistine, but you
could argue it.) Here, the presumption is that there is indeed an
underlying curve that generates this apparent list of similar and
dissimilar things---that there is a rhyme and reason, \textit{even
though you haven't said yet where it comes
from}{}---but I have unwittingly lost the rhythm and included some data
points from a different generator.}

{
 Long before you \textit{know} what it is that electricity and
magnetism have in common, you might still suspect---based on surface
appearances---that ``animal
magnetism'' does not belong on the list.}

{
 Once upon a time it was thought that the word
``fish'' included dolphins. Now you
could play the oh-so-clever arguer, and say, ``The
list: \{Salmon, guppies, sharks,
dolphins, trout\} is just a
list---you can't say that a list is \textit{wrong.} I
can prove in set theory that this list exists. So my definition of
\textit{fish}, which is simply this extensional list, cannot possibly
be `wrong' as you
claim.''}

{
 Or you could stop playing games and admit that dolphins
don't belong on the fish list.}

{
 You come up with a list of things that \textit{feel} similar, and
take a guess at why this is so. But when you finally discover what they
\textit{really} have in common, it may turn out that your guess was
wrong. It may even turn out that your list was wrong.}

{
 You cannot hide behind a comforting shield of
correct-by-definition. Both extensional definitions and intensional
definitions can be wrong, can fail to carve reality at the joints.}

{
 Categorizing is a guessing endeavor, in which you can make
mistakes; so it's wise to be able to admit, from a
theoretical standpoint, that your definition-guesses can be
``mistaken.''}

\myendsectiontext

\mysection{Entropy, and Short Codes}

{
 (If you aren't familiar with Bayesian inference,
this may be a good time to read An Intuitive Explanation of
Bayes's Theorem.) }

{
 Suppose you have a system $X$ that's equally likely
to be in any of 8 possible states:}

\begin{equation*}
\{X_{1},X_{2}, X_{3}, X_{4}, X_{5}, X_{6}, X_{7}, X_{8} \}.
\end{equation*}

{
 There's an extraordinarily ubiquitous
quantity---in physics, mathematics, and even biology---called
\textit{entropy}; and the entropy of $X$ is 3 bits. This means that, on
average, we'll have to ask 3 yes-or-no questions to
find out X's value. For example, someone could tell us
X's value using this code:}

\begin{equation*}
  \begin{array}{llll}
 X_{1} : 001 &  X_{2} : 010 &  X_{3} : 011 &  X_{4} : 100 \\
 X_{5} : 101 &  X_{6} : 110 &  X_{7} : 111 &  X_{8} : 000.
 \end{array}
\end{equation*}


\bigskip

{
 So if I asked ``Is the first symbol
1?'' and heard
``yes,'' then asked
``Is the second symbol 1?'' and
heard ``no,'' then asked
``Is the third symbol 1?'' and heard
``no,'' I would know that $X$ was in
state 4.}

{
 Now suppose that the system $Y$ has four possible states with the
following probabilities:}

\begin{equation*}
  \begin{array}{ll}
 Y_{1} : 1/2 (50\%) &  Y_{2} : 1/4 (25\%)\\
 Y_{3} : 1/8 (12.5\%) &  Y_{4} : 1/8 (12.5\%).
  \end{array}
\end{equation*}


{
 Then the entropy of $Y$ would be 1.75 bits, meaning that we can find
out its value by asking 1.75 yes-or-no questions.}

{
 What does it mean to talk about asking one and three-fourths of a
question? Imagine that we designate the states of $Y$ using the following
code:}

\begin{equation*}
  \begin{array}{llll}
 Y_{1} : 1 &  Y_{2} : 01 &  Y_{3} : 001 &  Y_{4} : 000.
  \end{array}
\end{equation*}


{
 First you ask, ``Is the first symbol
1?'' If the answer is
``yes,'' you're
done: $Y$ is in state 1. This happens half the time, so 50\% of the time,
it takes 1 yes-or-no question to find out $Y$'s state.}

{
 Suppose that instead the answer is
``No.'' Then you ask,
``Is the second symbol 1?'' If the
answer is ``yes,''
you're done: $Y$ is in state 2. The system $Y$ is in state
2 with probability 1/4, and each time $Y$ is in state 2 we discover this
fact using two yes-or-no questions, so 25\% of the time it takes 2
questions to discover $Y$'s state.}

{
 If the answer is ``No'' twice
in a row, you ask ``Is the third symbol
1?'' If ``yes,''
you're done and $Y$ is in state 3; if
``no,'' you're done
and $Y$ is in state 4. The 1/8 of the time that $Y$ is in state 3, it takes
three questions; and the 1/8 of the time that $Y$ is in state 4, it takes
three questions.}

\begin{align*}
 (&1/2 \times 1) + (1/4 \time 2) + (1/8 \times 3) + (1/8 \times 3) \\
  &= 0.5 + 0.5 + 0.375 + 0.375 \\
 &= 1.75.
\end{align*}

{
 The general formula for the entropy $H(S)$ of a system $S$ is the sum,
over all $S_{i}$, of
$-P(S_{i})\log_{2}(P(S_{i}))$.}

{
 For example, the log (base 2) of 1/8 is -3. So $-(1/8
\times -3) = 0.375$ is the contribution of state
$S_{4}$ to the total entropy: 1/8 of the time, we have to
ask 3 questions.}

{
 You can't always devise a perfect code for a
system, but if you have to tell someone the state of arbitrarily many
copies of S in a single message, you can get arbitrarily close to a
perfect code. (Google ``arithmetic
coding'' for a simple method.)}

{
 Now, you might ask: ``Why not use the code 10 for
$Y_{4}$, instead of 000? Wouldn't that let
us transmit messages more quickly?''}

{
 But if you use the code 10 for $Y_{4}$, then when
someone answers ``Yes'' to the
question ``Is the first symbol 1?,''
you won't know yet whether the system state is
$Y_{1}$ (1) or $Y_{4}$ (10). In fact, if you
change the code this way, the whole system falls apart---because if you
hear ``1001,'' you
don't know if it means
``$Y_{4}$, followed by
$Y_{2}$'' or
``$Y_{1}$, followed by
$Y_{3}$.''}

{
 The moral is that \textit{short words are a conserved resource}.}

{
 The key to creating a good code---a code that transmits messages
as compactly as possible---is to reserve short words for things that
you'll need to say frequently, and use longer words for
things that you won't need to say as often.}

{
 When you take this art to its limit, the length of the message you
need to describe something corresponds exactly or almost exactly to its
probability. This is the Minimum Description Length or Minimum Message
Length formalization of Occam's Razor.}

{
 And so even the \textit{labels} that we use for words are not
quite arbitrary. The sounds that we attach to our concepts can be
better or worse, wiser or more foolish. Even apart from considerations
of common usage!}

{
 I say all this, because the idea that ``You can $X$
any way you like'' is a huge obstacle to learning how
to $X$ wisely. ``It's a free country; I
have a right to my own opinion'' obstructs the art of
finding truth. ``I can define a word any way I
like'' obstructs the art of carving reality at its
joints. And even the sensible-sounding ``The labels we
attach to words are arbitrary'' obstructs awareness
of compactness. Prosody too, for that matter---Tolkien once observed
what a beautiful sound the phrase ``cellar
door'' makes; that is the kind of awareness it takes
to use language like Tolkien.}

{
 The length of words also plays a nontrivial role in the cognitive
science of language:}

{
 Consider the phrases
``recliner,''
``chair,'' and
``furniture.'' Recliner is a more
specific category than chair; furniture is a more general category than
chair. But the vast majority of chairs have a common use---you use the
same sort of motor actions to sit down in them, and you sit down in
them for the same sort of purpose (to take your weight off your feet
while you eat, or read, or type, or rest). Recliners do not depart from
this theme. ``Furniture,'' on the
other hand, includes things like beds and tables which have different
uses, and call up different motor functions, from chairs.}

{
 In the terminology of cognitive psychology,
``chair'' is a \textit{basic-level
category.}}

{
 People have a tendency to talk, and presumably think, at the basic
level of categorization---to draw the boundary around
``chairs,'' rather than around the
more specific category ``recliner,''
or the more general category
``furniture.'' People are more
likely to say ``You can sit in that
chair'' than ``You can sit in that
recliner'' or ``You can sit in that
furniture.''}

{
 And it is no coincidence that the word for
``chair'' contains fewer syllables
than either ``recliner'' or
``furniture.'' Basic-level
categories, in general, tend to have short names; and nouns with short
names tend to refer to basic-level categories. Not a perfect rule, of
course, but a definite tendency. Frequent use goes along with short
words; short words go along with frequent use.}

{
 Or as Douglas Hofstadter put it, there's a reason
why the English language uses
``the'' to mean
``the'' and
``antidisestablishmentarianism'' to
mean
``antidisestablishmentarianism''
instead of antidisestablishmentarianism other way around.}

\myendsectiontext

\mysection{Mutual Information, and Density in Thingspace}

{
 Suppose you have a system $X$ that can be in any of 8 states, which
are all equally probable (relative to your current state of knowledge),
and a system $Y$ that can be in any of 4 states, all equally probable. }

{
 The entropy of $X$, as defined in the previous essay, is 3 bits;
we'll need to ask 3 yes-or-no questions to find out
$X$'s exact state. The entropy of $Y$ is 2 bits; we have to
ask 2 yes-or-no questions to find out $Y$'s exact state.
This may seem obvious since $2^3 = 8$ and
$2^2 = 4$, so 3 questions can distinguish 8
possibilities and 2 questions can distinguish 4 possibilities; but
remember that if the possibilities were not all equally likely, we
could use a more clever code to discover $Y$'s state
using e.g. 1.75 questions on average. In this case, though,
$X$'s \textit{probability mass} is \textit{evenly
distributed} over all its possible states, and likewise $Y$, so we
can't use any clever codes.}

{
 What is the entropy of the combined system $(X,Y)$?}

{
 You might be tempted to answer, ``It takes 3
questions to find out $X$, and then 2 questions to find out $Y$, so it
takes 5 questions total to find out the state of $X$ and
$Y$.''}

{
 But what if the two variables are entangled, so that learning the
state of $Y$ tells us something about the state of $X$?}

{
 In particular, let's suppose that $X$ and $Y$ are
either both odd or both even.}

{
 Now if we receive a 3-bit message (ask 3 questions) and learn that
$X$ is in state $X_{5}$, we know that $Y$ is in state
$Y_{1}$ or state $Y_{3}$, but not state
$Y_{2}$ or state $Y_{4}$. So the single
additional question ``Is $Y$ in state
$Y_{3}$?,'' answered
``No,'' tells us the entire state of
$(X,Y): X = X_{5},Y = Y_{1}$. And we learned
this with a total of 4 questions.}

{
 Conversely, if we learn that $Y$ is in state $Y_{4}$
using two questions, it will take us only an additional two questions
to learn whether $X$ is in state $X_{2}$, $X_{4}$,
$X_{6}$, or $X_{8}$. Again, four questions to
learn the state of the joint system.}

{
 The \textit{mutual information} of two variables is defined as the
difference between the entropy of the joint system and the entropy of
the independent systems: $I(X;Y) = H(X) + H(Y) - H(X,Y)$.}

{
 Here there is one bit of mutual information between the two
systems: Learning $X$ tells us one bit of information about $Y$ (cuts down
the space of possibilities from 4 possibilities to 2, a factor-of-2
decrease in the volume) and learning $Y$ tells us one bit of information
about $X$ (cuts down the possibility space from 8 possibilities to 4).}

{
 What about when probability mass is not evenly distributed? Last
essay, for example, we discussed the case in which $Y$ had the
probabilities 1/2, 1/4, 1/8, 1/8 for its four states. Let us take this
to be our probability distribution over $Y$, considered
independently---if we saw $Y$, without seeing anything else, this is what
we'd expect to see. And suppose the variable $Z$ has two
states, $Z_{1}$ and $Z_{2}$, with probabilities
3/8 and 5/8 respectively.}

{
 Then if and only if the joint distribution of $Y$ and $Z$ is as
follows, there is zero mutual information between $Y$ and $Z$:}

\whencolumns{
\begin{equation*}
  \begin{array}{llll}
 Z_{1}Y_{1} : 3/16 & Z_{1}Y_{2} : 3/32 & Z_{1}Y_{3} : 3/64 & Z_{1}Y_{4} : 3/64 \\
 Z_{2}Y_{1} : 5/16 & Z_{2}Y_{2} : 5/32 & Z_{2}Y_{3} : 5/64 & Z_{2}Y_{4} : 5/64.
  \end{array}
\end{equation*}
}{
\begin{equation*}
  \begin{array}{llll}
 Z_{1}Y_{1}:3/16 & Z_{1}Y_{2}:3/32 \\ Z_{1}Y_{3}:3/64 & Z_{1}Y_{4}:3/64 \\
 Z_{2}Y_{1}:5/16 & Z_{2}Y_{2}:5/32 \\ Z_{2}Y_{3}:5/64 & Z_{2}Y_{4}:5/64.
  \end{array}
\end{equation*}
}

{
 This distribution obeys the law}

\begin{equation*}
 P(Y,Z) = P(Y)P(Z).
\end{equation*}


{
 For example, $P(Z_{1}Y_{2}) =
P(Z_{1})P(Y_{2}) = 3/8 \times 1/4 =
3/32$. }

{
 And observe that we can recover the marginal (independent)
probabilities of Y and Z just by looking at the joint distribution:}

\begin{align*}
 P(Y_{1}) &= \whencolumns{\text{total probability of all the different
ways } Y_{1} \text{ can happen}}{\text{total probability of all the different
}\\ & \text{ways } Y_{1} \text{ can happen}} \\
 &= P (Z_{1}Y_{1}) + P(Z_{2}Y_{1}) \\
 &= 3/16 + 5/16 \\
 &= 1/2.
\end{align*}

{
 So, just by inspecting the joint distribution, we can determine
whether the marginal variables $Y$ and $Z$ are independent; that is,
whether the joint distribution factors into the product of the marginal
distributions; whether, for all $Y$ and $Z$, we have $P(Y,Z) = P(Y)P(Z)$.}

{
 This last is significant because, by Bayes's
Rule,}

\begin{align*}
 P(Z_{j}Y_{i}) &= P(Y_{i})P(Z_{j}) \\
 P(Z_{j}Y_{i})/P(Z_{j}) &= P(Y_{i}) \\
 P(Y_{i}|Z_{j}) &= P(Y_{i}).
\end{align*}

{
 In English: ``After you learn $Z_{j}$,
your belief about $Y_{i}$ is just what it was
before.''}

{
 So when the distribution factorizes---when $P(Y,Z) =
P(Y)P(Z)$---this is \textit{equivalent} to ``Learning
about $Y$ never tells us anything about $Z$ or vice
versa.''}

{
 From which you might suspect, correctly, that there is no mutual
information between $Y$ and $Z$. Where there is no mutual information,
there is no Bayesian evidence, and vice versa.}

{
 Suppose that in the distribution $(Y,Z)$ above, we treated each
possible combination of $Y$ and $Z$ as a separate event---so that the
distribution $(Y,Z)$ would have a total of 8 possibilities, with the
probabilities shown---and then we calculated the entropy of the
distribution $(Y,Z)$ the same way we would calculate the entropy of any
distribution:}

\whencolumns{
\begin{align*}
&P(Z_{1}Y_{1})\log_{2}(P(Z_{1}Y_{1}))+P(Z_{1}Y_{2})\log_{2}(P(Z_{1}Y_{2}))+\\
&P(Z_{1}Y_{3})\log_{2}(P(Z_{1}Y_{3}))+ \ldots + P(Z_{2}Y_{4})\log_{2}(P(Z_{2}Y_{4}))\\
  =&(3/16)\log_{2}(3/16) +(3/32)\log_{2}(3/32) + \\
  &(3/64)\log_{2}(3/64) + \ldots + (5/64)\log_{2}(5/64).
\end{align*}
}{
\begin{align*}
&P(Z_{1}Y_{1})\log_{2}(P(Z_{1}Y_{1}))+P(Z_{1}Y_{2})\log_{2}(P(Z_{1}Y_{2}))+\\
  &P(Z_{1}Y_{3})\log_{2}(P(Z_{1}Y_{3}))+ \ldots + \\
  &P(Z_{2}Y_{4})\log_{2}(P(Z_{2}Y_{4}))\\
  =&(3/16)\log_{2}(3/16) +(3/32)\log_{2}(3/32) + \\
  &(3/64)\log_{2}(3/64) + \ldots + (5/64)\log_{2}(5/64).
\end{align*}
}

{
 You would end up with the same total you would get if you
separately calculated the entropy of $Y$ plus the entropy of $Z$. There is
no mutual information between the two variables, so our uncertainty
about the joint system is not any less than our uncertainty about the
two systems considered separately. (I am not showing the calculations,
but you are welcome to do them; and I am not showing the proof that
this is true in general, but you are welcome to Google on
``Shannon entropy'' and
``mutual information.'')}

{
 What if the joint distribution doesn't factorize?
For example:}

\whencolumns{
\begin{equation*}
  \begin{array}{llll}
 Z_{1}Y_{1} : 12/64 & Z_{1}Y_{2} : 8/64 & Z_{1}Y_{3} : 1/64 & Z_{1}Y_{4} : 3/64\\
 Z_{2}Y_{1} : 20/64 & Z_{2}Y_{2} : 8/64 & Z_{2}Y_{3} : 7/64 & Z_{2}Y_{4} : 5/64.
  \end{array}
\end{equation*}
}{
\begin{equation*}
  \begin{array}{llll}
 Z_{1}Y_{1} : 12/64 & Z_{1}Y_{2} : 8/64 \\ Z_{1}Y_{3} : 1/64 & Z_{1}Y_{4} : 3/64\\
 Z_{2}Y_{1} : 20/64 & Z_{2}Y_{2} : 8/64 \\ Z_{2}Y_{3} : 7/64 & Z_{2}Y_{4} : 5/64.
  \end{array}
\end{equation*}
}

{
 If you add up the joint probabilities to get marginal
probabilities, you should find that $P(Y_{1}) = 1/2$,
$P(Z_{1}) = 3/8$, and so on---the marginal probabilities
are the same as before.}

{
 But the joint probabilities do not always equal the product of the
marginal probabilities. For example, the probability
$P(Z_{1}Y_{2})$ equals 8/64, where
$P(Z_{1})P(Y_{2})$ would equal $3/8
\times 1/4 = 6/64$. That is, the probability of running into
$Z_{1}Y_{2}$ together is greater than
you'd expect based on the probabilities of running into
$Z_{1}$ or $Y_{2}$ separately.}

{
 Which in turn implies:}

\begin{align*}
 P(Z_{1}Y_{2}) &> P(Z_{1})P(Y_{2}) \\
 P(Z_{1}Y_{2})/P(Y_{2}) &> P(Z_{1}) \\
 P(Z_{1}|Y_{2}) &> P(Z_{1}).
\end{align*}

{
 Since there's an ``unusually
high'' probability for
$P(Z_{1}Y_{2})$---defined as a probability
higher than the marginal probabilities would indicate by default---it
follows that observing $Y_{2}$ is evidence that increases
the probability of $Z_{1}$. And by a symmetrical argument,
observing $Z_{1}$ must favor $Y_{2}$.}

{
 As there are at least some values of $Y$ that tell us about $Z$ (and
vice versa) there must be mutual information between the two variables;
and so you will find---I am confident, though I haven't
actually checked---that calculating the entropy of $(Y,Z)$ yields less
total uncertainty than the sum of the independent entropies of $Y$ and $Z$.
That is, $H(Y,Z) = H(Y) + H(Z) - I(Y;Z)$, with all quantities necessarily
positive.}

{
 (I digress here to remark that the symmetry of the expression for
the mutual information shows that $Y$ \textit{must} tell us as much about
$Z$, on average, as $Z$ tells us about $Y$. I leave it as an exercise to the
reader to reconcile this with anything they were taught in logic class
about how, if all ravens are black, being allowed to reason Raven($x$)
$\Rightarrow $ Black($x$) doesn't mean
you're allowed to reason Black($x$) $\Rightarrow $
Raven($x$). How different seem the symmetrical probability flows of the
Bayesian, from the sharp lurches of logic---even though the latter is
just a degenerate case of the former.)}

{
 ``But,'' you ask,
``what has all this to do with the proper use of
words?''}

{
 In Empty Labels and then Replace the Symbol with the Substance, we
saw the technique of replacing a word with its definition---the example
being given:}

\begin{verse}
 All [mortal, {\textlnot}feathers, bipedal] are mortal.\\
 Socrates is a [mortal, {\textlnot}feathers, bipedal].\\
 Therefore, Socrates is mortal.\\
\end{verse}

{
 Why, then, would you even want to have a word for
``human''? Why not just say
``Socrates is a mortal featherless
biped''?}

{
 Because it's helpful to have shorter words for
things that you encounter often. If your code for describing single
properties is already efficient, then there will not be an advantage to
having a special word for a conjunction---like
``human'' for
``mortal featherless
biped''---unless things that are mortal \textit{and}
featherless \textit{and} bipedal, are found \textit{more often} than
the marginal probabilities would lead you to expect.}

{
 In efficient codes, word length corresponds to probability---so
the code for $Z_{1}Y_{2}$ will be just as long
as the code for $Z_{1}$ plus the code for
$Y_{2}$, unless $P(Z_{1}Y_{2})
> P(Z_{1})P(Y_{2})$, in which
case the code for the word can be shorter than the codes for its
parts.}

{
 And this in turn corresponds exactly to the case where we can
infer some of the properties of the thing from seeing its other
properties. It must be more likely than the default that featherless
bipedal things will also be mortal.}

{
 Of course the word ``human''
really describes many, many more properties---when you see a
human-shaped entity that talks and wears clothes, you can infer whole
hosts of biochemical and anatomical and cognitive facts about it. To
replace the word ``human'' with a
description of everything we know about humans would require us to
spend an inordinate amount of time talking. But this is true
\textit{only} because a featherless talking biped is far more likely
than default to be poisonable by hemlock, or have broad nails, or be
overconfident.}

{
 Having a word for a thing, rather than just listing its
properties, is a more compact code precisely in those cases where we
can infer some of those properties from the other properties. (With the
exception perhaps of very primitive words, like
``red,'' that we would use to send
an entirely uncompressed description of our sensory experiences. But by
the time you encounter a bug, or even a rock, you're
dealing with nonsimple property collections, far above the primitive
level.)}

{
 So having a word ``wiggin'' for
green-eyed black-haired people is more useful than just saying
``green-eyed black-haired person''
precisely when:}

\begin{enumerate}
  \item {
 Green-eyed people are more likely than average to be black-haired
(and vice versa), meaning that we can probabilistically infer green
eyes from black hair or vice versa; \textit{or}}

\item{
 Wiggins share other properties that can be inferred at
greater-than-default probability. In this case we have to separately
observe the green eyes and black hair; but then, after observing both
these properties independently, we can probabilistically infer other
properties (like a taste for ketchup).}
\end{enumerate}

{
 One may even consider the act of defining a word as a promise to
this effect. Telling someone, ``I define the word
`wiggin' to mean a person with green
eyes and black hair,'' by Gricean implication,
asserts that the word ``wiggin''
will somehow help you make inferences / shorten your messages.}

{
 If green-eyes and black hair have no greater than default
probability to be found together, nor does any other property occur at
greater than default probability along with them, then the word
``wiggin'' is a lie: The word claims
that certain people are worth distinguishing as a group, but
they're not.}

{
 In this case the word
``wiggin'' does not help describe
reality more compactly---it is not defined by someone sending the
shortest message---it has no role in the simplest explanation.
Equivalently, the word ``wiggin''
will be of no help to you in doing any Bayesian inference. Even if you
do not call the word a lie, it is surely an error.}

{
 And the way to carve reality at its joints is to draw your
boundaries around concentrations of unusually high probability density
in Thingspace.}

\myendsectiontext

\mysection{Superexponential Conceptspace, and Simple Words}

{
 Thingspace, you might think, is a rather huge space. Much larger
than reality, for where reality only contains things that actually
exist, Thingspace contains everything that \textit{could} exist. }

{
 Actually, the way I ``defined''
Thingspace to have dimensions for every possible attribute---including
correlated attributes like density and volume and mass---Thingspace may
be too poorly defined to have anything you could call a \textit{size}.
But it's important to be able to visualize Thingspace
\textit{anyway}. Surely, no one can \textit{really} understand a flock
of sparrows if all they see is a cloud of flapping cawing things,
rather than a cluster of points in Thingspace.}

{
 But as vast as Thingspace may be, it doesn't hold
a candle to the size of Conceptspace.}

{
 ``Concept,'' in machine
learning, means a rule that includes or excludes examples. If you see
the data \texttt{\{2:+, 3:-, 14:+, 23:-, 8:+,
9:-\}} then you might guess that the
concept was ``even numbers.'' There
is a rather large literature (as one might expect) on how to learn
concepts from data \ldots given random examples, given chosen examples
\ldots given possible errors in classification \ldots and most
importantly, given different spaces of possible rules.}

{
 Suppose, for example, that we want to learn the concept
``good days on which to play
tennis.'' The possible attributes of Days are}

\begin{verse}
\texttt{ Sky: \{Sunny, Cloudy,Rainy\} }\\
\texttt{ AirTemp: \{Warm,Cold\} }\\
\texttt{ Humidity: \{Normal,High\} }\\
\texttt{ Wind: \{Strong,Weak\}. }\\
\end{verse}

{
 We're then presented with the following data,
where + indicates a positive example of the concept, and - indicates a
negative classification:}

\begin{verse}
\texttt{ + Sky: Sunny; AirTemp: Warm; Humidity: High; Wind: Strong. } \\
\texttt{ - Sky: Rainy; AirTemp: Cold; Humidity: High; Wind: Strong. } \\
\texttt{ + Sky: Sunny; AirTemp: Warm; Humidity: High; Wind: Weak. } \\
\end{verse}

{
 What should an algorithm infer from this?}

{
 A machine learner might represent \textit{one} concept that fits
this data as follows:}

\begin{verse}
  \texttt{ \{ Sky: ?; AirTemp: Warm; Humidity: High; Wind: ?\}. }\\
\end{verse}

{
 In this format, to determine whether this concept accepts or
rejects an example, we compare element-by-element: \texttt{?} accepts anything,
but a specific value accepts only that specific value.}

{
 So the concept above will accept only \texttt{Days} with \texttt{AirTemp = Warm} and
\texttt{Humidity = High}, but the \texttt{Sky} and the \texttt{Wind} can take on any value. This
fits both the negative and the positive classifications in the data so
far---though it isn't the \textit{only} concept that
does so.}

{
 We can also simplify the above concept representation to}

\begin{verse}
\texttt{ \{?, Warm, High, ?\}.}
\end{verse}


{
 Without going into details, the classic algorithm would be:}

\begin{itemize}
\item{
 Maintain the set of the most general hypotheses that fit the
data---those that positively classify as many examples as possible,
while still fitting the facts.}

\item{
 Maintain another set of the most specific hypotheses that fit the
data---those that negatively classify as many examples as possible,
while still fitting the facts.}

\item{
 Each time we see a new negative example, we strengthen all the
most general hypotheses as little as possible, so that the new set is
again as general as possible while fitting the facts.}

\item{
 Each time we see a new positive example, we relax all the most
specific hypotheses as little as possible, so that the new set is again
as specific as possible while fitting the facts.}

\item{
 We continue until we have only a single hypothesis left. This will
be the answer \textit{if} the target concept was in our hypothesis
space at all.}
\end{itemize}

{
 In the case above, the set of most general hypotheses would be}

\begin{verse}
\texttt{  \{\{?,Warm, ?,?\},\{Sunny,?, ?,?\}\},}\\
\end{verse}

{
 while the set of most specific hypotheses contains the single
member \texttt{\{Sunny, Warm, High,?\}}. }

{
 Any other concept you can find that fits the data will be strictly
more specific than one of the most general hypotheses, and strictly
more general than the most specific hypothesis.}

{
 (For more on this, I recommend Tom Mitchell's
\textit{Machine Learning}, from which this example was
adapted.\footnote{Tom M. Mitchell, \textit{Machine Learning} (McGraw-Hill
Science/Engineering/Math, 1997).\comment{1}})}

{
 Now you may notice that the format above \textit{cannot} represent
all possible concepts. E.g., ``Play tennis when the
sky is sunny \textit{or} the air is warm.'' That fits
the data, but in the concept representation defined above,
there's no quadruplet of values that describes the
rule.}

{
 Clearly our machine learner is not very general. Why not allow it
to represent \textit{all possible} concepts, so that it can learn with
the greatest possible flexibility?}

{
 \texttt{Day}s are composed of these four variables, one variable with 3
values and three variables with 2 values. So there are $3 \times 2
\times 2 \times 2 = 24$ possible \texttt{Day}s that we could
encounter.}

{
 The format given for representing Concepts allows us to require
any of these values for a variable, or leave the variable open. So
there are $4 \times 3 \times 3 \times 3 = 108$ concepts
in that representation. For the most-general/most-specific algorithm to
work, we need to start with the most specific hypothesis
``no example is ever positively
classified.'' If we add that, it makes a total of 109
concepts.}

{
 Is it suspicious that there are more possible concepts than
possible \texttt{Day}s? Surely not: After all, a concept can be viewed as a
\textit{collection} of \texttt{Day}s. A concept can be viewed as the set of days
that it classifies positively, or isomorphically, the set of days that
it classifies negatively.}

{
 So the space of \textit{all possible} concepts that classify Days
is the set of all possible sets of Days, whose size is
$2^{24} = 16,777,216$.}

{
 This complete space includes all the concepts we have discussed so
far. But it also includes concepts like ``Positively
classify only the examples \texttt{\{Sunny, Warm, High, Strong\}} and
\texttt{\{Sunny, Warm, High, Weak\}} and reject everything
else'' or ``Negatively classify only
the example \texttt{\{Rainy, Cold, High,Strong\}} and accept everything
else.'' It includes concepts with no compact
representation, just a flat list of what is and isn't
allowed.}

{
 That's the problem with trying to build a
``fully general'' inductive learner:
They can't learn concepts until they've
seen every possible example in the instance space.}

{
 If we add on more attributes to \texttt{Day}s---like the \texttt{Water} temperature,
or the \texttt{Forecast} for tomorrow---then the number of possible days will
grow exponentially in the number of attributes. But this
isn't a problem with our restricted concept space,
because you can narrow down a large space using a logarithmic number of
examples.}

{
  Let's say we add the \texttt{Water: \{Warm, Cold\}}
  attribute to days, which will
make for 48 possible \texttt{Day}s and 325 possible concepts.
Let's say that each \texttt{Day} we see is, usually, classified
positive by around half of the currently-plausible concepts, and
classified negative by the other half. Then when we learn the actual
classification of the example, it will cut the space of compatible
concepts in half. So it might only take 9 examples
$(2^9 = 512)$ to narrow 325 possible concepts down to
one.}

{
 Even if \texttt{Day}s had forty binary attributes, it should still only
take a manageable amount of data to narrow down the possible concepts
to one. Sixty-four examples, if each example is classified positive by
half the remaining concepts. \textit{Assuming}, of course, that the
\textit{actual} rule is one we can represent at all!}

{
 If you want to think of all the possibilities, well, good luck
with that. The space of \textit{all possible} concepts grows
\textit{super}exponentially in the number of attributes.}

{
 By the time you're talking about data with forty
binary attributes, the number of possible examples is past a
trillion---but the number of possible \textit{concepts} is past
two-to-the-trillionth-power. To narrow down that
\textit{super}exponential concept space, you'd have to
see over a trillion examples before you could say what was In, and what
was Out. You'd have to see every possible example, in
fact.}

{
 That's with forty binary attributes, mind you.
Forty bits, or 5 bytes, to be classified simply
``Yes'' or
``No.'' Forty bits implies
$2^{40}$ possible examples, and $2^{240}$
possible concepts that classify those examples as positive or
negative.}

{
 So, here in the real world, where objects take more than 5 bytes
to describe \textit{and} a trillion examples are not available
\textit{and} there is noise in the training data, we only even
\textit{think} about \textit{highly regular} concepts. A human
mind---or the whole observable universe---is not nearly large enough to
consider all the other hypotheses.}

{
 From this perspective, learning doesn't just
\textit{rely on} inductive bias, it is \textit{nearly all} inductive
bias---when you compare the number of concepts ruled out a priori, to
those ruled out by mere evidence.}

{
 But what has this (you inquire) to do with the proper use of
words?}

{
 It's the whole reason that words have intensions
as well as extensions.}

{
 In the last essay, I concluded:}

\begin{quote}
 The way to carve reality at its joints is to draw boundaries
 around concentrations of unusually high probability density.
\end{quote}

{
 I deliberately left out a key qualification in that (slightly
edited) statement, because I couldn't explain it until
now. A better statement would be:}

\begin{quote}
 The way to carve reality at its joints, is to draw \textit{simple}
boundaries around concentrations of unusually high probability density
in Thingspace.
\end{quote}

{
 Otherwise you would just gerrymander Thingspace. You would create
really odd noncontiguous boundaries that collected the observed
examples, examples that couldn't be described in any
shorter message than your observations themselves, and say:
``This is what I've seen before, and
what I expect to see more of in the future.''}

{
 In the real world, nothing above the level of molecules repeats
itself \textit{exactly.} Socrates is shaped a lot like all those other
humans who were vulnerable to hemlock, but he isn't
shaped \textit{exactly} like them. So your guess that Socrates is a
``human'' relies on drawing
\textit{simple} boundaries around the human cluster in Thingspace.
Rather than, ``Things shaped exactly like [5-megabyte
shape specification 1] and with [lots of other characteristics],
\textit{or} exactly like [5-megabyte shape specification 2] and [lots
of other characteristics], \ldots~, are human.''}

{
 If you don't draw \textit{simple} boundaries
around your experiences, you can't do inference with
them. So you try to describe ``art''
with intensional definitions like ``that which is
intended to inspire any complex emotion for the sake of inspiring
it,'' rather than just pointing at a long list of
things that are, or aren't art.}

{
 In fact, the above statement about ``how to carve
reality at its joints'' is a bit chicken-and-eggish:
You can't assess the \textit{density} of actual
observations until you've already done at least a
little carving. And the probability distribution comes from drawing the
boundaries, not the other way around---if you already \textit{had} the
probability distribution, you'd have everything
necessary for inference, so why would you bother drawing boundaries?}

{
 And this suggests another---yes, yet another---reason to be
suspicious of the claim that ``you can define a word
any way you like.'' When you consider the
superexponential size of Conceptspace, it becomes clear that singling
out one particular concept for consideration is an act of no small
audacity---not just for us, but for any mind of bounded computing
power.}

{
 Presenting us with the word
``wiggin,'' defined as
``a black-haired green-eyed
person,'' without some reason for raising
\textit{this particular concept} to the level of our deliberate
attention, is rather like a detective saying: ``Well,
I haven't the slightest shred of support one way or the
other for who could've murdered those orphans \ldots not
even an intuition, mind you \ldots but have we considered John Q.
Wiffleheim of 1234 Norkle Rd as a suspect?''}

\myendsectiontext


\bigskip

\mysection{Conditional Independence, and Naive Bayes}

{
 Previously I spoke of mutual information between $X$ and $Y$, written
$I(X;Y)$, which is the difference between the entropy of the joint
probability distribution, $H(X,Y)$, and the entropies of the marginal
distributions, $H(X) + H(Y)$. }

{
 I gave the example of a variable $X$, having eight states,
$X_{1}$ through $X_{8}$, which are all equally
probable if we have not yet encountered any evidence; and a variable $Y$,
with states $Y_{1}$ through $Y_{4}$, which are
all equally probable if we have not yet encountered any evidence. Then
if we calculate the marginal entropies $H(X)$ and $H(Y)$, we will find that
$X$ has 3 bits of entropy, and $Y$ has 2 bits.}

{
 However, we also know that $X$ and $Y$ are both even or both odd; and
this is all we know about the relation between them. So for the joint
distribution $(X,Y)$ there are only 16 possible states, all equally
probable, for a joint entropy of 4 bits. This is a 1-bit entropy
defect, compared to 5 bits of entropy if $X$ and $Y$ were independent. This
entropy defect is the mutual information---the information that $X$ tells
us about $Y$, or vice versa, so that we are not as uncertain about one
after having learned the other.}

{
 Suppose, however, that there exists a third variable $Z$. The
variable $Z$ has two states, ``even''
and ``odd,'' perfectly correlated to
the evenness or oddness of $(X,Y)$. In fact, we'll
suppose that $Z$ is just the question ``Are $X$ and $Y$ even
or odd?''}

{
 If we have no evidence about $X$ and $Y$, then $Z$ itself necessarily
has 1 bit of entropy on the information given. There is 1 bit of mutual
information between $Z$ and $X$, and 1 bit of mutual information between $Z$
and $Y$. And, as previously noted, 1 bit of mutual information between $X$
and $Y$. So how much entropy for the whole system $(X,Y,Z)$? You might
naively expect that}

\whencolumns{
\begin{equation*}
 H(X,Y,Z) = H(X) + H(Y) + H(Z) - I(X;Z) - I(Z;Y) - I(X;Y),
\end{equation*}
}{
\begin{align*}
  &H(X,Y,Z) = H(X) + H(Y) + H(Z)\\
  &- I(X;Z) - I(Z;Y) - I(X;Y),
\end{align*}
}

{
 but this turns out not to be the case. }

{
 The joint system $(X,Y,Z)$ only has 16 possible states---since $Z$ is
just the question ``Are X and Y even or
odd?''---so $H(X,Y,Z) = 4$ bits.}

{
 But if you calculate the formula just given, you get}

\begin{equation*}
 (3 + 2 + 1 - 1 - 1 - 1) \text{ bits }= 3\text{ bits }=\text{ WRONG!}
\end{equation*}



{
 Why? Because if you have the mutual information between $X$ and $Z$,
and the mutual information between $Z$ and $Y$, that may include some of
the \textit{same} mutual information that we'll
calculate exists between $X$ and $Y$. In this case, for example, knowing
that $X$ is even tells us that $Z$ is even, and knowing that $Z$ is even
tells us that $Y$ is even, but this is the same information that $X$ would
tell us about $Y$. We double-counted some of our knowledge, and so came
up with too little entropy. }

{
 The correct formula is (I believe):}

\whencolumns{
\begin{equation*}
  H(X,Y,Z) = H(X) + H(Y) + H(Z) - I(X;Z) - I(Z;Y) - I(X;Y|Z).
\end{equation*}
}{
\begin{align*}
  &H(X,Y,Z) = H(X) + H(Y) + H(Z)\\
  &- I(X;Z) - I(Z;Y) - I(X;Y|Z).
\end{align*}
}


{
 Here the last term, $I(X;Y |Z)$, means,
``the information that $X$ tells us about $Y$, given that
we already know $Z$.'' In this case, $X$
doesn't tell us anything about $Y$, given that we already
know $Z$, so the term comes out as zero---and the equation gives the
correct answer. There, isn't that nice? }

{
 ``No,'' you correctly reply,
``for you have not told me how to \textit{calculate}
$I(X;Y|Z)$, only given me a verbal argument that it ought to be
zero.''}

{
 We calculate $I(X;Y|Z)$ just the way you would expect. We
know $I(X;Y) = H(X) + H(Y) - H(X,Y)$, so}

\begin{equation*}
  I(X;Y|Z) = H(X|Z) + H(Y|Z) - H(X,Y|Z).
\end{equation*}


{
 And now, I suppose, you want to know how to calculate the
conditional entropy? Well, the \textit{original} formula for the
entropy is}

\begin{equation*}
 H(S) = \sum_{i}{-P(S_{i}) \times \log_{2}(P(S_{i}))}.
\end{equation*}


{
 If we then learned a new fact $Z_{0}$, our remaining
uncertainty about $S$ would be}

\begin{equation*}
 H(S|Z_0) =\sum_{i}{-P(S_{i}|Z_{0})\log_{2}(P(S_{i}|Z_{0}))}.
\end{equation*}

{
 So if we're going to learn a new fact $Z$, but we
don't know which $Z$ yet, then, on average, we expect to
be around this uncertain of $S$ afterward:}

\begin{equation*}
 H(S|Z) =\sum_{j}\left(P(Z_{j})\sum_{i}{-P(S_{i}|Z_{j})\log_{2}(P(S_{i}|Z_{j}))}\right)
\end{equation*}

{
 And that's how one calculates conditional
entropies; from which, in turn, we can get the conditional mutual
information. }

{
 There are \textit{all sorts} of ancillary theorems here, like}

\begin{equation*}
 H(X|Y) = H (X,Y) - H(Y)
\end{equation*}


{
 and}

\begin{equation*}
\text{if }I(X;Z) = 0\text{ and }I(Y;X|Z) = 0\text{ then }I(X;Y) = 0,
\end{equation*}


{
 but I'm not going to go into those. }

{
 ``But,'' you ask,
``what does \textit{this} have to do with the nature
of words and their hidden Bayesian structure?''}

{
 I am just so \textit{unspeakably} glad that you asked that
question, because I was planning to tell you whether you liked it or
not. But first there are a couple more preliminaries.}

{
 You will remember---yes, you \textit{will} remember---that there
is a duality between mutual information and Bayesian evidence. Mutual
information is positive if and only if the probability of at least some
joint events $P(x,y)$ does not equal the product of the probabilities of
the separate events $P(x)P(y)$. This, in turn, is exactly equivalent to
the condition that Bayesian evidence exists between $x$ and $y$:}

\begin{align*}
 I(X;Y) &> 0 \Rightarrow \\
 P(x,y) &\neq P(x)P(y) \\
 P(x,y) / P(y) &\neq P(x) \\
 P(x|y) &\neq P(x).
\end{align*}


{
 If you're conditioning on $Z$, you just adjust the
whole derivation accordingly:}

\begin{align*}
 I(X;Y|Z) &> 0 \Rightarrow \\
 P(x,y|z) &\neq P(x|z)P(y|z) \\
 P(x,y|z) / P(y|z) &\neq P (x|z) \\
 (P(x,y,z)/P(z)) / (P(y,z)/P(z)) &\neq P(x|z) \\
 P(x,y,z) / P(y,z) &\neq P(x|z) \\
 P(x|y,z) &\neq P(x|z).
\end{align*}

{
 Which last line reads ``Even knowing $Z$, learning
$Y$ still changes our beliefs about $X$.''}

{
 Conversely, as in our original case of $Z$ being
``even'' or
``odd,'' $Z$ screens off $X$ from
$Y$---that is, if we know that $Z$ is
``even,'' learning that $Y$ is in
state $Y_{4}$ tells us \textit{nothing more} about whether $X$
is $X{2}$, $X_{4}$, $X_{6}$, or
$X_{8}$. Or if we know that $Z$ is
``odd,'' then learning that $X$ is
$X_{5}$ tells us nothing more about whether $Y$ is
$Y_{1}$ or $Y_{3}$. Learning $Z$ has rendered $X$ and
$Y$ \textit{conditionally independent}.}

{
 Conditional independence is a hugely important concept in
probability theory---to cite just one example, without conditional
independence, the universe would have no structure.}

{
 Here, though, I only intend to talk about one particular kind of
conditional independence---the case of a central variable that screens
off other variables surrounding it, like a central body with
tentacles.}

{
 Let there be five variables $U$,$V$, $W$, $X$, and $Y$; and moreover,
suppose that for every pair of these variables, one variable is
evidence about the other. If you select $U$ and $W$, for example, then
learning $U = U_{1}$ will tell you something you
didn't know before about the probability that $W =
W_{1}$.}

{
 An unmanageable inferential mess? Evidence gone wild? Not
necessarily.}

{
 Maybe $U$ is ``Speaks a
language,'' $V$ is ``Two arms and ten
digits,'' $W$ is ``Wears
clothes,'' $X$ is ``Poisonable by
hemlock,'' and $Y$ is ``Red
blood.'' Now if you encounter a thing-in-the-world,
that might be an apple and might be a rock, and you learn that this
thing speaks Chinese, you are liable to assess a much higher
probability that it wears clothes; and if you learn that the thing is
not poisonable by hemlock, you will assess a somewhat lower probability
that it has red blood.}

{
 Now some of these rules are stronger than others. There is the
case of Fred, who is missing a finger due to a volcano accident, and
the case of Barney the Baby who doesn't speak yet, and
the case of Irving the IRCBot who emits sentences but has no blood. So
if we learn that a certain thing is not wearing clothes, that
doesn't screen off everything that its speech
capability can tell us about its blood color. If the thing
doesn't wear clothes but \textit{does} talk, maybe
it's Nude Nellie.}

{
 This makes the case more interesting than, say, five integer
variables that are all odd or all even, but otherwise uncorrelated. In
that case, knowing \textit{any} one of the variables would screen off
everything that knowing a second variable could tell us about a third
variable.}

{
 But here, we have dependencies that don't go away
as soon as we learn just one variable, as the case of Nude Nellie
shows. So is it an unmanageable inferential inconvenience?}

{
 Fear not! For there may be some \textit{sixth} variable $Z$, which,
if we knew it, really \textit{would} screen off every pair of variables
from each other. There may be some variable $Z$---even if we have to
\textit{construct} $Z$ rather than observing it directly---such that:}

\begin{align*}
 P(U|V,W,X,Y,Z) &= P(U|Z) \\
 P(V|U,W,X,Y,Z) &= P(V|Z) \\
 P(W|U,V,X,Y,Z) &= P(W|Z) \\
 &\vdots
\end{align*}

{
 Perhaps, \textit{given that} a thing is
``human,'' then the probabilities of
it speaking, wearing clothes, and having the standard number of
fingers, are all independent. Fred may be missing a finger---but he is
no more likely to be a nudist than the next person; Nude Nellie never
wears clothes, but knowing this doesn't make it any
less likely that she speaks; and Baby Barney doesn't
talk yet, but is not missing any limbs.}

{
 This is called the ``Naive
Bayes'' method, because it usually
isn't quite true, but \textit{pretending} that
it's true can simplify the living daylights out of your
calculations. We don't keep separate track of the
influence of clothed-ness on speech capability given finger number. We
just use all the information we've observed to keep
track of the probability that this thingy is a human (or alternatively,
something else, like a chimpanzee or robot) and then use our beliefs
about the central class to predict anything we haven't
seen yet, like vulnerability to hemlock.}

{
 Any observations of $U$,$V$, $W$, $X$, and $Y$ just act as evidence for the
central class variable $Z$, and then we use the posterior distribution on
$Z$ to make any predictions that need making about unobserved variables
in $U$,$V$, $W$, $X$, and $Y$.}

{
 Sound familiar? It should; see Figure \ref{cinb_network_2}.}

%177.1
\myfigurec{images/img200.jpg}{cinb_network_2}{Network 2}

{
 As a matter of fact, if you use the right kind of neural network
units, this ``neural network'' ends
up \textit{exactly, mathematically} equivalent to Naive Bayes. The
central unit just needs a logistic threshold---an S-curve
response---and the weights of the inputs just need to match the
logarithms of the likelihood ratios, et cetera. In fact,
it's a good guess that this is one of the reasons why
logistic response often works so well in neural networks---it lets the
algorithm sneak in a little Bayesian reasoning while the designers
aren't looking.}

{
 Just because someone is presenting you with an algorithm that they
call a ``neural network'' with
buzzwords like ``scruffy'' and
``emergent'' plastered all over it,
disclaiming proudly that they have no idea how the learned network
works---well, don't assume that their little AI
algorithm \textit{really is} Beyond the Realms of Logic. For this
paradigm of adhockery, if it works, will turn out to have Bayesian
structure; it may even be exactly equivalent to an algorithm of the
sort called ``Bayesian.''}

{
 Even if it doesn't \textit{look} Bayesian, on the
surface.}

{
 And then you just \textit{know} that the Bayesians are going to
start explaining exactly how the algorithm works, what underlying
assumptions it reflects, which environmental regularities it exploits,
where it works and where it fails, and even attaching understandable
meanings to the learned network weights.}

{
 Disappointing, isn't it?}

\myendsectiontext

\mysection{Words as Mental Paintbrush Handles}

{
 Suppose I tell you: ``It's the
strangest thing: The lamps in this hotel have triangular
lightbulbs.''}

{
 You may or may not have visualized it---if you
haven't done it yet, do so now---what, in your
mind's eye, does a ``triangular
lightbulb'' look like?}

{
 In your mind's eye, did the glass have sharp
edges, or smooth?}

{
 When the phrase ``triangular
lightbulb'' first crossed my mind---no, the hotel
doesn't have them---then as best as my introspection
could determine, I first saw a pyramidal lightbulb with sharp edges,
then (almost immediately) the edges were smoothed, and then my mind
generated a loop of flourescent bulb in the shape of a smooth triangle
as an alternative.}

{
 As far as I can tell, no deliberative/verbal thoughts were
involved---just wordless reflex flinch away from the imaginary mental
vision of sharp glass, which design problem was solved before I could
even think in words.}

{
 Believe it or not, for some decades, there was a serious debate
about whether people \textit{really} had mental images in their
mind---an actual \textit{picture} of a chair somewhere---or if people
just naively \textit{thought} they had mental images (having been
misled by ``introspection,'' a very
bad forbidden activity), while actually just having a little
``chair'' label, like a LISP token,
active in their brain.}

{
 I am trying hard not to say anything like ``How
spectacularly silly,'' because there is always the
hindsight effect to consider, but: how spectacularly silly.}

{
 This academic paradigm, I think, was mostly a deranged legacy of
behaviorism, which denied the existence of thoughts in humans, and
sought to explain all human phenomena as
``reflex,'' including speech.
Behaviorism probably deserves its own write at some point, as it was a
perversion of rationalism; but this is not that write.}

{
 ``You call it
`silly,''' you
inquire, ``but how do you \textit{know} that your
brain represents visual images? Is it merely that you can close your
eyes and see them?''}

{
 This question \textit{used} to be harder to answer, back in the
day of the controversy. If you wanted to prove the existence of mental
imagery ``scientifically,'' rather
than just by introspection, you had to infer the existence of mental
imagery from experiments like this: Show subjects two objects and ask
them if one can be rotated into correspondence with the other. The
response time is linearly proportional to the angle of rotation
required. This is easy to explain if you are actually visualizing the
image and continuously rotating it at a constant speed, but hard to
explain if you are just checking propositional features of the image.}

{
 Today we can actually neuroimage the little pictures in the visual
cortex. So, yes, your brain really does represent a detailed image of
what it sees or imagines. See Stephen Kosslyn's
\textit{Image and Brain: The Resolution of the Imagery
Debate.}\footnote{Stephen M. Kosslyn, \textit{Image and Brain: The Resolution of
the Imagery Debate} (Cambridge, MA: MIT Press, 1994).\comment{1}}}

{
 Part of the reason people get in trouble with words, is that they
do not realize how much complexity lurks behind words.}

{
 Can you visualize a ``green
dog''? Can you visualize a ``cheese
apple''?}

{
 ``Apple'' isn't
just a sequence of two syllables or five letters.
That's a shadow. That's the tip of the
tiger's tail.}

{
 Words, or rather the concepts behind them, are paintbrushes---you
can use them to draw images in your own mind. Literally draw, if you
employ concepts to make a picture in your visual cortex. And by the use
of shared labels, you can reach into someone else's
mind, and grasp their paintbrushes to draw pictures in \textit{their}
minds---sketch a little green dog in their visual cortex.}

{
 But don't think that, because you send syllables
through the air, or letters through the Internet, it is the syllables
or the letters that draw pictures in the visual cortex. That takes some
complex instructions that wouldn't fit in the sequence
of letters. ``Apple'' is 5 bytes,
and drawing a picture of an apple from scratch would take more data
than that.}

{
 ``Apple'' is merely the tag
attached to the true and wordless \textit{apple} concept, which can
paint a picture in your visual cortex, or collide with
``cheese,'' or recognize an apple
when you see one, or taste its archetype in apple pie, maybe even send
out the motor behavior for eating an apple \ldots}

{
 And it's not as simple as just calling up a
picture from memory. Or how would you be able to visualize combinations
like a ``triangular
lightbulb''---imposing triangleness on lightbulbs,
keeping the essence of both, even if you've never seen
such a thing in your life?}

{
 Don't make the mistake the behaviorists made.
There's far more to speech than sound in air. The
labels are just pointers---``look in memory area
1387540.'' Sooner or later, when
you're handed a pointer, it comes time to dereference
it, and actually look in memory area 1387540.}

{
 What does a word point to?}

\myendsectiontext


\bigskip

\mysection{Variable Question Fallacies}

\begin{quotation}
{
 ALBERT: ``Every time I've
listened to a tree fall, it made a sound, so I'll guess
that other trees falling also make sounds. I don't
believe the world changes around when I'm not
looking.''}

{
 BARRY: ``Wait a minute. If no one hears it, how
 can it be a sound?''}
\end{quotation}

{
 While writing the dialogue of Albert and Barry in their dispute
over whether a falling tree in a deserted forest makes a sound, I
sometimes found myself losing empathy with my characters. I would start
to lose the gut feel of why \textit{anyone} would ever argue like that,
even though I'd seen it happen many times.}

{
 On these occasions, I would repeat to myself,
``Either the falling tree makes a sound, or it does
not!'' to restore my borrowed sense of indignation.}

{
 (P or {\textlnot}P) is not always a reliable heuristic, if you
substitute arbitrary English sentences for P. ``This
sentence is false'' cannot be consistently viewed as
true or false. And then there's the old classic,
``Have you stopped beating your
wife?''}

{
 Now if you are a mathematician, and one who believes in classical
(rather than intuitionistic) logic, there are ways to continue
insisting that (P or {\textlnot}P) is a theorem: for example, saying
that ``This sentence is false'' is
not a sentence.}

{
 But such resolutions are subtle, which suffices to demonstrate a
need for subtlety. You cannot just bull ahead on every occasion with
``Either it does or it
doesn't!''}

{
 So does the falling tree make a sound, or not, or \ldots ?}

{
 Surely, 2 + 2 = X or it does not? Well, maybe, \textit{if}
it's \textit{really} the same X, the same 2, and the
same + and = . If X evaluates to 5 on some occasions and 4 on another,
your indignation may be misplaced.}

{
 To even begin claiming that (P or {\textlnot}P) ought to be a
necessary truth, the symbol P must stand for \textit{exactly} the same
thing in both halves of the dilemma. ``Either the fall
makes a sound, or not!''---but if Albert::sound is
not the same as Barry::sound, there is nothing paradoxical about the
tree making an Albert::sound but not a Barry::sound.}

{
 (The :: idiom is something I picked up in my C++ days for avoiding
namespace collisions. If you've got two different
packages that define a class Sound, you can write Package1::Sound to
specify which Sound you mean. The idiom is not widely known, I think;
which is a pity, because I often wish I could use it in writing.)}

{
 The variability may be subtle: Albert and Barry may carefully
verify that it is the same tree, in the same forest, and the same
occasion of falling, just to ensure that they really do have a
substantive disagreement about exactly the same event. And then forget
to check that they are matching this event against exactly the same
concept.}

{
 Think about the grocery store that you visit most often: Is it on
the left side of the street, or the right? But of course there is no
``\textit{the} left side'' of the
street, only \textit{your} left side, as you travel along it from some
particular direction. Many of the words we use are really functions of
implicit variables supplied by context.}

{
 It's actually one heck of a pain, requiring one
heck of a lot of work, to handle this kind of problem in an Artificial
Intelligence program intended to parse language---the phenomenon going
by the name of ``speaker deixis.''}

{
 ``Martin told Bob the building was on his
left.'' But
``left'' is a function-word that
evaluates with a speaker-dependent variable invisibly grabbed from the
surrounding context. Whose ``left''
is meant, Bob's or Martin's?}

{
 The variables in a variable question fallacy often
aren't neatly labeled---it's not as
simple as ``Say, do you think Z + 2 equals
6?''}

{
 If a namespace collision introduces two different concepts that
look like ``the same concept''
because they have the same name---or a map compression introduces two
different events that look like the same event because they
don't have separate mental files---or the same function
evaluates in different contexts---then reality itself becomes protean,
changeable. At least that's what the algorithm feels
like from inside. Your mind's eye sees the map, not the
territory directly.}

{
 If you have a question with a hidden variable, that evaluates to
different expressions in different contexts, it \textit{feels like}
reality itself is unstable---what your mind's eye sees,
shifts around depending on where it looks.}

{
 This often confuses undergraduates (and postmodernist professors)
who discover a sentence with more than one interpretation; they think
they have discovered an unstable portion of reality.}

{
 ``Oh my gosh! `The Sun goes around
the Earth' is true for Hunga Huntergatherer, but for
Amara Astronomer, `The Sun goes around the
Earth' is false! There is no fixed
truth!'' The deconstruction of this sophomoric
nitwittery is left as an exercise to the reader.}

{
 And yet, even I initially found myself writing
``If X is 5 on some occasions and 4 on another, the
sentence `2 + 2 = X' may have no fixed
truth-value.'' There is not \textit{one} sentence
with a \textit{variable} truth-value. ``2 + 2 =
X'' \textit{has no} truth-value. It is not a
\textit{proposition}, not yet, not as mathematicians define
proposition-ness, any more than ``2 + 2
='' is a proposition, or ``Fred
jumped over the'' is a grammatical sentence.}

{
 But this fallacy tends to sneak in, even when you allegedly know
better, because, well, that's how the algorithm feels
from inside.}

\myendsectiontext

\mysection{37 Ways That Words Can Be Wrong}

{
 Some reader is bound to declare that a better title for this essay
would be ``37 Ways That You Can Use Words
Unwisely,'' or ``37 Ways That
Suboptimal Use Of Categories Can Have Negative Side Effects On Your
Cognition.'' }

{
 But one of the primary lessons of this gigantic list is that
saying ``There's no way my choice of X
can be `wrong''' is
nearly always an error in practice, whatever the theory. You can always
be wrong. Even when it's theoretically impossible to be
wrong, you can still be wrong. There is never a Get Out of Jail Free
card for anything you do. That's life.}

{
 Besides, I can define the word
``wrong'' to mean anything I
like---it's not like a word can be \textit{wrong.}}

{
 Personally, I think it quite justified to use the word
``wrong'' when:}

\begin{enumerate}
\item {
 \textit{A word fails to connect to reality in the first place.} Is
Socrates a framster? Yes or no? (The Parable of the Dagger)}

\item {
 \textit{Your argument, if it worked, could coerce reality to go a
different way by choosing a different word definition.} Socrates is a
human, and humans, by definition, are mortal. So if you defined humans
to not be mortal, would Socrates live forever? (The Parable of
Hemlock)}

\item {
 \textit{You try to establish any sort of empirical proposition as
being true ``by definition.''}
Socrates is a human, and humans, by definition, are mortal. So is it a
\textit{logical truth} if we empirically predict that Socrates should
keel over if he drinks hemlock? It seems like there are logically
possible, non-self-contradictory worlds where Socrates
doesn't keel over---where he's immune
to hemlock by a quirk of biochemistry, say. Logical truths are true in
all possible worlds, and so never tell you \textit{which} possible
world you live in---and anything you can establish
``by definition'' is a logical
truth. (The Parable of Hemlock)}

\item {
 \textit{You unconsciously slap the conventional label on
something, without actually using the verbal definition you just gave.}
You know perfectly well that Bob is
``human,'' even though, by your
definition, you can never call Bob
``human'' without first observing
him to be mortal. (The Parable of Hemlock)}

\item {
 \textit{The act of labeling something with a word disguises a
challengable inductive inference you are making.} If the last 11
egg-shaped objects drawn have been blue, and the last 8 cubes drawn
have been red, it is a matter of induction to say this rule will hold
in the future. But if you call the blue eggs
``bleggs'' and the red cubes
``rubes,'' you may reach into the
barrel, feel an egg shape, and think ``Oh, a
blegg.'' (Words as Hidden Inferences)}

\item {
 \textit{You try to define a word using words, in turn defined with
ever-more-abstract words, without being able to point to an example.}
``What is red?''
``Red is a color.''
``What's a color?''
``It's a property of a
thing.'' ``What's a
thing? What's a property?'' It never
occurs to you to point to a stop sign and an apple. (Extensions and
Intensions)}

\item {
 \textit{The extension doesn't match the
intension.} We aren't consciously aware of our
identification of a red light in the sky as
``Mars,'' which will probably happen
regardless of your attempt to define
``Mars'' as ``The
God of War.'' (Extensions and Intensions)}

\item {
 \textit{Your verbal definition doesn't capture
more than a tiny fraction of the category's shared
characteristics, but you try to reason as if it does.} When the
philosophers of Plato's Academy claimed that the best
definition of a human was a ``featherless
biped,'' Diogenes the Cynic is said to have exhibited
a plucked chicken and declared ``Here is
Plato's Man.'' The Platonists
promptly changed their definition to ``a featherless
biped with broad nails.'' (Similarity Clusters)}

\item {
 \textit{You try to treat category membership as all-or-nothing,
ignoring the existence of more and less typical subclusters.} Ducks and
penguins are less typical birds than robins and pigeons. Interestingly,
a between-groups experiment showed that subjects thought a disease was
more likely to spread from robins to ducks on an island, than from
ducks to robins. (Typicality and Asymmetrical Similarity)}

\item {
 \textit{A verbal definition works well enough in practice to point
out the intended cluster of similar things, but you nitpick
exceptions.} Not every human has ten fingers, or wears clothes, or uses
language; but if you look for an empirical cluster of things which
share these characteristics, you'll get enough
information that the occasional nine-fingered human
won't fool you. (The Cluster Structure of Thingspace)}

\item {
 \textit{You ask whether something
``is'' or ``is
not'' a category member but can't
name the question you really want answered.} What is a
``man''? Is Barney the Baby Boy a
``man''? The
``correct'' answer may depend
considerably on whether the query you \textit{really} want answered is
``Would hemlock be a good thing to feed
Barney?'' or ``Will Barney make a
good husband?'' (Disguised Queries)}

\item {
 \textit{You treat intuitively perceived hierarchical categories
like the only correct way to parse the world, without realizing that
other forms of statistical inference are possible even though your
brain doesn't use them.} It's much
easier \textit{for a human} to notice whether an object is a
``blegg'' or
``rube''; than \textit{for a human}
to notice that red objects never glow in the dark, but red furred
objects have all the other characteristics of bleggs. Other statistical
algorithms work differently. (Neural Categories)}

\item {
 \textit{You talk about categories as if they are manna fallen from
the Platonic Realm, rather than inferences implemented in a real
brain.} The ancient philosophers said ``Socrates is a
man,'' not, ``My brain perceptually
classifies Socrates as a match against the
`human' concept.''
(How An Algorithm Feels From Inside)}

\item {
 \textit{You argue about a category membership even after screening
off all questions that could possibly depend on a category-based
inference.} After you observe that an object is blue, egg-shaped,
furred, flexible, opaque, luminescent, and palladium-containing,
what's \textit{left} to ask by arguing,
``Is it a blegg?'' But if your
brain's categorizing neural network contains a
(metaphorical) central unit corresponding to the inference of
blegg-ness, it may still \textit{feel} like there's a
leftover question. (How An Algorithm Feels From Inside)}

\item {
 \textit{You allow an argument to slide into being about
definitions, even though it isn't what you originally
wanted to argue about.} If, before a dispute started about whether a
tree falling in a deserted forest makes a
``sound,'' you asked the two
soon-to-be arguers whether they thought a
``sound'' should be defined as
``acoustic vibrations'' or
``auditory experiences,''
they'd probably tell you to flip a coin. Only after the
argument starts does the definition of a word become politically
charged. (Disputing Definitions)}

\item {
 \textit{You think a word has a meaning, as a property of the word
itself; rather than there being a label that your brain associates to a
particular concept.} When someone shouts ``Yikes! A
tiger!,'' evolution would not favor an organism that
thinks, ``Hm \ldots I have just heard the syllables
`Tie' and
`Grr' which my fellow tribemembers
associate with their internal analogues of my own \textit{tiger}
concept and which \textit{aiiieeee} CRUNCH CRUNCH
GULP.'' So the brain takes a shortcut, and it seems
that the meaning of tigerness is a property of the label itself. People
argue about the \textit{correct meaning} of a label like
``sound.'' (Feel the Meaning)}

\item {
 \textit{You argue over the meanings of a word, even after all
sides understand perfectly well what the other sides are trying to
say.} The human ability to associate labels to concepts is a tool for
communication. When people \textit{want} to communicate,
we're hard to stop; if we have no common language,
we'll draw pictures in sand. When you each understand
what is in the other's mind, you are \textit{done}.
(The Argument From Common Usage)}

\item {
 \textit{You pull out a dictionary in the middle of an empirical or
moral argument.} Dictionary editors are historians of usage, not
legislators of language. If the common definition contains a
problem---if ``Mars'' is defined as
the God of War, or a ``dolphin'' is
defined as a kind of fish, or
``Negroes'' are defined as a
separate category from humans, the dictionary will reflect the standard
mistake. (The Argument From Common Usage)}

\item {
 \textit{You pull out a dictionary in the middle of any argument
ever.} Seriously, what the heck makes you think that dictionary editors
are an authority on whether
``atheism'' is a
``religion'' or whatever? If you
have any substantive issue whatsoever at stake, do you really think
dictionary editors have access to ultimate wisdom that settles the
argument? (The Argument From Common Usage)}

\item {
 \textit{You defy common usage without a reason, making it
gratuitously hard for others to understand you.} Fast stand up
plutonium, with bagels without handle. (The Argument From Common
Usage)}

\item {
 \textit{You use complex renamings to create the illusion of
inference.} Is a ``human'' defined
as a ``mortal featherless biped''?
Then write: ``All [mortal featherless bipeds] are
mortal; Socrates is a [mortal featherless biped]; therefore, Socrates
is mortal.'' Looks less impressive that way,
doesn't it? (Empty Labels)}

\item {
 \textit{You get into arguments that you could avoid if you just
didn't use the word.} If Albert and Barry
aren't allowed to use the word
``sound,'' then Albert will have to
say ``A tree falling in a deserted forest generates
acoustic vibrations,'' and Barry will say
``A tree falling in a deserted forest generates no
auditory experiences.'' When a word poses a problem,
the simplest solution is to eliminate the word and its synonyms. (Taboo
Your Words)}

\item {
 \textit{The existence of a neat little word prevents you from
seeing the details of the thing you're trying to think
about.} What actually goes on in schools once you stop calling it
``education''?
What's a degree, once you stop calling it a
``degree''? If a coin lands
``heads,'' what's
its radial orientation? What is
``truth,'' if you
can't say
``accurate'' or
``correct'' or
``represent'' or
``reflect'' or
``semantic'' or
``believe'' or
``knowledge'' or
``map'' or
``real'' or any other simple term?
(Replace the Symbol with the Substance)}

\item {
 \textit{You have only one word, but there are two or more
different things-in-reality, so that all the facts about them get
dumped into a single undifferentiated mental bucket.}
It's part of a detective's ordinary
work to observe that Carol wore red last night, or that she has black
hair; and it's part of a detective's
ordinary work to wonder if maybe Carol dyes her hair. But it takes a
subtler detective to wonder if there are two Carols, so that the Carol
who wore red is not the same as the Carol who had black hair.
(Fallacies of Compression)}

\item {
 \textit{You see patterns where none exist, harvesting other
characteristics from your }\textit{definitions even when there is no
similarity along that dimension.} In Japan, it is thought that people
of blood type A are earnest and creative, blood type Bs are wild and
cheerful, blood type Os are agreeable and sociable, and blood type ABs
are cool and controlled. (Categorizing Has Consequences)}

\item {
 \textit{You try to sneak in the connotations of a word, by arguing
from a definition that doesn't include the
connotations.} A ``wiggin'' is
defined in the dictionary as a person with green eyes and black hair.
The word ``wiggin'' also carries the
connotation of someone who commits crimes and launches cute baby
squirrels, but that part isn't in the dictionary. So
you point to someone and say: ``Green eyes? Black
hair? See, told you he's a wiggin! Watch, next
he's going to steal the silverware.''
(Sneaking in Connotations)}

\item {
 \textit{You claim ``X, by definition, is a
Y!'' On such occasions you're almost
certainly trying to sneak in a connotation of Y that
wasn't in your given definition.} You define
``human'' as a
``featherless biped,'' and point to
Socrates and say, ``No feathers---two legs---he must
be human!'' But what you \textit{really} care about
is something else, like mortality. If what was in dispute was
Socrates's number of legs, the other fellow would just
reply, ``Whaddaya mean, Socrates's got
two legs? That's what we're arguing
about in the first place!'' (Arguing
``By Definition'')}

\item {
 \textit{You claim ``Ps, by definition, are
Qs!''} If you see Socrates out in the field with some
biologists, gathering herbs that might confer resistance to hemlock,
there's no point in arguing ``Men, by
definition, are mortal!'' The main time you feel the
need to tighten the vise by insisting that something is true
``by definition'' is when
there's other information that calls the default
inference into doubt. (Arguing ``By
Definition'')}

\item {
 \textit{You try to establish membership in an empirical cluster
``by definition.''} You
wouldn't feel the need to say,
``Hinduism, \textit{by definition,} is a
religion!'' because, well, of course Hinduism is a
religion. It's not just a religion
``by definition,''
it's, like, an \textit{actual} religion. Atheism does
not resemble the central members of the
``religion'' cluster, so if it
wasn't for the fact that atheism is a religion
\textit{by definition,} you might go around thinking that atheism
\textit{wasn't} a religion. That's why
you've got to crush all opposition by pointing out that
``Atheism is a religion'' is true
\textit{by definition,} because it isn't true any other
way. (Arguing ``By Definition'')}

\item {
 \textit{Your definition draws a boundary around things that
don't really belong together.} You can claim, if you
like, that you are \textit{defining} the word
``fish'' to refer to salmon,
guppies, sharks, dolphins, and trout, but not jellyfish or algae. You
can claim, if you like, that this is merely a list, and there is no way
a list can be ``wrong.'' Or you can
stop playing games and admit that you made a mistake and that dolphins
don't belong on the fish list. (Where to Draw the
Boundary?)}

\item {
 \textit{You use a short word for something that you
won't need to describe often, or a long word for
something you'll need to describe often. This can
result in inefficient thinking, or even misapplications of
Occam's Razor, if your mind thinks that short sentences
sound ``simpler.''} Which sounds
more plausible, ``God did a
miracle'' or ``A supernatural
universe-creating entity temporarily suspended the laws of
physics''? (Entropy, and Short Codes)}

\item {
 \textit{You draw your boundary around a volume of space where
there is no greater-than-usual density, meaning that the associated
word does not correspond to any performable Bayesian inferences.} Since
green-eyed people are not more likely to have black hair, or vice
versa, and they don't share any other characteristics
in common, why have a word for
``wiggin''? (Mutual Information, and
Density in Thingspace)}

\item {
 \textit{You draw an unsimple boundary without any reason to do
so.} The act of defining a word to refer to all humans, except black
people, seems kind of suspicious. If you don't present
reasons to draw that particular boundary, trying to create an
``arbitrary'' word in that location
is like a detective saying: ``Well, I
haven't the slightest shred of support one way or the
other for who could've murdered those orphans \ldots but
have we considered John Q. Wiffleheim as a suspect?''
(Superexponential Conceptspace, and Simple Words)}

\item {
 \textit{You use categorization to make inferences about properties
that don't have the appropriate empirical structure,
namely, conditional independence given knowledge of the class, to be
well-approximated by Naive Bayes.} No way am I trying to summarize this
one. Just read the essay. (Conditional Independence, and Naive Bayes)}

\item {
 \textit{You think that words are like tiny little LISP symbols in
your mind, rather than words being labels that act as handles to direct
complex mental paintbrushes that can paint detailed pictures in your
sensory workspace.} Visualize a ``triangular
lightbulb.'' What did you see? (Words as Mental
Paintbrush Handles)}

\item {
 \textit{You use a word that has different meanings in different
places as though it meant the same thing on each occasion, possibly
creating the illusion of something protean and shifting.}
``Martin told Bob the building was on his
left.'' But
``left'' is a function-word that
evaluates with a speaker-dependent variable grabbed from the
surrounding context. Whose ``left''
is meant, Bob's or Martin's? (Variable
Question Fallacies)}

\item {
 \textit{You think that definitions can't be
``wrong,'' or that
``I can define a word any way I
like!''} This kind of attitude teaches you to
indignantly defend your past actions, instead of paying attention to
their consequences, or fessing up to your mistakes. (37 Ways That
Suboptimal Use Of Categories Can Have Negative Side Effects On Your
Cognition)}
\end{enumerate}

{
 Everything you do in the mind has an effect, and your brain races
ahead unconsciously without your supervision.}

{
 Saying ``Words are arbitrary; I can define a word
any way I like'' makes around as much sense as
driving a car over thin ice with the accelerator floored and saying,
``Looking at this steering wheel, I
can't see why one radial angle is special---so I can
turn the steering wheel any way I like.''}

{
 If you're trying to go anywhere, or even just
trying to \textit{survive}, you had better start paying attention to
the three or six dozen optimality criteria that control how you use
words, definitions, categories, classes, boundaries, labels, and
concepts.}

\myendsectiontext

\mysectionnn{Interlude An Intuitive Explanation of Bayes's Theorem}
\label{intuitive_bayesian}

{
\textbf{\textit{Editor's
Note:}}\textit{ This is an abridgement of the
 original\footnote{\url{http://yudkowsky.net/rational/bayes}} version of this essay, which contained many
interactive elements.}}

{
\textit{ ~}}

{
\textit{\ } Your friends and colleagues are talking about something
called ``Bayes's
Theorem'' or
``Bayes's Rule,'' or
something called Bayesian reasoning. They sound really enthusiastic
about it, too, so you google and find a web page about
Bayes's Theorem and \ldots}

{
 It's this equation. That's all.
Just one equation. The page you found gives a definition of it, but it
doesn't say what it is, or why it's
useful, or why your friends would be interested in it. It looks like
this random statistics thing.}

{
 Why does a mathematical concept generate this strange enthusiasm
in its students? What is the so-called Bayesian Revolution now sweeping
through the sciences, which claims to subsume even the experimental
method itself as a special case? What is the secret that the adherents
of Bayes know? What is the light that they have seen?}

{
 Soon you will know. Soon you will be one of us.}

{
 While there are a few existing online explanations of
Bayes's Theorem, my experience with trying to introduce
people to Bayesian reasoning is that the existing online explanations
are too abstract. Bayesian reasoning is very \textit{counterintuitive}.
People do not employ Bayesian reasoning intuitively, find it very
difficult to learn Bayesian reasoning when tutored, and rapidly forget
Bayesian methods once the tutoring is over. This holds equally true for
novice students and highly trained professionals in a field. Bayesian
reasoning is apparently one of those things which, like quantum
mechanics or the Wason Selection Test, is inherently difficult for
humans to grasp with our built-in mental faculties.}

{
 Or so they claim. Here you will find an attempt to offer an
\textit{intuitive} explanation of Bayesian reasoning---an
excruciatingly gentle introduction that invokes all the human ways of
grasping numbers, from natural frequencies to spatial visualization.
The intent is to convey, not abstract rules for manipulating numbers,
but what the numbers mean, and why the rules are what they are (and
cannot possibly be anything else). When you are finished reading this,
you will see Bayesian problems in your dreams.}

{
 And let's begin.}

\hr

{
 Here's a story problem about a situation that
doctors often encounter:}

\begin{quote}
{
 1\% of women at age forty who participate in routine screening
have breast cancer. 80\% of women with breast cancer will get positive
mammographies. 9.6\% of women without breast cancer will also get
positive mammographies. A woman in this age group had a positive
mammography in a routine screening. What is the probability that she
actually has breast cancer?}
\end{quote}

{
 What do you think the answer is? If you haven't
encountered this kind of problem before, please take a moment to come
up with your own answer before continuing.}

\hr

{
 Next, suppose I told you that most doctors get the same wrong
answer on this problem---usually, only around 15\% of doctors get it
right. (``Really? 15\%? Is that a real number, or an
urban legend based on an Internet poll?''
It's a real number. See Casscells, Schoenberger, and
Graboys 1978;\footnote{Ward Casscells, Arno Schoenberger, and Thomas Graboys,
``Interpretation by Physicians of Clinical Laboratory
Results,'' \textit{New England Journal of Medicine}
299 (1978): 999--1001.\comment{1}} Eddy 1982;\footnote{David M. Eddy, ``Probabilistic Reasoning in
Clinical Medicine: Problems and Opportunities,'' in
\textit{Judgement Under Uncertainty: Heuristics and Biases}, ed. Daniel
Kahneman, Paul Slovic, and Amos Tversky (Cambridge University Press,
1982).\comment{2}}
Gigerenzer and Hoffrage 1995;\footnote{Gerd Gigerenzer and Ulrich Hoffrage, ``How to
Improve Bayesian Reasoning without Instruction: Frequency
Formats,'' \textit{Psychological Review} 102 (1995):
684--704.\comment{3}} and many other
studies. It's a surprising result which is easy to
replicate, so it's been extensively replicated.)}

{
 On the story problem above, most doctors estimate the probability
to be between 70\% and 80\%, which is wildly incorrect.}

{
 Here's an alternate version of the problem on
which doctors fare somewhat better:}

\begin{quote}
{
 10 out of 1,000 women at age forty who participate in routine
screening have breast cancer. 800 out of 1,000 women with breast cancer
will get positive mammographies. 96 out of 1,000 women without breast
cancer will also get positive mammographies. If 1,000 women in this age
group undergo a routine screening, about what fraction of women with
positive mammographies will actually have breast cancer?}
\end{quote}

{
 And finally, here's the problem on which doctors
fare best of all, with 46\%---nearly half---arriving at the correct
answer:}

\begin{quote}
{
 100 out of 10,000 women at age forty who participate in routine
screening have breast cancer. 80 of every 100 women with breast cancer
will get a positive mammography. 950 out of 9,900 women without breast
cancer will also get a positive mammography. If 10,000 women in this
age group undergo a routine screening, about what fraction of women
with positive mammographies will actually have breast cancer?}
\end{quote}

\hr

{
 The correct answer is 7.8\%, obtained as follows: Out of 10,000
women, 100 have breast cancer; 80 of those 100 have positive
mammographies. From the same 10,000 women, 9,900 will not have breast
cancer and of those 9,900 women, 950 will also get positive
mammographies. This makes the total number of women with positive
mammographies 950 + 80 or 1,030. Of those 1,030 women with positive
mammographies, 80 will have cancer. Expressed as a proportion, this is
80/1,030 or 0.07767 or 7.8\%.}

{
 To put it another way, before the mammography screening, the
10,000 women can be divided into two groups:}

\begin{itemize}
\item {
 Group 1: 100 women \textit{with} breast cancer.}

\item {
 Group 2: 9,900 women \textit{without} breast cancer.}
\end{itemize}

{
 Summing these two groups gives a total of 10,000 patients,
confirming that none have been lost in the math. After the mammography,
the women can be divided into four groups:}


\begin{itemize}
\item{
 Group $A$: 80 women \textit{with} breast cancer and a
\textit{positive} mammography.}

\item{
 Group $B$: 20 women \textit{with} breast cancer and a
\textit{negative} mammography.}

\item{
 Group $C$: 950 women \textit{without} breast cancer and a
\textit{positive} mammography.}

\item {
 Group $D$: 8,950 women \textit{without} breast cancer and a
 \textit{negative} mammography.}
\end{itemize}

{
 The sum of groups $A$ and $B$, the groups with breast cancer,
corresponds to group 1; and the sum of groups $C$ and $D$, the groups
without breast cancer, corresponds to group 2. If you administer a
mammography to 10,000 patients, then out of the 1,030 with positive
mammographies, eighty of those positive-mammography patients will have
cancer. This is the correct answer, the answer a doctor should give a
positive-mammography patient if she asks about the chance she has
breast cancer; if thirteen patients ask this question, roughly one out
of those thirteen will have cancer.}

\hr

{
 The most common mistake is to ignore the original fraction of
women with breast cancer, and the fraction of women without breast
cancer who receive false positives, and focus only on the fraction of
women with breast cancer who get positive results. For example, the
vast majority of doctors in these studies seem to have thought that if
around 80\% of women with breast cancer have positive mammographies,
then the probability of a women with a positive mammography having
breast cancer must be around 80\%.}

{
 Figuring out the final answer always requires \textit{all three}
pieces of information---the percentage of women with breast cancer, the
percentage of women without breast cancer who receive false positives,
and the percentage of women with breast cancer who receive (correct)
positives.}

{
 The original proportion of patients with breast cancer is known as
the \textit{prior probability}. The chance that a patient with breast
cancer gets a positive mammography, and the chance that a patient
without breast cancer gets a positive mammography, are known as the two
\textit{conditional probabilities}. Collectively, this initial
information is known as \textit{the priors}. The final answer---the
estimated probability that a patient has breast cancer, given that we
know she has a positive result on her mammography---is known as the
\textit{revised probability} or the \textit{posterior probability}.
What we've just seen is that the posterior probability
depends in part on the prior probability.}

{
 To see that the final answer always depends on the original
fraction of women with breast cancer, consider an alternate universe in
which only one woman out of a million has breast cancer. Even if
mammography in this world detects breast cancer in 8 out of 10 cases,
while returning a false positive on a woman without breast cancer in
only 1 out of 10 cases, there will still be a hundred thousand false
positives for every real case of cancer detected. The original
probability that a woman has cancer is so extremely low that, although
a positive result on the mammography does \textit{increase} the
estimated probability, the probability isn't increased
to certainty or even ``a noticeable
chance''; the probability goes from 1:1,000,000 to
1:100,000.}

{
 What this demonstrates is that the mammography result
doesn't \textit{replace} your old information about the
patient's chance of having cancer; the mammography
\textit{slides} the estimated probability in the direction of the
result. A positive result slides the original probability upward; a
negative result slides the probability downward. For example, in the
original problem where 1\% of the women have cancer, 80\% of women with
cancer get positive mammographies, and 9.6\% of women without cancer
get positive mammographies, a positive result on the mammography
\textit{slides} the 1\% chance upward to 7.8\%.}

{
 Most people encountering problems of this type for the first time
carry out the mental operation of \textit{replacing} the original 1\%
probability with the 80\% probability that a woman with cancer gets a
positive mammography. It may seem like a good idea, but it just
doesn't work. ``The probability that a
woman with a positive mammography has breast cancer''
is not at all the same thing as ``the probability that
a woman with breast cancer has a positive
mammography''; they are as unlike as apples and
cheese.}

\hr

\begin{quote}
{
 \textbf{Q. Why did the Bayesian reasoner cross the road?}}

{
  A. You need more information to answer this question.}
\end{quote}

\hr

{
 Suppose that a barrel contains many small plastic eggs. Some eggs
are painted red and some are painted blue. 40\% of the eggs in the bin
contain pearls, and 60\% contain nothing. 30\% of eggs containing
pearls are painted blue, and 10\% of eggs containing nothing are
painted blue. What is the probability that a blue egg contains a pearl?
For this example the arithmetic is simple enough that you may be able
to do it in your head, and I would suggest trying to do so.}

{
 A more compact way of specifying the problem:}

\begin{align*}
 P(\text{pearl}) &= 40\% \\
 P(\text{blue}|\text{pearl}) &= 30\% \\
 P(\text{blue}|\lnot\text{pearl}) &= 10\% \\
 P(\text{pearl}|\text{blue}) &= ?
\end{align*}


{
 The symbol ``{\textlnot}'' is
shorthand for ``not,'' so
{\textlnot}pearl reads ``not
pearl.''}

{
 The notation $P(\text{blue}|\text{pearl})$ is shorthand for
``the probability of blue given
pearl'' or ``the probability that an
egg is painted blue, given that the egg contains a
pearl.'' The item on the right side is what you
\textit{already know} or the \textit{premise}, and the item on the left
side is the \textit{implication} or \textit{conclusion}. If we have
$P(\text{blue}|\text{pearl}) = 30\%$, and we \textit{already know} that some
egg contains a pearl, then we can \textit{conclude} there is a 30\%
chance that the egg is painted blue. Thus, the final fact
we're looking for---``the chance that
a blue egg contains a pearl'' or
``the probability that an egg contains a pearl, if we
know the egg is painted blue''---reads
$P(\text{pearl}|\text{blue})$.}

{
 40\% of the eggs contain pearls, and 60\% of the eggs contain
nothing. 30\% of the eggs containing pearls are painted blue, so 12\%
of the eggs altogether contain pearls and are painted blue. 10\% of the
eggs containing nothing are painted blue, so altogether 6\% of the eggs
contain nothing and are painted blue. A total of 18\% of the eggs are
painted blue, and a total of 12\% of the eggs are painted blue and
contain pearls, so the chance a blue egg contains a pearl is 12/18 or
2/3 or around 67\%.}

{
 As before, we can see the necessity of all three pieces of
information by considering extreme cases. In a (large) barrel in which
only one egg out of a thousand contains a pearl, knowing that an egg is
painted blue slides the probability from 0.1\% to 0.3\% (instead of
sliding the probability from 40\% to 67\%). Similarly, if 999 out of
1,000 eggs contain pearls, knowing that an egg is blue slides the
probability from 99.9\% to 99.966\%; the probability that the egg does
\textit{not} contain a pearl goes from 1/1,000 to around 1/3,000.}

{
 On the pearl-egg problem, most respondents unfamiliar with
Bayesian reasoning would probably respond that the probability a blue
egg contains a pearl is 30\%, or perhaps 20\% (the 30\% chance of a
true positive minus the 10\% chance of a false positive). Even if this
mental operation seems like a good idea at the time, it makes no sense
in terms of the question asked. It's like the
experiment in which you ask a second-grader: ``If
eighteen people get on a bus, and then seven more people get on the
bus, how old is the bus driver?'' Many second-graders
will respond: ``Twenty-five.'' They
understand when they're being prompted to carry out a
particular mental procedure, but they haven't quite
connected the procedure to reality. Similarly, to find the probability
that a woman with a positive mammography has breast cancer, it makes no
sense whatsoever to \textit{replace} the original probability that the
woman has cancer with the probability that a woman with breast cancer
gets a positive mammography. Neither can you subtract the probability
of a false positive from the probability of the true positive. These
operations are as wildly irrelevant as adding the number of people on
the bus to find the age of the bus driver.}

\hr

{
 A study by Gigerenzer and Hoffrage in 1995 showed that some ways
of phrasing story problems are much more evocative of correct Bayesian
reasoning.\footnote{Ibid.\comment{4}} The \textit{least} evocative phrasing
used probabilities. A slightly more evocative phrasing used frequencies
instead of probabilities; the problem remained the same, but instead of
saying that 1\% of women had breast cancer, one would say that 1 out of
100 women had breast cancer, that 80 out of 100 women with breast
cancer would get a positive mammography, and so on. Why did a higher
proportion of subjects display Bayesian reasoning on this problem?
Probably because saying ``1 out of 100
women'' encourages you to concretely visualize X
women with cancer, leading you to visualize X women with cancer and a
positive mammography, etc.}

{
 The most effective presentation found so far is
what's known as \textit{natural frequencies}{}---saying
that 40 out of 100 eggs contain pearls, 12 out of 40 eggs containing
pearls are painted blue, and 6 out of 60 eggs containing nothing are
painted blue. A \textit{natural frequencies} presentation is one in
which the information about the prior probability is included in
presenting the conditional probabilities. If you were just learning
about the eggs' conditional probabilities through
natural experimentation, you would---in the course of cracking open a
hundred eggs---crack open around 40 eggs containing pearls, of which 12
eggs would be painted blue, while cracking open 60 eggs containing
nothing, of which about 6 would be painted blue. In the course of
learning the conditional probabilities, you'd see
examples of blue eggs containing pearls about twice as often as you saw
examples of blue eggs containing nothing.}

{
 Unfortunately, while natural frequencies are a step in the right
direction, it probably won't be enough. When problems
are presented in natural frequencies, the proportion of people using
Bayesian reasoning rises to around half. A big improvement, but not big
enough when you're talking about real doctors and real
patients.}

\hr

\begin{quote}
{
 \textbf{Q. How can I find the priors for a problem?}\newline
 A. Many commonly used priors are listed in the \textit{Handbook of
Chemistry and Physics}.}

{
 \textbf{Q. Where do priors }\textbf{\textit{originally}}\textbf{
come from?}\newline
 A. Never ask that question.}

{
 \textbf{Q. Uh huh. Then where do scientists get their
priors?}\newline
 A. Priors for scientific problems are established by annual vote of the
AAAS. In recent years the vote has become fractious and controversial,
with widespread acrimony, factional polarization, and several outright
assassinations. This may be a front for infighting within the Bayes
Council, or it may be that the disputants have too much spare time. No
one is really sure.}

{
 \textbf{Q. I see. And where does everyone else get their
priors?}\newline
 A. They download their priors from Kazaa.}

{
 \textbf{Q. What if the priors I want aren't
available on Kazaa?}\newline
 A. There's a small, cluttered antique shop in a back
alley of San Francisco's Chinatown.
\textit{Don't ask about the bronze rat.}}
\end{quote}


{
 Actually, priors are true or false just like the final
answer---they reflect reality and can be judged by comparing them
against reality. For example, if you think that 920 out of 10,000 women
in a sample have breast cancer, and the actual number is 100 out of
10,000, then your priors are wrong. For our particular problem, the
priors might have been established by three studies---a study on the
case histories of women with breast cancer to see how many of them
tested positive on a mammography, a study on women without breast
cancer to see how many of them test positive on a mammography, and an
epidemiological study on the prevalence of breast cancer in some
specific demographic.}

\hr

{
 The probability $P(A,B)$ is the same as $P(B,A)$, but $P(A|B)$
is not the same thing as $P(B|A)$, and $P(A,B)$ is completely
different from $P(A|B)$. It's a common confusion
to mix up some or all of these quantities.}

{
 To get acquainted with all the relationships between them,
we'll play ``follow the degrees of
freedom.'' For example, the two quantities $P(\text{cancer})$
and $P(\lnot\text{cancer})$ have one degree of freedom between them,
because of the general law $P(A) + P(\lnot A) = 1$. If you know that
$P(\lnot\text{cancer}) = 0.99$, you can obtain $P(\text{cancer}) = 1 -
P(\lnot\text{cancer}) = 0.01$.}

{
 The quantities $P(\text{positive}|\text{cancer})$ and
$P(\lnot\text{positive}|\text{cancer})$ also have only one degree of
freedom between them; either a woman with breast cancer gets a positive
mammography or she doesn't. On the other hand,
$P(\text{positive}|\text{cancer})$ and $P(\text{positive}|\lnot\text{cancer})$
have \textit{two} degrees of freedom. You can have a mammography test
that returns positive for 80\% of cancer patients and 9.6\% of healthy
patients, or that returns positive for 70\% of cancer patients and 2\%
of healthy patients, or even a health test that returns
``positive'' for 30\% of cancer
patients and 92\% of healthy patients. The two quantities, the output
of the mammography test for cancer patients and the output of the
mammography test for healthy patients, are in mathematical terms
independent; one cannot be obtained from the other in any way, and so
they have two degrees of freedom between them.}

{
 What about $P(\text{positive}, \text{cancer})$, $P(\text{positive}|\text{cancer})$, and
$P(\text{cancer})$? Here we have three quantities; how many degrees of freedom
are there? In this case the equation that must hold is}

\begin{equation*}
 P(\text{positive}, \text{cancer}) = P(\text{positive}|\text{cancer}) \times
P(\text{cancer}).
\end{equation*}

{
 This equality reduces the degrees of freedom by one. If we know
the fraction of patients with cancer, and the chance that a cancer
patient has a positive mammography, we can deduce the fraction of
patients who have breast cancer \textit{and} a positive mammography by
multiplying.}

{
 Similarly, if we know the number of patients with breast cancer
and positive mammographies, and also the number of patients with breast
cancer, we can estimate the chance that a woman with breast cancer gets
a positive mammography by dividing: $P(\text{positive}|\text{cancer}) =
P(\text{positive}, \text{cancer})/P(\text{cancer})$. In fact, this is exactly how such
medical diagnostic tests are calibrated; you do a study on 8,520 women
with breast cancer and see that there are 6,816 (or thereabouts) women
with breast cancer and positive mammographies, then divide 6,816 by
8,520 to find that 80\% of women with breast cancer had positive
mammographies. (Incidentally, if you accidentally divide 8,520 by 6,816
instead of the other way around, your calculations will start doing
strange things, such as insisting that 125\% of women with breast
cancer and positive mammographies have breast cancer. This is a common
mistake in carrying out Bayesian arithmetic, in my experience.) And
finally, if you know $P(\text{positive}, \text{cancer})$ and
$P(\text{positive}|\text{cancer})$, you can deduce how many cancer patients
there must have been originally. There are two degrees of freedom
shared out among the three quantities; if we know any two, we can
deduce the third.}

{
 How about $P(\text{positive})$, $P(\text{positive}, \text{cancer})$, and
$P(\text{positive},\lnot\text{cancer})$? Again there are only two degrees of
freedom among these three variables. The equation occupying the extra
degree of freedom is}

\whencolumns{
\begin{equation*}
 P(\text{positive}) = P(\text{positive}, \text{cancer}) + P(\text{positive},\lnot\text{cancer}).
\end{equation*}
}{
\begin{align*}
  &P(\text{positive}) = \\
  &P(\text{positive}, \text{cancer}) + P(\text{positive},\lnot\text{cancer}).
\end{align*}
}

{
 This is how $P(\text{positive})$ is computed to begin with; we figure out
the number of women with breast cancer who have positive mammographies,
and the number of women without breast cancer who have positive
mammographies, then add them together to get the total number of women
with positive mammographies. It would be very strange to go out and
conduct a study to determine the number of women with positive
mammographies---just that one number and nothing else---but in theory
you could do so. And if you then conducted another study and found the
number of those women who had positive mammographies \textit{and}
breast cancer, you would also know the number of women with positive
mammographies and \textit{no} breast cancer---either a woman with a
positive mammography has breast cancer or she doesn't.
In general, $P(A,B) + P(A,\lnot B) = P(A)$. Symmetrically, $P(A,B) +
P(\lnot A,B) = P(B)$.}

{
 What about $P(\text{positive}, \text{cancer})$, $P(\text{positive},\lnot\text{cancer})$,
$P(\lnot\text{positive}, \text{cancer})$, and
$P(\lnot\text{positive},\lnot\text{cancer})$? You might at first be tempted
to think that there are only two degrees of freedom for these four
quantities---that you can, for example, get
$P(\text{positive},\lnot\text{cancer})$ by multiplying $P(\text{positive}) \times
P(\lnot\text{cancer})$, and thus that all four quantities can be found
given only the two quantities $P(\text{positive})$ and $P(\text{cancer})$. This is not
the case! $P(\text{positive},\lnot\text{cancer}) = P(\text{positive}) \times
P(\lnot\text{cancer})$ only if the two probabilities are
\textit{statistically independent}{}---if the chance that a woman has
breast cancer has no bearing on whether she has a positive mammography.
This amounts to requiring that the two conditional probabilities be
equal to each other---a requirement which would eliminate one degree of
freedom. If you remember that these four quantities are the groups $A$,
$B$, $C$, and $D$, you can look over those four groups and realize that, in
theory, you can put any number of people into the four groups. If you
start with a group of 80 women with breast cancer and positive
mammographies, there's no reason why you
can't add another group of 500 women with breast cancer
and negative mammographies, followed by a group of 3 women without
breast cancer and negative mammographies, and so on. So now it seems
like the four quantities have four degrees of freedom. And they would,
except that in expressing them as \textit{probabilities}, we need to
normalize them to \textit{fractions} of the complete group, which adds
the constraint that $P(\text{positive},
\text{cancer})+P(\text{positive},\lnot\text{cancer})+P(\lnot\text{positive},
\text{cancer})+P(\lnot\text{positive},\lnot\text{cancer}) = 1$. This equation
takes up one degree of freedom, leaving three degrees of freedom among
the four quantities. If you specify the \textit{fractions} of women in
groups $A$, $B$, and $D$, you can deduce the fraction of women in group $C$.}

{
 Given the four groups $A$, $B$, $C$, and $D$, it is very straightforward
to compute everything else:}

\begin{align*}
 P(\text{cancer}) &= \frac{A + B}{A + B + C + D} \\
 P(\lnot\text{positive}|\text{cancer}) &= \frac{B}{A + B},
\end{align*}

{
 and so on. Since $\{A,B,C,D\}$
contains three degrees of freedom, it follows that the entire set of
probabilities relating cancer rates to test results contains only three
degrees of freedom. Remember that in our problems we always needed
\textit{three} pieces of information---the prior probability and the
two conditional probabilities---which, indeed, have three degrees of
freedom among them. Actually, for Bayesian problems, \textit{any} three
quantities with three degrees of freedom between them should logically
specify the entire problem.}

\hr

{
 \textit{The probability that a test gives a true positive} divided
by \textit{the probability that a }\textit{test gives a false positive}
is known as the \textit{likelihood ratio} of that test. The likelihood
ratio for a positive result summarizes how much a positive result will
slide the prior probability. Does the likelihood ratio of a medical
test then sum up everything there is to know about the usefulness of
the test?}

{
 No, it does not! The likelihood ratio sums up everything there is
to know about the \textit{meaning} of a \textit{positive} result on the
medical test, but the meaning of a \textit{negative} result on the test
is not specified, nor is the frequency with which the test is useful.
For example, a mammography with a hit rate of 80\% for patients with
breast cancer and a false positive rate of 9.6\% for healthy patients
has the same likelihood ratio as a test with an 8\% hit rate and a
false positive rate of 0.96\%. Although these two tests have the same
likelihood ratio, the first test is more useful in every way---it
detects disease more often, and a negative result is stronger evidence
of health.}

\hr
{
 Suppose that you apply \textit{two} tests for breast cancer in
succession---say, a standard mammography and also some other test which
is \textit{independent} of mammography. Since I don't
know of any such test that is independent of mammography,
I'll invent one for the purpose of this problem, and
call it the Tams-Braylor Division Test, which checks to see if any
cells are dividing more rapidly than other cells. We'll
suppose that the Tams-Braylor gives a true positive for 90\% of
patients with breast cancer, and gives a false positive for 5\% of
patients without cancer. Let's say the prior prevalence
of breast cancer is 1\%. If a patient gets a positive result on her
mammography \textit{and} her Tams-Braylor, what is the revised
probability she has breast cancer?}

{
 One way to solve this problem would be to take the revised
probability for a positive mammography, which we already calculated as
7.8\%, and plug that into the Tams-Braylor test as the new prior
probability. If we do this, we find that the result comes out to 60\%.}

{
 Suppose that the prior prevalence of breast cancer in a
demographic is 1\%. Suppose that we, as doctors, have a repertoire of
three independent tests for breast cancer. Our first test, test $A$, a
mammography, has a likelihood ratio of 80\%/9.6\% = 8.33. The second
test, test $B$, has a likelihood ratio of 18.0 (for example, from 90\%
versus 5\%); and the third test, test $C$, has a likelihood ratio of 3.5
(which could be from 70\% versus 20\%, or from 35\% versus 10\%; it
makes no difference). Suppose a patient gets a positive result on all
three tests. What is the probability the patient has breast cancer?}

{
 Here's a fun trick for simplifying the
bookkeeping. If the prior prevalence of breast cancer in a demographic
is 1\%, then 1 out of 100 women have breast cancer, and 99 out of 100
women do not have breast cancer. So if we rewrite the
\textit{probability} of 1\% as an \textit{odds ratio}, the odds are
1:99.}

{
 And the likelihood ratios of the three tests A, B, and C are:}

\begin{align*}
 8.33:1 &= 25:3\\
 18.0:1 &= 18:1\\
 3.5:1 &= 7.5:2.
\end{align*}

{
 The \textit{odds} for women with breast cancer who score positive
on all three tests, versus women without breast cancer who score
positive on all three tests, will equal:}

\begin{equation*}
 1 \times 25 \times 18 \times 7 : 99 \times 3 \times 1 \times 2 = 3150 : 594.
\end{equation*}

{
 To recover the probability from the odds, we just write:}

\begin{equation*}
 3150 / (3150 + 594) = 84\%.
\end{equation*}


{
 This always works regardless of how the odds ratios are written;
i.e., 8.33:1 is just the same as 25:3 or 75:9. It
doesn't matter in what order the tests are
administered, or in what order the results are computed. The proof is
left as an exercise for the reader.}

\hr

{
 E. T. Jaynes, in \textit{Probability Theory With Applications in
Science and Engineering}, suggests that credibility and evidence should
be measured in decibels.\footnote{Edwin T. Jaynes, ``Probability Theory, with
Applications in Science and Engineering,''
Unpublished manuscript (1974).\comment{5}}}

{
 Decibels?}

{
 Decibels are used for measuring exponential differences of
intensity. For example, if the sound from an automobile horn carries
10,000 times as much energy (per square meter per second) as the sound
from an alarm clock, the automobile horn would be 40 decibels louder.
The sound of a bird singing might carry 1,000 times less energy than an
alarm clock, and hence would be 30 decibels softer. To get the number
of decibels, you take the logarithm base 10 and multiply by 10:}

\begin{equation*}
  \text{decibels} = 10\log_{10}(\text{intensity})
\end{equation*}

{
 or}

\begin{equation*}
  \text{intensity} = 10^{\text{decibels}/10}.
\end{equation*}

{
 Suppose we start with a prior probability of 1\% that a woman has
breast cancer, corresponding to an odds ratio of 1:99. And then we
administer three tests of likelihood ratios 25:3, 18:1, and 7:2. You
\textit{could} multiply those numbers \ldots or you could just add their
logarithms:}

\begin{align*}
 10\log_{10}(1/99) &\approx -20 \\
 10\log_{10}(25/3) &\approx 9 \\
 10\log_{10}(18/1) &\approx 13 \\
 10\log_{10}(7/2) &\approx 5.
\end{align*}

{
 It starts out as fairly unlikely that a woman has breast
cancer---our credibility level is at -20 decibels. Then three test
results come in, corresponding to 9, 13, and 5 decibels of evidence.
This raises the credibility level by a total of 27 decibels, meaning
that the prior credibility of -20 decibels goes to a posterior
credibility of 7 decibels. So the odds go from 1:99 to 5:1, and the
probability goes from 1\% to around 83\%.}

\hr

\begin{quote}
{
 You are a mechanic for gizmos. When a gizmo stops working, it is
due to a blocked hose 30\% of the time. If a gizmo's
hose is blocked, there is a 45\% probability that prodding the gizmo
will produce sparks. If a gizmo's hose is unblocked,
there is only a 5\% chance that prodding the gizmo will produce sparks.
A customer brings you a malfunctioning gizmo. You prod the gizmo and
find that it produces sparks. What is the probability that a
spark-producing gizmo has a blocked hose?}
\end{quote}

{
 What is the sequence of arithmetical operations that you performed
to solve this problem?}

\begin{equation*}
 (45\% \times 30\%) / (45\% \times 30\% + 5\% \times 70\%)
\end{equation*}

{
 Similarly, to find the chance that a woman with positive
mammography has breast cancer, we computed:}

\begin{equation*}
  \frac{P(\text{positive}|\text{cancer})\times P(\text{cancer})}
       {\left(\begin{array}{rcl}
           P(\text{positive}|\text{cancer}) & \times & P(\text{cancer}) \\
           +\, P(\text{positive}|\lnot \text{cancer}) & \times & P(\lnot \text{cancer}) \\
         \end{array}
         \right)}
\end{equation*}


{
 which is}

\begin{equation*}
  \frac{P(\text{positive},\text{cancer})}
  {P(\text{positive},\text{cancer}) + P(\text{positive},\lnot\text{cancer})}
\end{equation*}

{
 which is}

\begin{equation*}
  P(\text{positive},\text{cancer}) / P(\text{positive})
\end{equation*}


{
 which is}

\begin{equation*}
  P(\text{cancer}|\text{positive}).
\end{equation*}

{
 The fully general form of this calculation is known as
\textit{Bayes's Theorem} or
\textit{Bayes's Rule}.}

{\centering
\mygraphics{images/img227.jpg}
\par}

\begin{equation*}
  P(A|X) = \frac{P(X|A) \times P(A)}
  {P(X|A) \times P(A) + P(X|\lnot A)\times P(\lnot A)} .
\end{equation*}



{
 When there is some phenomenon $A$ that we want to investigate, and
an observation $X$ that is evidence about $A$---for example, in the
previous example, $A$ is breast cancer and $X$ is a positive
mammography---Bayes's Theorem tells us how we should
\textit{update} our probability of $A$, given the \textit{new evidence}
$X$.}

{
 By this point, Bayes's Theorem may seem blatantly
obvious or even tautological, rather than exciting and new. If so, this
introduction has \textit{entirely succeeded} in its purpose.}

\hr

{
 Bayes's Theorem describes what makes something
``evidence'' and how much evidence
it is. Statistical models are judged by comparison to the
\textit{Bayesian method} because, in statistics, the Bayesian method is
as good as it gets---the Bayesian method defines the maximum amount of
mileage you can get out of a given piece of evidence, in the same way
that thermodynamics defines the maximum amount of work you can get out
of a temperature differential. This is why you hear cognitive
scientists talking about \textit{Bayesian reasoners}. In cognitive
science, \textit{Bayesian reasoner} is the technically precise code
word that we use to mean \textit{rational mind}.}

{
 There are also a number of general heuristics about human
reasoning that you can learn from looking at Bayes's
Theorem.}

{
 For example, in many discussions of Bayes's
Theorem, you may hear cognitive psychologists saying that people
\textit{do not take prior frequencies sufficiently into account},
meaning that when people approach a problem where
there's some evidence $X$ indicating that condition $A$
might hold true, they tend to judge $A$'s likelihood
solely by how well the evidence $X$ seems to match $A$, without taking into
account the prior frequency of $A$. If you think, for example, that under
the mammography example, the woman's chance of having
breast cancer is in the range of 70\%--80\%, then this kind of
reasoning is insensitive to the prior frequency given in the problem;
it doesn't notice whether 1\% of women or 10\% of women
start out having breast cancer. ``Pay more attention
to the prior frequency!'' is one of the many things
that humans need to bear in mind to partially compensate for our
built-in inadequacies.}

{
 A related error is to pay too much attention to $P(X|A)$
and not enough to $P(X|\lnot A)$ when determining how much
evidence $X$ is for $A$. The degree to which a result $X$ is \textit{evidence
for $A$} depends not only on the strength of the statement
\textit{we'd expect to see result $X$ if $A$ were true},
but also on the strength of the statement \textit{we
}\textbf{\textit{wouldn't}}\textit{ expect to see
result $X$ if $A$ weren't true.} For example, if it is
raining, this very strongly implies the grass is
wet---$P(\text{wetgrass}|\text{rain}) \approx 1$---but seeing that the
grass is wet doesn't necessarily mean that it has just
rained; perhaps the sprinkler was turned on, or you're
looking at the early morning dew. Since
$P(\text{wetgrass}|\lnot\text{rain})$ is substantially greater than
zero, $P(\text{rain}|\text{wetgrass})$ is substantially less than one. On the
other hand, if the grass was \textit{never} wet when it
wasn't raining, then knowing that the grass was wet
would \textit{always} show that it was raining,
$P(\text{rain}|\text{wetgrass}) \approx 1$, even if
$P(\text{wetgrass}|\text{rain}) = 50\%$; that is, even if the grass only got
wet 50\% of the times it rained. Evidence is always the result of the
\textit{differential} between the two conditional probabilities.
\textit{Strong} evidence is not the product of a very high probability
that $A$ leads to $X$, but the product of a very \textit{low} probability
that \textit{not}{}-$A$ could have led to $X$.}

{
 The \textit{Bayesian revolution in the sciences} is fueled, not
only by more and more cognitive scientists suddenly noticing that
mental phenomena have Bayesian structure in them; not only by
scientists in every field learning to judge their statistical methods
by comparison with the Bayesian method; but also by the idea that
\textit{science itself is a special case of Bayes's
Theorem; experimental evidence is Bayesian evidence}. The Bayesian
revolutionaries hold that when you perform an experiment and get
evidence that ``confirms'' or
``disconfirms'' your theory, this
confirmation and disconfirmation is governed by the Bayesian rules. For
example, you have to take into account not only whether your theory
predicts the phenomenon, but whether other possible explanations also
predict the phenomenon.}

{
 Previously, the most popular philosophy of science was probably
Karl Popper's \textit{falsificationism}{}---this is the
old philosophy that the Bayesian revolution is currently dethroning.
Karl Popper's idea that theories can be definitely
falsified, but never definitely confirmed, is yet another special case
of the Bayesian rules; if $P(X|A) \approx 1$---if the theory
makes a definite prediction---then observing $\lnot X$ very strongly
falsifies $A$. On the other hand, if $P(X|A) \approx 1$, and
we observe $X$, this doesn't definitely confirm the
theory; there might be some other condition $B$ such that $P(X|B)
\approx 1$, in which case observing $X$ doesn't favor
$A$ over $B$. For observing $X$ to definitely confirm $A$, we would have to
know, not that $P(X|A) \approx 1$, but that
$P(X|\lnot A) \approx 0$, which is something that we
can't know because we can't range over
all possible alternative explanations. For example, when
Einstein's theory of General Relativity toppled
Newton's incredibly well-confirmed theory of gravity,
it turned out that all of Newton's predictions were
just a special case of Einstein's predictions.}

{
 You can even formalize Popper's philosophy
mathematically. The likelihood ratio for $X$, the quantity
$P(X|A)/P(X|\lnot A)$, determines how much
observing $X$ slides the probability for $A$; the likelihood ratio is what
says \textit{how strong} $X$ is as evidence. Well, in your theory $A$, you
can predict $X$ with probability 1, if you like; but you
can't control the denominator of the likelihood ratio,
$P(X|\lnot A)$---there will always be some alternative
theories that also predict $X$, and while we go with the simplest theory
that fits the current evidence, you may someday encounter some evidence
that an alternative theory predicts but your theory does not.
That's the hidden gotcha that toppled
Newton's theory of gravity. So there's
a limit on how much mileage you can get from successful predictions;
there's a limit on how high the likelihood ratio goes
for \textit{confirmatory} evidence.}

{
 On the other hand, if you encounter some piece of evidence $Y$ that
is definitely \textit{not} predicted by your theory, this is
\textit{enormously} strong evidence against your theory. If
$P(Y|A)$ is infinitesimal, then the likelihood ratio will also
be infinitesimal. For example, if $P(Y|A)$ is 0.0001\%, and
$P(Y|\lnot A)$ is 1\%, then the likelihood ratio
$P(Y|A)/P(Y|\lnot A)$ will be 1:10,000.
That's -40 decibels of evidence! Or, flipping the
likelihood ratio, if $P(Y|A)$ is \textit{very small}, then
$P(Y|\lnot A)/P(Y|A)$ will be very large, meaning
that observing $Y$ greatly favors $\lnot A$ over $A$. Falsification is
much stronger than confirmation. This is a consequence of the earlier
point that \textit{very strong} evidence is not the product of a very
high probability that $A$ leads to $X$, but the product of a very
\textit{low} probability that \textit{not-A} could have led to $X$. This
is the precise Bayesian rule that underlies the heuristic value of
Popper's falsificationism.}

{
 Similarly, Popper's dictum that an idea must be
falsifiable can be interpreted as a manifestation of the Bayesian
conservation-of-probability rule; if a result $X$ is positive evidence
for the theory, then the result $\lnot X$ would have disconfirmed
the theory to some extent. If you try to interpret both $X$ and
$\lnot X$ as ``confirming'' the
theory, the Bayesian rules say this is impossible! To increase the
probability of a theory you \textit{must} expose it to tests that can
potentially decrease its probability; this is not just a rule for
detecting would-be cheaters in the social process of science, but a
consequence of Bayesian probability theory. On the other hand,
Popper's idea that there is \textit{only} falsification
and \textit{no such thing} as confirmation turns out to be incorrect.
Bayes's Theorem shows that falsification is
\textit{very strong} evidence compared to confirmation, but
falsification is still probabilistic in nature; it is not governed by
fundamentally different rules from confirmation, as Popper argued.}

{
 So we find that many phenomena in the cognitive sciences, plus the
statistical methods used by scientists, plus the scientific method
itself, are all turning out to be special cases of
Bayes's Theorem. Hence the Bayesian revolution.}

\hr

{
 Having introduced Bayes's Theorem explicitly, we
can explicitly discuss its components.}

\begin{equation*}
  P(A|X) = \frac{P(X|A) \times P(A)}
  {P(X|A) \times P(A) + P(X|\lnot A)\times P(\lnot A)}
\end{equation*}


{
 We'll start with $P(A|X)$. If you ever find
yourself getting confused about what's $A$ and
what's $X$ in Bayes's Theorem, start with
$P(A|X)$ on the left side of the equation;
that's the simplest part to interpret. In
$P(A|X)$, $A$ is the thing we want to know about. $X$ is how
we're observing it; $X$ is the evidence
we're using to make inferences about $A$. Remember that
for every expression $P(Q|P)$, we want to know about the
probability for $Q$ given $P$, the degree to which $P$ implies $Q$---a more
sensible notation, which it is now too late to adopt, would be $P(Q \leftarrow P)$.}

{
 $P(Q|P)$ is closely related to $P(Q,P)$, but they are not
identical. Expressed as a probability or a fraction, $P(Q,P)$ is the
proportion of things that have property $Q$ and property $P$ among all
things; e.g., the proportion of ``women with breast
cancer and a positive mammography'' within the group
of all women. If the total number of women is 10,000, and 80 women have
breast cancer and a positive mammography, then $P(Q,P)$ is 80/10,000 =
0.8\%. You might say that the absolute quantity, 80, is being
normalized to a probability relative to the group of all women. Or to
make it clearer, suppose that there's a group of 641
women with breast cancer and a positive mammography within a total
sample group of 89,031 women. Six hundred and forty-one is the absolute
quantity. If you pick out a random woman from the entire sample, then
the probability you'll pick a woman with breast cancer
and a positive mammography is $P(Q,P)$, or 0.72\% (in this example).}

{
 On the other hand, $P(Q|P)$ is the proportion of things
that have property $Q$ and property $P$ among \textit{all things that have
P}; e.g., the proportion of women with breast cancer and a positive
mammography within the group of \textit{all women with positive
mammographies}. If there are 641 women with breast cancer and positive
mammographies, 7,915 women with positive mammographies, and 89,031
women, then $P(Q,P)$ is the probability of getting one of those 641 women
if you're picking at random from the entire group of
89,031, while $P(Q|P)$ is the probability of getting one of
those 641 women if you're picking at random from the
smaller group of 7,915.}

{
 In a sense, $P(Q|P)$ really means $P(Q,P|P)$, but
specifying the extra $P$ all the time would be redundant. You already
\textit{know} it has property $P$, so the property you're
\textit{investigating} is $Q$---even though you're
looking at the size of group $(Q,P)$ within group $P$, not the size of
group $Q$ within group $P$ (which would be nonsense). This is what it means
to take the property on the right-hand side as \textit{given}; it means
you know you're working only within the group of things
that have property $P$. When you constrict your focus of attention to see
only this smaller group, many other probabilities change. If
you're taking $P$ as \textit{given}, then $P(Q,P)$ equals
just $P(Q)$---at least, \textit{relative to the group $P$}. The
\textit{old} $P(Q)$, the frequency of ``things that have
property $Q$ within the entire sample,'' is revised to
the new frequency of ``things that have property $Q$
within the subsample of things that have property
$P$.'' If $P$ is \textit{given}, if $P$ is our entire
world, then looking for $(Q,P)$ is the same as looking for just $Q$.}

{
 If you constrict your focus of attention to only the population of
eggs that are painted blue, then suddenly ``the
probability that an egg contains a pearl'' becomes a
different number; this proportion is different for the population of
blue eggs than the population of all eggs. The \textit{given}, the
property that constricts our focus of attention, is always on the
\textit{right} side of $P(Q|P)$; the $P$ becomes our world, the
entire thing we see, and on the other side of the
``given'' $P$ always has probability
1---that is what it means to take $P$ as given. So $P(Q|P)$ means
``If P has probability 1, what is the probability of
$Q$?'' or ``If we constrict our
attention to only things or events where $P$ is true, what is the
probability of $Q$?'' The statement $Q$, on the other
side of the given, is \textit{not} certain---its probability may be
10\% or 90\% or any other number. So when you use
Bayes's Theorem, and you write the part on the left
side as $P(A|X)$---how to \textit{update} the probability of $A$
after seeing $X$, the new probability of $A$ \textit{given} that we know $X$,
the degree to which $X$ \textit{implies} $A$---you can tell that $X$ is
always the \textit{observation} or the \textit{evidence}, and $A$ is the
property being investigated, the thing you want to know about.}

\hr

{
 The right side of Bayes's Theorem is derived from
the left side through these steps:}

\begin{align*}
  P(A|X) &= P(A|X) \\
  P(A|X) &= \frac{P(X,A)}{P(X)} \\
  P(A|X) &= \frac{P(X,A)}{P(X,A)+P(X,\lnot A)} \\
  P(A|X) &= \frac{P(X|A) \times P(A)}
  {P(X|A) \times P(A) + P(X|\lnot A) \times P(\lnot A)}.
\end{align*}


{
 Once the derivation is finished, all the implications on the right
side of the equation are of the form $P(X|A)$ or
$P(X|\lnot A)$, while the implication on the left side is
$P(A|X)$. The symmetry arises because the elementary
\textit{causal relations} are generally implications from facts to
observations, e.g., from breast cancer to positive mammography. The
elementary \textit{steps in reasoning} are generally implications from
observations to facts, e.g., from a positive mammography to breast
cancer. The left side of Bayes's Theorem is an
elementary \textit{inferential} step from the observation of positive
mammography to the conclusion of an increased probability of breast
cancer. Implication is written right-to-left, so we write
$P(\text{cancer}|\text{positive})$ on the left side of the equation. The right
side of Bayes's Theorem describes the elementary
\textit{causal} steps---for example, from breast cancer to a positive
mammography---and so the implications on the right side of
Bayes's Theorem take the form
$P(\text{positive}|\text{cancer})$ or $P(\text{positive}|\lnot \text{cancer})$.}

{
 And that's Bayes's Theorem.
Rational inference on the left end, physical causality on the right
end; an equation with mind on one side and reality on the other.
Remember how the scientific method turned out to be a special case of
Bayes's Theorem? If you wanted to put it poetically,
you could say that Bayes's Theorem binds reasoning into
the physical universe.}

{
 Okay, we're done.}

{\centering
 Reverend Bayes says:\newline
 
\mygraphics{images/img231.jpg}
 \newline
 You are now an initiate of the Bayesian Conspiracy.
\par}


\bigskip

\myendsectiontext


\bigskip




