\renewcommand\thechapter{\Alph{chapter}}

\newcounter{mysection}
\renewcommand\thesection{\arabic{mysection}}

%Per section figure numbering instead of per chapter.
\counterwithin{figure}{section}

\newcommand{\mysection}[1]{
  \stepcounter{mysection}
  \section{#1}
  \setcounter{footnote}{0}
}

%Section with no number
\newcommand{\mysectionnn}[1]{
  \section*{#1}
  \addcontentsline{toc}{section}{#1}
  \markright{\MakeUppercase{#1}}
  \setcounter{footnote}{0}
}

%Section with no number and short title
\newcommand{\mysectiontwo}[2]{
  \section*{#2}
  \addcontentsline{toc}{section}{#1}
  \setcounter{footnote}{0}
}


%No caption figure
\newcommand{\myfigure}[2]{
  \begin{figure}
    \centering
    \mygraphics{#1}
    \caption{}
    \label{#2}
  \end{figure}
}

%Captioned figure
\newcommand{\myfigurec}[3]{
  \begin{figure}
    \centering
    \mygraphics{#1}
    \caption{#3}
    \label{#2}
  \end{figure}
}

\newcommand{\hr}{
  \begin{center}
    \rule{0.5\linewidth}{0.4pt}
  \end{center}
}

\newcommand{\myendsectiontext}{
%  {\centering
% \ ~
%\par}
%
%{\centering
% *
%\par}
}

%comma in a superscript
\newcommand{\supercomma}{\textsuperscript{,}}

%Degree sign
\newcommand{\degree}{$^\circ$}

%move the footnote counter back
\newcommand{\footback}[1]{\addtocounter{footnote}{-#1}}

%move the footnote counter +1
\newcommand{\footnext}{\addtocounter{footnote}{1}}

%\DeclareNewFootnote{A}
%\DeclareNewFootnote{B}[alph]
\newcommand{\comment}[1]{
}


%\newcommand\textsubscript[1]{\ensuremath{{}_{\text{#1}}}}
% Outline numbering
\setcounter{secnumdepth}{1}


%Use \whencolumns like:
%\whencolumns{ONE COLUMN}{TWO COLUMN}

\makeatletter
\if@twocolumn
%TWO COLUMN
%Used for graphics
\newcommand{\mygraphics}[1]{
  \includegraphics[scale=0.125]{#1}
}

\newcommand{\mygraphicss}[2]{
  \scalebox{0.5}{\includegraphics[scale=#2]{#1}}
}

\newcommand{\whencolumns}[2]{
#2
}

\else
%ONE COLUMN
%Used for graphics
\newcommand{\mygraphics}[1]{
  \includegraphics[scale=0.25]{#1}
}

\newcommand{\mygraphicss}[2]{
  \includegraphics[scale=#2]{#1}
}

\newcommand{\whencolumns}[2]{
#1
}

\fi
\makeatother
