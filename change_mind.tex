\part{How to Actually Change Your Mind}



\mysectiontwo{Rationality: An Introduction}{Rationality: An Introduction\newline by Rob Bensinger}

{
 What should I believe?}

{
 As it turns out, that question has a right answer.}

{
 It has a right answer when you're wracked with
uncertainty, not just when you have a conclusive proof. There is always
a correct amount of confidence to have in a statement, even when it
looks like a ``personal belief'' and
not like an expert-verified
``fact.''}

{
 Yet we often talk as though the existence of uncertainty and
disagreement make beliefs a mere matter of taste. We say
``that's just my
opinion'' or
``you're entitled to your
opinion,'' as though the assertions of science and
math existed on a different and higher plane than beliefs that are
merely ``private'' or
``subjective.'' But, writes Robin
Hanson:\textsuperscript{1}}

{
 You are never entitled to your opinion. Ever! You are not even
entitled to ``I don't
know.'' You are entitled to your desires, and
sometimes to your choices. You might own a choice, and if you can
choose your preferences, you may have the right to do so. But your
beliefs are not about you; beliefs are about the world. Your beliefs
should be your best available estimate of the way things are; anything
else is a lie. [~\ldots~]}

{
 It is true that some topics give experts stronger mechanisms for
resolving disputes. On other topics our biases and the complexity of
the world make it harder to draw strong conclusions. [~\ldots~]}

{
 But never forget that on any question about the way things are (or
should be), and in any information situation, there \textit{is} always
a best estimate. You are only entitled to your best honest effort to
find that best estimate; anything else is a lie.}

{
 Suppose you find out that one of six people has a crush on
you---perhaps you get a letter from a secret admirer and
you're sure it's from one of those
six---but you have no idea which of those six it is. Your classmate Bob
is one of the six candidates, but you have no special evidence for or
against him being the one with the crush. In that case, the odds that
Bob is the one with the crush are 1:5.}

{
 Because there are six possibilities, a wild guess would result in
you getting it right once for every five times you got it wrong, on
average. This is what we mean by ``the odds are
1:5.'' You can't say,
``Well, I have no idea who has a crush on me; maybe
it's Bob, or maybe it's not. So
I'll just say the odds are
fifty-fifty.'' Even if you'd rather
say ``I don't know''
or ``Maybe'' and stop there, the
answer is still 1:5.\textsuperscript{2}}

{
 Suppose also that you've noticed you get winked at
by people ten times as often when they have a crush on you. If Bob then
winks at you, that's a new piece of evidence. In that
case, it would be a mistake to stay skeptical about whether Bob is your
secret admirer; the 10:1 odds in favor of ``a random
person who winks at me has a crush on me'' outweigh
the 1:5 odds against ``Bob has a crush on
me.''}

{
 It would \textit{also} be a mistake to say,
``That evidence is so strong, it's a
sure bet that he's the one who has the crush on me!
I'll just assume from now on that Bob is into
me.'' Overconfidence is just as bad as
underconfidence.}

{
 In fact, there's only one possible answer to this
question that's mathematically consistent. To change
our mind from the 1:5 prior odds based on the
evidence's 10:1 likelihood ratio, we multiply the left
sides together and the right sides together, getting 10:5 posterior
odds, or 2:1 odds in favor of ``Bob has a crush on
me.'' Given our assumptions and the available
evidence, guessing that Bob has a crush on you will turn out to be
correct 2 times for every 1 time it turns out to be wrong.
Equivalently: the probability that he's attracted to
you is 2/3. Any other confidence level would be inconsistent.}

{
 Our culture hasn't internalized the lessons of
probability theory---that the correct answer to questions like
``How sure can I be that Bob has a crush on
me?'' is just as logically constrained as the correct
answer to a question on an algebra quiz or in a geology textbook. Our
clichés are out of step with the discovery that ``what
beliefs should I hold?'' has an objectively right
answer, whether your question is ``does my classmate
have a crush on me?'' or ``do I have
an immortal soul?'' There really is a right way to
change your mind. And it's a \textit{precise} way.}

{
 ~}

\subsection{How to Not Actually Change Your Mind}

{
 Revising our beliefs in anything remotely like this idealized way
is a tricky task, however.}

{
 In the first volume of \textit{Rationality: From AI to Zombies},
we discussed the value of ``proper''
beliefs. There's nothing intrinsically wrong with
expressing your support for something you care about---like a group you
identify with, or a spiritual experience you find exalting. When we
conflate cheers with factual beliefs, however, those misunderstood
cheers can help shield an entire ideology from contamination by the
evidence.}

{
 Even beliefs that seem to elegantly explain our observations
aren't immune to this problem. It's all
too easy for us to see a vaguely scientific-sounding (or otherwise
authoritative) phrase and conclude that it has
``explained'' something, even when
it doesn't affect the odds we implicitly assign to our
possible future experiences.}

{
 Worst of all, prosaic beliefs---beliefs that \textit{are} in
principle falsifiable, beliefs that \textit{do} constrain what we
expect to see---can still get stuck in our heads, reinforced by a
network of illusions and biases.}

{
 In 1951, a football game between Dartmouth and Princeton turned
unusually rough. Psychologists Hastorf and Cantril asked students from
each school who had started the rough play. Nearly all agreed that
Princeton hadn't started it; but 86\% of Princeton
students believed that Dartmouth had started it, whereas only 36\% of
Dartmouth students blamed Dartmouth. (Most Dartmouth students believed
``both started it.'')}

{
 There's no reason to think this was a cheer, as
opposed to a real belief. The students were probably led by their
different beliefs to make different predictions about the behavior of
players in future games. And yet somehow the perfectly ordinary factual
beliefs at Dartmouth were wildly different from the perfectly ordinary
factual beliefs at Princeton.}

{
 Can we blame this on the different sources Dartmouth and Princeton
students had access to? On its own, bias in the different news sources
that groups rely on is a pretty serious problem.}

{
 However, there is more than that at work in this case. When
actually \textit{shown} a film of the game later and asked to count the
infractions they saw, Dartmouth students claimed to see a mean of 4.3
infractions by the Dartmouth team (and identified half as
``mild''), whereas Princeton
students claimed to see a mean of 9.8 Dartmouth infractions (and
identified a third as ``mild'').}

{
 Never mind getting rival factions to agree about complicated
propositions in national politics or moral philosophy; students with
different group loyalties couldn't even agree on what
they were \textit{seeing}.\textsuperscript{3}}

{
 When something we care about is threatened---our world-view, our
in-group, our social standing, or anything else---our thoughts and
perceptions rally to their defense.\textsuperscript{4,5} Some
psychologists these days go so far as to hypothesize that our ability
to come up with explicit justifications for our conclusions evolved
\textit{specifically} to help us win arguments.\textsuperscript{6}}

{
 One of the defining insights of 20th-century psychology, animating
everyone from the disciples of Freud to present-day cognitive
psychologists, is that human behavior is often driven by sophisticated
unconscious processes, and the stories we tell ourselves about our
motives and reasons are much more biased and confabulated than we
realize.}

{
 We often fail, in fact, to realize that we're
doing any story-telling. When we seem to ``directly
perceive'' things about ourselves in introspection,
it often turns out to rest on tenuous implicit causal
models.\textsuperscript{7,8} When we try to argue for our beliefs, we
can come up with shaky reasoning bearing no relation to how we first
arrived at the belief.\textsuperscript{9} Rather than judging our
explanations by their predictive power, we tell stories to make sense
of what we think we know.}

{
 How can we do better? How can we arrive at a realistic view of the
world, when our minds are so prone to rationalization? How can we come
to a realistic view of our mental lives, when our thoughts
\textit{about} thinking are also suspect? How can we become less
biased, when our efforts to debias ourselves can turn out to have
biases of their own?}

{
 What's the \textit{least} shaky place we could put
our weight down?}

{
 ~}

\subsection{The Mathematics of Rationality}

{
 At the turn of the 20th century, coming up with simple (e.g.,
set-theoretic) axioms for arithmetic gave mathematicians a clearer
standard by which to judge the correctness of their conclusions. If a
human or calculator outputs ``2 + 2 =
4,'' we can now do more than just say
``that seems intuitively right.'' We
can explain \textit{why} it's right, and we can prove
that its rightness is tied in systematic ways to the rightness of the
rest of arithmetic.}

{
 But mathematics and logic let us model the behaviors of physical
systems that are a lot more interesting than a pocket calculator. We
can also formalize \textit{rational belief in general}, using
probability theory to pick out features held in common by all
successful forms of inference. We can even formalize \textit{rational
behavior in general} by drawing upon decision theory.}

{
 Probability theory defines how we would ideally reason in the face
of uncertainty, if we had the time, the computing power, and the
self-control. Given some background knowledge (\textit{priors}) and a
new piece of evidence, probability theory uniquely defines the best set
of new beliefs (\textit{posterior}) I could adopt. Likewise, decision
theory defines what action I should take based on my beliefs. For any
consistent set of beliefs and preferences I could have about Bob, there
is a decision-theoretic answer to how I should then act in order to
satisfy my preferences.}

{
 Humans aren't perfect reasoners or perfect
decision-makers, any more than we're perfect
calculators. Our brains are kludges slapped together by natural
selection. Even at our best, we don't compute the
\textit{exact} right answer to ``what should I
think?'' and ``what should I
do?'' We lack the time and computing power, and
evolution lacked the engineering expertise and foresight, to iron out
all our bugs.}

{
 A maximally efficient bug-free reasoner in the real world, in
fact, would still need to rely on heuristics and approximations. The
optimal computationally tractable algorithms for changing beliefs fall
short of probability theory's consistency.}

{
 And yet, knowing we can't become \textit{fully}
consistent, we can certainly still get better. Knowing that
there's an ideal standard we can compare ourselves
to---what researchers call ``Bayesian
rationality''---can guide us as we improve our
thoughts and actions. Though we'll never be perfect
Bayesians, the mathematics of rationality can help us understand
\textit{why} a certain answer is correct, and help us spot exactly
where we messed up.}

{
 Imagine trying to learn math through rote memorization alone. You
might be told that ``10 + 3 = 13,''
``31 + 108 = 139,'' and so on, but
it won't do you a lot of good unless you understand the
pattern behind the squiggles. It can be a lot harder to seek out
methods for improving your rationality when you don't
have a general framework for judging a method's
success. The purpose of this book is to help people build for
themselves such frameworks.}

{
 ~}

\subsection{Rationality Applied}

{
 In a blog post discussing how rationality-enthusiast
``rationalists'' differ from
anti-empiricist ``rationalists,''
Scott Alexander observed:\textsuperscript{10}}

{
 [O]bviously it's useful to have as much evidence
as possible, in the same way it's useful to have as
much money as possible. But equally obviously it's
useful to be able to use a limited amount of evidence wisely, in the
same way it's useful to be able to use a limited amount
of money wisely.}

{
 Rationality techniques help us get more mileage out of the
evidence we have, in cases where the evidence is inconclusive or our
biases and attachments are distorting how we interpret the evidence.
This applies to our personal lives, as in the tale of Bob. It applies
to disagreements between political factions (and between sports fans).
And it applies to technological and philosophical puzzles, as in
debates over transhumanism, the position that we should use technology
to radically refurbish the human condition. Recognizing that the same
mathematical rules apply to each of these domains---and that the same
cognitive biases in many cases hold sway---\textit{How to Actually
Change Your Mind} draws on a wide range of example problems.}

{
 The first sequence of essays in \textit{How to Actually Change
Your Mind}, ``Overly Convenient
Excuses,'' focuses on questions that are as
probabilistically clear-cut as questions get. The Bayes-optimal answer
is often infeasible to compute, but errors like confirmation bias can
take root even in cases where the available evidence is overwhelming
and we have plenty of time to think things over.}

{
 From there, we move into murkier waters with a sequence on
``Politics and Rationality.''
Mainstream national politics, as debated by TV pundits, is famous for
its angry, unproductive discussions. On the face of it,
there's something surprising about that. Why do we take
political disagreements so personally, even when the machinery and
effects of national politics are so distant from us in space or in
time? For that matter, why do we not become \textit{more} careful and
rigorous with the evidence when we're dealing with
issues we deem important?}

{
 The Dartmouth-Princeton game hints at an answer. Much of our
reasoning process is really rationalization---story-telling that makes
our current beliefs feel more coherent and justified, without
necessarily improving their accuracy. ``Against
Rationalization'' speaks to this problem, followed by
``Against Doublethink'' (on
self-deception) and ``Seeing with Fresh
Eyes'' (on the challenge of recognizing evidence that
doesn't fit our expectations and assumptions).}

{
 Leveling up in rationality means encountering a lot of interesting
and powerful new ideas. In many cases, it also means making friends who
you can bounce ideas off of and finding communities that encourage you
to better yourself. ``Death
Spirals'' discusses some important hazards that can
afflict groups united around common interests and amazing shiny ideas,
which will need to be overcome if we're to get the full
benefits out of rationalist communities. \textit{How to Actually Change
Your Mind} then concludes with a sequence on ``Letting
Go.''}

{
 Our natural state \textit{isn't} to change our
minds like a Bayesian would. Getting the Dartmouth and Princeton
students to \textit{notice what they're really seeing}
won't be as easy as reciting the axioms of probability
theory to them. As Luke Muehlhauser writes, in The Power of
Agency:\textsuperscript{11}}

{
 You are not a Bayesian homunculus whose reasoning is
``corrupted'' by cognitive biases.}

{
 You just \textit{are} cognitive biases.}

{
 Confirmation bias, status quo bias, correspondence bias, and the
like are not tacked on to our reasoning; they are its very substance.}

{
 That doesn't mean that debiasing is impossible. We
aren't perfect calculators underneath all our
arithmetic errors, either. Many of our mathematical limitations result
from very deep facts about how the human brain works. Yet we can train
our mathematical abilities; we can learn when to trust and distrust our
mathematical intuitions, and share our knowledge, and help one another;
we can shape our environments to make things easier on us, and build
tools to offload much of the work.}

{
 Our biases are part of us. But there is a shadow of Bayesianism
present in us as well, a flawed apparatus that really can bring us
closer to truth. No homunculus---but still, some truth. Enough,
perhaps, to get started.}

{
 ~}

{\centering
 *
\par}


\bigskip

{
 1. Robin Hanson, ``You Are Never Entitled to Your
Opinion,'' \textit{Overcoming Bias (blog)} (2006),
http://www.overcomingbias.com/2006/12/you\_are\_never\_e.html.}

{
 2. This follows from the assumption that there are six
possibilities and you have no reason to favor one of them over any of
the others. We're also assuming, unrealistically, that
you can really be certain the admirer is one of those six people, and
that you aren't neglecting other possibilities. (What
if more than one of the six people has a crush on you?)}

{
 3. Albert Hastorf and Hadley Cantril, ``They Saw
a Game: A Case Study,'' \textit{Journal of Abnormal
and Social Psychology} 49 (1954): 129--134,
http://www2.psych.ubc.ca/\~{}schaller/Psyc590Readings/Hastorf1954.pdf.}

{
 4. Pronin, ``How We See Ourselves and How We See
Others.''}

{
 5. Robert P. Vallone, Lee Ross, and Mark R. Lepper,
``The Hostile Media Phenomenon: Biased Perception and
Perceptions of Media Bias in Coverage of the Beirut
Massacre,'' \textit{Journal of Personality and Social
Psychology} 49 (1985): 577--585,
http://ssc.wisc.edu/\~{}jpiliavi/965/hwang.pdf.}

{
 6. Hugo Mercier and Dan Sperber, ``Why Do Humans
Reason? Arguments for an Argumentative Theory,''
\textit{Behavioral and Brain Sciences} 34 (2011): 57--74,
https://hal.archives-ouvertes.fr/file/index/docid/904097/filename/MercierSperberWhydohumansreason.pdf.}

{
 7. Richard E. Nisbett and Timothy D. Wilson,
``Telling More than We Can Know: Verbal Reports on
Mental Processes,'' \textit{Psychological Review} 84
(1977): 231--259,
http://people.virginia.edu/\~{}tdw/nisbett\&wilson.pdf.}

{
 8. Eric Schwitzgebel, \textit{Perplexities of Consciousness} (MIT
Press, 2011).}

{
 9. Jonathan Haidt, ``The Emotional Dog and Its
Rational Tail: A Social Intuitionist Approach to Moral
Judgment,'' \textit{Psychological Review} 108, no. 4
(2001): 814--834, doi:10.1037/0033-295X.108.4.814.}

{
 10. Scott Alexander, ``Why I Am Not Rene
Descartes,'' \textit{Slate Star Codex (blog)} (2014),
http://slatestarcodex.com/2014/11/27/why-i-am-not-rene-descartes/.}

{
 11. Luke Muehlhauser, ``The Power of
Agency,'' \textit{Less Wrong (blog)} (2011),
http://lesswrong.com/lw/5i8/the\_power\_of\_agency/.}

\chapter{Overly Convenient Excuses}

\mysection{The Proper Use of Humility}

{
 It is widely recognized that good science requires some kind of
humility. \textit{What sort} of humility is more controversial. }

{
 Consider the creationist who says: ``But who can
really know whether evolution is correct? It is just a theory. You
should be more humble and open-minded.'' Is this
humility? The creationist practices a very selective underconfidence,
refusing to integrate massive weights of evidence in favor of a
conclusion they find uncomfortable. I would say that whether you call
this ``humility'' or not, it is the
wrong step in the dance.}

{
 What about the engineer who humbly designs fail-safe mechanisms
into machinery, even though they're damn sure the
machinery won't fail? This seems like a good kind of
humility to me. Historically, it's not unheard-of for
an engineer to be damn sure a new machine won't fail,
and then it fails anyway.}

{
 What about the student who humbly double-checks the answers on
their math test? Again I'd categorize that as good
humility.}

{
 What about a student who says, ``Well, no matter
how many times I check, I can't ever be
\textit{certain} my test answers are correct,'' and
therefore doesn't check even once? Even if this choice
stems from an emotion similar to the emotion felt by the previous
student, it is less wise.}

{
 You suggest studying harder, and the student replies:
``No, it wouldn't work for me;
I'm not one of the smart kids like you; nay, one so
lowly as myself can hope for no better lot.'' This is
social modesty, not humility. It has to do with regulating status in
the tribe, rather than scientific process. If you ask someone to
``be more humble,'' by default
they'll associate the words to social modesty---which
is an intuitive, everyday, ancestrally relevant concept. Scientific
humility is a more recent and rarefied invention, and it is not
inherently social. Scientific humility is something you would practice
even if you were alone in a spacesuit, light years from Earth with no
one watching. Or even if you received an absolute guarantee that no one
would ever criticize you again, no matter what you said or thought of
yourself. You'd still double-check your calculations if
you were wise.}

{
 The student says: ``But I've seen
other students double-check their answers and then they still turned
out to be wrong. Or what if, by the problem of induction, 2 + 2 = 5
this time around? No matter what I do, I won't be sure
of myself.'' It sounds very profound, and very
modest. But it is not coincidence that the student wants to hand in the
test quickly, and go home and play video games.}

{
 The end of an era in physics does not always announce itself with
thunder and trumpets; more often it begins with what seems like a
small, small flaw \ldots But because physicists have this arrogant idea
that their models should work \textit{all} the time, not just
\textit{most} of the time, they follow up on small flaws. Usually, the
small flaw goes away under closer inspection. Rarely, the flaw widens
to the point where it blows up the whole theory. Therefore it is
written: ``If you do not seek perfection you will halt
before taking your first steps.''}

{
 But think of the social audacity of trying to be right
\textit{all} the time! I seriously suspect that if Science claimed that
evolutionary theory is true most of the time but not all of the
time---or if Science conceded that maybe on some days the Earth
\textit{is} flat, but who really knows---then scientists would have
better social reputations. Science would be viewed as less
confrontational, because we wouldn't have to argue with
people who say the Earth is flat---there would be room for compromise.
When you argue a lot, people look upon you as confrontational. If you
repeatedly refuse to compromise, it's even worse.
Consider it as a question of tribal status: scientists have certainly
earned some extra status in exchange for such socially useful tools as
medicine and cellphones. But this social status does not justify their
insistence that \textit{only} scientific ideas on evolution be taught
in public schools. Priests also have high social status, after all.
Scientists are getting above themselves---they won a little status, and
now they think they're chiefs of the whole tribe! They
ought to be more humble, and compromise a little.}

{
 Many people seem to possess rather hazy views of
``rationalist humility.'' It is
dangerous to have a prescriptive principle which you only vaguely
comprehend; your mental picture may have so many degrees of freedom
that it can adapt to justify almost any deed. Where people have vague
mental models that can be used to argue anything, they usually end up
believing whatever they started out wanting to believe. This is so
convenient that people are often reluctant to give up vagueness. But
the purpose of our ethics is to move us, not be moved by us.}

{
 ``Humility'' is a virtue that
is often misunderstood. This doesn't mean we should
discard the concept of humility, but we should be careful using it. It
may help to look at the \textit{actions} recommended by a
``humble'' line of thinking, and
ask: ``Does acting this way make you stronger, or
weaker?'' If you think about the problem of induction
as applied to a bridge that needs to stay up, it may sound reasonable
to conclude that nothing is certain no matter what precautions are
employed; but if you consider the real-world difference between adding
a few extra cables, and shrugging, it seems clear enough what makes the
stronger bridge.}

{
 The vast majority of appeals that I witness to
``rationalist's
humility'' are excuses to shrug. The one who buys a
lottery ticket, saying, ``But you
can't \textit{know} that I'll
lose.'' The one who disbelieves in evolution, saying,
``But you can't \textit{prove} to me
that it's true.'' The one who refuses
to confront a difficult-looking problem, saying,
``It's probably too hard to
solve.'' The problem is motivated skepticism a.k.a.
disconfirmation bias---more heavily scrutinizing assertions that we
don't want to believe. Humility, in its most commonly
misunderstood form, is a fully general excuse not to believe something;
since, after all, you can't be \textit{sure}. Beware of
fully general excuses!}

{
 A further problem is that humility is all too easy to
\textit{profess.} Dennett, in \textit{Breaking the Spell: Religion as a
Natural Phenomenon}, points out that while many religious assertions
are very hard to believe, it is easy for people to believe that they
\textit{ought} to believe them. Dennett terms this
``belief in belief.'' What would it
mean to really assume, to really believe, that three is equal to one?
It's a lot easier to believe that you \textit{should},
somehow, believe that three equals one, and to make this response at
the appropriate points in church. Dennett suggests that much
``religious belief'' should be
studied as ``religious
profession''---what people think they should believe
and what they know they ought to say.}

{
 It is all too easy to meet every counterargument by saying,
``Well, of course I could be
wrong.'' Then, having dutifully genuflected in the
direction of Modesty, having made the required obeisance, you can go on
about your way without changing a thing.}

{
 The temptation is always to claim the most points with the least
effort. The temptation is to carefully integrate all incoming news in a
way that lets us change our beliefs, and above all our
\textit{actions}, as little as possible. John Kenneth Galbraith said:
``Faced with the choice of changing
one's mind and proving that there is no need to do so,
almost everyone gets busy on the
proof.''\textsuperscript{1} And the greater the
\textit{inconvenience} of changing one's mind, the more
effort people will expend on the proof.}

{
 But y'know, if you're gonna
\textit{do} the same thing anyway, there's no point in
going to such incredible lengths to rationalize it. Often I have
witnessed people encountering new information, apparently accepting it,
and then carefully explaining why they are going to do exactly the same
thing they planned to do previously, but with a different
justification. The point of thinking is to \textit{shape} our plans; if
you're going to keep the same plans anyway, why bother
going to all that work to justify it? When you encounter new
information, the hard part is to \textit{update}, to \textit{react},
rather than just letting the information disappear down a black hole.
And humility, properly misunderstood, makes a wonderful black
hole---all you have to do is admit you could be wrong. Therefore it is
written: ``To be humble is to take specific actions in
anticipation of your own errors. To confess your fallibility and then
do nothing about it is not humble; it is boasting of your
modesty.''}

\myendsectiontext


\bigskip

{
 1. John Kenneth Galbraith, \textit{Economics, Peace and Laughter}
(Plume, 1981), 50.}

\mysection{The Third Alternative}


{
 \textit{Believing in Santa Claus gives children a sense of wonder
and encourages them to behave well in hope of receiving presents. If
Santa-belief is destroyed by truth, the children will lose their sense
of wonder and stop behaving nicely. Therefore, even though Santa-belief
is false-to-fact, it is a Noble Lie whose net benefit should be
preserved for utilitarian reasons.}}

{
 Classically, this is known as a false dilemma, the fallacy of the
excluded middle, or the package-deal fallacy. Even if we accept the
underlying factual and moral premises of the above argument, it does
not carry through. Even supposing that the Santa policy (encourage
children to believe in Santa Claus) is better than the null policy (do
nothing), it does not follow that Santa-ism is the \textit{best of all
possible alternatives.} Other policies could also supply children with
a sense of wonder, such as taking them to watch a Space Shuttle launch
or supplying them with science fiction novels. Likewise (if I recall
correctly), offering children bribes for good behavior encourages the
children to behave well \textit{only} when adults are watching, while
praise without bribes leads to unconditional good behavior.}

{
 Noble Lies are generally package-deal fallacies; and the response
to a package-deal fallacy is that if we really need the supposed gain,
we can construct a Third Alternative for getting it.}

{
 How can we obtain Third Alternatives? The first step in obtaining
a Third Alternative is deciding to look for one, and the last step is
the decision to accept it. This sounds obvious, and yet most people
fail on these two steps, rather than within the search process. Where
do false dilemmas come from? Some arise honestly, because superior
alternatives are cognitively hard to see. But one factory for false
dilemmas is justifying a questionable policy by pointing to a supposed
benefit over the null action. In this case, the justifier \textit{does
not want} a Third Alternative; finding a Third Alternative would
destroy the justification. The last thing a Santa-ist wants to hear is
that praise works better than bribes, or that spaceships can be as
inspiring as flying reindeer.}

{
 The best is the enemy of the good. If the goal is \textit{really}
to help people, then a superior alternative is cause for
celebration---once we find this better strategy, we can help people
more effectively. But if the goal is to justify a particular strategy
\textit{by claiming that it helps people}, a Third Alternative is an
enemy argument, a competitor.}

{
 Modern cognitive psychology views decision-making as a search for
alternatives. In real life, it's not enough to compare
options; you have to generate the options in the first place. On many
problems, the number of alternatives is huge, so you need a stopping
criterion for the search. When you're looking to buy a
house, you can't compare every house in the city; at
some point you have to stop looking and decide.}

{
 But what about when our conscious motives for the search---the
criteria we can admit to ourselves---don't square with
subconscious influences? When we are carrying out an allegedly
altruistic search, a search for an altruistic policy, and we find a
strategy that benefits others but disadvantages ourselves---well, we
don't stop looking \textit{there}; we go on looking.
Telling ourselves that we're looking for a strategy
that brings greater altruistic benefit, of course. But suppose we find
a policy that has some defensible benefit, and \textit{also} just
happens to be personally convenient? Then we stop the search at once!
In fact, we'll probably \textit{resist} any suggestion
that we start looking again---pleading lack of time, perhaps. (And yet
somehow, we always have cognitive resources for coming up with
justifications for our current policy.)}

{
 Beware when you find yourself arguing that a policy is
\textit{defensible} rather than \textit{optimal}; or that it has some
benefit compared to the null action, rather than the best benefit of
any action.}

{
 False dilemmas are often presented to justify unethical policies
that are, by some vast coincidence, very convenient. Lying, for
example, is often much more convenient than telling the truth; and
believing whatever you started out with is more convenient than
updating. Hence the popularity of arguments for Noble Lies; it serves
as a defense of a pre-existing belief---one does not find Noble Liars
who calculate an optimal new Noble Lie; they keep whatever lie they
started with. Better stop that search fast!}

{
 To do better, ask yourself straight out: \textit{If I saw that
there was a superior alternative to my current policy, would I be glad
in the depths of my heart, or would I feel a tiny flash of reluctance
before I let go?} If the answers are
``no'' and
``yes,'' beware that you may not
have searched for a Third Alternative.}

{
 Which leads into another good question to ask yourself straight
out: \textit{Did I spend five minutes with my eyes closed,
brainstorming wild and creative options, trying to think of a better
alternative?} It has to be five minutes by the clock, because otherwise
you blink---close your eyes and open them again---and say,
``Why, yes, I searched for alternatives, but there
weren't any.'' Blinking makes a good
black hole down which to dump your duties. An actual, physical clock is
recommended.}

{
 And those wild and creative options---were you careful not to
think of a good one? Was there a secret effort from the corner of your
mind to ensure that every option considered would be obviously bad?}

{
 It's amazing how many Noble Liars and their ilk
are eager to embrace ethical violations---with all due bewailing of
their agonies of conscience---when they haven't spent
even five minutes by the clock looking for an alternative. There are
some mental searches that we secretly wish would fail; and when the
prospect of success is uncomfortable, people take the earliest possible
excuse to give up.}

\myendsectiontext


\mysection{Lotteries: A Waste of Hope}

{
 The classic criticism of the lottery is that the people who play
are the ones who can least afford to lose; that the lottery is a sink
of money, draining wealth from those who most need it. Some lottery
advocates, and even some commentors on \textit{Overcoming Bias}, have
tried to defend lottery-ticket buying as a \textit{rational purchase of
fantasy}{}---paying a dollar for a day's worth of
pleasant anticipation, imagining yourself as a millionaire. }

{
 But consider exactly what this implies. It would mean that
you're occupying your valuable brain with a fantasy
whose real probability is nearly zero---a tiny line of likelihood which
you, yourself, can do nothing to realize. The lottery balls will decide
your future. The fantasy is of wealth that arrives without
effort{}---without conscientiousness, learning, charisma, or even
patience.}

{
 Which makes the lottery another kind of sink: a sink of emotional
energy. It encourages people to invest their dreams, their hopes for a
better future, into an infinitesimal probability. If not for the
lottery, maybe they would fantasize about going to technical school, or
opening their own business, or getting a promotion at work---things
they might be able to actually \textit{do}, hopes that would make them
want to become stronger. Their dreaming brains might, in the 20th
visualization of the pleasant fantasy, notice a way to really do it.
Isn't that what dreams and brains are \textit{for}? But
how can such reality-limited fare compete with the artificially
sweetened prospect of instant wealth---not after herding a dot-com
startup through to IPO, but on Tuesday?}

{
 Seriously, why can't we just say that buying
lottery tickets is stupid? Human beings \textit{are} stupid, from time
to time---it shouldn't be so surprising a hypothesis.}

{
 Unsurprisingly, the human brain doesn't do 64-bit
floating-point arithmetic, and it can't devalue the
emotional force of a pleasant anticipation by a factor of 0.00000001
without dropping the line of reasoning entirely. Unsurprisingly, many
people don't realize that a numerical calculation of
expected utility ought to \textit{override} or \textit{replace} their
imprecise financial instincts, and instead treat the calculation as
merely one \textit{argument} to be balanced against their pleasant
anticipations---an emotionally weak argument, since
it's made up of mere squiggles on paper, instead of
visions of fabulous wealth.}

{
 This seems sufficient to explain the popularity of lotteries. Why
do so many arguers feel impelled to defend this classic form of
self-destruction?}

{
 The process of overcoming bias requires (1) first noticing the
bias, (2) analyzing the bias in detail, (3) deciding that the bias is
bad, (4) figuring out a workaround, and then (5) implementing it.
It's unfortunate how many people get through steps 1
and 2 and then bog down in step 3, which by rights should be the
easiest of the five. Biases are lemons, not lemonade, and we
shouldn't try to make lemonade out of them---just burn
those lemons \textit{down}.}

\myendsectiontext

\mysection{New Improved Lottery}

{
 People are still suggesting that the lottery is not a waste of
hope, but a service which enables purchase of
fantasy---``daydreaming about becoming a millionaire
for much less money than daydreaming about hollywood stars in
movies.'' One commenter wrote:
``There is a big difference between zero chance of
becoming wealthy, and epsilon. Buying a ticket allows your dream of
riches to bridge that gap.'' }

{
 Actually, one of the points I was trying to make is that between
zero chance of becoming wealthy, and epsilon chance, there is an
order-of-epsilon difference. If you doubt this, let epsilon equal one
over googolplex.}

{
 Anyway, if we pretend that the lottery sells epsilon hope, this
suggests a design for a New Improved Lottery. The New Improved Lottery
pays out every five years on average, at a random time---determined,
say, by the decay of a not-very-radioactive element. You buy in once,
for a single dollar, and get not just a few days of epsilon chance of
becoming rich, but a few \textit{years} of epsilon. Not only that, your
wealth could strike at any time! At \textit{any minute}, the phone
could ring to inform you that \textit{you, yes, you} are a
millionaire!}

{
 Think of how much better this would be than an ordinary lottery
drawing, which only takes place at defined times, a few times per week.
Let's say the boss comes in and demands you rework a
proposal, or restock inventory, or something similarly annoying.
Instead of getting to work, you could turn to the phone and stare,
hoping for that call---because there would be epsilon chance that,
\textit{at that exact moment, you yes you} would be awarded the Grand
Prize! And even if it doesn't happen \textit{this}
minute, why, there's no need to be disappointed---it
might happen the \textit{next} minute!}

{
 Think of how many more fantasies this New Improved Lottery would
enable. You could shop at the store, adding expensive items to your
shopping cart---if your cellphone doesn't ring with
news of a lottery win, you could always put the items back, right?}

{
 Maybe the New Improved Lottery could even show a constantly
fluctuating probability distribution over the likelihood of a win
occurring, and the likelihood of particular numbers being selected,
with the overall expectation working out to the aforesaid Poisson
distribution. Think of how much fun \textit{that} would be! Oh,
goodness, right this minute the chance of a win occurring is nearly ten
times higher than usual! And look, the number 42 that I selected for
the Mega Ball has nearly twice the usual chance of winning! You could
feed it to a display on people's cellphones, so they
could just flip open the cellphone and see their chances of winning.
Think of how exciting \textit{that} would be! Much more exciting than
trying to balance your checkbook! Much more exciting than doing your
homework! This new dream would be so much tastier that it would compete
with, not only hopes of going to technical school, but even hopes of
getting home from work early. People could just stay glued to the
screen all day long, why, they wouldn't need to dream
about anything \textit{else}!}

{
 Yep, offering people tempting daydreams \textit{that will not
actually happen} sure is a valuable service, all right. People are
willing to pay; it must be valuable. The alternative is that consumers
are making mistakes, and we all know that can't
happen.}

{
 And yet current governments, with their vile monopoly on
lotteries, don't offer this simple and obvious service.
Why? Because they want to overcharge people. They want them to spend
money every week. They want them to spend a hundred dollars for the
thrill of believing their chance of winning is a hundred times as
large, instead of being able to stare at a cellphone screen waiting for
the likelihood to spike. So if you believe that the lottery is a
service, it is clearly an enormously overpriced service---charged to
the poorest members of society---and it is your solemn duty as a
citizen to demand the New Improved Lottery instead.}

\myendsectiontext

\mysection{But There's Still a Chance, Right?}

{
 Years ago, I was speaking to someone when he casually remarked
that he didn't believe in evolution. And I said,
``This is not the nineteenth century. When Darwin
first proposed evolution, it might have been reasonable to doubt it.
But this is the twenty-first century. We can \textit{read the genes.}
Humans and chimpanzees have 98\% shared DNA. We \textit{know} humans
and chimps are related. It's
\textit{over.}'' }

{
 He said, ``Maybe the DNA is just similar by
coincidence.''}

{
 I said, ``The odds of that are something like two
to the power of seven hundred and fifty million to
one.''}

{
 He said, ``But there's still a
chance, right?''}

{
 Now, there's a number of reasons my past self
cannot claim a strict moral victory in this conversation. One reason is
that I have no memory of whence I pulled that
2\textsuperscript{750,000,000} figure, though it's
probably the right meta-order of magnitude. The other reason is that my
past self didn't apply the concept of a calibrated
confidence. Of all the times over the history of humanity that a human
being has calculated odds of 2\textsuperscript{750,000,000}:1 against
something, they have undoubtedly been wrong more often than once in
2\textsuperscript{750,000,000} times. E.g. the shared genes estimate
was revised to 95\%, not 98\%---and that may even apply only to the
30,000 known genes and not the entire genome, in which case
it's the wrong meta-order of magnitude.}

{
 But I think the other guy's reply is still pretty
funny.}

{
 I don't recall what I said in further
response---probably something like
``\textbf{\textit{No}}''---but I
remember this occasion because it brought me several insights into the
laws of thought as seen by the unenlightened ones.}

{
 It first occurred to me that human intuitions were making a
qualitative distinction between ``No
chance'' and ``A very tiny chance,
but worth keeping track of.'' You can see this in the
\textit{Overcoming Bias} lottery debate, where someone said,
``There's a big difference between
zero chance of winning and epsilon chance of
winning,'' and I replied, ``No,
there's an order-of-epsilon difference; if you doubt
this, let epsilon equal one over googolplex.''}

{
 The problem is that probability theory sometimes lets us calculate
a chance which is, indeed, too tiny to be worth the mental space to
keep track of it---but by that time, you've already
calculated it. People mix up the map with the territory, so that on a
gut level, tracking a symbolically described probability feels like
``a chance worth keeping track of,''
even if the \textit{referent} of the symbolic description is a number
so tiny that if it was a dust speck, you couldn't see
it. We can use words to describe numbers that small, but not
feelings---a feeling that small doesn't exist,
doesn't fire enough neurons or release enough
neurotransmitters to be felt. This is why people buy lottery
tickets---no one can \textit{feel} the smallness of a probability that
small.}

{
 But what I found even more fascinating was the qualitative
distinction between ``certain'' and
``uncertain'' arguments, where if an
argument is not certain, you're allowed to ignore it.
Like, if the likelihood is zero, then you have to give up the belief,
but if the likelihood is one over googol, you're
allowed to keep it.}

{
 Now it's a free country and no one should put you
in jail for illegal reasoning, but if you're going to
ignore an argument that says the likelihood is one over googol, why not
also ignore an argument that says the likelihood is zero? I mean, as
long as you're ignoring the evidence anyway, why is it
so much worse to ignore certain evidence than uncertain evidence?}

{
 I have often found, in life, that I have learned from other
people's nicely blatant bad examples, duly generalized
to more subtle cases. In this case, the flip lesson is that, if you
can't ignore a likelihood of one over googol because
you want to, you can't ignore a likelihood of 0.9
because you want to. It's all the same slippery cliff.}

{
 Consider his example if you ever you find yourself thinking,
``But you can't \textit{prove} me
wrong.'' If you're going to ignore a
probabilistic counterargument, why not ignore a proof, too?}

\myendsectiontext

\mysection{The Fallacy of Gray}

{
 The Sophisticate: ``The world
isn't black and white. No one does pure good or pure
bad. It's all gray. Therefore, no one is better than
anyone else.''}

{
 The Zetet: ``Knowing only gray, you conclude that
all grays are the same shade. You mock the simplicity of the two-color
view, yet you replace it with a one-color view
\ldots''}

{\raggedleft
 {}---Marc Stiegler, \textit{David's
Sling}\textsuperscript{1}
\par}


\bigskip

{
 ~}

{
 I don't know if the Sophisticate's
mistake has an official name, but I call it the Fallacy of Gray. We saw
it manifested in the previous essay---the one who believed that odds of
two to the power of seven hundred and fifty millon to one, against,
meant ``there was still a chance.''
All probabilities, to him, were simply
``uncertain'' and that meant he was
licensed to ignore them if he pleased.}

{
 ``The Moon is made of green
cheese'' and ``the Sun is made of
mostly hydrogen and helium'' are both uncertainties,
but they are not the same uncertainty.}

{
 Everything is shades of gray, but there are shades of gray so
light as to be very nearly white, and shades of gray so dark as to be
very nearly black. Or even if not, we can still compare shades, and say
``it is darker'' or
``it is lighter.''}

{
 Years ago, one of the strange little formative moments in my
career as a rationalist was reading this paragraph from \textit{Player
of Games} by Iain M. Banks, especially the sentence in
bold:\textsuperscript{2}}

{
 A guilty system recognizes no innocents. As with any power
apparatus which thinks everybody's either for it or
against it, we're against it. You would be too, if you
thought about it. The very way you think places you amongst its
enemies. This might not be your fault, because \textbf{every society
imposes some of its values on those raised within it, but the point is
that some societies try to maximize that effect, and some try to
minimize it}. You come from one of the latter and
you're being asked to explain yourself to one of the
former. Prevarication will be more difficult than you might imagine;
neutrality is probably impossible. You cannot choose not to have the
politics you do; they are not some separate set of entities somehow
detachable from the rest of your being; they are a function of your
existence. I know that and they know that; you had better accept it.}

{
 Now, don't write angry comments saying that, if
societies impose fewer of their values, then each succeeding generation
has more work to start over from scratch. That's not
what I got out of the paragraph.}

{
 What I got out of the paragraph was something which seems so
obvious in retrospect that I could have conceivably picked it up in a
hundred places; but something about that one paragraph made it click
for me.}

{
 It was the whole notion of the Quantitative Way applied to
life-problems like moral judgments and the quest for personal
self-improvement. That, even if you couldn't switch
something from on to off, you could still tend to increase it or
decrease it.}

{
 Is this too obvious to be worth mentioning? I say it is not too
obvious, for many bloggers have said of \textit{Overcoming Bias}:
``It is impossible, no one can completely eliminate
bias.'' I don't care if the one is a
professional economist, it is clear that they have not yet grokked the
Quantitative Way as it applies to everyday life and matters like
personal self-improvement. That which I cannot \textit{eliminate} may
be well worth \textit{reducing}.}

{
 Or consider this exchange between Robin Hanson and Tyler Cowen.
Robin Hanson said that he preferred to put at least 75\% weight on the
prescriptions of economic theory versus his intuitions:
``I try to mostly just straightforwardly apply
economic theory, adding little personal or cultural
judgment.'' Tyler Cowen replied:}

{
 In my view there is no such thing as
``straightforwardly applying economic
theory'' \ldots theories are always applied through
our personal and cultural filters and there is no other way it can be.}

{
 Yes, but you can try to minimize that effect, or you can do things
that are bound to increase it. And \textit{if} you try to minimize it,
then in many cases I don't think it's
unreasonable to call the output
``straightforward''---even in
economics.}

{
 ``Everyone is imperfect.''
Mohandas Gandhi was imperfect and Joseph Stalin was imperfect, but they
were not the same shade of imperfection. ``Everyone is
imperfect'' is an excellent example of replacing a
two-color view with a one-color view. If you say, ``No
one is perfect, but \textit{some people are less imperfect than
others},'' you may not gain applause; but for those
who strive to do better, you have held out hope. No one is
\textit{perfectly} imperfect, after all.}

{
 (Whenever someone says to me, ``Perfectionism is
bad for you,'' I reply: ``I think
it's okay to be imperfect, but not so imperfect that
other people notice.'')}

{
 Likewise the folly of those who say, ``Every
scientific paradigm imposes some of its assumptions on how it
interprets experiments,'' and then act like
they'd proven science to occupy the same level with
witchdoctoring. Every worldview imposes some of its structure on its
observations, but the point is that there are worldviews which try to
minimize that imposition, and worldviews which glory in it. There is no
white, but there are shades of gray that are far lighter than others,
and it is folly to treat them as if they were all on the same level.}

{
 If the Moon has orbited the Earth these past few billion years, if
you have seen it in the sky these last years, and you expect to see it
in its appointed place and phase tomorrow, then that is not a
certainty. And if you expect an invisible dragon to heal your daughter
of cancer, that too is not a certainty. But they are rather different
degrees of uncertainty---this business of expecting things to happen
yet again in the same way you have previously predicted to twelve
decimal places, versus expecting something to happen that
\textit{violates} the order previously observed. Calling them both
``faith'' seems a little too
un-narrow.}

{
 It's a most peculiar psychology---this business of
``Science is based on faith too, so
there!'' Typically this is said by people who claim
that faith is a \textit{good} thing. Then why do they say
``Science is based on faith too!''
in that angry-triumphal tone, rather than as a compliment? And a rather
\textit{dangerous} compliment to give, one would think, from their
perspective. If science is based on
``faith,'' then science is of the
same kind as religion---directly comparable. If science is a religion,
it is the religion that heals the sick and reveals the secrets of the
stars. It would make sense to say, ``The priests of
science can blatantly, publicly, verifiably walk on the Moon as a
faith-based miracle, and your priests' faith
can't do the same.'' Are you sure you
wish to go there, oh faithist? Perhaps, on further reflection, you
would prefer to retract this whole business of
``Science is a religion too!''}

{
 There's a strange dynamic here: You try to purify
your shade of gray, and you get it to a point where
it's pretty light-toned, and someone stands up and says
in a deeply offended tone, ``But it's
not white! It's gray!''
It's one thing when someone says,
``This isn't as light as you think,
because of specific problems X, Y, and Z.''
It's a different matter when someone says angrily
``It's not white! It's
gray!'' without pointing out any specific dark
spots.}

{
 In this case, I begin to suspect psychology that is more imperfect
than usual---that someone may have made a devil's
bargain with their own mistakes, and now refuses to hear of any
possibility of improvement. When someone finds an excuse not to try to
do better, they often refuse to concede that anyone else \textit{can}
try to do better, and every mode of improvement is thereafter their
enemy, and every claim that it is possible to move forward is an
offense against them. And so they say in one breath proudly,
``I'm glad to be
gray,'' and in the next breath angrily,
``And \textit{you're gray
too!}''}

{
 If there is no black and white, there is yet lighter and darker,
and not all grays are the same.}

{
 G2 points us to Asimov's ``The
Relativity of Wrong'':\textsuperscript{3}}

{
 When people thought the earth was flat, they were wrong. When
people thought the earth was spherical, they were wrong. But if you
think that thinking the earth is spherical is just as wrong as thinking
the earth is flat, then your view is wronger than both of them put
together.}

\myendsectiontext


\bigskip

{
 1. Marc Stiegler, \textit{David's Sling} (Baen,
1988).}

{
 2. Iain Banks, \textit{The Player of Games} (Orbit, 1989).}

{
 3. Isaac Asimov, \textit{The Relativity of Wrong} (Oxford
University Press, 1989).}

\mysection{Absolute Authority}

{
 The one comes to you and loftily says: ``Science
doesn't really \textit{know} anything. All you have are
\textit{theories}{}---you can't know for
\textit{certain} that you're right. You scientists
changed your minds about how gravity works---who's to
say that tomorrow you won't change your minds about
evolution?'' }

{
 Behold the abyssal cultural gap. If you think you can cross it in
a few sentences, you are bound to be sorely disappointed.}

{
 In the world of the unenlightened ones, there is authority and
un-authority. What can be trusted, can be trusted; what cannot be
trusted, you may as well throw away. There are good sources of
information and bad sources of information. If scientists have changed
their stories ever in their history, then science cannot be a true
Authority, and can never again be trusted---like a witness caught in a
contradiction, or like an employee found stealing from the till.}

{
 Plus, the one takes for granted that a proponent of an idea is
expected to defend it against every possible counterargument and
confess nothing. All claims are discounted accordingly. If even the
\textit{proponent} of science admits that science is less than perfect,
why, it must be pretty much worthless.}

{
 When someone has lived their life accustomed to certainty, you
can't just say to them, ``Science is
probabilistic, just like all other knowledge.'' They
will accept the first half of the statement as a confession of guilt;
and dismiss the second half as a flailing attempt to accuse everyone
else to avoid judgment.}

{
 You have admitted you are not trustworthy---so begone, Science,
and trouble us no more!}

{
 One obvious source for this pattern of thought is religion, where
the scriptures are alleged to come from God; therefore to confess any
flaw in them would destroy their authority utterly; so any trace of
doubt is a sin, and claiming certainty is \textit{mandatory} whether
you're certain or not.}

{
 But I suspect that the traditional school regimen also has
something to do with it. The teacher tells you certain things, and you
have to believe them, and you have to recite them back on the test. But
when a student makes a suggestion in class, you don't
have to go along with it---you're free to agree or
disagree (it seems) and no one will punish you.}

{
 This experience, I fear, maps the domain of belief onto the social
domains of \textit{authority}, of \textit{command}, of \textit{law.} In
the social domain, there is a qualitative difference between absolute
laws and nonabsolute laws, between commands and suggestions, between
authorities and unauthorities. There seems to be strict knowledge and
unstrict knowledge, like a strict regulation and an unstrict
regulation. Strict authorities must be yielded to, while unstrict
suggestions can be obeyed or discarded as a matter of personal
preference. And Science, since it confesses itself to have a
possibility of error, must belong in the second class.}

{
 (I note in passing that I see a certain similarity to they who
think that if you don't get an Authoritative
probability written on a piece of paper from the teacher in class, or
handed down from some similar Unarguable Source, then your uncertainty
is not a matter for Bayesian probability theory. Someone
might---\textit{gasp!}{}---argue with your estimate of the prior
probability. It thus seems to the not-fully-enlightened ones that
Bayesian priors belong to the class of beliefs proposed by students,
and not the class of beliefs commanded you by teachers---it is not
proper \textit{knowledge.})}

{
 The abyssal cultural gap between the Authoritative Way and the
Quantitative Way is rather annoying to those of us staring across it
from the rationalist side. Here is someone who believes they have
knowledge \textit{more} reliable than science's mere
probabilistic guesses---such as the guess that the Moon will rise in
its appointed place and phase tomorrow, just like it has every observed
night since the invention of astronomical record-keeping, and just as
predicted by physical theories whose previous predictions have been
successfully confirmed to fourteen decimal places. And what is this
knowledge that the unenlightened ones set above ours, and why?
It's probably some musty old scroll that has been
contradicted eleventeen ways from Sunday, and from Monday, and from
every day of the week. Yet this is more reliable than Science (they
say) because it never admits to error, never changes its mind, no
matter how often it is contradicted. They toss around the word
``certainty'' like a tennis ball,
using it as lightly as a feather---while scientists are weighed down by
dutiful doubt, struggling to achieve even a modicum of probability.
``I'm perfect,''
they say without a care in the world, ``I must be so
far above \textit{you}, who must still struggle to improve
yourselves.''}

{
 There is nothing simple you can say to them---no \textit{fast}
crushing rebuttal. By thinking carefully, you may be able to win over
the audience, if this is a public debate. Unfortunately you cannot just
blurt out, ``Foolish mortal, the Quantitative Way is
beyond your comprehension, and the beliefs you lightly name
`certain' are less assured than the
least of our mighty hypotheses.''
It's a difference of \textit{life-gestalt} that
isn't easy to describe in words at all, let alone
quickly.}

{
 What might you try, rhetorically, in front of an audience? Hard to
say \ldots maybe:}

{
 ``The power of science comes from having the
ability to change our minds and admit we're wrong. If
you've never admitted you're wrong, it
doesn't mean you've made fewer
mistakes.''}

{
 ``Anyone can \textit{say} they're
absolutely certain. It's a bit harder to never, ever
make any mistakes. Scientists understand the difference, so they
don't say they're absolutely certain.
That's all. It doesn't mean that they
have any specific reason to doubt a theory---absolutely every scrap of
evidence can be going the same way, all the stars and planets lined up
like dominos in support of a single hypothesis, and the scientists
still won't say they're absolutely
sure, because they've just got higher standards. It
doesn't mean scientists are less \textit{entitled} to
certainty than, say, the politicians who always seem so sure of
everything.''}

{
 ``Scientists don't use the phrase
`not absolutely certain' the way
you're used to from regular conversation. I mean,
suppose you went to the doctor, and got a blood test, and the doctor
came back and said, `We ran some tests, and
it's not absolutely certain that you're
not made out of cheese, and there's a non-zero chance
that twenty fairies made out of sentient chocolate are singing the
``I love you'' song from Barney
inside your lower intestine.' Run for the hills, your
doctor needs a doctor. When a scientist says the same thing, it means
that they think the probability is so tiny that you
couldn't see it with an electron microscope, but the
scientist is willing to see the evidence in the extremely unlikely
event that you have it.''}

{
 ``Would you be willing to change your mind about
the things you call `certain' if you saw
enough evidence? I mean, suppose that God himself descended from the
clouds and told you that your whole religion was true except for the
Virgin Birth. If that would change your mind, you can't
say you're absolutely certain of the Virgin Birth. For
technical reasons of probability theory, if it's
theoretically possible for you to change your mind about something, it
can't have a probability exactly equal to one. The
uncertainty might be smaller than a dust speck, but it has to be there.
And if you wouldn't change your mind even if God told
you otherwise, then you have a problem with refusing to admit
you're wrong that transcends anything a mortal like me
can say to you, I guess.''}

{
 But, in a way, the more interesting question is what you say to
someone \textit{not} in front of an audience. How do you begin the long
process of teaching someone to live in a universe without certainty?}

{
 I think the first, beginning step should be understanding that you
\textit{can} live without certainty---that \textit{if,}
\textit{hypothetically speaking,} you couldn't be
certain of anything, it would not deprive you of the ability to make
moral or factual distinctions. To paraphrase Lois Bujold,
``Don't push harder, lower the
resistance.''}

{
 One of the common \textit{defenses} of Absolute Authority is
something I call ``The Argument From The Argument From
Gray,'' which runs like this:}

{
  \textit{Moral relativists say:} The world isn't
black and white, therefore: Everything is gray, therefore: No one is
better than anyone else, therefore: I can do whatever I want and you
can't stop me bwahahaha. }

{
 But we've got to be able to stop people from
committing murder.}

{
 Therefore there has to be some way of being absolutely certain, or
the moral relativists win.}

{
 Reversed stupidity is not intelligence. You can't
arrive at a correct answer by reversing \textit{every single} line of
an argument that ends with a bad conclusion---it gives the fool too
much detailed control over you. Every single line must be correct for a
mathematical argument to carry. And it doesn't follow,
from the fact that moral relativists say ``The world
isn't black and white,'' that this is
false, any more than it follows, from Stalin's belief
that 2 + 2 = 4, that ``2 + 2 = 4''
is false. The error (and it only takes one) is in the leap from the
two-color view to the single-color view, that all grays are the same
shade.}

{
 It would concede far too much (indeed, concede the whole argument)
to agree with the premise that you need absolute knowledge of
absolutely good options and absolutely evil options in order to be
moral. You can have uncertain knowledge of relatively better and
relatively worse options, and still choose. It should be routine, in
fact, not something to get all dramatic about.}

{
 I mean, yes, if you have to choose between two alternatives A and
B, and you somehow succeed in establishing knowably certain
well-calibrated 100\% confidence that A is absolutely and entirely
desirable and that B is the sum of everything evil and disgusting, then
this is a \textit{sufficient} condition for choosing A over B. It is
not a \textit{necessary} condition.}

{
 Oh, and: Logical fallacy: Appeal to consequences of belief.}

{
 Let's see, what else do they need to know? Well,
there's the entire rationalist culture which says that
doubt, questioning, and confession of error are not terrible shameful
things.}

{
 There's the whole notion of gaining information by
\textit{looking at things}, rather than being proselytized. When you
look at things harder, sometimes you find out that
they're different from what you thought they were at
first glance; but it doesn't mean that Nature lied to
you, or that you should give up on seeing.}

{
 Then there's the concept of a calibrated
confidence---that ``probability''
isn't the same concept as the little progress bar in
your head that measures your emotional commitment to an idea.
It's more like a measure of how often, pragmatically,
in real life, people in a certain state of belief say things that are
actually true. If you take one hundred people and ask them each to make
a statement of which they are ``absolutely
certain,'' how many of these statements will be
correct? Not one hundred.}

{
 If anything, the statements that people are really fanatic about
are \textit{far less} likely to be correct than statements like
``the Sun is larger than the Moon''
that seem too obvious to get excited about. For every statement you can
find of which someone is ``absolutely
certain,'' you can probably find someone
``absolutely certain'' of its
opposite, because such fanatic professions of belief do not arise in
the absence of opposition. So the little progress bar in
people's heads that measures their emotional commitment
to a belief does not translate well into a calibrated confidence---it
doesn't even behave monotonically.}

{
 As for ``absolute
certainty''---well, if you say that something is
99.9999\% probable, it means you think you could make \textit{one
million} equally strong independent statements, \textit{one after the
other}, over the course of a solid year or so, and be wrong, on
average, around once. This is incredible enough. (It's
amazing to realize we can actually \textit{get} that level of
confidence for ``Thou shalt not win the
lottery.'') So let us say nothing of probability 1.0.
Once you realize you don't \textit{need} probabilities
of 1.0 to get along in life, you'll realize how
absolutely ridiculous it is to think you could ever get to 1.0 with a
human brain. A probability of 1.0 isn't just certainty,
it's \textit{infinite certainty.}}

{
 In fact, it seems to me that to prevent public misunderstanding,
maybe scientists should go around saying ``We are not
INFINITELY certain'' rather than
``We are not certain.'' For the
latter case, in ordinary discourse, suggests you know some specific
reason for doubt.}

\myendsectiontext

\mysection{How to Convince Me That 2 + 2 = 3}

{
 In What is Evidence? I wrote:}

{
 This is why rationalists put such a heavy premium on the
paradoxical-seeming claim that a belief is only really
\textit{worthwhile} if you could, in principle, be persuaded to believe
otherwise. If your retina ended up in the same state regardless of what
light entered it, you would be blind \ldots Hence the phrase,
``blind faith.'' If what you believe
doesn't depend on what you see, you've
been blinded as effectively as by poking out your eyeballs.}

{
 Cihan Baran replied:}

{
 I can not conceive of a situation that would make 2 + 2 = 4 false.
Perhaps for that reason, my belief in 2 + 2 = 4 is unconditional.}

{
 I admit, I cannot conceive of a
``situation'' that would
\textit{make} 2 + 2 = 4 false. (There are redefinitions, but those are
not ``situations,'' and then
you're no longer talking about 2, 4, =, or +.) But that
doesn't make my belief unconditional. I find it quite
easy to imagine a situation which would \textit{convince} me that 2 + 2
= 3.}

{
 Suppose I got up one morning, and took out two earplugs, and set
them down next to two other earplugs on my nighttable, and noticed that
there were now three earplugs, without any earplugs having appeared or
disappeared---in contrast to my stored memory that 2 + 2 was supposed
to equal 4. Moreover, when I visualized the process in my own mind, it
seemed that making XX and XX come out to XXXX required an extra X to
appear from nowhere, and was, moreover, inconsistent with other
arithmetic I visualized, since subtracting XX from XXX left XX, but
subtracting XX from XXXX left XXX. This would conflict with my stored
memory that 3 - 2 = 1, but memory would be absurd in the face of
physical and mental confirmation that XXX - XX = XX.}

{
 I would also check a pocket calculator, Google, and perhaps my
copy of 1984 where Winston writes that ``Freedom is
the freedom to say two plus two equals three.'' All
of these would naturally show that the rest of the world agreed with my
current visualization, and disagreed with my memory, that 2 + 2 = 3.}

{
 How could I possibly have ever been so deluded as to believe that
2 + 2 = 4? Two explanations would come to mind: First, a neurological
fault (possibly caused by a sneeze) had made all the additive sums in
my stored memory go up by one. Second, someone was messing with me, by
hypnosis or by my being a computer simulation. In the second case, I
would think it more likely that they had messed with my arithmetic
\textit{recall} than that 2 + 2 \textit{actually} equalled 4. Neither
of these plausible-sounding explanations would prevent me from noticing
that I was very, very, \textit{very} confused.}

{
 What would convince me that 2 + 2 = 3, in other words, is exactly
the same kind of evidence that currently convinces me that 2 + 2 = 4:
The evidential crossfire of physical observation, mental visualization,
and social agreement.}

{
 There was a time when I had no idea that 2 + 2 = 4. I did not
arrive at this \textit{new} belief by random processes---then there
would have been no particular reason for my brain to end up storing
``2 + 2 = 4'' instead of
``2 + 2 = 7.'' The fact that my
brain stores an answer surprisingly similar to what happens when I lay
down two earplugs alongside two earplugs, calls forth an explanation of
what entanglement produces this strange mirroring of mind and reality.}

{
 There's really only two possibilities, for a
belief of fact---either the belief got there via a mind-reality
entangling process, or not. If not, the belief can't be
correct except by coincidence. For beliefs with the slightest shred of
internal complexity (requiring a computer program of more than 10 bits
to simulate), the space of possibilities is large enough that
coincidence vanishes.}

{
 Unconditional facts are not the same as unconditional beliefs. If
entangled evidence convinces me that a fact is unconditional, this
doesn't mean I always believed in the fact without need
of entangled evidence.}

{
 I believe that 2 + 2 = 4, and I find it quite easy to conceive of
a situation which would convince me that 2 + 2 = 3. Namely, the same
sort of situation that currently convinces me that 2 + 2 = 4. Thus I do
not fear that I am a victim of blind faith.}

{
 If there are any Christians in the audience \textit{who know
Bayes's Theorem} (no numerophobes, please), might I
inquire of you what situation would convince you of the truth of Islam?
Presumably it would be the same sort of situation causally responsible
for producing your current belief in Christianity: We would push you
screaming out of the uterus of a Muslim woman, and have you raised by
Muslim parents who continually told you that it is good to believe
unconditionally in Islam. Or is there more to it than that? If so, what
situation would convince you of Islam, or at least, non-Christianity?}

\myendsectiontext

\mysection{Infinite Certainty}

{
 In Absolute Authority, I argued that you don't
\textit{need} infinite certainty:}

{
 If you have to choose between two alternatives A and B, and you
somehow succeed in establishing knowably certain well-calibrated 100\%
confidence that A is absolutely and entirely desirable and that B is
the sum of everything evil and disgusting, then this is a
\textit{sufficient} condition for choosing A over B. It is not a
\textit{necessary} condition \ldots You can have uncertain knowledge of
relatively better and relatively worse options, and still choose. It
should be routine, in fact.}

{
 Concerning the proposition that 2 + 2 = 4, we must distinguish
between the map and the territory. Given the seeming absolute stability
and universality of physical laws, it's possible that
never, in the whole history of the universe, has any particle exceeded
the local lightspeed limit. That is, the lightspeed limit may be, not
just true 99\% of the time, or 99.9999\% of the time, or (1 -
1/googolplex) of the time, but simply \textit{always and absolutely
true}.}

{
 But whether we can ever have \textit{absolute confidence} in the
lightspeed limit is a whole 'nother question. The map
is not the territory.}

{
 It may be entirely and wholly true that a student plagiarized
their assignment, but whether you have any knowledge of this fact at
all---let alone \textit{absolute} confidence in the belief---is a
separate issue. If you flip a coin and then don't look
at it, it may be completely true that the coin is showing heads, and
you may be completely unsure of whether the coin is showing heads or
tails. A degree of uncertainty is not the same as a degree of truth or
a frequency of occurrence.}

{
 The same holds for mathematical truths. It's
questionable whether the statement ``2 + 2 =
4'' or ``In Peano arithmetic, SS0 +
SS0 = SSSS0'' can be said to be \textit{true} in any
purely abstract sense, apart from physical systems that seem to behave
in ways similar to the Peano axioms. Having said this, I will charge
right ahead and guess that, in whatever sense ``2 + 2
= 4'' is true at all, it is always and precisely
true, not just roughly true (``2 + 2 actually equals
4.0000004'') or true 999,999,999,999 times out of
1,000,000,000,000.}

{
 I'm not totally sure what
``true'' should mean in this case,
but I stand by my guess. The credibility of ``2 + 2 =
4 is always true'' far exceeds the credibility of any
particular philosophical position on what
``true,''
``always,'' or
``is'' means in the statement
above.}

{
 This doesn't mean, though, that I have
\textit{absolute confidence} that 2 + 2 = 4. See the previous
discussion on how to convince me that 2 + 2 = 3, which could be done
using much the same sort of evidence that convinced me that 2 + 2 = 4
in the first place. I could have hallucinated all that previous
evidence, or I could be misremembering it. In the annals of neurology
there are stranger brain dysfunctions than this.}

{
 So if we attach some probability to the statement
``2 + 2 = 4,'' then what should the
probability be? What you seek to attain in a case like this is good
calibration---statements to which you assign ``99\%
probability'' come true 99 times out of 100. This is
actually a hell of a lot more difficult than you might think. Take a
hundred people, and ask each of them to make ten statements of which
they are ``99\% confident.'' Of the
1,000 statements, do you think that around 10 will be wrong?}

{
 I am not going to discuss the actual experiments that have been
done on calibration---you can find them in my book chapter
``Cognitive biases potentially affecting judgment of
global risks''---because I've seen
that when I blurt this out to people without proper preparation, they
thereafter use it as a Fully General Counterargument, which somehow
leaps to mind whenever they have to discount the confidence of someone
whose opinion they dislike, and fails to be available when they
consider their own opinions. So I try not to talk about the experiments
on calibration except as part of a structured presentation of
rationality that includes warnings against motivated skepticism.}

{
 But the observed calibration of human beings who say they are
``99\% confident'' is not 99\%
accuracy.}

{
 Suppose you say that you're 99.99\% confident that
2 + 2 = 4. Then you have just asserted that you could make 10,000
\textit{independent} statements, in which you repose equal confidence,
and be wrong, on average, around once. Maybe for 2 + 2 = 4 this
extraordinary degree of confidence would be possible:
``2 + 2 = 4'' is extremely simple,
and mathematical as well as empirical, and widely believed socially
(not with passionate affirmation but just quietly taken for granted).
So maybe you really could get up to 99.99\% confidence on this one.}

{
 I don't think you could get up to 99.99\%
confidence for assertions like ``53 is a prime
number.'' Yes, it seems likely, but by the time you
tried to set up protocols that would let you assert 10,000
\textit{independent} statements of this sort---that is, not just a set
of statements about prime numbers, but a new protocol each time---you
would fail more than once. Peter de Blanc has an amusing anecdote on
this point. (I told him not to do it again.)}

{
 Yet the map is not the territory: if I say that I am 99\%
confident that 2 + 2 = 4, it doesn't mean that I think
``2 + 2 = 4'' is true to within 99\%
precision, or that ``2 + 2 = 4'' is
true 99 times out of 100. The proposition in which I repose my
confidence is the proposition that ``2 + 2 = 4 is
always and exactly true,'' not the proposition
``2 + 2 = 4 is mostly and usually
true.''}

{
 As for the notion that you could get up to 100\% confidence in a
mathematical proposition---well, really now! If you say 99.9999\%
confidence, you're implying that you could make
\textit{one million} equally fraught statements, one after the other,
and be wrong, on average, about once. That's around a
solid year's worth of talking, if you can make one
assertion every 20 seconds and you talk for 16 hours a day.}

{
 Assert 99.9999999999\% confidence, and you're
taking it up to a trillion. Now you're going to talk
for a hundred human lifetimes, and not be wrong even once?}

{
 Assert a confidence of (1 - 1/googolplex) and your ego far exceeds
that of mental patients who think they're God.}

{
 And a googolplex is a lot smaller than even relatively small
inconceivably huge numbers like 3 $\uparrow \uparrow \uparrow $ 3. But
even a confidence of (1 - 1$/$3 $\uparrow \uparrow \uparrow $ 3)
isn't all that much closer to \textbf{PROBABILITY 1}
than being 90\% sure of something.}

{
 If all else fails, the hypothetical Dark Lords of the Matrix, who
are \textit{right now} tampering with your brain's
credibility assessment of \textit{this very sentence}, will bar the
path and defend us from the scourge of infinite certainty.}

{
 Am I absolutely sure of that?}

{
 Why, of course not.}

{
 As Rafal Smigrodski once said:}

{
 I would say you should be able to assign a less than 1 certainty
level to the mathematical concepts which are necessary to derive
Bayes's rule itself, and still practically use it. I am
not totally sure I have to be always unsure. Maybe I could be
legitimately sure about something. But once I assign a probability of 1
to a proposition, I can never undo it. No matter what I see or learn, I
have to reject everything that disagrees with the axiom. I
don't like the idea of not being able to change my
mind, ever.}

\myendsectiontext

\mysection{0 And 1 Are Not Probabilities}

{
 One, two, and three are all integers, and so is negative four. If
you keep counting up, or keep counting down, you're
bound to encounter a whole lot more integers. You will not, however,
encounter anything called ``positive
infinity'' or ``negative
infinity,'' so these are not integers. }

{
 Positive and negative infinity are not integers, but rather
special symbols for talking about the behavior of integers. People
sometimes say something like, ``5 + infinity =
infinity,'' because if you start at 5 and keep
counting up without ever stopping, you'll get higher
and higher numbers without limit. But it doesn't follow
from this that ``infinity - infinity =
5.'' You can't count up from 0
without ever stopping, and then count down without ever stopping, and
then find yourself at 5 when you're done.}

{
 From this we can see that infinity is not only not-an-integer, it
doesn't even \textit{behave} like an integer. If you
unwisely try to mix up infinities with integers, you'll
need all sorts of special new inconsistent-seeming behaviors which you
don't need for 1, 2, 3 and other \textit{actual}
integers.}

{
 Even though infinity isn't an integer, you
don't have to worry about being left at a loss for
numbers. Although people have seen five sheep, millions of grains of
sand, and septillions of atoms, no one has ever counted an infinity of
anything. The same with continuous quantities---people have measured
dust specks a millimeter across, animals a meter across, cities
kilometers across, and galaxies thousands of lightyears across, but no
one has ever measured anything an infinity across. In the real world,
you don't \textit{need} a whole lot of infinity.}

{
 (I should note for the more sophisticated readers in the audience
that they do not need to write me with elaborate explanations of, say,
the difference between ordinal numbers and cardinal numbers. Yes, I
possess various advanced set-theoretic definitions of infinity, but I
don't see a good use for them in probability theory.
See below.)}

{
 In the usual way of writing probabilities, probabilities are
between 0 and 1. A coin might have a probability of 0.5 of coming up
tails, or the weatherman might assign probability 0.9 to rain
tomorrow.}

{
 This isn't the only way of writing probabilities,
though. For example, you can transform probabilities into odds via the
transformation O = (P$/$(1 - P)). So a probability of 50\% would go to
odds of 0.5/0.5 or 1, usually written 1:1, while a probability of 0.9
would go to odds of 0.9/0.1 or 9, usually written 9:1. To take odds
back to probabilities you use P = (O$/$(1 + O)), and this is perfectly
reversible, so the transformation is an isomorphism---a two-way
reversible mapping. Thus, probabilities and odds are isomorphic, and
you can use one or the other according to convenience.}

{
 For example, it's more convenient to use odds when
you're doing Bayesian updates. Let's
say that I roll a six-sided die: If any face except 1 comes up,
there's a 10\% chance of hearing a bell, but if the
face 1 comes up, there's a 20\% chance of hearing the
bell. Now I roll the die, and hear a bell. What are the \textit{odds}
that the face showing is 1? Well, the prior odds are 1:5 (corresponding
to the real number 1/5 = 0.20) and the likelihood ratio is 0.2:0.1
(corresponding to the real number 2) and I can just multiply these two
together to get the posterior odds 2:5 (corresponding to the real
number 2/5 or 0.40). Then I convert back into a probability, if I like,
and get (0.4$/$1.4) = 2$/$7 = \~{}29\%.}

{
 So odds are more manageable for Bayesian updates---if you use
probabilities, you've got to deploy
Bayes's Theorem in its complicated version. But
probabilities are more convenient for answering questions like
``If I roll a six-sided die, what's
the chance of seeing a number from 1 to 4?'' You can
add up the probabilities of 1/6 for each side and get 4/6, but you
can't add up the odds ratios of 0.2 for each side and
get an odds ratio of 0.8.}

{
 Why am I saying all this? To show that ``odd
ratios'' are just as legitimate a way of mapping
uncertainties onto real numbers as
``probabilities.'' Odds ratios are
more convenient for some operations, probabilities are more convenient
for others. A famous proof called Cox's Theorem (plus
various extensions and refinements thereof) shows that all ways of
representing uncertainties that obey some reasonable-sounding
constraints, end up isomorphic to each other.}

{
 Why does it matter that odds ratios are just as legitimate as
probabilities? Probabilities as ordinarily written are between 0 and 1,
and both 0 and 1 look like they ought to be readily reachable
quantities---it's easy to see 1 zebra or 0 unicorns.
But when you transform probabilities onto odds ratios, 0 goes to 0, but
1 goes to positive infinity. Now absolute truth doesn't
look like it should be so easy to reach.}

{
 A representation that makes it even simpler to do Bayesian updates
is the log odds---this is how E. T. Jaynes recommended thinking about
probabilities. For example, let's say that the prior
probability of a proposition is 0.0001---this corresponds to a log odds
of around -40 decibels. Then you see evidence that seems 100 times more
likely if the proposition is true than if it is false. This is 20
decibels of evidence. So the posterior odds are around -40 dB + 20 dB =
-20 dB, that is, the posterior probability is \~{}0.01.}

{
 When you transform probabilities to log odds, 0 goes onto negative
infinity and 1 goes onto positive infinity. Now both infinite certainty
and infinite improbability seem a bit more out-of-reach.}

{
 In probabilities, 0.9999 and 0.99999 seem to be only 0.00009
apart, so that 0.502 is much further away from 0.503 than 0.9999 is
from 0.99999. To get to probability 1 from probability 0.99999, it
seems like you should need to travel a distance of merely 0.00001.}

{
 But when you transform to odds ratios, 0.502 and 0.503 go to 1.008
and 1.012, and 0.9999 and 0.99999 go to 9,999 and 99,999. And when you
transform to log odds, 0.502 and 0.503 go to 0.03 decibels and 0.05
decibels, but 0.9999 and 0.99999 go to 40 decibels and 50 decibels.}

{
 When you work in log odds, \textbf{the distance between any two
degrees of uncertainty equals the amount of evidence you would need to
go from one to the other}. That is, the log odds gives us a natural
measure of spacing among degrees of confidence.}

{
 Using the log odds exposes the fact that reaching infinite
certainty requires infinitely strong evidence, just as infinite
absurdity requires infinitely strong counterevidence.}

{
 Furthermore, all sorts of standard theorems in probability have
special cases if you try to plug 1s or 0s into them---like what happens
if you try to do a Bayesian update on an observation to which you
assigned probability 0.}

{
 So I propose that it makes sense to say that 1 and 0 are not in
the probabilities; just as negative and positive infinity, which do not
obey the field axioms, are not in the real numbers.}

{
 The main reason this would upset probability theorists is that we
would need to rederive theorems previously obtained by assuming that we
can marginalize over a joint probability by adding up all the pieces
and having them sum to 1.}

{
 However, in the real world, when you roll a die, it
doesn't literally have infinite certainty of coming up
some number between 1 and 6. The die might land on its edge; or get
struck by a meteor; or the Dark Lords of the Matrix might reach in and
write ``37'' on one side.}

{
 If you made a magical symbol to stand for ``all
possibilities I haven't considered,''
then you could marginalize over the events including this magical
symbol, and arrive at a magical symbol
``T'' that stands for infinite
certainty.}

{
 But I would rather ask whether there's some way to
derive a theorem without using magic symbols with special behaviors.
That would be more elegant. Just as there are mathematicians who refuse
to believe in the law of the excluded middle or infinite sets, I would
like to be a probability theorist who doesn't believe
in absolute certainty.}

\myendsectiontext

\mysection{Your Rationality Is My Business}

{
 Some responses to Lotteries: A Waste of Hope chided me for daring
to criticize others' decisions; if someone else chooses
to buy lottery tickets, who am I to disagree? This is a special case of
a more general question: What business is it of mine, if someone else
chooses to believe what is pleasant rather than what is true?
Can't we each choose for ourselves whether to care
about the truth? }

{
 An obvious snappy comeback is: ``Why do
\textit{you} care whether \textit{I} care whether someone \textit{else}
cares about the truth?'' It is somewhat inconsistent
for your utility function to contain a negative term for anyone
else's utility function having a term for someone
else's utility function. But that is only a snappy
comeback, not an answer.}

{
 So here then is my answer: I believe that it is right and proper
for me, as a human being, to have an interest in the future, and what
human civilization becomes in the future. One of those interests is the
human pursuit of truth, which has strengthened slowly over the
generations (for there was not always Science). I wish to strengthen
that pursuit further, in \textit{this} generation. That is a wish of
mine, for the Future. For we are all of us players upon that vast
gameboard, whether we accept the responsibility or not.}

{
 And that makes \textit{your} rationality \textit{my} business.}

{
 Is this a dangerous idea? Yes, and not just pleasantly edgy
``dangerous.'' People have been
burned to death because some priest decided that they
didn't think the way they should. Deciding to burn
people to death because they ``don't
think properly''---that's a revolting
kind of reasoning, isn't it? You
wouldn't want people to think that way, why,
it's \textit{disgusting.} People who think like that,
well, we'll have to do something about them \ldots}

{
 I agree! Here's my proposal: Let's
argue against bad ideas but \textit{not} set their bearers on fire.}

{
 The syllogism we desire to avoid runs: ``I think
Susie said a bad thing, \textit{therefore,} Susie should be set on
fire.'' Some try to avoid the syllogism by labeling
it improper to think that Susie said a bad thing. No one should judge
anyone, ever; anyone who judges is committing a terrible sin, and
should be publicly pilloried for it.}

{
 As for myself, I deny the \textit{therefore.} My syllogism runs,
``I think Susie said something wrong,
\textit{therefore,} I will argue against what she said, but I will not
set her on fire, or try to stop her from talking by violence or
regulation \ldots''}

{
 We are all of us players upon that vast gameboard; and one of my
interests for the Future is to make the game fair. The counterintuitive
idea underlying science is that factual disagreements should be fought
out with experiments and mathematics, not violence and edicts. This
incredible notion can be extended beyond science, to a fair fight for
the whole Future. You should have to win by convincing people, and
should not be allowed to burn them. This is one of the principles of
Rationality, to which I have pledged my allegiance.}

{
 People who advocate relativism or selfishness do not appear to me
to be truly relativistic or selfish. If they were really relativistic,
they would not judge. If they were really selfish, they would get on
with making money instead of arguing passionately with others. Rather,
they have chosen the side of Relativism, whose goal upon that vast
gameboard is to prevent the players---\textit{all} the players---from
making certain kinds of judgments. Or they have chosen the side of
Selfishness, whose goal is to make \textit{all} players selfish. And
then they play the game, fairly or unfairly according to their wisdom.}

{
 If there are any true Relativists or Selfishes, we do not hear
them---they remain silent, non-players.}

{
 I cannot help but care how you think, because---as I cannot help
but see the universe---each time a human being turns away from the
truth, the unfolding story of humankind becomes a little darker. In
many cases, it is a small darkness only. (Someone
doesn't \textit{always} end up getting hurt.) Lying to
yourself, in the privacy of your own thoughts, does not shadow
humanity's history so much as telling public lies or
setting people on fire. Yet there is a part of me which cannot help but
mourn. And so long as I \textit{don't} try to set you
on fire---only argue with your ideas---I believe that it is right and
proper to me, as a human, that I care about my fellow humans. That,
also, is a position I defend into the Future.}

\myendsectiontext


\chapter{Politics and Rationality}

\mysection{Politics is the Mind{}-Killer}

{
 People go funny in the head when talking about politics. The
evolutionary reasons for this are so obvious as to be worth belaboring:
In the ancestral environment, politics was a matter of life and death.
And sex, and wealth, and allies, and reputation \ldots When, today, you
get into an argument about whether
``we'' ought to raise the minimum
wage, you're executing adaptations for an ancestral
environment where being on the wrong side of the argument could get you
killed. Being on the \textit{right} side of the argument could let
\textit{you} kill your hated rival! }

{
 If you want to make a point about science, or rationality, then my
advice is to not choose a domain from \textit{contemporary} politics if
you can possibly avoid it. If your point is inherently about politics,
then talk about Louis XVI during the French Revolution. Politics is an
important domain to which we should individually apply our
rationality---but it's a terrible domain in which to
\textit{learn} rationality, or discuss rationality, unless all the
discussants are already rational.}

{
 Politics is an extension of war by other means. Arguments are
soldiers. Once you know which side you're on, you must
support all arguments of that side, and attack all arguments that
appear to favor the enemy side; otherwise it's like
stabbing your soldiers in the back---providing aid and comfort to the
enemy. People who would be level-headed about evenhandedly weighing all
sides of an issue in their professional life as scientists, can
suddenly turn into slogan-chanting zombies when there's
a Blue or Green position on an issue.}

{
 In Artificial Intelligence, and particularly in the domain of
nonmonotonic reasoning, there's a standard problem:
``All Quakers are pacifists. All Republicans are not
pacifists. Nixon is a Quaker and a Republican. Is Nixon a
pacifist?''}

{
 What on Earth was the point of choosing this as an example? To
rouse the political emotions of the readers and distract them from the
main question? To make Republicans feel unwelcome in courses on
Artificial Intelligence and discourage them from entering the field?
(And no, I am not a Republican. Or a Democrat.)}

{
 Why would anyone pick such a \textit{distracting} example to
illustrate nonmonotonic reasoning? Probably because the author just
couldn't resist getting in a good, solid dig at those
hated Greens. It feels so \textit{good} to get in a hearty punch,
y'know, it's like trying to resist a
chocolate cookie.}

{
 As with chocolate cookies, not everything that feels pleasurable
is good for you.}

{
 I'm not saying that I think we should be
apolitical, or even that we should adopt Wikipedia's
ideal of the Neutral Point of View. But try to resist getting in those
good, solid digs if you can possibly avoid it. If your topic
legitimately relates to attempts to ban evolution in school curricula,
then go ahead and talk about it---but don't blame it
explicitly on the whole Republican Party; some of your readers may be
Republicans, and they may feel that the problem is a few rogues, not
the entire party. As with Wikipedia's NPOV, it
doesn't matter whether (you think) the Republican Party
really \textit{is} at fault. It's just better for the
spiritual growth of the community to discuss the issue without invoking
color politics.}

\myendsectiontext

\mysection{Policy Debates Should Not Appear One{}-Sided}

{
 Robin Hanson proposed stores where banned products could be sold.
There are a number of excellent arguments for such a policy---an
inherent right of individual liberty, the career incentive of
bureaucrats to prohibit \textit{everything}, legislators being just as
biased as individuals. But even so (I replied), \textit{some} poor,
honest, not overwhelmingly educated mother of five children is going to
go into these stores and buy a ``Dr.
Snakeoil's Sulfuric Acid Drink'' for
her arthritis and die, leaving her orphans to weep on national
television. }

{
 I was just making a simple factual observation. Why did some
people think it was an argument in favor of regulation?}

{
 On questions of simple fact (for example, whether Earthly life
arose by natural selection) there's a legitimate
expectation that the argument should be a one-sided battle; the facts
themselves are either one way or another, and the so-called
``balance of evidence'' should
reflect this. Indeed, under the Bayesian definition of evidence,
``strong evidence'' is just that
sort of evidence which we only expect to find on one side of an
argument.}

{
 But there is no reason for complex actions with many consequences
to exhibit this onesidedness property. Why do people seem to want their
\textit{policy} debates to be one-sided?}

{
 Politics is the mind-killer. Arguments are soldiers. Once you know
which side you're on, you must support all arguments of
that side, and attack all arguments that appear to favor the enemy
side; otherwise it's like stabbing your soldiers in the
back. If you abide within that pattern, policy debates will also appear
one-sided to you---the costs and drawbacks of your favored policy are
enemy soldiers, to be attacked by any means necessary.}

{
 One should also be aware of a related failure pattern, thinking
that the course of Deep Wisdom is to compromise with perfect evenness
between whichever two policy positions receive the most airtime. A
policy may legitimately have \textit{lopsided} costs or benefits. If
policy questions were not tilted one way or the other, we would be
unable to make decisions about them. But there is also a human tendency
to deny all costs of a favored policy, or deny all benefits of a
disfavored policy; and people will therefore tend to think policy
tradeoffs are tilted much further than they actually are.}

{
 If you allow shops that sell otherwise banned products, some poor,
honest, poorly educated mother of five kids is going to buy something
that kills her. This is a prediction about a factual consequence, and
as a factual question it appears rather straightforward---a sane person
should readily confess this to be true regardless of which stance they
take on the policy issue. You may \textit{also} think that making
things illegal just makes them more expensive, that regulators will
abuse their power, or that her individual freedom trumps your desire to
meddle with her life. But, as a matter of simple fact,
she's still going to die.}

{
 We live in an unfair universe. Like all primates, humans have
strong negative reactions to perceived unfairness; thus we find this
fact stressful. There are two popular methods of dealing with the
resulting cognitive dissonance. First, one may change
one's view of the facts---deny that the unfair events
took place, or edit the history to make it appear fair. (This is
mediated by the affect heuristic and the just-world fallacy.) Second,
one may change one's morality---deny that the events
are unfair.}

{
 Some libertarians might say that if you go into a
``banned products shop,'' passing
clear warning labels that say THINGS IN THIS STORE MAY KILL YOU, and
buy something that kills you, then it's your own fault
and you deserve it. If that were a moral truth, there would be
\textit{no downside} to having shops that sell banned products. It
wouldn't just be a \textit{net benefit}, it would be a
\textit{one-sided} tradeoff with no drawbacks.}

{
 Others argue that regulators can be trained to choose rationally
and in harmony with consumer interests; if those were the facts of the
matter then (in their moral view) there would be \textit{no downside}
to regulation.}

{
 Like it or not, there's a birth lottery for
intelligence---though this is one of the cases where the
universe's unfairness is so extreme that many people
choose to deny the facts. The experimental evidence for a purely
genetic component of 0.6--0.8 is overwhelming, but even if this were to
be denied, you don't choose your parental upbringing or
your early schools either.}

{
 I was raised to believe that denying reality is a \textit{moral
wrong.} If I were to engage in wishful optimism about how Sulfuric Acid
Drink was likely to benefit me, I would be doing something that I was
\textit{warned} against and raised to regard as unacceptable. Some
people are born into environments---we won't discuss
their genes, because that part is too unfair---where the local witch
doctor tells them that it is \textit{right} to have faith and
\textit{wrong} to be skeptical. In all goodwill, they follow this
advice and die. Unlike you, they weren't raised to
believe that people are responsible for their individual choices to
follow society's lead. Do you really think
you're so smart that you would have been a proper
scientific skeptic even if you'd been born in 500 CE?
Yes, there is a birth lottery, no matter what you believe about genes.}

{
 Saying ``People who buy dangerous products
deserve to get hurt!'' is not tough-minded. It is a
way of refusing to live in an unfair universe. Real tough-mindedness is
saying, ``Yes, sulfuric acid is a horrible painful
death, and no, that mother of five children didn't
deserve it, but we're going to keep the shops open
anyway because we did this cost-benefit
calculation.'' Can you imagine a politician saying
that? Neither can I. But insofar as economists have the power to
influence policy, it might help if they could think it
privately---maybe even say it in journal articles, suitably dressed up
in polysyllabismic obfuscationalization so the media
can't quote it.}

{
 I don't think that when someone makes a stupid
choice and dies, this is a cause for celebration. I count it as a
tragedy. It is not always helping people, to save them from the
consequences of their own actions; but I draw a moral line at capital
punishment. If you're dead, you can't
learn from your mistakes.}

{
 Unfortunately the universe doesn't agree with me.
We'll see which one of us is still standing when this
is over.}

\myendsectiontext

\mysection{The Scales of Justice, the Notebook of Rationality}

{
 Lady Justice is widely depicted as carrying scales. A set of
scales has the property that whatever pulls one side down pushes the
other side up. This makes things very convenient and easy to track.
It's also usually a gross distortion. }

{
 In human discourse there is a natural tendency to treat discussion
as a form of combat, an extension of war, a sport; and in sports you
only need to keep track of how many points have been scored by each
team. There are only two sides, and every point scored against one side
is a point in favor of the other. Everyone in the audience keeps a
mental running count of how many points each speaker scores against the
other. At the end of the debate, the speaker who has scored more points
is, obviously, the winner; so everything that speaker says must be
true, and everything the loser says must be wrong.}

{
 ``The Affect Heuristic in Judgments of Risks and
Benefits'' studied whether subjects mixed up their
judgments of the possible benefits of a technology (e.g., nuclear
power), and the possible risks of that technology, into a single
overall good or bad feeling about the technology.\textsuperscript{1}
Suppose that I first tell you that a particular kind of nuclear reactor
generates less nuclear waste than competing reactor designs. But then I
tell you that the reactor is more unstable than competing designs, with
a greater danger of melting down if a sufficient number of things go
wrong simultaneously.}

{
 If the reactor is more likely to melt down, this seems like a
``point against'' the reactor, or a
``point against'' someone who argues
for building the reactor. And if the reactor produces less waste, this
is a ``point for'' the reactor, or a
``point for'' building it. So are
these two facts opposed to each other? No. In the real world, no. These
two facts may be cited by different sides of the same debate, but they
are logically distinct; the facts don't know whose side
they're on.}

{
 If it's a physical fact about a reactor design
that it's passively safe (won't go
supercritical even if the surrounding coolant systems and so on break
down), this doesn't imply that the reactor will
necessarily generate less waste, or produce electricity at a lower
cost. All these things would be good, but they are not the same good
thing. The amount of waste produced by the reactor arises from the
properties of that reactor. Other physical properties of the reactor
make the nuclear reaction more unstable. Even if some of the same
design properties are involved, you have to separately consider the
probability of meltdown, and the expected annual waste generated. These
are two different physical questions with two different factual
answers.}

{
 But studies such as the above show that people tend to judge
technologies---and many other problems---by an overall good or bad
feeling. If you tell people a reactor design produces less waste, they
rate its probability of meltdown as lower. This means getting the
\textit{wrong answer} to physical questions with definite factual
answers, because you have mixed up logically distinct
questions---treated facts like human soldiers on different sides of a
war, thinking that any soldier on one side can be used to fight any
soldier on the other side.}

{
 A set of scales is not wholly inappropriate for Lady Justice if
she is investigating a strictly factual question of guilt or innocence.
Either John Smith killed John Doe, or not. We are taught (by E. T.
Jaynes) that all Bayesian evidence consists of probability flows
\textit{between} hypotheses; there is no such thing as evidence that
``supports'' or
``contradicts'' a single hypothesis,
except insofar as other hypotheses do worse or better. So long as Lady
Justice is investigating a \textit{single}, strictly \textit{factual}
question with a \textit{binary} answer space, a set of scales would be
an appropriate tool. If Justitia must consider any more complex issue,
she should relinquish her scales or relinquish her sword.}

{
 Not all arguments reduce to mere up or down. Lady Rationality
carries a notebook, wherein she writes down all the facts that
aren't on anyone's side.}

\myendsectiontext


\bigskip

{
 1. Melissa L. Finucane et al., ``The Affect
Heuristic in Judgments of Risks and Benefits,''
\textit{Journal of Behavioral Decision Making} 13, no. 1 (2000):
1--17.}

\mysection{Correspondence Bias}

{
 The correspondence bias is the tendency to draw inferences about a
person's unique and enduring dispositions from
behaviors that can be entirely explained by the situations in which
they occur.}

{\raggedleft
 {}---Gilbert and Malone\textsuperscript{1}
\par}


\bigskip

{
 ~}

{
 We tend to see far too direct a correspondence between
others' actions and personalities. When we see someone
else kick a vending machine for no visible reason, we assume they are
``an angry person.'' But when you
yourself kick the vending machine, it's because the bus
was late, the train was early, your report is overdue, and now the
damned vending machine has eaten your lunch money for the second day in
a row. \textit{Surely,} you think to yourself, \textit{anyone would
kick the vending machine, in that situation.}}

{
 We attribute our own actions to our \textit{situations}, seeing
our behaviors as perfectly normal responses to experience. But when
someone else kicks a vending machine, we don't see
their past history trailing behind them in the air. We just see the
kick, for no reason \textit{we} know about, and we think this must be a
naturally angry person---since they lashed out without any
provocation.}

{
 Yet consider the prior probabilities. There are more late buses in
the world, than mutants born with unnaturally high anger levels that
cause them to sometimes spontaneously kick vending machines. Now the
average human is, in fact, a mutant. If I recall correctly, an average
individual has two to ten somatically expressed mutations. But any
\textit{given} DNA location is very unlikely to be affected. Similarly,
any given aspect of someone's disposition is probably
not very far from average. To suggest otherwise is to shoulder a burden
of improbability.}

{
 Even when people are informed explicitly of situational causes,
they don't seem to properly discount the observed
behavior. When subjects are told that a pro-abortion or anti-abortion
speaker was \textit{randomly assigned} to give a speech on that
position, subjects still think the speakers harbor leanings in the
direction randomly assigned.\textsuperscript{2}}

{
 It seems quite intuitive to explain rain by water spirits; explain
fire by a fire-stuff (phlogiston) escaping from burning matter; explain
the soporific effect of a medication by saying that it contains a
``dormitive potency.'' Reality
usually involves more complicated mechanisms: an evaporation and
condensation cycle underlying rain, oxidizing combustion underlying
fire, chemical interactions with the nervous system for soporifics. But
mechanisms sound more complicated than essences; they are harder to
think of, less available. So when someone kicks a vending machine, we
think they have an innate vending-machine-kicking-tendency.}

{
 Unless the ``someone'' who
kicks the machine is us---in which case we're behaving
perfectly normally, given our situations; surely anyone else would do
the same. Indeed, we overestimate how likely others are to respond the
same way we do---the ``false consensus
effect.'' Drinking students considerably overestimate
the fraction of fellow students who drink, but nondrinkers considerably
underestimate the fraction. The ``fundamental
attribution error'' refers to our tendency to
overattribute others' behaviors to their dispositions,
while reversing this tendency for ourselves.}

{
 \textit{To understand why people act the way they do, we must
first realize that everyone sees themselves as behaving normally.}
Don't ask what strange, mutant disposition they were
born with, which directly corresponds to their surface behavior.
Rather, ask what situations people see themselves as being in. Yes,
people do have dispositions---but there are not \textit{enough}
heritable quirks of disposition to directly account for all the surface
behaviors you see.}

{
 Suppose I gave you a control with two buttons, a red button and a
green button. The red button destroys the world, and the green button
stops the red button from being pressed. Which button would you press?
The green one. Anyone who gives a different answer is probably
overcomplicating the question.}

{
 And yet people sometimes ask me why I want to save the world. Like
I must have had a traumatic childhood or something. Really, it seems
like a pretty obvious decision \ldots if you see the situation in those
terms.}

{
 I may have non-average views which call for explanation---why do I
believe such things, when most people don't?---but
given those beliefs, my \textit{reaction} doesn't seem
to call forth an exceptional explanation. Perhaps I am a victim of
false consensus; perhaps I overestimate how many people would press the
green button if they saw the situation in those terms. But
y'know, I'd still bet
there'd be at least a \textit{substantial minority.}}

{
 Most people see themselves as perfectly normal, from the inside.
Even people you hate, people who do terrible things, are not
exceptional mutants. No mutations are required, alas. When you
understand this, you are ready to stop being surprised by human
events.}

\myendsectiontext


\bigskip

{
 1. Daniel T. Gilbert and Patrick S. Malone, ``The
Correspondence Bias,'' \textit{Psychological
Bulletin} 117, no. 1 (1995): 21--38,
http://www.wjh.harvard.edu/\~{}dtg/Gilbert\%20\&\%20Malone\%20(CORRESPONDENCE\%20BIAS).pdf.}

{
 2. Edward E. Jones and Victor A. Harris, ``The
Attribution of Attitudes,'' \textit{Journal of
Experimental Social Psychology} 3 (1967): 1--24,
http://www.radford.edu/\~{}jaspelme/443/spring-2007/Articles/Jones\_n\_Harris\_1967.pdf.}

\mysection{Are Your Enemies Innately Evil?}

{
 We see far too direct a correspondence between
others' actions and their inherent dispositions. We see
unusual dispositions that exactly match the unusual behavior, rather
than asking after real situations or imagined situations that could
explain the behavior. We hypothesize mutants. }

{
 When someone actually \textit{offends} us---commits an action of
which we (rightly or wrongly) disapprove---then, I observe, the
correspondence bias redoubles. There seems to be a \textit{very} strong
tendency to blame evil deeds on the Enemy's mutant,
evil disposition. Not as a moral point, but as a strict question of
prior probability, we should ask what the Enemy might believe about
their situation that would reduce the seeming bizarrity of their
behavior. This would allow us to hypothesize a less exceptional
disposition, and thereby shoulder a lesser burden of improbability.}

{
 On September 11th, 2001, nineteen Muslim males hijacked four jet
airliners in a deliberately suicidal effort to hurt the United States
of America. Now why do you suppose they might have done that? Because
they saw the USA as a beacon of freedom to the world, but were born
with a mutant disposition that made them hate freedom?}

{
 \textit{Realistically}, most people don't
construct their life stories with themselves as the villains. Everyone
is the hero of their own story. The Enemy's story, as
seen by the Enemy, \textit{is not going to make the Enemy look bad.} If
you try to construe motivations that \textit{would} make the Enemy look
bad, you'll end up flat wrong about what actually goes
on in the Enemy's mind.}

{
 But politics is the mind-killer. Debate is war; arguments are
soldiers. Once you know which side you're on, you must
support all arguments of that side, and attack all arguments that
appear to favor the opposing side; otherwise it's like
stabbing your soldiers in the back.}

{
 If the Enemy did have an evil disposition, that would be an
argument in favor of your side. And \textit{any} argument that favors
your side must be supported, no matter how silly---otherwise
you're letting up the pressure somewhere on the
battlefront. Everyone strives to outshine their neighbor in patriotic
denunciation, and no one dares to contradict. Soon the Enemy has horns,
bat wings, flaming breath, and fangs that drip corrosive venom. If you
deny any aspect of this on merely factual grounds, you are arguing the
Enemy's side; you are a traitor. Very few people will
understand that you aren't defending the Enemy, just
defending the truth.}

{
 If it took a mutant to do monstrous things, the history of the
human species would look very different. Mutants would be rare.}

{
 Or maybe the fear is that understanding will lead to forgiveness.
It's easier to shoot down evil mutants. It is a more
inspiring battle cry to scream, ``Die, vicious
scum!'' instead of ``Die, people who
could have been just like me but grew up in a different
environment!'' You might feel guilty killing people
who \textit{weren't} pure darkness.}

{
 This looks to me like the deep-seated yearning for a one-sided
policy debate in which the best policy has \textit{no} drawbacks. If an
army is crossing the border or a lunatic is coming at you with a knife,
the policy alternatives are (a) defend yourself or (b) lie down and
die. If you defend yourself, you may have to kill. If you kill someone
who could, in another world, have been your friend, that is a tragedy.
And it \textit{is} a tragedy. The other option, lying down and dying,
is also a tragedy. Why must there be a non-tragic option? Who says that
the best policy available must have no downside? If someone has to die,
it may as well be the initiator of force, to discourage future violence
and thereby minimize the total sum of death.}

{
 If the Enemy has an average disposition, and is acting from
beliefs about their situation that would make violence a typically
human response, then that doesn't mean their beliefs
are factually accurate. It doesn't mean
they're justified. It means you'll have
to shoot down someone who is the hero of their own story, and in their
novel the protagonist will die on page 80. That is a tragedy, but it is
better than the alternative tragedy. It is the choice that every police
officer makes, every day, to keep our neat little worlds from
dissolving into chaos.}

{
 When you accurately estimate the Enemy's
psychology---when you know what is really in the
Enemy's mind---that knowledge won't
feel like landing a delicious punch on the opposing side. It
won't give you a warm feeling of righteous indignation.
It won't make you feel good about yourself. If your
estimate makes you feel unbearably sad, you may be seeing the world as
it really is. More rarely, an accurate estimate may send shivers of
serious horror down your spine, as when dealing with true psychopaths,
or neurologically intact people with beliefs that have utterly
destroyed their sanity (Scientologists or Jesus Campers).}

{
 So let's come right out and say it---the 9/11
hijackers weren't evil mutants. They did not hate
freedom. They, too, were the heroes of their own stories, and they died
for what they believed was right---truth, justice, and the Islamic way.
If the hijackers saw themselves that way, it doesn't
mean their beliefs were true. If the hijackers saw themselves that way,
it doesn't mean that we have to agree that what they
did was justified. If the hijackers saw themselves that way, it
doesn't mean that the passengers of United Flight 93
should have stood aside and let it happen. It does mean that in another
world, if they had been raised in a different environment, those
hijackers might have been police officers. And that is indeed a
tragedy. Welcome to Earth.}

\myendsectiontext

\mysection{Reversed Stupidity Is Not Intelligence}

{
 ``\ldots then our people on that time-line went to
work with corrective action. Here.''}

{
 He wiped the screen and then began punching combinations. Page
after page appeared, bearing accounts of people who had claimed to have
seen the mysterious disks, and each report was more fantastic than the
last.}

{
 ``The standard smother-out
technique,'' Verkan Vall grinned.
``I only heard a little talk about the
`flying saucers,' and all of that was in
joke. In that order of culture, you can always discredit one true story
by setting up ten others, palpably false, parallel to
it.''}

{\raggedleft
 {}---H. Beam Piper, \textit{Police Operation}\textsuperscript{1}
\par}


\bigskip

{
 ~}

{
 Piper had a point. Pers'nally, I
don't believe there are any poorly hidden aliens
infesting these parts. But my disbelief has nothing to do with the
awful embarrassing irrationality of flying saucer cults---at least, I
hope not.}

{
 You and I believe that flying saucer cults arose in the total
absence of any flying saucers. Cults can arise around almost any idea,
thanks to human silliness. This silliness operates
\textit{orthogonally} to alien intervention: We would expect to see
flying saucer cults whether or not there were flying saucers. Even if
there were poorly hidden aliens, it would not be any \textit{less}
likely for flying saucer cults to arise. The conditional probability
P(cults{\textbar}aliens) isn't less than
P(cults{\textbar}{\textlnot}aliens), unless you suppose that poorly
hidden aliens would deliberately suppress flying saucer cults. By the
Bayesian definition of evidence, the observation
``flying saucer cults exist'' is not
evidence \textit{against} the existence of flying saucers.
It's not much evidence one way or the other.}

{
 This is an application of the general principle that, as Robert
Pirsig puts it, ``The world's greatest
fool may say the Sun is shining, but that doesn't make
it dark out.''\textsuperscript{2}}

{
 If you knew someone who was wrong 99.99\% of the time on yes-or-no
questions, you could obtain 99.99\% accuracy just by reversing their
answers. They would need to do all the work of obtaining good evidence
entangled with reality, and processing that evidence coherently, just
to \textit{anticorrelate} that reliably. They would have to be
superintelligent to be that stupid.}

{
 A car with a broken engine cannot drive backward at 200 mph, even
if the engine is \textit{really really broken.}}

{
 If stupidity does not reliably anticorrelate with truth, how much
less should human evil anticorrelate with truth? The converse of the
halo effect is the horns effect: All perceived negative qualities
correlate. If Stalin is evil, then everything he says should be false.
You wouldn't want to agree with \textit{Stalin}, would
you?}

{
 Stalin also believed that 2 + 2 = 4. Yet if you defend any
statement made by Stalin, even ``2 + 2 =
4,'' people will see only that you are
``agreeing with Stalin''; you must
be on his side.}

{
 Corollaries of this principle:}

{
 To argue against an idea honestly, you should argue against the
best arguments of the strongest advocates. Arguing against weaker
advocates proves \textit{nothing}, because even the strongest idea will
attract weak advocates. If you want to argue against transhumanism or
the intelligence explosion, you have to directly challenge the
arguments of Nick Bostrom or Eliezer Yudkowsky post-2003. The least
convenient path is the only valid one.}

{
 Exhibiting sad, pathetic lunatics, driven to madness by their
apprehension of an Idea, is no evidence against that Idea. Many New
Agers have been made crazier by their personal apprehension of quantum
mechanics.}

{
 Someone once said, ``Not all conservatives are
stupid, but most stupid people are conservatives.''
If you cannot place yourself in a state of mind where this statement,
true or false, seems \textit{completely irrelevant} as a critique of
conservatism, you are not ready to think rationally about politics.}

{
 Ad hominem argument is not valid.}

{
 You need to be able to argue against genocide without saying
``Hitler wanted to exterminate the
Jews.'' If Hitler \textit{hadn't}
advocated genocide, would it thereby become okay?}

{
 In Hansonian terms: Your instinctive willingness to believe
something will change along with your willingness to \textit{affiliate}
with people who are known for believing it---quite apart from whether
the belief is actually \textit{true.} Some people may be reluctant to
believe that God does not exist, not because there is evidence that God
\textit{does} exist, but rather because they are reluctant to affiliate
with Richard Dawkins or those darned
``strident'' atheists who go around
publicly saying ``God does not
exist.''}

{
 If your current computer stops working, you can't
conclude that everything about the current system is wrong and that you
need a new system without an AMD processor, an ATI video card, a Maxtor
hard drive, or case fans---even though your current system has all
these things and it doesn't work. Maybe you just need a
new power cord.}

{
 If a hundred inventors fail to build flying machines using metal
and wood and canvas, it doesn't imply that what you
really need is a flying machine of bone and flesh. If a thousand
projects fail to build Artificial Intelligence using electricity-based
computing, this doesn't mean that electricity is the
source of the problem. Until you understand the problem, hopeful
reversals are exceedingly unlikely to hit the solution.}

\myendsectiontext


\bigskip

{
 1. Henry Beam Piper, ``Police
Operation,'' \textit{Astounding Science Fiction}
(July 1948).}

{
 2. Robert M. Pirsig, \textit{Zen and the Art of Motorcycle
Maintenance: An Inquiry Into Values}, 1st ed. (New York: Morrow,
1974).}

\mysection{Argument Screens Off Authority}

{
 Scenario 1: Barry is a famous geologist. Charles is a
fourteen-year-old juvenile delinquent with a long arrest record and
occasional psychotic episodes. Barry flatly asserts to Arthur some
counterintuitive statement about rocks, and Arthur judges it 90\%
probable. Then Charles makes an equally counterintuitive flat assertion
about rocks, and Arthur judges it 10\% probable. Clearly, Arthur is
taking the speaker's \textit{authority} into account in
deciding whether to believe the speaker's assertions. }

{
 Scenario 2: David makes a counterintuitive statement about physics
and gives Arthur a detailed explanation of the arguments, including
references. Ernie makes an equally counterintuitive statement, but
gives an unconvincing argument involving several leaps of faith. Both
David and Ernie assert that this is the best explanation they can
possibly give (to anyone, not just Arthur). Arthur assigns 90\%
probability to David's statement after hearing his
explanation, but assigns a 10\% probability to Ernie's
statement.}

{
 It might seem like these two scenarios are roughly symmetrical:
both involve taking into account useful evidence, whether strong versus
weak authority, or strong versus weak argument.}

{
 But now suppose that Arthur asks Barry and Charles to make full
technical cases, with references; and that Barry and Charles present
equally good cases, and Arthur looks up the references and they check
out. Then Arthur asks David and Ernie for their credentials, and it
turns out that David and Ernie have roughly the same
credentials---maybe they're both clowns, maybe
they're both physicists.}

{
 Assuming that Arthur is knowledgeable enough to understand all the
technical arguments---otherwise they're just impressive
noises---it seems that Arthur should view David as having a great
advantage in plausibility over Ernie, while Barry has at best a minor
advantage over Charles.}

{
 Indeed, if the technical arguments are good enough,
Barry's advantage over Charles may not be worth
tracking. A good technical argument is one that \textit{eliminates}
reliance on the personal authority of the speaker.}

{
 Similarly, if we really believe Ernie that the argument he gave is
the best argument he \textit{could} give, which includes all of the
inferential steps that Ernie executed, and all of the support that
Ernie took into account---citing any authorities that Ernie may have
listened to himself---then we can pretty much ignore any information
about Ernie's credentials. Ernie can be a physicist or
a clown, it shouldn't matter. (Again, this assumes we
have enough technical ability to process the argument. Otherwise, Ernie
is simply uttering mystical syllables, and whether we
``believe'' these syllables depends
a great deal on his authority.)}

{
 So it seems there's an asymmetry between argument
and authority. If we know authority we are still interested in hearing
the arguments; but if we know the arguments fully, we have very little
left to learn from authority.}

{
 Clearly (says the novice) authority and argument are fundamentally
different kinds of evidence, a difference unaccountable in the boringly
clean methods of Bayesian probability theory. For while the strength of
the evidences---90\% versus 10\%---is just the same in both cases, they
do not behave similarly when combined. How will we account for this?}

{
 Here's half a technical demonstration of how to
represent this difference in probability theory. (The rest you can take
on my personal authority, or look up in the references.)}

{
 If P(H{\textbar}E\textsubscript{1}) = 90\% and
P(H{\textbar}E\textsubscript{2}) = 9\%, what is the probability
P(H{\textbar}E\textsubscript{1},E\textsubscript{2})? If learning
E\textsubscript{1} is true leads us to assign 90\% probability to H,
and learning E\textsubscript{2} is true leads us to assign 9\%
probability to H, then what probability should we assign to H if we
learn both E\textsubscript{1} and E\textsubscript{2}? This is simply
not something you can calculate in probability theory from the
information given. No, the missing information is not the prior
probability of H. The events E\textsubscript{1} and E\textsubscript{2}
may not be independent of each other.}

{
 Suppose that H is ``My sidewalk is
slippery,'' E\textsubscript{1} is
``My sprinkler is running,'' and
E\textsubscript{2} is ``It's
night.'' The sidewalk is slippery starting from one
minute after the sprinkler starts, until just after the sprinkler
finishes, and the sprinkler runs for ten minutes. So if we know the
sprinkler is on, the probability is 90\% that the sidewalk is slippery.
The sprinkler is on during 10\% of the nighttime, so if we know that
it's night, the probability of the sidewalk being
slippery is 9\%. If we know that it's night and the
sprinkler is on---that is, if we know both facts---the probability of
the sidewalk being slippery is 90\%.}

{
 We can represent this in a graphical model as follows:}

{
 ~}

\mygraphics{Rationality20From20AI20to20Zombies2020Eliezer20Yudkowsky-img82.jpg}


{
 ~}

{
 Whether or not it's Night \textit{causes} the
Sprinkler to be on or off, and whether the Sprinkler is on
\textit{causes} the sidewalk to be Slippery or unSlippery.}

{
 The direction of the arrows is meaningful. Say we had:}

{
 ~}

\mygraphics{Rationality20From20AI20to20Zombies2020Eliezer20Yudkowsky-img83.jpg}

{
 ~}

{
 This would mean that, if I \textit{didn't} know
anything about the sprinkler, the probability of Nighttime and
Slipperiness would be independent of each other. For example, suppose
that I roll Die One and Die Two, and add up the showing numbers to get
the Sum:}

{
 ~}

\mygraphics{Rationality20From20AI20to20Zombies2020Eliezer20Yudkowsky-img84.jpg}


{
 ~}

{
 If you don't tell me the sum of the two numbers,
and you tell me the first die showed 6, this doesn't
tell me anything about the result of the second die, yet. But if you
now also tell me the sum is 7, I know the second die showed 1.}

{
 Figuring out when various pieces of information are dependent or
independent of each other, given various background knowledge, actually
turns into a quite technical topic. The books to read are Judea
Pearl's \textit{Probabilistic Reasoning in Intelligent
Systems: Networks of Plausible Inference}\textsuperscript{1} and
\textit{Causality: Models, Reasoning, and
Inference}.\textsuperscript{2} (If you only have time to read one book,
read the first one.)}

{
 If you know how to read causal graphs, then you look at the
dice-roll graph and immediately see:}

{\centering
 P(Die 1, Die 2) = P(Die 1) {\texttimes} P(Die 2)
\par}


\bigskip

{\centering
 P(Die 1, Die 2{\textbar}Sum) ${\neq}$ P(Die 1{\textbar}Sum)
{\texttimes} P(Die 2{\textbar}Sum)
\par}


\bigskip

{
 If you look at the correct sidewalk diagram, you see facts like:}

{\centering
 P(Slippery{\textbar}Night) ${\neq}$ P(Slippery)
\par}


\bigskip

{\centering
 P(Slippery{\textbar}Sprinkler) ${\neq}$ P(Slippery)
\par}


\bigskip

{\centering
 P(Slippery{\textbar}Night, Sprinkler) =
P(Slippery{\textbar}Sprinkler).
\par}


\bigskip

{
 That is, the probability of the sidewalk being Slippery, given
knowledge about the Sprinkler and the Night, is the same probability we
would assign if we knew only about the Sprinkler. Knowledge of the
Sprinkler has made knowledge of the Night irrelevant to inferences
about Slipperiness.}

{
 This is known as \textit{screening off}, and the criterion that
lets us read such conditional independences off causal graphs is known
as \textit{D-separation}.}

{
 For the case of argument and authority, the causal diagram looks
like this:}

{
 ~}

\mygraphics{Rationality20From20AI20to20Zombies2020Eliezer20Yudkowsky-img85.jpg}


{
 ~}

{
 If something is true, then it therefore tends to have arguments in
favor of it, and the experts therefore observe these evidences and
change their opinions. (In theory!)}

{
 If we see that an expert believes something, we infer back to the
existence of evidence-in-the-abstract (even though we
don't know what that evidence is exactly), and from the
existence of this abstract evidence, we infer back to the truth of the
proposition.}

{
 But if we know the value of the Argument node, this D-separates
the node ``Truth'' from the node
``Expert Belief'' by blocking all
paths between them, according to certain technical criteria for
``path blocking'' that seem pretty
obvious in this case. So even without checking the exact probability
distribution, we can read off from the graph that:}

{\centering
 P(truth{\textbar}argument, expert) = P(truth{\textbar}argument).
\par}


\bigskip

{
 This does not represent a contradiction of ordinary probability
theory. It's just a more compact way of expressing
certain probabilistic facts. You could read the same equalities and
inequalities off an unadorned probability distribution---but it would
be harder to see it by eyeballing. Authority and argument
don't need two different kinds of probability, any more
than sprinklers are made out of ontologically different stuff than
sunlight. }

{
 In practice you can never \textit{completely} eliminate reliance
on authority. Good authorities are more likely to know about any
counterevidence that exists and should be taken into account; a lesser
authority is less likely to know this, which makes their arguments less
reliable. This is not a factor you can eliminate merely by hearing the
evidence they \textit{did} take into account.}

{
 It's also very hard to reduce arguments to
\textit{pure} math; and otherwise, judging the strength of an
inferential step may rely on intuitions you can't
duplicate without the same thirty years of experience.}

{
 There is an ineradicable legitimacy to assigning \textit{slightly}
higher probability to what E. T. Jaynes tells you about Bayesian
probability, than you assign to Eliezer Yudkowsky making the exact same
statement. Fifty additional years of experience should not count for
literally \textit{zero} influence.}

{
 But this slight strength of authority is only \textit{ceteris
paribus}, and can easily be overwhelmed by stronger arguments. I have a
minor erratum in one of Jaynes's books---because
algebra trumps authority.}

\myendsectiontext


\bigskip

{
 1. Pearl, \textit{Probabilistic Reasoning in Intelligent
Systems}.}

{
 2. Judea Pearl, \textit{Causality: Models, Reasoning, and
Inference}, 2nd ed. (New York: Cambridge University Press, 2009).}

\mysection{Hug the Query}

{
 In the art of rationality there is a discipline of
\textit{closeness-to-the-issue}{}---trying to observe evidence that is
as near to the original question as possible, so that it screens off as
many other arguments as possible. }

{
 The Wright Brothers say, ``My plane will
fly.'' If you look at their authority (bicycle
mechanics who happen to be excellent amateur physicists) then you will
compare their authority to, say, Lord Kelvin, and you will find that
Lord Kelvin is the greater authority.}

{
 If you demand to see the Wright Brothers'
calculations, and you can follow them, and you demand to see Lord
Kelvin's calculations (he probably
doesn't have any apart from his own incredulity), then
authority becomes much less relevant.}

{
 If you actually \textit{watch the plane fly}, the calculations
themselves become moot for many purposes, and Kelvin's
authority not even worth considering.}

{
 The more \textit{directly} your arguments bear on a question,
without intermediate inferences---the closer the observed nodes are to
the queried node, in the Great Web of Causality---the more powerful the
evidence. It's a theorem of these causal graphs that
you can never get \textit{more} information from distant nodes, than
from strictly closer nodes that screen off the distant ones.}

{
 Jerry Cleaver said: ``What does you in is not
failure to apply some high-level, intricate, complicated technique.
It's overlooking the basics. Not keeping your eye on
the ball.''\textsuperscript{1}}

{
 Just as it is superior to argue physics than credentials, it is
also superior to argue physics than rationality. Who was more rational,
the Wright Brothers or Lord Kelvin? If we can check their calculations,
we don't have to care! The virtue of a rationalist
cannot \textit{directly} cause a plane to fly.}

{
 If you forget this principle, learning about more biases will hurt
you, because it will distract you from more direct arguments.
It's all too easy to argue that someone is exhibiting
Bias \#182 in your repertoire of fully generic accusations, but you
can't \textit{settle} a factual issue without closer
evidence. If there are biased reasons to say the Sun is shining, that
doesn't make it dark out.}

{
 Just as you can't always experiment today, you
can't always check the calculations today. Sometimes
you don't know enough background material, sometimes
there's private information, sometimes there just
isn't time. There's a sadly large
number of times when it's worthwhile to judge the
speaker's rationality. You should always do it with a
hollow feeling in your heart, though, a sense that
something's missing.}

{
 Whenever you can, dance as near to the original question as
possible---press yourself up against it---get close enough to
\textit{hug the query!}}

\myendsectiontext


\bigskip

{
 1. Jerry Cleaver, \textit{Immediate Fiction: A Complete Writing
Course} (Macmillan, 2004).}

\mysection{Rationality and the English Language}

{
 Responding to my discussion of applause lights, someone said that
my writing reminded them of George Orwell's Politics
and the English Language.\textsuperscript{1} I was honored. Especially
since I'd already thought of today's
topic. }

{
 If you \textit{really} want an artist's
perspective on rationality, then read Orwell; he is mandatory reading
for rationalists as well as authors. Orwell was not a scientist, but a
writer; his tools were not numbers, but words; his adversary was not
Nature, but human evil. If you wish to imprison people for years
without trial, you must think of some other way to say it than
``I'm going to imprison Mr. Jennings
for years without trial.'' You must muddy the
listener's thinking, prevent clear images from
outraging conscience. You say, ``Unreliable elements
were subjected to an alternative justice process.''}

{
 Orwell was the outraged opponent of totalitarianism and the muddy
thinking in which evil cloaks itself---which is how
Orwell's writings on language ended up as classic
rationalist documents on a level with Feynman, Sagan, or Dawkins.}

{
 ``Writers are told to avoid usage of the passive
voice.'' A rationalist whose background comes
\textit{exclusively} from science may fail to see the flaw in the
previous sentence; but anyone who's done a little
writing should see it right away. I wrote the sentence in the passive
voice, without telling you \textit{who} tells authors to avoid passive
voice. Passive voice removes the actor, leaving only the acted-upon.
``Unreliable elements were subjected to an alternative
justice process''---subjected by \textit{whom}? What
does an ``alternative justice
process'' \textit{do}? With enough static noun
phrases, you can keep anything unpleasant from actually
\textit{happening}.}

{
 Journal articles are often written in passive voice. (Pardon me,
\textit{some scientists} write their journal articles in passive voice.
It's not as if the articles are being written by no
one, with no one to blame.) It sounds more authoritative to say
``The subjects were administered
Progenitorivox'' than ``I gave each
college student a bottle of 20 Progenitorivox, and told them to take
one every night until they were gone.'' If you remove
the scientist from the description, that leaves only the all-important
data. But in reality the scientist \textit{is} there, and the subjects
\textit{are} college students, and the Progenitorivox
wasn't
``administered'' but handed over
with instructions. Passive voice obscures reality.}

{
 Judging from the comments I get, someone will protest that using
the passive voice in a journal article is hardly a sin---after all, if
you \textit{think} about it, you can realize the scientist is there. It
doesn't seem like a logical flaw. And this is why
rationalists need to read Orwell, not just Feynman or even Jaynes.}

{
 Nonfiction conveys \textit{knowledge}, fiction conveys
\textit{experience.} Medical science can extrapolate what would happen
to a human unprotected in a vacuum. Fiction can make you live through
it.}

{
 Some rationalists will try to analyze a misleading phrase, try to
see if there \textit{might possibly} be anything meaningful to it, try
to \textit{construct} a logical interpretation. They will be
charitable, give the author the benefit of the doubt. Authors, on the
other hand, are trained \textit{not} to give themselves the benefit of
the doubt. Whatever the audience \textit{thinks} you said \textit{is}
what you said, whether you meant to say it or not; you
can't argue with the audience no matter how clever your
justifications.}

{
 A writer knows that readers will \textit{not} stop for a minute to
think. A fictional experience is a continuous stream of first
impressions. A writer-rationalist pays attention to the
\textit{experience} words create. If you are evaluating the public
rationality of a statement, and you analyze the words deliberatively,
rephrasing propositions, trying out different meanings, searching for
nuggets of truthiness, then you're losing track of the
first impression---what the audience \textit{sees}, or rather
\textit{feels.}}

{
 A novelist would notice the screaming wrongness of
``The subjects were administered
Progenitorivox.'' What life is here for a reader to
live? This sentence creates a distant feeling of authoritativeness, and
that's \textit{all}{}---the \textit{only} experience is
the feeling of being told something reliable. A novelist would see
nouns too abstract to show what actually happened---the postdoc with
the bottle in their hand, trying to look stern; the student listening
with a nervous grin.}

{
 My point is not to say that journal articles should be written
like novels, but that a rationalist should become consciously aware of
the \textit{experiences} which words create. A rationalist must
understand the mind and how to operate it. That includes the stream of
consciousness, the part of yourself that unfolds in language. A
rationalist must become consciously aware of the actual, experiential
impact of phrases, beyond their mere propositional semantics.}

{
 Or to say it more bluntly: \textit{Meaning does not excuse
impact!}}

{
 I don't care what rational interpretation you can
\textit{construct} for an applause light like ``AI
should be developed through democratic processes.''
That cannot excuse its irrational impact of signaling the audience to
applaud, not to mention its cloudy question-begging vagueness.}

{
 Here is Orwell, railing against the \textit{impact} of cliches,
their effect on the experience of thinking:}

{
 When one watches some tired hack on the platform mechanically
repeating the familiar phrases---BESTIAL, ATROCITIES, IRON HEEL,
BLOODSTAINED TYRANNY, FREE PEOPLES OF THE WORLD, STAND SHOULDER TO
SHOULDER---one often has a curious feeling that one is not watching a
live human being but some kind of dummy \ldots A speaker who uses that
kind of phraseology has gone some distance toward turning himself into
a machine. The appropriate noises are coming out of his larynx, but his
brain is not involved, as it would be if he were choosing his words for
himself \ldots}

{
 What is above all needed is to let the meaning choose the word,
and not the other way around. In prose, the worst thing one can do with
words is surrender to them. When you think of a concrete object, you
think wordlessly, and then, if you want to describe the thing you have
been visualising you probably hunt about until you find the exact words
that seem to fit it. When you think of something abstract you are more
inclined to use words from the start, and unless you make a conscious
effort to prevent it, the existing dialect will come rushing in and do
the job for you, at the expense of blurring or even changing your
meaning. Probably it is better to put off using words as long as
possible and get one's meaning as clear as one can
through pictures and sensations.}

{
 Charles Sanders Peirce might have written that last paragraph.
More than one path can lead to the Way.}

\myendsectiontext


\bigskip

{
 1. George Orwell, ``Politics and the English
Language,'' \textit{Horizon} (April 1946).}

\mysection{Human Evil and Muddled Thinking}

{
 George Orwell saw the descent of the civilized world into
totalitarianism, the conversion or corruption of one country after
another; the boot stamping on a human face, forever, and remember that
it is forever. You were born too late to remember a time when the rise
of totalitarianism seemed unstoppable, when one country after another
fell to secret police and the thunderous knock at midnight, while the
professors of free universities hailed the Soviet
Union's purges as progress. It feels as alien to you as
fiction; it is hard for you to take seriously. Because, in your branch
of time, the Berlin Wall fell. And if Orwell's name is
not carved into one of those stones, it should be. }

{
 Orwell saw the destiny of the human species, and he put forth a
convulsive effort to wrench it off its path. Orwell's
weapon was clear writing. Orwell knew that muddled language is muddled
thinking; he knew that human evil and muddled thinking intertwine like
conjugate strands of DNA:\textsuperscript{1}}

{
 In our time, political speech and writing are largely the defence
of the indefensible. Things like the continuance of British rule in
India, the Russian purges and deportations, the dropping of the atom
bombs on Japan, can indeed be defended, but only by arguments which are
too brutal for most people to face, and which do not square with the
professed aims of the political parties. Thus political language has to
consist largely of euphemism, question-begging and sheer cloudy
vagueness. Defenceless villages are bombarded from the air, the
inhabitants driven out into the countryside, the cattle machine-gunned,
the huts set on fire with incendiary bullets: this is called
PACIFICATION \ldots}

{
 Orwell was clear on the goal of his clarity:}

{
 If you simplify your English, you are freed from the worst follies
of orthodoxy. You cannot speak any of the necessary dialects, and when
you make a stupid remark its stupidity will be obvious, even to
yourself.}

{
 To make our stupidity obvious, even to ourselves---this is the
heart of \textit{Overcoming Bias}.}

{
 Evil sneaks, hidden, through the unlit shadows of the mind. We
look back with the clarity of history, and weep to remember the planned
famines of Stalin and Mao, which killed tens of millions. We call this
evil, because it was done by deliberate human intent to inflict pain
and death upon innocent human beings. We call this evil, because of the
revulsion that we feel against it, looking back with the clarity of
history. For perpetrators of evil to avoid its natural opposition, the
revulsion must remain latent. Clarity must be avoided at any cost. Even
as humans of clear sight tend to oppose the evil that they see; so too
does human evil, wherever it exists, set out to muddle thinking.}

{
 \textit{1984} sets this forth starkly: Orwell's
ultimate villains are cutters and airbrushers of photographs (based on
historical cutting and airbrushing in the Soviet Union). At the peak of
all darkness in the Ministry of Love, O'Brien tortures
Winston to admit that two plus two equals five:\textsuperscript{2}}

{
 ``Do you remember,'' he went
on, ``writing in your diary, `Freedom
is the freedom to say that two plus two make
four'?''}

{
 ``Yes,'' said Winston.}

{
 O'Brien held up his left hand, its back towards
Winston, with the thumb hidden and the four fingers extended.}

{
 ``How many fingers am I holding up,
Winston?''}

{
 ``Four.''}

{
 ``And if the party says that it is not four but
five---then how many?''}

{
 ``Four.''}

{
 The word ended in a gasp of pain. The needle of the dial had shot
up to fifty-five. The sweat had sprung out all over
Winston's body. The air tore into his lungs and issued
again in deep groans which even by clenching his teeth he could not
stop. O'Brien watched him, the four fingers still
extended. He drew back the lever. This time the pain was only slightly
eased.}

{
 I am continually aghast at apparently intelligent folks---such as
Robin Hanson's colleague Tyler Cowen{}---who
don't think that overcoming bias is important. This is
your \textit{mind} we're talking about. Your human
intelligence. It separates you from an ape. It built this world. You
don't think how the mind works is important? You
don't think the mind's systematic
malfunctions are important? Do you think the Inquisition would have
tortured witches, if all were ideal Bayesians?}

{
 Tyler Cowen apparently feels that overcoming bias is just as
biased as bias: ``I view Robin's blog
as exemplifying bias, and indeed showing that bias can be very
useful.'' I \textit{hope} this is only the result of
thinking too abstractly while trying to sound clever. Does Tyler
seriously think that scope insensitivity to the value of human life is
on the same level with trying to create plans that will \textit{really}
save as many lives as possible?}

{
 Orwell was forced to fight a similar attitude---that to admit to
any distinction is youthful naiveté:}

{
 Stuart Chase and others have come near to claiming that all
abstract words are meaningless, and have used this as a pretext for
advocating a kind of political quietism. Since you
don't know what Fascism is, how can you struggle
against Fascism?}

{
 Maybe overcoming bias doesn't look quite exciting
enough, if it's framed as a struggle against mere
accidental mistakes. Maybe it's harder to get excited
if there isn't some clear evil to oppose. So let us be
absolutely clear that where there is human evil in the world, where
there is cruelty and torture and deliberate murder, there are biases
enshrouding it. Where people of clear sight oppose these biases, the
concealed evil fights back. The truth \textit{does} have enemies. If
\textit{Overcoming Bias} were a newsletter in the old Soviet Union,
every poster and commenter of \textit{Overcoming Bias} would have been
shipped off to labor camps.}

{
 In all human history, every great leap forward has been driven by
a new clarity of thought. Except for a few natural catastrophes, every
great woe has been driven by a stupidity. Our last enemy is ourselves;
and this is a war, and we are soldiers.}

\myendsectiontext


\bigskip

{
 1. Ibid.}

{
 2. George Orwell, \textit{1984} (Signet Classic, 1950).}


\chapter{Against Rationalization}

\mysection{Knowing About Biases Can Hurt People}

{
 Once upon a time I tried to tell my mother about the problem of
expert calibration, saying: ``So when an expert says
they're 99\% confident, it only happens about 70\% of
the time.'' Then there was a pause as, suddenly, I
realized I was talking to my mother, and I hastily added:
``Of course, you've got to make sure
to apply that skepticism evenhandedly, including to yourself, rather
than just using it to argue against anything you disagree
with---'' }

{
 And my mother said: ``Are you kidding? This is
great! I'm going to use it all the
time!''}

{
 Taber and Lodge's ``Motivated
skepticism in the evaluation of political beliefs''
describes the confirmation of six predictions:\textsuperscript{1}}

{
 Prior attitude effect. Subjects who feel strongly about an
issue---even when encouraged to be objective---will evaluate supportive
arguments more favorably than contrary arguments.}

{
 Disconfirmation bias. Subjects will spend more time and cognitive
resources denigrating contrary arguments than supportive arguments.}

{
 Confirmation bias. Subjects free to choose their information
sources will seek out supportive rather than contrary sources.}

{
 \textbf{Attitude polarization. Exposing subjects to an apparently
balanced set of pro and con arguments will exaggerate their initial
polarization.}}

{
 Attitude strength effect. Subjects voicing stronger attitudes will
be more prone to the above biases.}

{
 \textbf{Sophistication effect. Politically knowledgeable subjects,
because they possess greater ammunition with which to counter-argue
incongruent facts and arguments, will be more prone to the above
biases.}}

{
 If you're irrational to start with, having
\textit{more} knowledge can \textit{hurt} you. For a true Bayesian,
information would never have negative expected utility. But humans
aren't perfect Bayes-wielders; if we're
not careful, we can cut ourselves.}

{
 I've \textit{seen} people severely messed up by
their own knowledge of biases. They have more ammunition with which to
argue against anything they don't like. And that
problem---too much ready ammunition---is one of the primary ways that
people with high mental agility end up stupid, in
Stanovich's
``dysrationalia'' sense of
stupidity.}

{
 You can think of people who fit this description, right? People
with high g-factor who end up being \textit{less} effective because
they are too sophisticated as arguers? Do you think
you'd be helping them---making them more effective
rationalists---if you just told them about a list of classic biases?}

{
 I recall someone who learned about the calibration/overconfidence
problem. Soon after he said: ``Well, you
can't trust experts; they're wrong so
often---as experiments have shown. So therefore, when I predict the
future, I prefer to assume that things will continue historically as
they have---'' and went off into this whole complex,
error-prone, highly questionable extrapolation. Somehow, when it came
to trusting his own preferred conclusions, all those biases and
fallacies seemed much less \textit{salient}{}---leapt much less readily
to mind---than when he needed to counter-argue someone else.}

{
 I told the one about the problem of disconfirmation bias and
sophisticated argument, and lo and behold, the next time I said
something he didn't like, he accused me of being a
sophisticated arguer. He didn't try to point out any
particular sophisticated argument, any particular flaw---just shook his
head and sighed sadly over how I was apparently using my own
intelligence to defeat itself. He had acquired yet another Fully
General Counterargument.}

{
 Even the notion of a ``sophisticated
arguer'' can be deadly, if it leaps all too readily
to mind when you encounter a seemingly intelligent person who says
something you don't like.}

{
 I endeavor to learn from my mistakes. The last time I gave a talk
on heuristics and biases, I started out by introducing the general
concept by way of the conjunction fallacy and representativeness
heuristic. And then I moved on to confirmation bias, disconfirmation
bias, sophisticated argument, motivated skepticism, and other attitude
effects. I spent the next thirty minutes \textit{hammering} on that
theme, reintroducing it from as many different perspectives as I
could.}

{
 I wanted to get my audience interested in the subject. Well, a
simple description of conjunction fallacy and representativeness would
suffice for that. But suppose they did get interested. Then what? The
literature on bias is mostly cognitive psychology for cognitive
psychology's sake. I had to give my audience their dire
warnings during that one lecture, or they probably
wouldn't hear them at all.}

{
 Whether I do it on paper, or in speech, I now try to never mention
calibration and overconfidence unless I have first talked about
disconfirmation bias, motivated skepticism, sophisticated arguers, and
dysrationalia in the mentally agile. First, do no harm!}

\myendsectiontext


\bigskip

{
 1. Charles S. Taber and Milton Lodge, ``Motivated
Skepticism in the Evaluation of Political Beliefs,''
\textit{American Journal of Political Science} 50, no. 3 (2006):
755--769, doi:10.1111/j.1540-5907.2006.00214.x.}

\mysection{Update Yourself Incrementally}

{
 Politics is the mind-killer. Debate is war, arguments are
soldiers. There is the temptation to search for ways to interpret every
possible experimental result to confirm your theory, like securing a
citadel against every possible line of attack. This you cannot do. It
is mathematically impossible. For every expectation of evidence, there
is an equal and opposite expectation of counterevidence. }

{
 But it's okay if your cherished belief
isn't \textit{perfectly} defended. If the hypothesis is
that the coin comes up heads 95\% of the time, then one time in twenty
you will expect to see what looks like contrary evidence. This is okay.
It's normal. It's even expected, so
long as you've got nineteen supporting observations for
every contrary one. A probabilistic model can take a hit or two, and
still survive, so long as the hits don't \textit{keep
on} coming in.}

{
 Yet it is widely believed, especially in the court of public
opinion, that a true theory can have \textit{no} failures and a false
theory \textit{no} successes.}

{
 You find people holding up a single piece of what they conceive to
be evidence, and claiming that their theory can
``explain'' it, as though this were
all the support that any theory needed. Apparently a false theory can
have \textit{no} supporting evidence; it is impossible for a false
theory to fit even a single event. Thus, a single piece of confirming
evidence is all that any theory needs.}

{
 It is only slightly less foolish to hold up a single piece of
\textit{probabilistic} counterevidence as disproof, as though it were
impossible for a correct theory to have even a \textit{slight} argument
against it. But this is how humans have argued for ages and ages,
trying to defeat all enemy arguments, while denying the enemy even a
single shred of support. People want their debates to be one-sided;
they are accustomed to a world in which their preferred theories have
not one iota of antisupport. Thus, allowing a single item of
probabilistic counterevidence would be the end of the world.}

{
 I just know someone in the audience out there is going to say,
``But you \textit{can't} concede even
a single point if you want to win debates in the real world! If you
concede that any counterarguments exist, the Enemy will harp on them
over and over---you can't let the Enemy do that!
You'll \textit{lose!} What could be more viscerally
terrifying than \textit{that?}''}

{
 Whatever. Rationality is not for winning debates, it is for
deciding which side to join. If you've already decided
which side to argue for, the work of rationality is \textit{done}
within you, whether well or poorly. But how can you, yourself, decide
which side to argue? If \textit{choosing the wrong side} is viscerally
terrifying, even just a little viscerally terrifying,
you'd best integrate \textit{all} the evidence.}

{
 Rationality is not a walk, but a dance. On each step in that dance
your foot should come down in exactly the correct spot, neither to the
left nor to the right. Shifting belief upward with each iota of
confirming evidence. Shifting belief downward with each iota of
contrary evidence. Yes, \textit{down.} Even with a correct model, if it
is not an exact model, you will sometimes need to revise your belief
\textit{down.}}

{
 If an iota or two of evidence happens to countersupport your
belief, that's okay. It happens, sometimes, with
probabilistic evidence for non-exact theories. (If an exact theory
fails, you \textit{are} in trouble!) Just shift your belief downward a
little---the probability, the odds ratio, or even a nonverbal weight of
credence in your mind. Just shift downward a little, and wait for more
evidence. If the theory is true, supporting evidence will come in
shortly, and the probability will climb again. If the theory is false,
you don't really want it anyway.}

{
 The problem with using black-and-white, binary, qualitative
reasoning is that any single observation either destroys the theory or
it does not. When not even a single contrary observation is allowed, it
creates cognitive dissonance and has to be argued away. And this rules
out incremental progress; it rules out correct integration of all the
evidence. Reasoning probabilistically, we realize that on average, a
correct theory will generate a greater weight of support than
countersupport. And so you can, \textit{without fear,} say to yourself:
``This is gently contrary evidence, I will shift my
belief downward.'' Yes, \textit{down.} It does not
destroy your cherished theory. That is qualitative reasoning; think
quantitatively.}

{
 For every expectation of evidence, there is an equal and opposite
expectation of counterevidence. On every occasion, you must, on
average, anticipate revising your beliefs downward as much as you
anticipate revising them upward. If you think you already know what
evidence will come in, then you must already be fairly sure of your
theory---probability close to 1---which doesn't leave
much room for the probability to go further upward. And however
unlikely it seems that you will encounter disconfirming evidence, the
resulting downward shift must be large enough to precisely balance the
anticipated gain on the other side. The weighted mean of your expected
posterior probability must equal your prior probability.}

{
 How silly is it, then, to be terrified of revising your
probability downward, if you're bothering to
investigate a matter at all? On average, you must anticipate as much
downward shift as upward shift from every individual observation.}

{
 It may perhaps happen that an iota of antisupport comes in again,
and again and again, while new support is slow to trickle in. You may
find your belief drifting downward and further downward. Until,
finally, you realize from which quarter the winds of evidence are
blowing against you. In that moment of realization, there is no point
in constructing excuses. In that moment of realization, you have
\textit{already relinquished} your cherished belief. Yay! Time to
celebrate! Pop a champagne bottle or send out for pizza! You
can't become stronger by keeping the beliefs you
started with, after all.}

\myendsectiontext

\mysection{One Argument Against An Army}

{
 I talked about a style of reasoning in which not a single contrary
argument is allowed, with the result that every non-supporting
observation has to be argued away. Here I suggest that when people
encounter a contrary argument, they prevent themselves from
downshifting their confidence by \textit{rehearsing} already-known
support. }

{
 Suppose the country of Freedonia is debating whether its neighbor,
Sylvania, is responsible for a recent rash of meteor strikes on its
cities. There are several pieces of evidence suggesting this: the
meteors struck cities close to the Sylvanian border; there was unusual
activity in the Sylvanian stock markets \textit{before} the strikes;
and the Sylvanian ambassador Trentino was heard muttering about
``heavenly vengeance.''}

{
 Someone comes to you and says: ``I
don't think Sylvania is responsible for the meteor
strikes. They have trade with us of billions of dinars
annually.''
``Well,'' you reply,
``the meteors struck cities close to Sylvania, there
was suspicious activity in their stock market, and their ambassador
spoke of heavenly vengeance afterward.'' Since these
three arguments outweigh the first, you \textit{keep} your belief that
Sylvania is responsible---you believe rather than disbelieve,
qualitatively. Clearly, the balance of evidence weighs against
Sylvania.}

{
 Then another comes to you and says: ``I
don't think Sylvania is responsible for the meteor
strikes. Directing an asteroid strike is really hard. Sylvania
doesn't even have a space program.''
You reply, ``But the meteors struck cities close to
Sylvania, and their investors knew it, and the ambassador came right
out and admitted it!'' Again, these three arguments
outweigh the first (by three arguments against one argument), so you
keep your belief that Sylvania is responsible.}

{
 Indeed, your convictions are \textit{strengthened.} On two
separate occasions now, you have evaluated the balance of evidence, and
both times the balance was tilted against Sylvania by a ratio of 3 to
1.}

{
 You encounter further arguments by the pro-Sylvania
traitors---again, and again, and a hundred times again---but each time
the new argument is handily defeated by 3 to 1. And on every occasion,
you feel yourself becoming more confident that Sylvania was indeed
responsible, shifting your prior according to the felt balance of
evidence.}

{
 The problem, of course, is that by \textit{rehearsing} arguments
you \textit{already knew}, you are double-counting the evidence. This
would be a grave sin even if you double-counted \textit{all} the
evidence. (Imagine a scientist who does an experiment with 50 subjects
and fails to obtain statistically significant results, so the scientist
counts all the data twice.)}

{
 But to selectively double-count \textit{only some} evidence is
sheer farce. I remember seeing a cartoon as a child, where a villain
was dividing up loot using the following algorithm:
``One for you, one for me. One for you, one-two for
me. One for you, one-two-three for me.''}

{
 As I emphasized in the last essay, even if a cherished belief is
\textit{true}, a rationalist may sometimes need to downshift the
probability while integrating \textit{all} the evidence. Yes, the
balance of support may still favor your cherished belief. But you still
have to shift the probability \textit{down}{}---yes,
\textit{down}{}---from whatever it was before you heard the contrary
evidence. It does no good to \textit{rehearse} supporting arguments,
because you have already taken those into account.}

{
 And yet it does appear to me that when people are confronted by a
\textit{new} counterargument, they search for a justification not to
downshift their confidence, and of course they find supporting
arguments they \textit{already know.} I have to keep constant vigilance
not to do this myself! It feels as natural as parrying a sword-strike
with a handy shield.}

{
 With the right kind of wrong reasoning, a handful of support---or
even a single argument---can stand off an army of contradictions.}

\myendsectiontext

\mysection{The Bottom Line}

{
 There are two sealed boxes up for auction, box A and box B. One
and only one of these boxes contains a valuable diamond. There are all
manner of signs and portents indicating whether a box contains a
diamond; but I have no sign which I \textit{know} to be perfectly
reliable. There is a blue stamp on one box, for example, and I know
that boxes which contain diamonds are more likely than empty boxes to
show a blue stamp. Or one box has a shiny surface, and I have a
suspicion---I am not sure---that no diamond-containing box is ever
shiny. }

{
 Now suppose there is a clever arguer, holding a sheet of paper,
and they say to the owners of box A and box B: ``Bid
for my services, and whoever wins my services, I shall argue that their
box contains the diamond, so that the box will receive a higher
price.'' So the box-owners bid, and box
B's owner bids higher, winning the services of the
clever arguer.}

{
 The clever arguer begins to organize their thoughts. First, they
write, ``And \textit{therefore}, box B contains the
diamond!'' at the bottom of their sheet of paper.
Then, at the top of the paper, the clever arguer writes,
``Box B shows a blue stamp,'' and
beneath it, ``Box A is shiny,'' and
then, ``Box B is lighter than box
A,'' and so on through many signs and portents; yet
the clever arguer neglects all those signs which might argue in favor
of box A. And then the clever arguer comes to me and recites from their
sheet of paper: ``Box B shows a blue stamp, and box A
is shiny,'' and so on, until they reach:
``and \textit{therefore}, box B contains the
diamond.''}

{
 But consider: At the moment when the clever arguer wrote down
their conclusion, at the moment they put ink on their sheet of paper,
the evidential entanglement of that physical ink with the physical
boxes became fixed.}

{
 It may help to visualize a collection of worlds---Everett branches
or Tegmark duplicates{}---within which there is some objective
frequency at which box A or box B contains a diamond.
There's likewise some objective frequency within the
subset ``worlds with a shiny box A''
where box B contains the diamond; and some objective frequency in
``worlds with shiny box A and blue-stamped box
B'' where box B contains the diamond.}

{
 The ink on paper is formed into odd shapes and curves, which look
like this text: ``And \textit{therefore}, box B
contains the diamond.'' If you happened to be a
literate English speaker, you might become confused, and think that
this shaped ink somehow \textit{meant} that box B contained the
diamond. Subjects instructed to say the color of printed pictures and
shown the picture GREEN often say
``green'' instead of
``red.'' It helps to be illiterate,
so that you are not confused by the shape of the ink.}

{
 To us, the true import of a thing is its entanglement with other
things. Consider again the collection of worlds, Everett branches or
Tegmark duplicates. At the moment when all clever arguers in all worlds
put ink to the bottom line of their paper---let us suppose this is a
single moment---it fixed the correlation of the ink with the boxes. The
clever arguer writes in non-erasable pen; the ink will not change. The
boxes will not change. Within the subset of worlds where the ink says
``And therefore, box B contains the
diamond,'' there is already some fixed percentage of
worlds where box A contains the diamond. This will not change
regardless of what is written in on the blank lines above.}

{
 So the evidential entanglement of the ink is fixed, and I leave to
you to decide what it might be. Perhaps box owners who believe a better
case can be made for them are more liable to hire advertisers; perhaps
box owners who fear their own deficiencies bid higher. If the box
owners do not themselves understand the signs and portents, then the
ink will be completely unentangled with the boxes'
contents, though it may tell you something about the
owners' finances and bidding habits.}

{
 Now suppose another person present is genuinely curious, and they
\textit{first} write down all the distinguishing signs of \textit{both}
boxes on a sheet of paper, and then apply their knowledge and the laws
of probability and write down at the bottom:
``\textit{Therefore,} I estimate an 85\% probability
that box B contains the diamond.'' Of what is this
handwriting evidence? Examining the chain of cause and effect leading
to this physical ink on physical paper, I find that the chain of
causality wends its way through all the signs and portents of the
boxes, and is dependent on these signs; for in worlds with different
portents, a different probability is written at the bottom.}

{
 So the handwriting of the curious inquirer is entangled with the
signs and portents and the contents of the boxes, whereas the
handwriting of the clever arguer is evidence only of which owner paid
the higher bid. There is a great difference in the indications of ink,
though one who foolishly read aloud the ink-shapes might think the
English words sounded similar.}

{
 Your effectiveness as a rationalist is determined by whichever
algorithm actually writes the bottom line of your thoughts. If your car
makes metallic squealing noises when you brake, and you
aren't willing to face up to the financial cost of
getting your brakes replaced, you can decide to look for reasons why
your car might not need fixing. But the actual percentage of you that
survive in Everett branches or Tegmark worlds---which we will take to
describe your effectiveness as a rationalist---is determined by the
algorithm that decided \textit{which} conclusion you would seek
arguments for. In this case, the real algorithm is
``Never repair anything expensive.''
If this is a good algorithm, fine; if this is a bad algorithm, oh well.
The arguments you write afterward, above the bottom line, will not
change anything either way.}

{
 This is intended as a caution for your own thinking, not a Fully
General Counterargument against conclusions you don't
like. For it is indeed a clever argument to say ``My
opponent is a clever arguer,'' if you are paying
yourself to retain whatever beliefs you had at the start. The
world's cleverest arguer may point out that the Sun is
shining, and yet it is still probably daytime.}

\myendsectiontext

\mysection{What Evidence Filtered Evidence?}

{
 I discussed the dilemma of the clever arguer, hired to sell you a
box that may or may not contain a diamond. The clever arguer points out
to you that the box has a blue stamp, and it is a valid known fact that
diamond-containing boxes are more likely than empty boxes to bear a
blue stamp. What happens at this point, from a Bayesian perspective?
Must you helplessly update your probabilities, as the clever arguer
wishes? }

{
 If you can look at the box yourself, you can add up all the signs
yourself. What if you can't look? What if the only
evidence you have is the word of the clever arguer, who is legally
constrained to make only true statements, but does not tell you
everything they know? Each statement that the clever arguer makes is
valid evidence---how could you \textit{not} update your probabilities?
Has it ceased to be true that, in such-and-such a proportion of Everett
branches or Tegmark duplicates in which box B has a blue stamp, box B
contains a diamond? According to Jaynes, a Bayesian must always
condition on all known evidence, on pain of paradox. But then the
clever arguer can make you believe anything they choose, if there is a
sufficient variety of signs to selectively report. That
doesn't sound right.}

{
 Consider a simpler case, a biased coin, which may be biased to
come up 2/3 heads and 1/3 tails, or 1/3 heads and 2/3 tails, both cases
being equally likely a priori. Each H observed is 1 bit of evidence for
an H-biased coin; each T observed is 1 bit of evidence for a T-biased
coin. I flip the coin ten times, and then I tell you,
``The 4th flip, 6th flip, and 9th flip came up
heads.'' What is your posterior probability that the
coin is H-biased?}

{
 And the answer is that it could be almost anything, depending on
what chain of cause and effect lay behind my utterance of those
words---my selection of which flips to report.}

{
 I might be following the algorithm of reporting the result of the
4th, 6th, and 9th flips, regardless of the result of those and all
other flips. If you know that I used this algorithm, the posterior odds
are 8:1 in favor of an H-biased coin.}

{
 I could be reporting on all flips, and only flips, that came up
heads. In this case, you know that all 7 other flips came up tails, and
the posterior odds are 1:16 against the coin being H-biased.}

{
 I could have decided in advance to say the result of the 4th, 6th,
and 9th flips only if the probability of the coin being H-biased
exceeds 98\%. And so on.}

{
 Or consider the Monty Hall problem:}

{
 On a game show, you are given the choice of three doors leading to
three rooms. You know that in one room is \$100,000, and the other two
are empty. The host asks you to pick a door, and you pick door \#1.
Then the host opens door \#2, revealing an empty room. Do you want to
switch to door \#3, or stick with door \#1?}

{
 The answer depends on the host's algorithm. If the
host always opens a door and always picks a door leading to an empty
room, then you should switch to door \#3. If the host always opens door
\#2 regardless of what is behind it, \#1 and \#3 both have 50\%
probabilities of containing the money. If the host only opens a door,
at all, if you initially pick the door with the money, then you should
definitely stick with \#1.}

{
 You shouldn't just condition on \#2 being empty,
but this fact plus the fact of the host \textit{choosing} to open door
\#2. Many people are confused by the standard Monty Hall problem
because they update only on \#2 being empty, in which case \#1 and \#3
have equal probabilities of containing the money. This is why Bayesians
are commanded to condition on all of their knowledge, on pain of
paradox.}

{
 When someone says, ``The 4th coinflip came up
heads,'' we are not conditioning on the 4th coinflip
having come up heads---we are not taking the subset of all possible
worlds where the 4th coinflip came up heads---rather we are
conditioning on the subset of all possible worlds where a speaker
following some particular algorithm \textit{said}
``The 4th coinflip came up heads.''
The spoken sentence is not the fact itself; don't be
led astray by the mere meanings of words.}

{
 Most legal processes work on the theory that every case has
exactly two opposed sides and that it is easier to find two biased
humans than one unbiased one. Between the prosecution and the defense,
\textit{someone} has a motive to present any given piece of evidence,
so the court will see all the evidence; that is the theory. If there
are two clever arguers in the box dilemma, it is not quite as good as
one curious inquirer, but it is almost as good. But that is with two
boxes. Reality often has many-sided problems, and deep problems, and
nonobvious answers, which are not readily found by Blues and Greens
screaming at each other.}

{
 Beware lest you abuse the notion of evidence-filtering as a Fully
General Counterargument to exclude all evidence you
don't like: ``That argument was
filtered, therefore I can ignore it.'' If
you're ticked off by a contrary argument, then you are
familiar with the case, and care enough to take sides. You probably
already know your own side's strongest arguments. You
have no reason to infer, from a contrary argument, the existence of new
favorable signs and portents which you have not yet seen. So you are
left with the uncomfortable facts themselves; a blue stamp on box B is
still evidence.}

{
 But if you are hearing an argument for the first time, and you are
only hearing one side of the argument, then indeed you should beware!
In a way, no one can \textit{really} trust the theory of natural
selection until after they have listened to creationists for five
minutes; and \textit{then} they know it's solid.}

\myendsectiontext

\mysection{Rationalization}

{
 In The Bottom Line, I presented the dilemma of two boxes, only one
of which contains a diamond, with various signs and portents as
evidence. I dichotomized the curious inquirer and the clever arguer.
The curious inquirer writes down all the signs and portents, and
processes them, and finally writes down
``\textit{Therefore,} I estimate an 85\% probability
that box B contains the diamond.'' The clever arguer
works for the highest bidder, and begins by writing,
``\textit{Therefore,} box B contains the
diamond,'' and then selects favorable signs and
portents to list on the lines above. }

{
 The first procedure is rationality. The second procedure is
generally known as
``rationalization.''}

{
 ``Rationalization.'' What a
curious term. I would call it a \textit{wrong word.} You cannot
``rationalize'' what is not already
rational. It is as if ``lying'' were
called ``truthization.''}

{
 On a purely computational level, there is a rather large
difference between:}

{
 Starting from evidence, and then crunching probability flows, in
order to output a probable conclusion. (Writing down all the signs and
portents, and then flowing forward to a probability on the bottom line
which depends on those signs and portents.)}

{
 Starting from a conclusion, and then crunching probability flows,
in order to output evidence apparently favoring that conclusion.
(Writing down the bottom line, and then flowing backward to select
signs and portents for presentation on the lines above.)}

{
 What fool devised such confusingly similar words,
``rationality'' and
``rationalization,'' to describe
such extraordinarily different mental processes? I would prefer terms
that made the algorithmic difference obvious, like
``rationality'' versus
``giant sucking cognitive black
hole.''}

{
 Not every change is an improvement, but every improvement is
necessarily a change. You cannot obtain more truth for a fixed
proposition by arguing it; you can make more people believe it, but you
cannot make it more \textit{true}. To improve our beliefs, we must
necessarily change our beliefs. Rationality is the operation that we
use to obtain more accuracy for our beliefs by changing them.
Rationalization operates to fix beliefs in place; it would be better
named ``anti-rationality,'' both for
its pragmatic results and for its reversed algorithm.}

{
 ``Rationality'' is the
\textit{forward} flow that gathers evidence, weighs it, and outputs a
conclusion. The curious inquirer used a forward-flow algorithm:
\textit{first} gathering the evidence, writing down a list of all
visible signs and portents, which they then processed \textit{forward}
to obtain a previously unknown probability for the box containing the
diamond. During the entire time that the rationality-process was
running forward, the curious inquirer did not yet know their
destination, which was why they were \textit{curious.} In the Way of
Bayes, the prior probability equals the expected posterior probability:
If you know your destination, you are already there.}

{
 ``Rationalization'' is a
\textit{backward} flow from conclusion to selected evidence. First you
write down the bottom line, which is known and fixed; the purpose of
your processing is to find out which arguments you should write down on
the lines above. This, not the bottom line, is the variable unknown to
the running process.}

{
 I fear that Traditional Rationality does not properly sensitize
its users to the difference between forward flow and backward flow. In
Traditional Rationality, there is nothing wrong with the scientist who
arrives at a pet hypothesis and then sets out to find an experiment
that proves it. A Traditional Rationalist would look at this
approvingly, and say, ``This pride is the engine that
drives Science forward.'' Well, it \textit{is} the
engine that drives Science forward. It is easier to find a prosecutor
and defender biased in opposite directions, than to find a single
unbiased human.}

{
 But just because everyone does something, doesn't
make it okay. It would be better yet if the scientist, arriving at a
pet hypothesis, set out to \textit{test} that hypothesis for the sake
of \textit{curiosity}{}---creating experiments that would drive their
own beliefs in an unknown direction.}

{
 If you genuinely don't know where you are going,
you will probably feel quite curious about it. Curiosity is the first
virtue, without which your questioning will be purposeless and your
skills without direction.}

{
 Feel the flow of the Force, and make sure it isn't
flowing backwards.}

\myendsectiontext

\mysection{A Rational Argument}

{
 You are, by occupation, a campaign manager, and
you've just been hired by Mortimer Q. Snodgrass, the
Green candidate for Mayor of Hadleyburg. As a campaign manager reading
a book on rationality, one question lies foremost on your mind:
``How can I construct an impeccable rational argument
that Mortimer Q. Snodgrass is the best candidate for Mayor of
Hadleyburg?'' }

{
 Sorry. It can't be done.}

{
 ``What?'' you cry.
``But what if I use only valid support to construct my
structure of reason? What if every fact I cite is true to the best of
my knowledge, and relevant evidence under Bayes's
Rule?''}

{
 Sorry. It still can't be done. You defeated
yourself the instant you specified your argument's
conclusion in advance.}

{
 This year, the \textit{Hadleyburg Trumpet} sent out a 16-item
questionnaire to all mayoral candidates, with questions like
``Can you paint with all the colors of the
wind?'' and ``Did you
inhale?'' Alas, the
\textit{Trumpet's} offices are destroyed by a meteorite
before publication. It's a pity, since your own
candidate, Mortimer Q. Snodgrass, compares well to his opponents on 15
out of 16 questions. The only sticking point was Question 11,
``Are you now, or have you ever been, a
supervillain?''}

{
 So you are tempted to publish the questionnaire as part of your
own campaign literature \ldots with the 11th question omitted, of
course.}

{
 Which crosses the line between \textit{rationality} and
\textit{rationalization.} It is no longer possible for the voters to
condition on the facts alone; they must condition on the additional
fact of their presentation, and infer the existence of hidden
evidence.}

{
 Indeed, you crossed the line at the point where you considered
whether the questionnaire was favorable or unfavorable to your
candidate, before deciding whether to publish it.
``What!'' you cry.
``A campaign should publish facts unfavorable to their
candidate?'' But put yourself in the shoes of a
voter, still trying to select a candidate---why would you censor useful
information? You wouldn't, if you were genuinely
curious. If you were flowing \textit{forward} from the evidence to an
unknown choice of candidate, rather than flowing \textit{backward} from
a fixed candidate to determine the arguments.}

{
 A ``logical'' argument is one
that follows from its premises. Thus the following argument is
\textit{illogical}:}

{
 All rectangles are quadrilaterals.}

{
 All squares are quadrilaterals.}

{
 \textit{Therefore}, all squares are rectangles.}

{
 This syllogism is not rescued from illogic by the truth of its
premises or even the truth of its conclusion. It is worth
distinguishing logical deductions from illogical ones, and to refuse to
excuse them even if their conclusions happen to be true. For one thing,
the distinction may affect how we revise our beliefs in light of future
evidence. For another, sloppiness is habit-forming.}

{
 Above all, the syllogism fails to state the real explanation.
Maybe all squares are rectangles, but, if so, it's not
\textit{because} they are both quadrilaterals. You might call it a
hypocritical syllogism---one with a disconnect between its stated
reasons and real reasons.}

{
 If you really want to present an honest, rational argument
\textit{for your candidate}, in a political campaign, there is only one
way to do it:}

{
 \textit{Before anyone hires you,} gather up all the evidence you
can about the different candidates.}

{
 Make a checklist which you, yourself, will use to decide which
candidate seems best.}

{
 Process the checklist.}

{
 Go to the winning candidate.}

{
 Offer to become their campaign manager.}

{
 When they ask for campaign literature, print out your checklist.}

{
 Only in this way can you offer a \textit{rational} chain of
argument, one whose bottom line was written flowing \textit{forward}
from the lines above it. Whatever \textit{actually} decides your bottom
line, is the only thing you can \textit{honestly} write on the lines
above.}

\myendsectiontext

\mysection{Avoiding Your Belief's Real Weak Points}

{
 A few years back, my great-grandmother died, in her nineties,
after a long, slow, and cruel disintegration. I never knew her as a
person, but in my distant childhood, she cooked for her family; I
remember her gefilte fish, and her face, and that she was kind to me.
At her funeral, my grand-uncle, who had taken care of her for years,
spoke. He said, choking back tears, that God had called back his mother
piece by piece: her memory, and her speech, and then finally her smile;
and that when God finally took her smile, he knew it
wouldn't be long before she died, because it meant that
she was almost entirely gone. }

{
 I heard this and was puzzled, because it was an unthinkably
horrible thing to happen to \textit{anyone}, and therefore I would not
have expected my grand-uncle to attribute it to God. Usually, a Jew
would somehow just-not-think-about the logical implication that God had
permitted a tragedy. According to Jewish theology, God continually
sustains the universe and chooses every event in it; but ordinarily,
drawing logical implications from this belief is reserved for happier
occasions. By saying ``God did it!''
only when you've been blessed with a baby girl, and
just-not-thinking ``God did it!''
for miscarriages and stillbirths and crib deaths, you can build up
quite a lopsided picture of your God's benevolent
personality.}

{
 Hence I was surprised to hear my grand-uncle attributing the slow
disintegration of his mother to a deliberate, strategically planned act
of God. It violated the rules of religious self-deception as I
understood them.}

{
 If I had noticed my own confusion, I could have made a successful
surprising prediction. Not long afterward, my grand-uncle left the
Jewish religion. (The only member of my extended family besides myself
to do so, as far as I know.)}

{
 Modern Orthodox Judaism is like no other religion I have ever
heard of, and I don't know how to describe it to anyone
who hasn't been forced to study Mishna and Gemara.
There is a tradition of questioning, but the \textit{kind} of
questioning \ldots It would not be at all surprising to hear a rabbi, in
his weekly sermon, point out the conflict between the seven days of
creation and the 13.7 billion years since the Big Bang---because he
thought he had a really clever explanation for it, involving three
other Biblical references, a Midrash, and a half-understood article in
\textit{Scientific American.} In Orthodox Judaism
you're allowed to notice inconsistencies and
contradictions, but only for purposes of explaining them away, and
whoever comes up with the most complicated explanation gets a prize.}

{
 There is a tradition of inquiry. But you only attack targets for
purposes of defending them. You only attack targets you know you can
defend.}

{
 In Modern Orthodox Judaism I have not heard much emphasis of the
virtues of blind faith. You're allowed to doubt.
You're just not allowed to \textit{successfully}
doubt.}

{
 I expect that the vast majority of educated Orthodox Jews have
questioned their faith at some point in their lives. But the
questioning probably went something like this:
``According to the skeptics, the Torah says that the
universe was created in seven days, which is not scientifically
accurate. But would the original tribespeople of Israel, gathered at
Mount Sinai, have been able to understand the scientific truth, even if
it had been presented to them? Did they even have a word for
`billion'? It's easier
to see the seven-days story as a metaphor---first God created light,
which represents the Big Bang \ldots''}

{
 Is this the weakest point at which to attack one's
own Judaism? Read a bit further on in the Torah, and you can find God
killing the first-born male children of Egypt to convince an unelected
Pharaoh to release slaves who logically could have been teleported out
of the country. An Orthodox Jew is most certainly familiar with this
episode, because they are supposed to read through the entire Torah in
synagogue once per year, and this event has an associated major
holiday. The name ``Passover''
(``Pesach'') comes from God
\textit{passing over} the Jewish households while killing every male
firstborn in Egypt.}

{
 Modern Orthodox Jews are, by and large, kind and civilized people;
far more civilized than the several editors of the Old Testament. Even
the old rabbis were more civilized. There's a ritual in
the Seder where you take ten drops of wine from your cup, one drop for
each of the Ten Plagues, to emphasize the suffering of the Egyptians.
(Of course, you're supposed to be sympathetic to the
suffering of the Egyptians, but not \textit{so} sympathetic that you
stand up and say, ``This is not right! It is
\textit{wrong} to do such a thing!'') It shows an
interesting contrast---the rabbis were sufficiently kinder than the
compilers of the Old Testament that they saw the harshness of the
Plagues. But Science was weaker in these days, and so rabbis could
ponder the more unpleasant aspects of Scripture without fearing that it
would break their faith entirely.}

{
 You don't even \textit{ask} whether the incident
reflects poorly on God, so there's no need to quickly
blurt out ``The ways of God are
mysterious!'' or
``We're not wise enough to question
God's decisions!'' or
``Murdering babies is okay when God does
it!'' That part of the question is
just-not-thought-about.}

{
 The reason that educated religious people stay religious, I
suspect, is that when they doubt, they are subconsciously very careful
to attack their own beliefs only at the strongest points---places where
they know they can defend. Moreover, places where rehearsing the
standard defense will feel strengthening.}

{
 It probably feels really good, for example, to rehearse
one's prescripted defense for
``Doesn't Science say that the
universe is just meaningless atoms bopping around?,''
because it confirms the meaning of the universe and how it flows from
God, etc. Much more comfortable to think about than an illiterate
Egyptian mother wailing over the crib of her slaughtered son. Anyone
who \textit{spontaneously} thinks about the latter, when questioning
their faith in Judaism, is \textit{really} questioning it, and is
probably not going to stay Jewish much longer.}

{
 My point here is not just to beat up on Orthodox Judaism.
I'm sure that there's some reply or
other for the Slaying of the Firstborn, and probably a dozen of them.
My point is that, when it comes to spontaneous self-questioning, one is
much more likely to spontaneously self-attack strong points with
comforting replies to rehearse, then to spontaneously self-attack the
weakest, most vulnerable points. Similarly, one is likely to stop at
the first reply and be comforted, rather than further criticizing the
reply. A better title than ``Avoiding Your
Belief's Real Weak Points'' would be
``Not Spontaneously Thinking About Your
Belief's Most Painful Weaknesses.''}

{
 More than anything, the grip of religion is sustained by people
just-not-thinking-about the real weak points of their religion. I
don't think this is a matter of training, but a matter
of instinct. People don't think about the real weak
points of their beliefs for the same reason they don't
touch an oven's red-hot burners; it's
\textit{painful.}}

{
 To do better: When you're doubting one of your
most cherished beliefs, close your eyes, empty your mind, grit your
teeth, and deliberately think about whatever hurts the most.
Don't rehearse standard objections whose standard
counters would make you feel better. Ask yourself what \textit{smart}
people who disagree would say to your first reply, and your second
reply. Whenever you catch yourself flinching away from an objection you
fleetingly thought of, drag it out into the forefront of your mind.
Punch yourself in the solar plexus. Stick a knife in your heart, and
wiggle to widen the hole. In the face of the pain, rehearse only this:}

{
 What is true is already so.}

{
 Owning up to it doesn't make it worse.}

{
 Not being open about it doesn't make it go away.}

{
 And because it's true, it is what is there to be
interacted with.}

{
 Anything untrue isn't there to be lived.}

{
 People can stand what is true,}

{
 for they are already enduring it.}

{\raggedleft
 {}---Eugene Gendlin\textsuperscript{1}
\par}


\bigskip

{
 (Hat tip to Stephen Omohundro.)}

\myendsectiontext


\bigskip

{
 1. Eugene T. Gendlin, \textit{Focusing} (Bantam Books, 1982).}

\mysection{Motivated Stopping and Motivated Continuation}

{
 While I disagree with some views of the Fast and Frugal crowd---in
my opinion they make a few \textit{too} many lemons into lemonade---it
also seems to me that they tend to develop the most
\textit{psychologically realistic} models of any school of decision
theory. Most experiments present the subjects with options, and the
subject chooses an option, and that's the experimental
result. The frugalists realized that in real life, you have to
\textit{generate} your options, and they studied how subjects did
\textit{that.} }

{
 Likewise, although many experiments present evidence on a silver
platter, in real life you have to gather evidence, which may be costly,
and at some point decide that you have enough evidence to stop and
choose. When you're buying a house, you
don't get exactly ten houses to choose from, and you
aren't led on a guided tour of all of them before
you're allowed to decide anything. You look at one
house, and another, and compare them to each other; you adjust your
aspirations---reconsider how much you really need to be close to your
workplace and how much you're really willing to pay;
you decide which house to look at next; and at some point you decide
that you've seen enough houses, and choose.}

{
 Gilovich's distinction between \textit{motivated
skepticism} and \textit{motivated credulity} highlights how conclusions
a person does not want to believe are held to a higher standard than
conclusions a person wants to believe. A motivated skeptic asks if the
evidence \textit{compels} them to accept the conclusion; a motivated
credulist asks if the evidence \textit{allows} them to accept the
conclusion.}

{
 I suggest that an analogous bias in psychologically realistic
search is \textit{motivated stopping} and \textit{motivated
continuation}: when we have a \textit{hidden} motive for choosing the
``best'' current option, we have a
hidden motive to stop, and choose, and reject consideration of any more
options. When we have a hidden motive to reject the current best
option, we have a hidden motive to suspend judgment pending additional
evidence, to generate more options---to find something, anything, to do
\textit{instead} of coming to a conclusion.}

{
 A major historical scandal in statistics was R. A. Fisher, an
eminent founder of the field, insisting that no \textit{causal} link
had been established between smoking and lung cancer.
``Correlation is not causation,'' he
testified to Congress. Perhaps smokers had a gene which both
predisposed them to smoke and predisposed them to lung cancer.}

{
 Or maybe Fisher's being employed as a consultant
for tobacco firms gave him a hidden motive to decide that the evidence
already gathered was insufficient to come to a conclusion, and it was
better to keep looking. Fisher was also a smoker himself, and died of
colon cancer in 1962.}

{
 (Ad hominem note: Fisher was a frequentist. Bayesians are more
reasonable about inferring probable causality.)}

{
 Like many other forms of motivated skepticism, motivated
continuation can try to disguise itself as virtuous rationality. Who
can argue against gathering more evidence? I can. Evidence is often
costly, and worse, slow, and there is certainly nothing virtuous about
refusing to integrate the evidence you already have. You can always
change your mind later. (Apparent contradiction resolved as follows:
Spending \textit{one hour} discussing the problem, with your mind
carefully cleared of all conclusions, is different from waiting ten
years on another \$20 million study.)}

{
 As for motivated stopping, it appears in every place a third
alternative is feared, and wherever you have an argument whose obvious
counterargument you would rather not see, and in other places as well.
It appears when you pursue a course of action that makes you feel good
just for acting, and so you'd rather not investigate
how well your plan \textit{really} worked, for fear of destroying the
warm glow of moral satisfaction you paid good money to purchase. It
appears wherever your beliefs and anticipations get out of sync, so you
have a reason to fear any new evidence gathered.}

{
 The moral is that the decision to terminate a search procedure
(temporarily or permanently) is, like the search procedure itself,
subject to bias and hidden motives. You should suspect motivated
stopping when you close off search, after coming to a comfortable
conclusion, and yet there's a lot of fast cheap
evidence you haven't gathered yet---there are websites
you could visit, there are counter-counter arguments you could
consider, or you haven't closed your eyes for five
minutes by the clock trying to think of a better option. You should
suspect motivated continuation when some evidence is leaning in a way
you don't like, but you decide that more evidence is
needed---\textit{expensive} evidence that you know you
can't gather anytime soon, as opposed to something
you're going to look up on Google in thirty
minutes---before you'll have to do anything
uncomfortable.}

\myendsectiontext

\mysection{Fake Justification}

{
 Many Christians who've stopped really believing
now insist that they revere the Bible as a source of ethical advice.
The standard atheist reply is given by Sam Harris:
``You and I both know that it would take us five
minutes to produce a book that offers a more coherent and compassionate
morality than the Bible does.'' Similarly, one may
try to insist that the Bible is valuable as a literary work. Then why
not revere \textit{Lord of the Rings}, a vastly superior literary work?
And despite the standard criticisms of Tolkien's
morality, \textit{Lord of the Rings} is at least superior to the Bible
as a source of ethics. So why don't people wear little
rings around their neck, instead of crosses? Even \textit{Harry Potter}
is superior to the Bible, both as a work of literary art and as moral
philosophy. If I really wanted to be cruel, I would compare the Bible
to Jacqueline Carey's \textit{Kushiel} series. }

{
 ``How can you justify buying a \$1 million
gem-studded laptop,'' you ask your friend,
``when so many people have no laptops at
all?'' And your friend says, ``But
think of the employment that this will provide---to the laptop maker,
the laptop maker's advertising agency---and then
they'll buy meals and haircuts---it will stimulate the
economy and eventually many people will get their own
laptops.'' But it would be even \textit{more}
efficient to buy 5,000 One Laptop Per Child laptops, thus providing
employment to the OLPC manufacturers \textit{and} giving out laptops
directly.}

{
 I've touched before on the failure to look for
third alternatives. But this is not really motivated stopping. Calling
it ``motivated stopping'' would
imply that there was a search carried out in the first place.}

{
 In The Bottom Line, I observed that only the real determinants of
our beliefs can ever influence our real-world accuracy, only the real
determinants of our actions can influence our effectiveness in
achieving our goals. Someone who buys a million-dollar laptop was
really thinking, ``Ooh, shiny,'' and
that was the one true causal history of their decision to buy a laptop.
No amount of ``justification'' can
change this, unless the justification is a genuine, newly running
search process that can change the conclusion. \textit{Really} change
the conclusion. Most criticism carried out from a sense of duty is more
of a token inspection than anything else. Free elections in a one-party
country.}

{
 To genuinely justify the Bible as a lauding-object by reference to
its literary quality, you would have to somehow perform a neutral
reading through candidate books until you found the book of highest
literary quality. Renown is one reasonable criteria for generating
candidates, so I suppose you could legitimately end up reading
Shakespeare, the Bible, and \textit{Gödel, Escher, Bach}. (Otherwise it
would be quite a coincidence to find the Bible as a candidate, among a
million other books.) The real difficulty is in that
``neutral reading'' part. Easy
enough if you're not a Christian, but if you are \ldots}

{
 But of course nothing like this happened. No search ever occurred.
Writing the justification of ``literary
quality'' above the bottom line of
``I {\textless}heart{\textgreater} the
Bible'' is a historical misrepresentation of how the
bottom line really got there, like selling cat milk as cow milk. That
is just not where the bottom line really came from. That is just not
what originally happened to produce that conclusion.}

{
 If you genuinely subject your conclusion to a criticism that can
potentially de-conclude it---if the criticism \textit{genuinely} has
that power---then that does modify ``the real
algorithm behind'' your conclusion. It changes the
entanglement of your conclusion over possible worlds. But people
overestimate, by far, how likely they \textit{really} are to change
their minds.}

{
 With all those open minds out there, you'd think
there'd be more belief-updating.}

{
 Let me guess: Yes, you admit that you originally decided you
wanted to buy a million-dollar laptop by thinking,
``Ooh, shiny.'' Yes, you concede
that this isn't a decision process consonant with your
stated goals. But since then, you've decided that you
really ought to spend your money in such fashion as to provide laptops
to as many laptopless wretches as possible. And yet you just
\textit{couldn't} find any more efficient way to do
this than buying a million-dollar diamond-studded laptop---because,
hey, you're giving money to a laptop store and
stimulating the economy! Can't beat that!}

{
 My friend, I am damned suspicious of this amazing coincidence. I
am damned suspicious that the best answer under this lovely, rational,
altruistic criterion X, is also the idea that just happened to
originally pop out of the unrelated indefensible process Y. If you
don't think that rolling dice would have been likely to
produce the correct answer, then how likely is it to pop out of any
other irrational cognition?}

{
 It's improbable that you used mistaken reasoning,
yet made no mistakes.}

\myendsectiontext

\mysection{Is That Your True Rejection?}

{
 It happens every now and then, that the one encounters some of my
transhumanist-side beliefs---as opposed to my ideas having to do with
human rationality---strange, exotic-sounding ideas like
superintelligence and Friendly AI. And the one rejects them. }

{
 If the one is called upon to explain the rejection, not uncommonly
the one says, ``Why should I believe anything
Yudkowsky says? He doesn't have a
PhD!''}

{
 And occasionally someone else, hearing, says,
``Oh, you should get a PhD, so that people will listen
to you.'' Or this advice may even be offered by the
same one who disbelieved, saying, ``Come back when you
have a PhD.''}

{
 Now there are good and bad reasons to get a PhD, but this is one
of the bad ones.}

{
 There's many reasons why someone \textit{actually}
has an adverse reaction to transhumanist theses. Most are matters of
pattern recognition, rather than verbal thought: the thesis matches
against ``strange weird idea'' or
``science fiction'' or
``end-of-the-world cult'' or
``overenthusiastic youth.''}

{
 So immediately, at the speed of perception, the idea is rejected.
If, afterward, someone says ``Why
not?,'' this launches a search for justification. But
this search will not necessarily hit on the true reason---by
``true reason'' I mean not the
\textit{best} reason that could be offered, but rather, whichever
causes were decisive as a matter of historical fact, at the
\textit{very first} moment the rejection occurred.}

{
 Instead, the search for justification hits on the
justifying-sounding fact, ``This speaker does not have
a PhD.''}

{
 But I also don't have a PhD when I talk about
human rationality, so why is the same objection not raised there?}

{
 And more to the point, if I \textit{had} a PhD, people would not
treat this as a decisive factor indicating that they ought to believe
everything I say. Rather, the same initial rejection would occur, for
the same reasons; and the search for justification, afterward, would
terminate at a different stopping point.}

{
 They would say, ``Why should I believe
\textit{you}? You're just some guy with a PhD! There
are lots of those. Come back when you're well-known in
your field and tenured at a major university.''}

{
 But do people \textit{actually} believe arbitrary professors at
Harvard who say weird things? Of course not. (But if I were a professor
at Harvard, it would in fact be easier to get \textit{media attention.}
Reporters initially disinclined to believe me---who would probably be
equally disinclined to believe a random PhD-bearer---would still report
on me, because it would be news that a Harvard professor believes such
a weird thing.)}

{
 If you are saying things that sound \textit{wrong} to a novice, as
opposed to just rattling off magical-sounding technobabble about
leptical quark braids in N + 2 dimensions; and the hearer is a
stranger, unfamiliar with you personally \textit{and} with the subject
matter of your field; then I suspect that the point at which the
average person will \textit{actually} start to grant credence
overriding their initial impression, purely \textit{because} of
academic credentials, is somewhere around the Nobel Laureate level. If
that. Roughly, you need whatever level of academic credential qualifies
as ``beyond the mundane.''}

{
 This is more or less what happened to Eric Drexler, as far as I
can tell. He presented his vision of nanotechnology, and people said,
``Where are the technical details?''
or ``Come back when you have a
PhD!'' And Eric Drexler spent six years writing up
technical details and got his PhD under Marvin Minsky for doing it. And
\textit{Nanosystems} is a great book. But did the same people who said,
``Come back when you have a PhD,''
actually change their minds at all about molecular nanotechnology? Not
so far as I ever heard.}

{
 It has similarly been a general rule with the Machine Intelligence
Research Institute that, whatever it is we're supposed
to do to be more credible, when we actually do it, nothing much
changes. ``Do you do any sort of code development?
I'm not interested in supporting an organization that
doesn't develop code'' $\rightarrow $
OpenCog $\rightarrow $ nothing changes. ``Eliezer
Yudkowsky lacks academic credentials'' $\rightarrow $
Professor Ben Goertzel installed as Director of Research $\rightarrow $
nothing changes. The one thing that actually \textit{has} seemed to
raise credibility, is famous people associating with the organization,
like Peter Thiel funding us, or Ray Kurzweil on the Board.}

{
 This might be an important thing for young businesses and
new-minted consultants to keep in mind---that what your failed
prospects \textit{tell} you is the reason for rejection, may not make
the \textit{real} difference; and you should ponder that carefully
before spending huge efforts. If the venture capitalist says
``If only your sales were growing a little
faster!,'' or if the potential customer says
``It seems good, but you don't have
feature X,'' that may not be the \textit{true}
rejection. Fixing it may, or may not, change anything.}

{
 And it would also be something to keep in mind during
disagreements. Robin Hanson and I share a belief that two rationalists
should not agree to disagree: they should not have common knowledge of
epistemic disagreement unless something is very wrong.}

{
 I suspect that, in general, if two rationalists set out to resolve
a disagreement that persisted past the first exchange, they should
expect to find that the true sources of the disagreement are either
hard to communicate, or hard to expose. E.g.:}

{
 Uncommon, but well-supported, scientific knowledge or math;}

{
 Long inferential distances;}

{
 Hard-to-verbalize intuitions, perhaps stemming from specific
visualizations;}

{
 Zeitgeists inherited from a profession (that may have good reason
for it);}

{
 Patterns perceptually recognized from experience;}

{
 Sheer habits of thought;}

{
 Emotional commitments to believing in a particular outcome;}

{
 Fear of a past mistake being disproven;}

{
 Deep self-deception for the sake of pride or other personal
benefits.}

{
 If the matter were one in which \textit{all} the true rejections
could be \textit{easily} laid on the table, the disagreement would
probably be so straightforward to resolve that it would never have
lasted past the first meeting.}

{
 ``Is this my true rejection?''
is something that both disagreers should surely be asking
\textit{themselves}, to make things easier on the Other Fellow.
However, attempts to directly, publicly psychoanalyze the Other may
cause the conversation to degenerate \textit{very} fast, in my
observation.}

{
 Still---``Is that your true
rejection?'' should be fair game for Disagreers to
humbly ask, if there's any productive way to pursue
that sub-issue. Maybe the rule could be that you can openly ask,
``Is that simple straightforward-sounding reason your
\textit{true} rejection, or does it come from intuition-X or
professional-zeitgeist-Y?'' While the more
embarrassing possibilities lower on the table are left to the
Other's conscience, as their own responsibility to
handle.}

\myendsectiontext

\mysection{Entangled Truths, Contagious Lies}

{
 One of your very early philosophers came to the conclusion that a
fully competent mind, from a study of one fact or artifact belonging to
any given universe, could construct or visualize that universe, from
the instant of its creation to its ultimate end \ldots}

{\raggedleft
 {}---\textit{First Lensman}\textsuperscript{1}
\par}


\bigskip

{
 ~}

{
 If any one of you will concentrate upon one single fact, or small
object, such as a pebble or the seed of a plant or other creature, for
as short a period of time as one hundred of your years, you will begin
to perceive its truth.}

{\raggedleft
 {}---\textit{Gray Lensman}\textsuperscript{2}
\par}


\bigskip

{
 ~}

{
 I am reasonably sure that a single pebble, taken from a beach of
our own Earth, does not specify the continents and countries, politics
and people of this Earth. Other planets in space and time, other
Everett branches, would generate the same pebble. On the other hand,
the identity of a single pebble would seem to include our laws of
physics. In that sense the entirety of our Universe---\textit{all} the
Everett branches---would be implied by the pebble. (If, as seems
likely, there are no truly free variables.)}

{
 So a single pebble probably does not imply our whole Earth. But a
single pebble implies a very great deal. From the study of that single
pebble you could see the laws of physics and all they imply. Thinking
about those laws of physics, you can see that planets will form, and
you can guess that the pebble came from such a planet. The internal
crystals and molecular formations of the pebble formed under gravity,
which tells you something about the planet's mass; the
mix of elements in the pebble tells you something about the
planet's formation.}

{
 I am not a geologist, so I don't know to which
mysteries geologists are privy. But I find it very easy to imagine
showing a geologist a pebble, and saying, ``This
pebble came from a beach at Half Moon Bay,'' and the
geologist immediately says, ``I'm
confused'' or even ``You
liar.'' Maybe it's the wrong kind of
rock, or the pebble isn't worn enough to be from a
beach---I don't know pebbles well enough to guess the
linkages and signatures by which I might be caught, which is the
point.}

{
 ``Only God can tell a truly plausible
lie.'' I wonder if there was ever a religion that
developed this as a proverb? I would (falsifiably) guess not:
it's a rationalist sentiment, even if you cast it in
theological metaphor. Saying ``everything is
interconnected to everything else, because God made the whole world and
sustains it'' may generate some nice warm
'n' fuzzy feelings during the sermon,
but it doesn't get you very far when it comes to
assigning pebbles to beaches.}

{
 A penny on Earth exerts a gravitational acceleration on the Moon
of around 4.5 {\texttimes} 10\textsuperscript{{}-31}
m/s\textsuperscript{2}, so in one sense it's not too
far wrong to say that every event is entangled with its whole past
light cone. And since inferences can propagate backward and forward
through causal networks, \textit{epistemic} entanglements can easily
cross the borders of light cones. But I wouldn't want
to be the forensic astronomer who had to look at the Moon and figure
out whether the penny landed heads or tails---the influence is far less
than quantum uncertainty and thermal noise.}

{
 If you said ``Everything is entangled with
something else'' or ``Everything is
inferentially entangled and some entanglements are much stronger than
others,'' you might be really wise instead of just
Deeply Wise.}

{
 Physically, each event is in some sense the sum of its whole past
light cone, without borders or boundaries. But the list of
\textit{noticeable} entanglements is much shorter, and it gives you
something like a network. This high-level regularity is what I refer to
when I talk about the Great Web of Causality.}

{
 I use these Capitalized Letters somewhat tongue-in-cheek, perhaps;
but if anything at all is worth Capitalized Letters, surely the Great
Web of Causality makes the list.}

{
 ``Oh what a tangled web we weave, when first we
practise to deceive,'' said Sir Walter Scott. Not
\textit{all} lies spin out of control---we don't live
in so righteous a universe. But it does occasionally happen, that
someone lies about a fact, and then has to lie about an entangled fact,
and then another fact entangled with that one:}

{
 ``Where were you?''}

{
 ``Oh, I was on a business
trip.''}

{
 ``What was the business trip
about?''}

{
 ``I can't tell you that;
it's proprietary negotiations with a major
client.''}

{
 ``Oh---they're letting you in on
those? Good news! I should call your boss to thank him for adding
you.''}

{
 ``Sorry---he's not in the office
right now \ldots''}

{
 Human beings, who are not gods, often fail to \textit{imagine} all
the facts they would need to distort to tell a truly plausible lie.
``God made me pregnant'' sounded a
tad more likely in the old days before our models of the world
contained (quotations of) Y chromosomes. Many similar lies, today, may
blow up when genetic testing becomes more common. Rapists have been
convicted, and false accusers exposed, years later, based on evidence
they didn't realize they could leave. A student of
evolutionary biology can see the design signature of natural selection
on every wolf that chases a rabbit; and every rabbit that runs away;
and every bee that stings instead of broadcasting a polite
warning---but the deceptions of creationists sound plausible to
\textit{them}, I'm sure.}

{
 Not all lies are uncovered, not all liars are punished; we
don't live in that righteous a universe. But not all
lies are as safe as their liars believe. How many sins would become
known to a Bayesian superintelligence, I wonder, if it did a
(non-destructive?) nanotechnological scan of the Earth? At minimum, all
the lies of which any evidence still exists in any brain. Some such
lies may become known sooner than that, if the neuroscientists ever
succeed in building a really good lie detector via neuroimaging. Paul
Ekman (a pioneer in the study of tiny facial muscle movements) could
probably read off a sizeable fraction of the world's
lies right now, given a chance.}

{
 Not all lies are uncovered, not all liars are punished. But the
Great Web is very commonly underestimated. Just the knowledge that
humans have \textit{already accumulated} would take many human
lifetimes to learn. Anyone who thinks that a non-God can tell a
\textit{perfect} lie, risk-free, is underestimating the tangledness of
the Great Web.}

{
 Is honesty the best policy? I don't know if
I'd go that far: Even on my ethics,
it's sometimes okay to shut up. But compared to
outright lies, either honesty or silence involves less exposure to
recursively propagating risks you don't know
you're taking.}

\myendsectiontext


\bigskip

{
 1. Edward Elmer Smith and A. J. Donnell, \textit{First Lensman}
(Old Earth Books, 1997).}

{
 2. Edward Elmer Smith and Ric Binkley, \textit{Gray Lensman} (Old
Earth Books, 1998).}

\mysection{Of Lies and Black Swan Blowups}

{
 Judge Marcus Einfeld, age 70, Queen's Counsel
since 1977, Australian Living Treasure 1997, United Nations Peace Award
2002, founding president of Australia's Human Rights
and Equal Opportunities Commission, retired a few years back but
routinely brought back to judge important cases \ldots }

{
 \ldots went to jail for two years over a series of perjuries and
lies that started with a £36, 6-mph-over speeding ticket.}

{
 That whole \textit{suspiciously virtuous-sounding} theory about
honest people not being good at lying, and entangled traces being left
somewhere, and the entire thing blowing up in a Black Swan epic fail,
actually \textit{does} have a certain number of exemplars in real life,
though obvious selective reporting is at work in our hearing about this
one.}

\myendsectiontext

\mysection{Dark Side Epistemology}

{
 If you once tell a lie, the truth is ever after your enemy. }

{
 I have previously spoken of the notion that, the truth being
entangled, lies are contagious. If you pick up a pebble from the
driveway, and tell a geologist that you found it on a beach---well, do
\textit{you} know what a geologist knows about rocks? I
don't. But I can suspect that a water-worn pebble
wouldn't look like a droplet of frozen lava from a
volcanic eruption. Do you know where the pebble in your driveway really
came from? Things bear the marks of their places in a lawful universe;
in that web, a lie is out of place. (Actually, a geologist in the
comments says that most pebbles in driveways are taken \textit{from}
beaches, so they couldn't tell the difference between a
driveway pebble and a beach pebble, but they could tell the difference
between a mountain pebble and a driveway/beach pebble. Case in point
\ldots)}

{
 What sounds like an arbitrary truth to one mind---one that could
easily be replaced by a plausible lie---might be nailed down by a dozen
linkages to the eyes of greater knowledge. To a creationist, the idea
that life was shaped by ``intelligent
design'' instead of ``natural
selection'' might sound like a sports team to cheer
for. To a biologist, plausibly arguing that an organism was
intelligently designed would require lying about almost every facet of
the organism. To plausibly argue that
``humans'' were intelligently
designed, you'd have to lie about the design of the
human retina, the architecture of the human brain, the proteins bound
together by weak van der Waals forces instead of strong covalent bonds
\ldots}

{
 Or you could just lie about evolutionary theory, which is the path
taken by most creationists. Instead of lying about the connected nodes
in the network, they lie about the \textit{general} laws governing the
links.}

{
 And then to cover \textit{that} up, they lie about the rules of
science---like what it means to call something a
``theory,'' or what it means for a
scientist to say that they are not absolutely certain.}

{
 So they pass from lying about specific facts, to lying about
general laws, to lying about the rules of reasoning. To lie about
whether humans evolved, you must lie about evolution; and then you have
to lie about the rules of science that constrain our understanding of
evolution.}

{
 But how else? Just as a human would be out of place in a community
of \textit{actually} intelligently designed life forms, and you have to
lie about the rules of evolution to make it appear otherwise; so too,
beliefs about creationism are themselves out of place in science---you
wouldn't find them in a well-ordered mind any more than
you'd find palm trees growing on a glacier. And so you
have to disrupt the barriers that would forbid them.}

{
 Which brings us to the case of self-deception.}

{
 A single lie you tell \textit{yourself} may seem plausible enough,
when you don't know any of the rules governing
thoughts, or even that there \textit{are} rules; and the choice seems
as arbitrary as choosing a flavor of ice cream, as isolated as a pebble
on the shore \ldots}

{
 \ldots but then someone calls you on your belief, using the rules
of reasoning that \textit{they've} learned. They say,
``Where's your
evidence?''}

{
 And you say, ``What? Why do I need
evidence?''}

{
 So they say, ``In general, beliefs require
evidence.''}

{
 This argument, clearly, is a soldier fighting on the other side,
which you must defeat. So you say: ``I disagree! Not
all beliefs require evidence. In particular, beliefs about dragons
don't require evidence. When it comes to dragons,
you're allowed to believe anything you like. So I
don't need evidence to believe there's
a dragon in my garage.''}

{
 And the one says, ``Eh? You can't
just exclude dragons like that. There's a reason for
the rule that beliefs require evidence. To draw a correct map of the
city, you have to walk through the streets and make lines on paper that
correspond to what you see. That's not an arbitrary
legal requirement---if you sit in your living room and draw lines on
the paper at random, the map's going to be wrong. With
extremely high probability. That's as true of a map of
a dragon as it is of anything.''}

{
 So now \textit{this}, the explanation of \textit{why} beliefs
require evidence, is \textit{also} an opposing soldier. So you say:
``Wrong with extremely high probability? Then
there's still a chance, right? I don't
have to believe if it's not absolutely
certain.''}

{
 Or maybe you even begin to suspect, yourself, that
``beliefs require evidence.'' But
this threatens a lie you hold precious; so you reject the dawn inside
you, push the Sun back under the horizon.}

{
 Or you've previously heard the proverb
``beliefs require evidence,'' and it
sounded wise enough, and you endorsed it in public. But it never quite
occurred to you, until someone else brought it to your attention, that
this proverb could \textit{apply to} your belief that
there's a dragon in your garage. So you think fast and
say, ``The dragon is in a separate
magisterium.''}

{
 Having false beliefs isn't a good thing, but it
doesn't have to be permanently crippling---if, when you
discover your mistake, you get over it. The dangerous thing is to have
a false belief that you \textit{believe should be protected as a
belief---}a belief-in-belief, whether or not accompanied by actual
belief.}

{
 A single Lie That Must Be Protected can block
someone's progress into advanced rationality. No,
it's not harmless fun.}

{
 Just as the world itself is more tangled by far than it appears on
the surface; so too, there are stricter rules of reasoning,
constraining belief more strongly, than the untrained would suspect.
The world is woven tightly, governed by general laws, and so are
\textit{rational} beliefs.}

{
 Think of what it would take to deny evolution or
heliocentrism---all the connected truths and governing laws you
wouldn't be allowed to know. Then you can imagine how a
single act of self-deception can block off the whole meta-level of
truthseeking, once your mind begins to be threatened by seeing the
connections. Forbidding all the intermediate and higher levels of the
rationalist's Art. Creating, in its stead, a vast
complex of anti-law, rules of anti-thought, general justifications for
believing the untrue.}

{
 Steven Kaas said, ``Promoting less than maximally
accurate beliefs is an act of sabotage. Don't do it to
anyone unless you'd also slash their
tires.'' Giving someone a false belief \textit{to
protect---}convincing them that the \textit{belief itself} must be
defended from any thought that seems to threaten it---well, you
shouldn't do that to someone unless
you'd also give them a frontal lobotomy.}

{
 Once you tell a lie, the truth is your enemy; and every truth
connected to that truth, and every ally of truth in general; all of
these you must oppose, to protect the lie. Whether
you're lying to others, or to yourself.}

{
 You have to deny that beliefs require evidence, and then you have
to deny that maps should reflect territories, and then you have to deny
that truth is a good thing \ldots}

{
 Thus comes into being the Dark Side.}

{
 I worry that people aren't aware of it, or
aren't sufficiently wary---that as we wander through
our human world, we can expect to encounter \textit{systematically} bad
epistemology.}

{
 The ``how to think'' memes
floating around, the cached thoughts of Deep Wisdom---some of it will
be good advice devised by rationalists. But other notions were invented
to protect a lie or self-deception: spawned from the Dark Side.}

{
 ``Everyone has a right to their own
opinion.'' When you think about it, where was that
proverb generated? Is it something that someone would say in the course
of protecting a truth, or in the course of protecting \textit{from} the
truth? But people don't perk up and say,
``Aha! I sense the presence of the Dark
Side!'' As far as I can tell, it's
not widely realized that the Dark Side is out there.}

{
 But how else? Whether you're deceiving others, or
just yourself, the Lie That Must Be Protected will propagate
recursively through the network of empirical causality, and the network
of general empirical rules, and the rules of reasoning themselves, and
the understanding behind those rules. If there is \textit{good}
epistemology in the world, and also lies or self-deceptions that people
are trying to protect, then there will come into existence bad
epistemology to counter the good. We could hardly expect, in this
world, to find the Light Side without the Dark Side; there is the Sun,
and that which shrinks away and generates a cloaking Shadow.}

{
 Mind you, these are not necessarily \textit{evil} people. The vast
majority who go about repeating the Deep Wisdom are more duped than
duplicitous, more self-deceived than deceiving. I think.}

{
 And it's surely not my intent to offer you a Fully
General Counterargument, so that whenever someone offers you some
epistemology you don't like, you say:
``Oh, someone on the Dark Side made that
up.'' It's one of the rules of the
Light Side that you have to refute the proposition for itself, not by
accusing its inventor of bad intentions.}

{
 But the Dark Side is out there. Fear is the path that leads to it,
and one betrayal can turn you. Not all who wear robes are either Jedi
or fakes; there are also the Sith Lords, masters and unwitting
apprentices. Be warned, be wary.}

{
 As for listing common memes that were spawned by the Dark
Side---not random false beliefs, mind you, but bad epistemology, the
Generic Defenses of Fail---well, would you care to take a stab at it,
dear readers?}

\myendsectiontext

\chapter{Against Doublethink}

\mysection{Singlethink}

{
 I remember the exact moment when I began my journey as a
rationalist. }

{
 It was not while reading \textit{Surely You're
Joking, Mr. Feynman} or any existing work upon rationality; for these I
simply accepted as obvious. The journey begins when you see a great
flaw in your existing art, and discover a drive to improve, to create
\textit{new} skills beyond the helpful but inadequate ones you found in
books.}

{
 In the last moments of my first life, I was fifteen years old, and
rehearsing a pleasantly self-righteous memory of a time when I was much
younger. My memories this far back are vague; I have a mental image,
but I don't remember how old I was exactly. I think I
was six or seven, and that the original event happened during summer
camp.}

{
 What happened originally was that a camp counselor, a teenage
male, got us much younger boys to form a line, and proposed the
following game: the boy at the end of the line would crawl through our
legs, and we would spank him as he went past, and then it would be the
turn of the next eight-year-old boy at the end of the line. (Maybe
it's just that I've lost my youthful
innocence, but I can't help but wonder \ldots) I refused
to play this game, and was told to go sit in the corner.}

{
 This memory---of refusing to spank and be spanked---came to
symbolize to me that even at this very early age I had refused to take
joy in hurting others. That I would not purchase a spank on
another's butt, at the price of a spank on my own;
would not pay in hurt for the opportunity to inflict hurt. I had
refused to play a negative-sum game.}

{
 And then, at the age of fifteen, I suddenly realized that it
wasn't true. I \textit{hadn't} refused
out of a principled stand against negative-sum games. I found out about
the Prisoner's Dilemma pretty early in life, but not at
the age of seven. I'd refused simply because I
didn't want to get hurt, and standing in the corner was
an acceptable price to pay for not getting hurt.}

{
 More importantly, I realized that I had \textit{always} known
this---that the real memory had \textit{always} been lurking in a
corner of my mind, my mental eye glancing at it for a fraction of a
second and then looking away.}

{
 In my very first step along the Way, \textit{I caught the
feeling---}generalized over the subjective experience---and said,
``So \textit{that's} what it feels
like to shove an unwanted truth into the corner of my mind! Now
I'm going to notice every time I do that, and clean out
\textit{all} my corners!''}

{
 This discipline I named \textit{singlethink}, after
Orwell's doublethink. In doublethink, you forget, and
then forget you have forgotten. In singlethink, you notice you are
forgetting, and then you remember. You hold only a single
non-contradictory thought in your mind at once.}

{
 ``Singlethink'' was the first
\textit{new} rationalist skill I created, which I had not read about in
books. I doubt that it is original in the sense of academic priority,
but this is thankfully not required.}

{
 Oh, and my fifteen-year-old self liked to name things.}

{
 The terrifying depths of the confirmation bias go on and on. Not
forever, for the brain is of finite complexity, but long enough that it
feels like forever. You keep on discovering (or reading about) new
mechanisms by which your brain shoves things out of the way.}

{
 But my young self swept out quite a few corners with that first
broom.}

\myendsectiontext

\mysection{Doublethink (Choosing to be Biased)}

{
 An oblong slip of newspaper had appeared between
O'Brien's fingers. For perhaps five
seconds it was within the angle of Winston's vision. It
was a photograph, and there was no question of its identity. It was the
photograph. It was another copy of the photograph of Jones, Aaronson,
and Rutherford at the party function in New York, which he had chanced
upon eleven years ago and promptly destroyed. For only an instant it
was before his eyes, then it was out of sight again. But he had seen
it, unquestionably he had seen it! He made a desperate, agonizing
effort to wrench the top half of his body free. It was impossible to
move so much as a centimetre in any direction. For the moment he had
even forgotten the dial. All he wanted was to hold the photograph in
his fingers again, or at least to see it.}

{
 ``It exists!'' he cried.}

{
 ``No,'' said
O'Brien.}

{
 He stepped across the room.}

{
 There was a memory hole in the opposite wall.
O'Brien lifted the grating. Unseen, the frail slip of
paper was whirling away on the current of warm air; it was vanishing in
a flash of flame. O'Brien turned away from the wall.}

{
 ``Ashes,'' he said.
``Not even identifiable ashes. Dust. It does not
exist. It never existed.''}

{
 ``But it did exist! It does exist! It exists in
memory. I remember it. You remember it.''}

{
 ``I do not remember it,'' said
O'Brien.}

{
 Winston's heart sank. That was doublethink. He had
a feeling of deadly helplessness. If he could have been certain that
O'Brien was lying, it would not have seemed to matter.
But it was perfectly possible that O'Brien had really
forgotten the photograph. And if so, then already he would have
forgotten his denial of remembering it, and forgotten the act of
forgetting. How could one be sure that it was simple trickery? Perhaps
that lunatic dislocation in the mind could really happen: that was the
thought that defeated him.}

{\raggedleft
 {}---George Orwell, \textit{1984}\textsuperscript{1}
\par}


\bigskip

{
 ~}

{
 What if self-deception helps us be happy? What if just running out
and overcoming bias will make us---gasp!---\textit{unhappy}? Surely,
\textit{true} wisdom would be \textit{second-order} rationality,
choosing when to be rational. That way you can decide which cognitive
biases should govern you, to maximize your happiness.}

{
 Leaving the morality aside, I doubt such a lunatic dislocation in
the mind could really happen.}

{
 Second-order rationality implies that at some point, you will
think to yourself, ``And now, I will irrationally
believe that I will win the lottery, in order to make myself
happy.'' But we do not have such direct control over
our beliefs. You cannot make yourself believe the sky is green by an
act of will. You might be able to believe you believed it---though I
have just made that more difficult for you by pointing out the
difference. (You're welcome!) You might even
\textit{believe} you were happy and self-deceived; but you would not
\textit{in fact} be happy and self-deceived.}

{
 For second-order rationality to be genuinely \textit{rational},
you would first need a good model of reality, to extrapolate the
consequences of rationality and irrationality. If you then chose to be
first-order irrational, you would need to forget this accurate view.
And then forget the act of forgetting. I don't mean to
commit the logical fallacy of generalizing from fictional evidence, but
I think Orwell did a good job of extrapolating where this path leads.}

{
 You can't know the consequences of being biased,
until you have already debiased yourself. And then it is too late for
self-deception.}

{
 The other alternative is to choose blindly to remain biased,
without any clear idea of the consequences. This is not second-order
rationality. It is willful stupidity.}

{
 Be irrationally optimistic about your driving skills, and you will
be happily unconcerned where others sweat and fear. You
won't have to put up with the inconvenience of a seat
belt. You will be happily unconcerned for a day, a week, a year. Then
\textit{crash}, and spend the rest of your life wishing you could
scratch the itch in your phantom limb. Or paralyzed from the neck down.
Or dead. It's not inevitable, but it's
possible; how probable is it? You can't make that
tradeoff rationally unless you know your \textit{real} driving skills,
so you can figure out how much danger you're placing
yourself in. You can't make that tradeoff rationally
unless you know about biases like neglect of probability.}

{
 No matter how many days go by in blissful ignorance, it only takes
a single mistake to undo a human life, to outweigh every penny you
picked up from the railroad tracks of stupidity.}

{
 One of the chief pieces of advice I give to aspiring rationalists
is ``Don't try to be
clever.'' And, ``Listen to those
quiet, nagging doubts.'' If you don't
know, you don't know \textit{what} you
don't know, you don't know how
\textit{much} you don't know, and you
don't know how much you \textit{needed} to know.}

{
 There is no second-order rationality. There is only a blind leap
into what may or may not be a flaming lava pit. Once you \textit{know},
it will be too late for blindness.}

{
 But people neglect this, because they do not know what they do not
know. Unknown unknowns are not available. They do not focus on the
blank area on the map, but treat it as if it corresponded to a blank
territory. When they consider leaping blindly, they check their memory
for dangers, and find no flaming lava pits in the blank map. Why not
leap?}

{
 Been there. Tried that. Got burned. Don't try to
be clever.}

{
 I once said to a friend that I suspected the happiness of
stupidity was greatly overrated. And she shook her head seriously, and
said, ``No, it's not;
it's really not.''}

{
 Maybe there are stupid happy people out there. Maybe they are
happier than you are. And life isn't fair, and you
won't become happier by being jealous of what you
can't have. I suspect the vast majority of
\textit{Overcoming Bias} readers could not achieve the
``happiness of stupidity'' if they
tried. That way is closed to you. You can never achieve that degree of
ignorance, you cannot forget what you know, you cannot unsee what you
see.}

{
 The happiness of stupidity is closed to you. You will never have
it short of actual brain damage, and maybe not even then. You should
wonder, I think, whether the happiness of stupidity is
\textit{optimal}{}---if it is the \textit{most} happiness that a human
can aspire to---but it matters not. That way is closed to you, if it
was ever open.}

{
 All that is left to you now, is to aspire to such happiness as a
rationalist can achieve. I think it may prove greater, in the end.
There are bounded paths and open-ended paths; plateaus on which to
laze, and mountains to climb; and if climbing takes more effort, still
the mountain rises higher in the end.}

{
 Also there is more to life than happiness; and other happinesses
than your own may be at stake in your decisions.}

{
 But that is moot. By the time you realize you have a choice, there
is no choice. You cannot unsee what you see. The other way is closed.}

\myendsectiontext


\bigskip

{
 1. Orwell, \textit{1984}.}

\mysection{No, Really, I've Deceived Myself}

{
 I recently spoke with a person who \ldots it's
difficult to describe. Nominally, she was an Orthodox Jew. She was also
highly intelligent, conversant with some of the archaeological evidence
against her religion, and the shallow standard arguments against
religion that religious people know about. For example, she knew that
Mordecai, Esther, Haman, and Vashti were not in the Persian historical
records, but that there was a corresponding old Persian legend about
the Babylonian gods Marduk and Ishtar, and the rival Elamite gods
Humman and Vashti. She \textit{knows} this, and she still celebrates
Purim. One of those highly intelligent religious people who stew in
their own contradictions for years, elaborating and tweaking, until the
insides of their minds look like an M. C. Escher painting. }

{
 Most people like this will pretend that they are much too wise to
talk to atheists, but she was willing to talk with me for a few hours.}

{
 As a result, I now understand at least one more thing about
self-deception that I didn't explicitly understand
before---namely, that you don't have to \textit{really}
deceive yourself so long as you \textit{believe} you've
deceived yourself. Call it ``belief in
self-deception.''}

{
 When this woman was in high school, she thought she was an
atheist. But she decided, at that time, that she should act as if she
believed in God. And then---she told me earnestly---over time, she came
to really believe in God.}

{
 So far as I can tell, she is completely wrong about that. Always
throughout our conversation, she said, over and over,
``I \textit{believe} in God,'' never
once, ``There \textit{is} a God.''
When I asked her why she was religious, she never once talked about the
consequences of God existing, only about the consequences of believing
in God. Never, ``God will help me,''
always, ``my belief in God helps
me.'' When I put to her, ``Someone
who just wanted the truth and looked at our universe would not even
invent God as a hypothesis,'' she agreed outright.}

{
 She hasn't \textit{actually} deceived herself into
believing that God exists or that the Jewish religion is true. Not even
close, so far as I can tell.}

{
 On the other hand, I think she really \textit{does} believe she
has deceived herself.}

{
 So although she does not receive any benefit of believing in
God---because she doesn't---she honestly
\textit{believes} she has deceived herself into believing in God, and
so she honestly \textit{expects} to receive the benefits that she
associates with deceiving oneself into believing in God; and
\textit{that}, I suppose, ought to produce much the same placebo effect
as \textit{actually} believing in God.}

{
 And this may explain why she was motivated to earnestly defend the
statement that she \textit{believed} in God from my skeptical
questioning, while never saying ``Oh, and by the way,
God actually does exist'' or even seeming the
slightest bit interested in the proposition.}

\myendsectiontext

\mysection{Belief in Self{}-Deception}

{
 I spoke of my conversation with a nominally Orthodox Jewish woman
who vigorously defended the assertion that she believed in God, while
seeming not to actually believe in God at all. }

{
 While I was questioning her about the benefits that she thought
came from believing in God, I introduced the Litany of Tarski---which
is actually an infinite family of litanies, a specific example being:}

{
 \textit{If the sky is blue}}

{
 \textit{I desire to believe ``the sky is
blue''}}

{
 \textit{If the sky is not blue}}

{
 \textit{I desire to believe ``the sky is not
blue.''}}

{
 ``This is not my philosophy,''
she said to me.}

{
 ``I didn't think it
was,'' I replied to her.
``I'm just asking---assuming that God
does \textit{not} exist, and this is known, then should you still
believe in God?''}

{
 She hesitated. She seemed to really be trying to think about it,
which surprised me.}

{
 ``So it's a counterfactual
question \ldots'' she said slowly.}

{
 I thought at the time that she was having difficulty allowing
herself to visualize the world where God does not exist, because of her
attachment to a God-containing world.}

{
 Now, however, I suspect she was having difficulty visualizing a
contrast between the way the \textit{world} would look if God existed
or did not exist, because all her thoughts were about her
\textit{belief in God}, but her causal network modelling the world did
not contain God as a node. So she could easily answer
``How would the world look different if I
didn't believe in God?,'' but not
``How would the world look different if there was no
God?''}

{
 She didn't answer that question, at the time. But
she did produce a \textit{counterexample} to the Litany of Tarski:}

{
 She said, ``I believe that people are nicer than
they really are.''}

{
 I tried to explain that if you say, ``People are
bad,'' that means you believe people are bad, and if
you say, ``I believe people are
nice,'' that means you believe you believe people are
nice. So saying ``People are bad and I believe people
are nice'' means you believe people are bad but you
believe you believe people are nice.}

{
 I quoted to her:}

{
 If there were a verb meaning ``to believe
falsely,'' it would not have any significant first
person, present indicative.}

{\raggedleft
 {}---Ludwig Wittgenstein\textsuperscript{1}
\par}


\bigskip

{
 She said, smiling, ``Yes, I believe people are
nicer than, in fact, they are. I just thought I should put it that way
for you.''}

{
 ``I reckon Granny ought to have a good look at
you, Walter,'' said Nanny. ``I
reckon your mind's all tangled up like a ball of string
what's been dropped.''}

{\raggedleft
 {}---Terry Pratchett, \textit{Maskerade}\textsuperscript{2}
\par}


\bigskip

{
 And I can type out the words, ``Well, I guess she
didn't believe that her reasoning ought to be
consistent under reflection,'' but
I'm still having trouble coming to grips with it.}

{
 I can see the pattern in the words coming out of her lips, but I
can't understand the mind behind on an empathic level.
I can imagine myself into the shoes of baby-eating aliens and the Lady
3rd Kiritsugu, but I cannot imagine what it is like to be her. Or maybe
I just don't \textit{want} to?}

{
 This is why intelligent people only have a certain amount of time
(measured in subjective time spent thinking about religion) to become
atheists. After a certain point, if you're smart, have
spent time thinking about and defending your religion, and still
haven't escaped the grip of Dark Side Epistemology, the
inside of your mind ends up as an Escher painting.}

{
 (One of the other few moments that gave her pause---I mention
this, in case you have occasion to use it---is when she was talking
about how it's good to believe that someone cares
whether you do right or wrong---\textit{not}, of course, talking about
how there actually \textit{is} a God who cares whether you do right or
wrong, this proposition is not part of her religion---}

{
 And I said, ``But \textit{I} care whether you do
right or wrong. So what you're saying is that this
isn't enough, and you also need to believe in something
\textit{above} humanity that cares whether you do right or
wrong.'' So that stopped her, for a bit, because of
course she'd never thought of it in those terms before.
Just a standard application of the nonstandard toolbox.)}

{
 Later on, at one point, I was asking her if it would be good to do
\textit{anything} differently if there definitely was no God, and this
time, she answered, ``No.''}

{
 ``So,'' I said incredulously,
``if God exists or doesn't exist, that
has absolutely no effect on how it would be good for people to think or
act? I think even a rabbi would look a little askance at
that.''}

{
 Her religion seems to now consist \textit{entirely} of the worship
of worship. As the true believers of older times might have believed
that an all-seeing father would save them, she now believes that belief
in God will save her.}

{
 After she said ``I believe people are nicer than
they are,'' I asked, ``So, are you
consistently surprised when people undershoot your
expectations?'' There was a long silence, and then,
slowly: ``Well \ldots am I \textit{surprised} when
people \ldots undershoot my expectations?''}

{
 I didn't understand this pause at the time.
I'd intended it to suggest that if she was constantly
disappointed by reality, then this was a downside of believing falsely.
But she seemed, instead, to be taken aback at the implications of
\textit{not} being surprised.}

{
 I now realize that the whole essence of her philosophy was
\textit{her belief that she had deceived herself,} and the possibility
that her estimates of other people were \textit{actually accurate},
threatened the Dark Side Epistemology that she had built around beliefs
such as ``I benefit from believing people are nicer
than they actually are.''}

{
 She has taken the old idol off its throne, and replaced it with an
explicit worship of the Dark Side Epistemology that was once invented
to defend the idol; she worships her own attempt at self-deception. The
attempt failed, but she is honestly unaware of this.}

{
 And so humanity's token guardians of sanity
(motto: ``pooping your deranged little party since
Epicurus'') must now fight the active worship of
self-deception---the worship \textit{of the supposed benefits of
faith}, in place of God.}

{
 This actually explains a fact about \textit{myself} that I
didn't really understand earlier---the reason why
I'm annoyed when people talk as if self-deception is
\textit{easy}, and why I write entire essays arguing that making a
deliberate choice to believe the sky is green is harder to get away
with than people seem to think.}

{
 It's because---while you
\textit{can't} just choose to believe the sky is
green---if you don't \textit{realize} this fact, then
you actually \textit{can} fool yourself into believing that
you've successfully deceived yourself.}

{
 And since you then sincerely \textit{expect} to receive the
benefits that you think come from self-deception, you get the same sort
of placebo benefit that would actually come from a successful
self-deception.}

{
 So by going around explaining how \textit{hard} self-deception is,
I'm actually taking direct aim at the placebo benefits
that people get from believing that they've deceived
themselves, and targeting the new sort of religion that worships only
the worship of God.}

{
 Will this battle, I wonder, generate a new list of reasons why,
not belief, but belief in belief, is \textit{itself} a good thing? Why
people derive great benefits from worshipping their worship? Will we
have to do this over again with belief in belief in belief and worship
of worship of worship? Or will intelligent theists finally just give up
on that line of argument?}

{
 I wish I could believe that no one could possibly believe in
belief in belief in belief, but the Zombie World argument in philosophy
has gotten even more tangled than this and its proponents still
haven't abandoned it.}

\myendsectiontext


\bigskip

{
 1. Ludwig Wittgenstein, \textit{Philosophical Investigations},
trans. Gertrude E. M. Anscombe (Oxford: Blackwell, 1953).}

{
 2. Terry Pratchett, \textit{Maskerade}, Discworld Series (ISIS,
1997).}

\mysection{Moore's Paradox}

{
 Moore's Paradox is the standard term for saying
``It's raining outside but I
don't believe that it is.'' Hat tip
to painquale on MetaFilter. }

{
 I think I understand Moore's Paradox a bit better
now, after reading some of the comments on \textit{Less Wrong}.
Jimrandomh suggests:}

{
 Many people cannot distinguish between levels of indirection. To
them, ``I believe X'' and
``X'' are the same thing, and
therefore, reasons why it is beneficial to believe X are also reasons
why X is true.}

{
 I don't think this is correct---relatively young
children can understand the concept of having a false belief, which
requires separate mental buckets for the map and the territory. But it
points in the direction of a similar idea:}

{
 Many people may not consciously distinguish between
\textit{believing} something and \textit{endorsing} it.}

{
 After all---``I believe in
democracy'' means, colloquially, that you endorse the
concept of democracy, not that you believe democracy exists. The word
``belief,'' then, has more than one
meaning. We could be looking at a confused word that causes confused
thinking (or maybe it just reflects pre-existing confusion).}

{
 So: in the original example, ``I believe people
are nicer than they are,'' she came up with some
reasons why it would be good to believe people are nice---health
benefits and such---and since she now had some warm affect on
``believing people are nice,'' she
introspected on this warm affect and concluded, ``I
believe people are nice.'' That is, she mistook the
\textit{positive affect} attached to the quoted belief, as signaling
\textit{her belief in the proposition}. At the same time, the world
itself seemed like people weren't so nice. So she said,
``I believe people are nicer than they
are.''}

{
 And that verges on being an honest mistake---sort of---since
people are not taught explicitly how to know when they believe
something. As in the parable of the dragon in the garage; the one who
says ``There is a dragon in my garage---but
it's invisible,'' does not recognize
their \textit{anticipation} of seeing no dragon, as indicating that
they possess an (accurate) model with no dragon in it.}

{
 It's not as if people are \textit{trained} to
recognize when they believe something. It's not like
they're ever taught in high school:
``What it feels like to actually believe
something---to have that statement in your belief pool---is that it
just seems like the way the world \textit{is.} You should recognize
this feeling, which is actual (unquoted) belief, and distinguish it
from having good feelings about a belief that you recognize as a belief
(which means that it's in quote
marks).''}

{
 This goes a long way toward making this real-life case of
Moore's Paradox seem less alien, and providing another
mechanism whereby people can be simultaneously right and wrong.}

{
 Likewise Kurige, who wrote:}

{
 I believe that there is a God---and that He has instilled a sense
of right and wrong in us by which we are able to evaluate the world
around us. I also believe a sense of morality has been evolutionarily
programmed into us---a sense of morality that is most likely a result
of the formation of meta-political coalitions in Bonobo communities a
very, very long time ago. These two beliefs are not contradictory, but
the complexity lies in reconciling the two.}

{
 I suspect, Kurige, you have decided that you have \textit{reasons
to endorse} the quoted belief that God has instilled a sense of right
and wrong in us. And also that you have reasons to endorse the verdict
of science. They both seem like good communities to join, right? There
are benefits to both sets of beliefs? You introspect and find that you
feel good about both beliefs?}

{
 But you did \textit{not} say:}

{
 ``God instilled a sense of right and wrong in us,
and also a sense of morality has been evolutionarily programmed into
us. The two states of reality are not inconsistent, but the complexity
lies in reconciling the two.''}

{
 If you're reading this, Kurige, you should very
quickly say the above out loud, so you can notice that it seems at
least slightly harder to swallow---notice the \textit{subjective
difference}{}---before you go to the trouble of rerationalizing.}

{
 This is the subjective difference between having reasons to
endorse two different beliefs, and your mental model of a single world,
a single way-things-are.}

\myendsectiontext

\mysection{Don't Believe You'll Self{}-Deceive}

{
 I don't mean to seem like I'm
picking on Kurige, but I think you have to expect a certain amount of
questioning if you show up on \textit{Less Wrong} and say:}

{
 One thing I've come to realize that helps to
explain the disparity I feel when I talk with most other Christians is
the fact that somewhere along the way my world-view took a major shift
away from blind faith and landed somewhere in the vicinity of Orwellian
double-think.}

{
 ``If you \textit{know} it's
double-think \ldots}

{
 \ldots how can you still \textit{believe} it?'' I
helplessly want to say.}

{
 Or:}

{
 I chose to believe in the existence of God---deliberately and
consciously. This decision, however, has absolutely zero effect on the
actual existence of God.}

{
 If you \textit{know} your belief isn't correlated
to reality, how can you still believe it?}

{
 Shouldn't the \textit{gut-level} realization,
``Oh, wait, the sky really
\textit{isn't} green'' follow from
the realization ``My map that says `the
sky is green' has no reason to be correlated with the
territory''?}

{
 Well \ldots apparently not.}

{
 One part of this puzzle may be my explanation of
Moore's Paradox
(``It's raining, but I
don't believe it is'')---that people
introspectively mistake positive affect attached to a quoted belief,
for actual credulity.}

{
 But another part of it may just be that---contrary to the
indignation I initially wanted to put forward---it's
actually quite \textit{easy} not to make the jump from
``The map that reflects the territory would say
`X''' to actually
believing ``X.'' It takes some work
to \textit{explain} the ideas of minds as map-territory correspondence
builders, and even then, it may take more work to get the implications
on a \textit{gut level}.}

{
 I realize now that when I wrote ``You cannot make
yourself believe the sky is green by an act of
will,'' I wasn't just a dispassionate
reporter of the existing facts. I was also trying to instill a
self-fulfilling prophecy.}

{
 It may be wise to go around deliberately repeating
``I can't get away with
double-thinking! Deep down, I'll know
it's not true! If I know my map has no reason to be
correlated with the territory, that means I don't
believe it!''}

{
 Because that way---if you're ever tempted to
try---the thoughts ``But I know this
isn't really true!'' and
``I can't fool
myself!'' will always rise readily to mind; and that
way, you will indeed be less likely to fool yourself successfully.
You're more likely to get, on a gut level, that telling
yourself X doesn't make X true: and therefore, really
truly not-X.}

{
 If you keep telling yourself that you
\textit{can't} just deliberately choose to believe the
sky is green---then you're less likely to succeed in
fooling yourself on one level or another; either in the sense of really
believing it, or of falling into Moore's Paradox,
belief in belief, or belief in self-deception.}

{
 If you keep telling yourself that deep down you'll
know---}

{
 If you keep telling yourself that you'd just look
at your elaborately constructed false map, and just know that it was a
false map without any expected correlation to the territory, and
therefore, despite all its elaborate construction, you
wouldn't be able to invest any credulity in it---}

{
 If you keep telling yourself that reflective consistency will take
over and make you stop believing on the object level, once you come to
the meta-level realization that the map is not reflecting---}

{
 Then when push comes to shove---you may, indeed, fail.}

{
 When it comes to deliberate self-deception, you must
\textit{believe in your own inability!}}

{
 Tell yourself the effort is doomed---\textit{and it will be!}}

{
 Is that the power of positive thinking, or the power of negative
thinking? Either way, it seems like a wise precaution.}

\myendsectiontext

\chapter{Seeing with Fresh Eyes}

\mysection{Anchoring and Adjustment}

{
 Suppose I spin a Wheel of Fortune device as you watch, and it
comes up pointing to 65. Then I ask: Do you think the percentage of
African countries in the UN is above or below this number? What do you
think is the percentage of African countries in the UN? Take a moment
to consider these two questions yourself, if you like, and please
don't Google. }

{
 Also, try to guess, within \textit{five seconds}, the value of the
following arithmetical expression. Five seconds. Ready? Set \ldots
\textit{Go!}}

{\centering
 1 {\texttimes} 2 {\texttimes} 3 {\texttimes} 4 {\texttimes} 5
{\texttimes} 6 {\texttimes} 7 {\texttimes} 8
\par}


\bigskip

{
 Tversky and Kahneman recorded the estimates of subjects who saw
the Wheel of Fortune showing various numbers.\textsuperscript{1} The
median estimate of subjects who saw the wheel show 65 was 45\%; the
median estimate of subjects who saw 10 was 25\%. }

{
 The current theory for this and similar experiments is that
subjects take the initial, uninformative number as their starting point
or \textit{anchor}; and then they \textit{adjust} upward or downward
from their starting estimate until they reached an answer that
``sounded plausible''; and then they
stopped adjusting. This typically results in under-adjustment from the
anchor---more distant numbers could also be
``plausible,'' but one stops at the
first satisfying-sounding answer.}

{
 Similarly, students shown ``1 {\texttimes} 2
{\texttimes} 3 {\texttimes} 4 {\texttimes} 5 {\texttimes} 6
{\texttimes} 7 {\texttimes} 8'' made a median
estimate of 512, while students shown ``8 {\texttimes}
7 {\texttimes} 6 {\texttimes} 5 {\texttimes} 4 {\texttimes} 3
{\texttimes} 2 {\texttimes} 1'' made a median
estimate of 2,250. The motivating hypothesis was that students would
try to multiply (or guess-combine) the first few factors of the
product, then adjust upward. In both cases the adjustments were
insufficient, relative to the true value of 40,320; but the first set
of guesses were much more insufficient because they started from a
lower anchor.}

{
 Tversky and Kahneman report that offering payoffs for accuracy did
not reduce the anchoring effect.}

{
 Strack and Mussweiler asked for the year Einstein first visited
the United States.\textsuperscript{2} Completely implausible anchors,
such as 1215 or 1992, produced anchoring effects just as large as more
plausible anchors such as 1905 or 1939.}

{
 There are obvious applications in, say, salary negotiations, or
buying a car. I won't suggest that you exploit it, but
watch out for exploiters.}

{
 And watch yourself thinking, and try to notice when you are
\textit{adjusting} a figure in search of an estimate.}

{
 Debiasing manipulations for anchoring have generally proved not
very effective. I would suggest these two: First, if the initial guess
sounds implausible, try to throw it away entirely and come up with a
new estimate, rather than sliding from the anchor. But this in itself
may not be sufficient---subjects instructed to avoid anchoring still
seem to do so.\textsuperscript{3} So, second, even if you are trying
the first method, try also to think of an anchor in the opposite
direction---an anchor that is clearly too small or too large, instead
of too large or too small---and dwell on it briefly.}

\myendsectiontext


\bigskip

{
 1. Amos Tversky and Daniel Kahneman, ``Judgment
Under Uncertainty: Heuristics and Biases,''
\textit{Science} 185, no. 4157 (1974): 1124--1131,
doi:10.1126/science.185.4157.1124.}

{
 2. Fritz Strack and Thomas Mussweiler,
``Explaining the Enigmatic Anchoring Effect:
Mechanisms of Selective Accessibility,''
\textit{Journal of Personality and Social Psychology} 73, no. 3 (1997):
437--446.}

{
 3. George A. Quattrone et al., ``Explorations in
Anchoring: The Effects of Prior Range, Anchor Extremity, and Suggestive
Hints'' (Unpublished manuscript, Stanford University,
1981).}

\mysection{Priming and Contamination}

{
 Suppose you ask subjects to press one button if a string of
letters forms a word, and another button if the string does not form a
word (e.g., ``banack'' vs.
``banner''). Then you show them the
string ``water.'' Later, they will
more quickly identify the string
``drink'' as a word. This is known
as ``cognitive priming''; this
particular form would be ``semantic
priming'' or ``conceptual
priming.'' }

{
 The fascinating thing about priming is that it occurs at such a
low level---priming speeds up \textit{identifying letters as forming a
word}, which one would expect to take place \textit{before} you
deliberate on the word's meaning.}

{
 Priming also reveals the massive parallelism of spreading
activation: if seeing ``water''
activates the word ``drink,'' it
probably also activates ``river,''
or ``cup,'' or
``splash'' \ldots and this activation
spreads, from the semantic linkage of concepts, all the way back to
recognizing strings of letters.}

{
 Priming is subconscious and unstoppable, an artifact of the human
neural architecture. Trying to stop yourself from priming is like
trying to stop the spreading activation of your own neural circuits.
Try to say aloud the color---not the meaning, but the color---of the
following letter-string:\newline
}

{\centering
 GREEN
\par}


\bigskip

{
 In Mussweiler and Strack's experiment, subjects
were asked an anchoring question: ``Is the annual mean
temperature in Germany higher or lower than 5 C / 20
C?''\textsuperscript{1} Afterward, on a
word-identification task, subjects presented with the 5 C anchor were
faster on identifying words like
``cold'' and
``snow,'' while subjects with the
high anchor were faster to identify
``hot'' and
``sun.'' This shows a non-adjustment
mechanism for anchoring: priming compatible thoughts and memories.}

{
 The more general result is that \textit{completely uninformative},
\textit{known false}, or \textit{totally irrelevant}
``information'' can influence
estimates and decisions. In the field of heuristics and biases, this
more general phenomenon is known as
\textit{contamination}.\textsuperscript{2}}

{
 Early research in heuristics and biases discovered anchoring
effects, such as subjects giving lower (higher) estimates of the
percentage of UN countries found within Africa, depending on whether
they were first asked if the percentage was more or less than 10 (65).
This effect was originally attributed to subjects adjusting from the
anchor as a starting point, stopping as soon as they reached a
plausible value, and under-adjusting because they were stopping at one
end of a confidence interval.\textsuperscript{3}}

{
 Tversky and Kahneman's early hypothesis still
appears to be the correct explanation in some circumstances, notably
when subjects generate the initial estimate
themselves.\textsuperscript{4} But modern research seems to show that
most anchoring is actually due to contamination, not sliding
adjustment. (Hat tip to Unnamed for reminding me of
this---I'd read the Epley and Gilovich paper years ago,
as a chapter in \textit{Heuristics and Biases}, but forgotten it.)}

{
 Your grocery store probably has annoying signs saying
``Limit 12 per customer'' or
``5 for \$10.'' Are these signs
effective at getting customers to buy in larger quantities? You
probably think you're not influenced. But
\textit{someone} must be, because these signs have been shown to work,
which is why stores keep putting them up.\textsuperscript{5}}

{
 Yet the most fearsome aspect of contamination is that it serves as
yet another of the thousand faces of confirmation bias. Once an idea
gets into your head, it primes information compatible with it---and
thereby ensures its continued existence. Never mind the selection
pressures for winning political arguments; confirmation bias is built
directly into our hardware, associational networks priming compatible
thoughts and memories. An unfortunate side effect of our existence as
neural creatures.}

{
 A single fleeting image can be enough to prime associated words
for recognition. Don't think it takes anything more to
set confirmation bias in motion. All it takes is that one quick flash,
and the bottom line is already decided, for we change our minds less
often than we think \ldots}

\myendsectiontext


\bigskip

{
 1. Thomas Mussweiler and Fritz Strack,
``Comparing Is Believing: A Selective Accessibility
Model of Judgmental Anchoring,'' \textit{European
Review of Social Psychology} 10 (1 1999): 135--167,
doi:10.1080/14792779943000044.}

{
 2. Gretchen B. Chapman and Eric J. Johnson,
``Incorporating the Irrelevant: Anchors in Judgments
of Belief and Value,'' in Gilovich, Griffin, and
Kahneman, \textit{Heuristics and Biases}, 120--138.}

{
 3. Tversky and Kahneman, ``Judgment Under
Uncertainty.''}

{
 4. Nicholas Epley and Thomas Gilovich, ``Putting
Adjustment Back in the Anchoring and Adjustment Heuristic: Differential
Processing of Self-Generated and Experimentor-Provided
Anchors,'' \textit{Psychological Science} 12 (5
2001): 391--396, doi:10.1111/1467-9280.00372.}

{
 5. Brian Wansink, Robert J. Kent, and Stephen J. Hoch,
``An Anchoring and Adjustment Model of Purchase
Quantity Decisions,'' \textit{Journal of Marketing
Research} 35, no. 1 (1998): 71--81,
http://www.jstor.org/stable/3151931.}

\mysection{Do We Believe Everything We're Told?}

{
 Some early experiments on anchoring and adjustment tested whether
\textit{distracting} the subjects---rendering subjects cognitively
``busy'' by asking them to keep a
lookout for ``5'' in strings of
numbers, or some such---would decrease adjustment, and hence increase
the influence of anchors. Most of the experiments seemed to bear out
the idea that cognitive busyness increased anchoring, and more
generally contamination. }

{
 Looking over the accumulating experimental results---more and more
findings of contamination, exacerbated by cognitive busyness---Daniel
Gilbert saw a truly crazy pattern emerging: Do we believe
\textit{everything} we're told?}

{
 One might naturally think that on being told a proposition, we
would first \textit{comprehend} what the proposition meant, then
\textit{consider} the proposition, and finally \textit{accept} or
\textit{reject} it. This obvious-seeming model of cognitive process
flow dates back to Descartes. But Descartes's rival,
Spinoza, disagreed; Spinoza suggested that we first \textit{passively
accept a proposition in the course of comprehending it}, and only
afterward \textit{actively disbelieve} propositions which are rejected
by consideration.}

{
 Over the last few centuries, philosophers pretty much went along
with Descartes, since his view seemed more, y'know,
logical and intuitive. But Gilbert saw a way of testing
Descartes's and Spinoza's hypotheses
experimentally.}

{
 If Descartes is right, then distracting subjects should interfere
with both accepting true statements and rejecting false statements. If
Spinoza is right, then distracting subjects should cause them to
remember false statements as being true, but should not cause them to
remember true statements as being false.}

{
 Gilbert, Krull, and Malone bear out this result, showing that,
among subjects presented with novel statements labeled TRUE or FALSE,
distraction had no effect on identifying true propositions (55\%
success for uninterrupted presentations, vs. 58\% when interrupted);
but did affect identifying false propositions (55\% success when
uninterrupted, vs. 35\% when interrupted).\textsuperscript{1}}

{
 A much more dramatic illustration was produced in followup
experiments by Gilbert, Tafarodi, and Malone.\textsuperscript{2}
Subjects read aloud crime reports crawling across a video monitor, in
which the color of the text indicated whether a particular statement
was true or false. Some reports contained false statements that
exacerbated the severity of the crime, other reports contained false
statements that extenuated (excused) the crime. Some subjects also had
to pay attention to strings of digits, looking for a
``5,'' while reading the crime
reports---this being the distraction task to create cognitive busyness.
Finally, subjects had to recommend the length of prison terms for each
criminal, from 0 to 20 years.}

{
 Subjects in the cognitively busy condition recommended an average
of 11.15 years in prison for criminals in the
``exacerbating'' condition, that is,
criminals whose reports contained labeled false statements exacerbating
the severity of the crime. Busy subjects recommended an average of 5.83
years in prison for criminals whose reports contained labeled false
statements excusing the crime. This nearly twofold difference was, as
you might suspect, statistically significant.}

{
 Non-busy participants read exactly the same reports, with the same
labels, and the same strings of numbers occasionally crawling past,
except that they did not have to search for the number
``5.'' Thus, they could devote more
attention to ``unbelieving''
statements labeled false. These non-busy participants recommended 7.03
years versus 6.03 years for criminals whose reports falsely exacerbated
or falsely excused.}

{
 Gilbert, Tafarodi, and Malone's paper was entitled
``You Can't Not Believe Everything You
Read.''}

{
 This suggests---to say the very least---that we should be more
careful when we expose ourselves to unreliable information, especially
if we're doing something else at the time. Be careful
when you glance at that newspaper in the supermarket.}

{
 PS: According to an unverified rumor I just made up, people will
be less skeptical of this essay because of the distracting color
changes.}

\myendsectiontext


\bigskip

{
 1. Daniel T. Gilbert, Douglas S. Krull, and Patrick S. Malone,
``Unbelieving the Unbelievable: Some Problems in the
Rejection of False Information,'' \textit{Journal of
Personality and Social Psychology} 59 (4 1990): 601--613,
doi:10.1037/0022-3514.59.4.601.}

{
 2. Gilbert, Tafarodi, and Malone, ``You
Can't Not Believe Everything You
Read.''}

\mysection{Cached Thoughts}

{
 One of the single greatest puzzles about the human brain is how
the damn thing works \textit{at all} when most neurons fire 10--20
times per second, or 200Hz tops. In neurology, the
``hundred-step rule'' is that any
postulated operation has to complete in \textit{at most} 100 sequential
steps---you can be as parallel as you like, but you
can't postulate more than 100 (preferably fewer) neural
spikes one after the other. }

{
 Can you imagine having to program using 100Hz CPUs, no matter how
many of them you had? You'd also need a hundred billion
processors just to get \textit{anything} done in realtime.}

{
 If you did need to write realtime programs for a hundred billion
100Hz processors, one trick you'd use as heavily as
possible is caching. That's when you store the results
of previous operations and look them up next time, instead of
recomputing them from scratch. And it's a very
\textit{neural} idiom---recognition, association, completing the
pattern.}

{
 It's a good guess that the actual
\textit{majority} of human cognition consists of cache lookups.}

{
 This thought does tend to go through my mind at certain times.}

{
 There was a wonderfully illustrative story which I thought I had
bookmarked, but couldn't re-find: it was the story of a
man whose know-it-all neighbor had once claimed in passing that the
best way to remove a chimney from your house was to knock out the
fireplace, wait for the bricks to drop down one level, knock out those
bricks, and repeat until the chimney was gone. Years later, when the
man wanted to remove his own chimney, this cached thought was lurking,
waiting to pounce \ldots}

{
 As the man noted afterward---you can guess it
didn't go well---his neighbor was not particularly
knowledgeable in these matters, not a trusted source. If
he'd \textit{questioned} the idea, he probably would
have realized it was a poor one. Some cache hits we'd
be better off recomputing. But the brain completes the pattern
automatically---and if you don't consciously realize
the pattern needs correction, you'll be left with a
completed pattern.}

{
 I suspect that if the thought had occurred to the man himself---if
he'd \textit{personally} had this bright idea for how
to remove a chimney---he would have examined the idea more critically.
But if someone \textit{else} has already thought an idea through, you
can save on computing power by caching their
\textit{conclusion}{}---right?}

{
 In modern civilization particularly, no one can think fast enough
to think their own thoughts. If I'd been abandoned in
the woods as an infant, raised by wolves or silent robots, I would
scarcely be recognizable as human. No one can think fast enough to
recapitulate the wisdom of a hunter-gatherer tribe in one lifetime,
starting from scratch. As for the wisdom of a literate civilization,
forget it.}

{
 But the flip side of this is that I continually see people who
aspire to critical thinking, repeating back cached thoughts which were
not invented by critical thinkers.}

{
 A good example is the skeptic who concedes,
``Well, you can't prove or disprove a
religion by factual evidence.'' As I have pointed out
elsewhere, this is simply false as probability theory. And it is also
simply false relative to the real psychology of religion---a few
centuries ago, saying this would have gotten you burned at the stake. A
mother whose daughter has cancer prays, ``God, please
heal my daughter,'' not, ``Dear God,
I know that religions are not allowed to have any falsifiable
consequences, which means that you can't possibly heal
my daughter, so \ldots well, basically, I'm praying to
make myself feel better, instead of doing something that could actually
help my daughter.''}

{
 But people read ``You can't prove
or disprove a religion by factual evidence,'' and
then, the next time they see a piece of evidence disproving a religion,
their brain completes the pattern. Even some atheists repeat this
absurdity without hesitation. If they'd thought of the
idea themselves, rather than hearing it from someone else, they would
have been more skeptical.}

{
 Death. Complete the pattern: ``Death gives
meaning to life.''}

{
 It's frustrating, talking to good and decent
folk---people who would never in a thousand years
\textit{spontaneously} think of wiping out the human species---raising
the topic of existential risk, and hearing them say,
``Well, maybe the human species
doesn't deserve to survive.'' They
would never in a thousand years shoot their own child, who is a part of
the human species, but the brain completes the pattern.}

{
 What patterns are being completed, inside your mind, that you
never chose to be there?}

{
 Rationality. Complete the pattern: ``Love
isn't rational.''}

{
 If this idea had suddenly occurred to you personally, as an
entirely new thought, how would you examine it critically? I know what
\textit{I} would say, but what would \textit{you}? It can be hard to
see with fresh eyes. Try to keep your mind from completing the pattern
in the standard, unsurprising, already-known way. It may be that there
is no better answer than the standard one, but you
can't \textit{think} about the answer until you can
stop your brain from filling in the answer automatically.}

{
 Now that you've read this, the next time you hear
someone unhesitatingly repeating a meme you think is silly or false,
you'll think, ``Cached
thoughts.'' My belief is now there in your mind,
waiting to complete the pattern. But is it true? Don't
let your mind complete the pattern! \textit{Think!}}

\myendsectiontext

\mysection{The ``Outside the Box'' Box}

{
 Whenever someone exhorts you to ``think outside
the box,'' they usually, \textit{for your
convenience,} point out exactly where ``outside the
box'' is located. Isn't it funny how
nonconformists all dress the same \ldots }

{
 In Artificial Intelligence, everyone outside the field has a
cached result for \textit{brilliant new revolutionary AI
idea}{}---neural networks, which work just like the human brain! New AI
idea. Complete the pattern: ``Logical AIs, despite all
the big promises, have failed to provide real intelligence for
decades---what we need are neural networks!''}

{
 This cached thought has been around for three decades. Still no
general intelligence. But, somehow, everyone outside the field knows
that neural networks are the Dominant-Paradigm-Overthrowing New Idea,
ever since backpropagation was invented in the 1970s. Talk about your
aging hippies.}

{
 Nonconformist images, by their nature, permit no departure from
the norm. If you don't wear black, how will people know
you're a tortured artist? How will people recognize
uniqueness if you don't fit the standard pattern for
what uniqueness is supposed to look like? How will anyone recognize
you've got a revolutionary AI concept, if
it's not about neural networks?}

{
 Another example of the same trope is
``subversive'' literature, all of
which sounds the same, backed up by a tiny defiant league of rebels who
control the entire English Department. As Anonymous asks on Scott
Aaronson's blog:}

{
 Has any of the subversive literature you've read
caused you to modify any of your political views?}

{
 Or as Lizard observes:}

{
 Revolution has already been televised. Revolution has been
\textit{merchandised}. Revolution is a commodity, a packaged lifestyle,
available at your local mall. \$19.95 gets you the black mask, the
spray can, the ``Crush the
Fascists'' protest sign, and access to your blog
where you can write about the police brutality you suffered when you
chained yourself to a fire hydrant. Capitalism has learned how to sell
anti-capitalism.}

{
 Many in Silicon Valley have observed that the vast majority of
venture capitalists at any given time are all chasing the same
Revolutionary Innovation, and it's the Revolutionary
Innovation that IPO'd six months ago. This is an
\textit{especially} crushing observation in venture capital, because
there's a direct economic motive to not follow the
herd---either someone else is also developing the product, or someone
else is bidding too much for the startup. Steve Jurvetson once told me
that at Draper Fisher Jurvetson, only two partners need to agree in
order to fund any startup up to \$1.5 million. And if \textit{all} the
partners agree that something sounds like a good idea, they
won't do it. If only grant committees were this sane.}

{
 The problem with originality is that you actually have to
\textit{think} in order to attain it, instead of letting your brain
complete the pattern. There is no conveniently labeled
``Outside the Box'' to which you can
immediately run off. There's an almost Zen-like quality
to it---like the way you can't teach satori in words
because satori is the experience of words failing you. The more you try
to follow the Zen Master's instructions in words, the
further you are from attaining an empty mind.}

{
 There is a reason, I think, why people do not attain novelty by
striving for it. Properties like truth or good design are independent
of novelty: 2 + 2 = 4, yes, really, even though this is what everyone
else thinks too. People who strive to discover truth or to invent good
designs, may in the course of time attain creativity. Not every change
is an improvement, but every improvement is a change.}

{
 Every improvement is a change, but not every change is an
improvement. The one who says ``I want to build an
original mousetrap!,'' and not ``I
want to build an optimal mousetrap!,'' nearly always
wishes to be \textit{perceived} as original.
``Originality'' in this sense is
inherently social, because it can only be determined by comparison to
other people. So their brain simply completes the standard pattern for
what is perceived as ``original,''
and their friends nod in agreement and say it is subversive.}

{
 Business books always tell you, for your convenience, where your
cheese has been moved to. Otherwise the readers would be left around
saying, ``Where is this `Outside the
Box' I'm supposed to
go?''}

{
 \textit{Actually thinking,} like satori, is a wordless act of
mind.}

{
 The eminent philosophers of Monty Python said it best of all in
\textit{Life of Brian}:\textsuperscript{1}}

{
 ``You've got to think for
yourselves! You're all
individuals!''}

{
 ``Yes, we're all
individuals!''}

{
 ``You're all
different!''}

{
 ``Yes, we're all
different!''}

{
 ``You've all got to work it out
for yourselves!''}

{
 ``Yes, we've got to work it out
for ourselves!''}

\myendsectiontext


\bigskip

{
 1. Graham Chapman et al., \textit{Monty Python's
The Life of Brian (of Nazareth)} (Eyre Methuen, 1979).}

\mysection{Original Seeing}

{
 Since Robert Pirsig put this very well, I'll just
copy down what he said. I don't know if this story is
based on reality or not, but either way, it's
true.\textsuperscript{1}}

{
 He'd been having trouble with students who had
nothing to say. At first he thought it was laziness but later it became
apparent that it wasn't. They just
couldn't think of anything to say.}

{
 One of them, a girl with strong-lensed glasses, wanted to write a
five-hundred word essay about the United States. He was used to the
sinking feeling that comes from statements like this, and suggested
without disparagement that she narrow it down to just Bozeman.}

{
 When the paper came due she didn't have it and was
quite upset. She had tried and tried but she just
couldn't think of anything to say.}

{
 It just stumped him. Now \textit{he} couldn't
think of anything to say. A silence occurred, and then a peculiar
answer: ``Narrow it down to the \textit{main street}
of Bozeman.'' It was a stroke of insight.}

{
 She nodded dutifully and went out. But just before her next class
she came back in \textit{real} distress, tears this time, distress that
had obviously been there for a long time. She still
couldn't think of anything to say, and
couldn't understand why, if she
couldn't think of anything about \textit{all} of
Bozeman, she should be able to think of something about just one
street.}

{
 He was furious. ``You're not
\textit{looking!}'' he said. A memory came back of
his own dismissal from the University for having \textit{too much} to
say. For every fact there is an \textit{infinity} of hypotheses. The
more you \textit{look} the more you \textit{see.} She really
wasn't looking and yet somehow didn't
understand this.}

{
 He told her angrily, ``Narrow it down to the
\textit{front} of \textit{one} building on the main street of Bozeman.
The Opera House. Start with the upper left-hand
brick.''}

{
 Her eyes, behind the thick-lensed glasses, opened wide.}

{
 She came in the next class with a puzzled look and handed him a
five-thousand-word essay on the front of the Opera House on the main
street of Bozeman, Montana. ``I sat in the hamburger
stand across the street,'' she said,
``and started writing about the first brick, and the
second brick, and then by the third brick it all started to come and I
couldn't stop. They thought I was crazy, and they kept
kidding me, but here it all is. I don't understand
it.''}

{
 Neither did he, but on long walks through the streets of town he
thought about it and concluded she was evidently stopped with the same
kind of blockage that had paralyzed him on his first day of teaching.
She was blocked because she was trying to repeat, in her writing,
things she had already heard, just as on the first day he had tried to
repeat things he had already decided to say. She
couldn't think of anything to write about Bozeman
because she couldn't recall anything she had heard
worth repeating. She was strangely unaware that she could look and see
freshly for herself, as she wrote, without primary regard for what had
been said before. The narrowing down to one brick destroyed the
blockage because it was so obvious she \textit{had} to do some original
and direct seeing.}

{\raggedleft
 {}---Robert M. Pirsig,\newline
 \textit{Zen and the Art of Motorcycle Maintenance}
\par}


\bigskip

\myendsectiontext


\bigskip

{
 1. Pirsig, \textit{Zen and the Art of Motorcycle Maintenance}.}

\mysection{Stranger than History}

{
 Suppose I told you that I knew for a \textit{fact} that the
following statements were true:}

{
 If you paint yourself a certain \textit{exact} color between blue
and green, it will reverse the force of gravity on you and cause you to
fall upward.}

{
 In the future, the sky will be filled by billions of floating
black spheres. Each sphere will be larger than all the zeppelins that
have ever existed put together. If you offer a sphere money, it will
lower a male prostitute out of the sky on a bungee cord.}

{
 Your grandchildren will think it is not just foolish, but
\textit{evil}, to put thieves in jail instead of spanking them.}

{
 You'd think I was crazy, right?}

{
 Now suppose it were the year 1901, and you had to choose between
believing those statements I have just offered, and believing
statements like the following:}

{
 There is an absolute speed limit on how fast two objects can seem
to be traveling relative to each other, which is exactly 670,616,629.2
miles per hour. If you hop on board a train going almost this fast and
fire a gun out the window, the fundamental units of length change
around, so it looks to \textit{you} like the bullet is speeding ahead
of you, but other people see something different. Oh, and time changes
around too.}

{
 In the future, there will be a superconnected global network of
billions of adding machines, each one of which has more power than all
pre-1901 adding machines put together. One of the primary uses of this
network will be to transport moving pictures of lesbian sex by
pretending they are made out of numbers.}

{
 Your grandchildren will think it is not just foolish, but
\textit{evil}, to say that someone should not be President of the
United States because she is black.}

{
 Based on a comment of Robin Hanson's:
\textit{``I wonder if one could describe in enough
detail a fictional story of an alternative reality, a reality that our
ancestors could not distinguish from the truth, in order to make it
very clear how surprising the truth turned out to
be.''}}

\myendsectiontext

\mysection{The Logical Fallacy of Generalization from Fictional Evidence}

{
 When I try to introduce the subject of advanced AI,
what's the first thing I hear, more than half the time?
}

{
 ``Oh, you mean like the \textit{Terminator}
movies / \textit{The Matrix} / Asimov's
robots!''}

{
 And I reply, ``Well, no, not exactly. I try to
avoid the logical fallacy of generalizing from fictional
evidence.''}

{
 Some people get it right away, and laugh. Others defend their use
of the example, disagreeing that it's a fallacy.}

{
 What's wrong with using movies or novels as
starting points for the discussion? No one's claiming
that it's \textit{true}, after all. Where is the lie,
where is the rationalist sin? Science fiction represents the
author's attempt to visualize the future; why not take
advantage of the thinking that's already been done on
our behalf, instead of starting over?}

{
 Not every misstep in the precise dance of rationality consists of
outright belief in a falsehood; there are subtler ways to go wrong.}

{
 First, let us dispose of the notion that science fiction
represents a full-fledged rational attempt to forecast the future. Even
the most diligent science fiction writers are, first and foremost,
storytellers; the requirements of storytelling are not the same as the
requirements of forecasting. As Nick Bostrom points
out:\textsuperscript{1}}

{
 When was the last time you saw a movie about humankind suddenly
going extinct (without warning and without being replaced by some other
civilization)? While this scenario may be much more probable than a
scenario in which human heroes successfully repel an invasion of
monsters or robot warriors, it wouldn't be much fun to
watch.}

{
 So there are specific distortions in fiction. But trying to
correct for these specific distortions is not enough. A story is
\textit{never} a rational attempt at analysis, not even with the most
diligent science fiction writers, because stories don't
use probability distributions. I illustrate as follows:}

{
 Bob Merkelthud slid cautiously through the door of the alien
spacecraft, glancing right and then left (or left and then right) to
see whether any of the dreaded Space Monsters yet remained. At his side
was the only weapon that had been found effective against the Space
Monsters, a Space Sword forged of pure titanium with 30\% probability,
an ordinary iron crowbar with 20\% probability, and a shimmering black
discus found in the smoking ruins of Stonehenge with 45\% probability,
the remaining 5\% being distributed over too many minor outcomes to
list here.}

{
 Merklethud (though there's a significant chance
that Susan Wifflefoofer was there instead) took two steps forward or
one step back, when a vast roar split the silence of the black airlock!
Or the quiet background hum of the white airlock! Although Amfer and
Woofi (1997) argue that Merklethud is devoured at this point,
Spacklebackle (2003) points out that---}

{
 Characters can be ignorant, but the \textit{author}
can't say the three magic words ``I
don't know.'' The protagonist must
thread a single line through the future, full of the details that lend
flesh to the story, from Wifflefoofer's appropriately
futuristic attitudes toward feminism, down to the color of her
earrings.}

{
 Then all these burdensome details and questionable assumptions are
wrapped up and given a short label, creating the illusion that they are
a single package.}

{
 On problems with large answer spaces, the greatest difficulty is
not \textit{verifying} the correct answer but simply locating it in
answer space to begin with. If someone starts out by asking whether or
not AIs are gonna put us into capsules like in \textit{The Matrix},
they're jumping to a 100-bit proposition, without a
corresponding 98 bits of evidence to locate it in the answer space as a
possibility worthy of explicit consideration. It would only take a
handful more evidence after the first 98 bits to promote that
possibility to near-certainty, which tells you something about where
nearly all the work gets done.}

{
 The ``preliminary'' step of
locating possibilities worthy of explicit consideration includes steps
like: Weighing what you know and don't know, what you
can and can't predict, making a deliberate effort to
avoid absurdity bias and widen confidence intervals, pondering which
questions are the important ones, trying to adjust for possible Black
Swans and think of (formerly) unknown unknowns. Jumping to
``\textit{The Matrix}: Yes or No?''
skips over all of this.}

{
 Any professional negotiator knows that to control the terms of a
debate is very nearly to control the outcome of the debate. If you
start out by thinking of \textit{The Matrix}, it brings to mind
marching robot armies defeating humans after a long struggle---not a
superintelligence snapping nanotechnological fingers. It focuses on an
``Us vs. Them'' struggle, directing
attention to questions like ``Who will
win?'' and ``Who should
win?'' and ``Will AIs really be like
that?'' It creates a general atmosphere of
entertainment, of ``What is your amazing vision of the
future?''}

{
 Lost to the echoing emptiness are: considerations of more than one
possible mind design that an ``Artificial
Intelligence'' could implement; the
future's dependence on initial conditions; the power of
smarter-than-human intelligence and the argument for its
unpredictability; people taking the whole matter seriously and trying
to do something about it.}

{
 If some insidious corrupter of debates decided that \textit{their}
preferred outcome would be best served by forcing discussants to start
out by refuting \textit{Terminator}, they would have done well in
skewing the frame. Debating gun control, the NRA spokesperson does not
wish to be introduced as a ``shooting
freak,'' the anti-gun opponent does not wish to be
introduced as a ``victim disarmament
advocate.'' Why should you allow the same order of
frame-skewing by Hollywood scriptwriters, even accidentally?}

{
 Journalists don't tell me, ``The
future will be like \textit{2001}.'' But they ask,
``Will the future be like \textit{2001}, or will it be
like \textit{A.I.}?'' This is just as huge a framing
issue as asking ``Should we cut benefits for disabled
veterans, or raise taxes on the rich?''}

{
 In the ancestral environment, there were no moving pictures; what
you saw with your own eyes was true. A momentary glimpse of a single
word can prime us and make compatible thoughts more available, with
demonstrated strong influence on probability estimates. How much havoc
do you think a two-hour movie can wreak on your judgment? It will be
hard enough to undo the damage by deliberate concentration---why invite
the vampire into your house? In Chess or Go, every wasted move is a
loss; in rationality, any non-evidential influence is (on average)
entropic.}

{
 Do movie-viewers succeed in unbelieving what they see? So far as I
can tell, few movie viewers act as if they have \textit{directly}
observed Earth's future. People who watched the
\textit{Terminator} movies didn't hide in fallout
shelters on August 29, 1997. But those who commit the fallacy seem to
act as if they had seen the movie events occurring on \textit{some
other} planet; not Earth, but somewhere similar to Earth.}

{
 You say, ``Suppose we build a very smart
AI,'' and they say, ``But
didn't that lead to nuclear war in \textit{The
Terminator}?'' As far as I can tell,
it's identical reasoning, down to the tone of voice, of
someone who might say: ``But didn't
that lead to nuclear war on Alpha Centauri?'' or
``Didn't that lead to the fall of the
Italian city-state of Piccolo in the fourteenth
century?'' The movie is not believed, but it is
available. It is treated, not as a prophecy, but as an illustrative
historical case. Will history repeat itself? Who knows?}

{
 In a recent intelligence explosion discussion, someone mentioned
that Vinge didn't seem to think that brain-computer
interfaces would increase intelligence much, and cited \textit{Marooned
in Realtime} and Tunç Blumenthal, who was the most advanced traveller
but didn't seem all that powerful. I replied
indignantly, ``But Tunç lost most of his hardware! He
was crippled!'' And then I did a mental double-take
and thought to myself: What the \textit{hell} am I saying.}

{
 Does the issue not have to be argued in its own right, regardless
of how Vinge depicted his characters? Tunç Blumenthal is not
``crippled,'' he's
\textit{unreal.} I could say ``Vinge chose to depict
Tunç as crippled, for reasons that may or may not have had anything to
do with his personal best forecast,'' and that would
give his authorial choice an appropriate weight of evidence. I cannot
say ``Tunç was crippled.'' There is
no \textit{was} of Tunç Blumenthal.}

{
 I deliberately left in a mistake I made, in my first draft of the
beginning of this essay: ``Others defend their use of
the \textit{example}, disagreeing that it's a
fallacy.'' But \textit{The Matrix} is \textit{not} an
example!}

{
 A neighboring flaw is the logical fallacy of arguing from
imaginary evidence: ``Well, if you \textit{did} go to
the end of the rainbow, you \textit{would} find a pot of gold---which
just proves my point!'' (Updating on evidence
predicted, but not observed, is the mathematical mirror image of
hindsight bias.)}

{
 The brain has many mechanisms for generalizing from observation,
not just the availability heuristic. You see three zebras, you form the
category ``zebra,'' and this
category embodies an automatic perceptual inference. Horse-shaped
creatures with white and black stripes are classified as
``Zebras,'' therefore they are fast
and good to eat; they are expected to be similar to other zebras
observed.}

{
 So people see (moving pictures of) three Borg, their brain
automatically creates the category
``Borg,'' and they infer
automatically that humans with brain-computer interfaces are of class
``Borg'' and will be similar to
other Borg observed: cold, uncompassionate, dressing in black leather,
walking with heavy mechanical steps. Journalists don't
believe that the future \textit{will} contain Borg---they
don't believe \textit{Star Trek} is a prophecy. But
when someone talks about brain-computer interfaces, they think,
``Will the future contain Borg?''
Not, ``How do I know computer-assisted telepathy makes
people less nice?'' Not,
``I've never seen a Borg and never has
anyone else.'' Not,
``I'm forming a racial stereotype
based on \textit{literally} zero evidence.''}

{
 As George Orwell said of cliches:\textsuperscript{2}}

{
 What is above all needed is to let the meaning choose the word,
and not the other way around \ldots When you think of something abstract
you are more inclined to use words from the start, and unless you make
a conscious effort to prevent it, the existing dialect will come
rushing in and do the job for you, at the expense of blurring or even
changing your meaning.}

{
 Yet in my estimation, the \textit{most} damaging aspect of using
other authors' imaginations is that it stops people
from using their own. As Robert Pirsig said:\textsuperscript{3}}

{
 She was blocked because she was trying to repeat, in her writing,
things she had already heard, just as on the first day he had tried to
repeat things he had already decided to say. She
couldn't think of anything to write about Bozeman
because she couldn't recall anything she had heard
worth repeating. She was strangely unaware that she could look and see
freshly for herself, as she wrote, without primary regard for what had
been said before.}

{
 Remembered fictions rush in and do your thinking for you; they
substitute for \textit{seeing}{}---the deadliest convenience of all.}

\myendsectiontext


\bigskip

{
 1. Nick Bostrom, ``Existential Risks: Analyzing
Human Extinction Scenarios and Related Hazards,''
\textit{Journal of Evolution and Technology} 9 (2002),
http://www.jetpress.org/volume9/risks.html.}

{
 2. Orwell, ``Politics and the English
Language.''}

{
 3. Pirsig, \textit{Zen and the Art of Motorcycle Maintenance}.}

\mysection{The Virtue of Narrowness}

{
 What is true of one apple may not be true of another apple; thus
more can be said about a single apple than about all the apples in the
world.}

{\raggedleft
 {}---The Twelve Virtues of Rationality
\par}


\bigskip

{
 ~}

{
 Within their own professions, people grasp the importance of
narrowness; a car mechanic knows the difference between a carburetor
and a radiator, and would not think of them both as
``car parts.'' A hunter-gatherer
knows the difference between a lion and a panther. A janitor does not
wipe the floor with window cleaner, even if the bottles look similar to
one who has not mastered the art.}

{
 Outside their own professions, people often commit the misstep of
trying to broaden a word as widely as possible, to cover as much
territory as possible. Is it not more glorious, more wise, more
impressive, to talk about \textit{all} the apples in the world? How
much loftier it must be to \textit{explain human thought in general},
without being distracted by smaller questions, such as how humans
invent techniques for solving a Rubik's Cube. Indeed,
it scarcely seems necessary to consider \textit{specific} questions at
all; isn't a general theory a worthy enough
accomplishment on its own?}

{
 It is the way of the curious to lift up one pebble from among a
million pebbles on the shore, and see something new about it, something
interesting, something different. You call these pebbles
``diamonds,'' and ask what might be
special about them---what inner qualities they might have in common,
beyond the glitter you first noticed. And then someone else comes along
and says: ``Why not call \textit{this} pebble a
diamond too? And this one, and this one?'' They are
enthusiastic, and they mean well. For it seems undemocratic and
exclusionary and elitist and unholistic to call some pebbles
``diamonds,'' and others not. It
seems \ldots \textit{narrow-minded \ldots} if you'll
pardon the phrase. Hardly \textit{open}, hardly \textit{embracing},
hardly \textit{communal.}}

{
 You might think it poetic, to give one word many meanings, and
thereby spread shades of connotation all around. But even poets, if
they are good poets, must learn to see the world precisely. It is not
enough to compare love to a flower. Hot jealous unconsummated love is
not the same as the love of a couple married for decades. If you need a
flower to symbolize jealous love, you must go into the garden, and
look, and make subtle distinctions---find a flower with a heady scent,
and a bright color, and thorns. Even if your intent is to shade
meanings and cast connotations, you must keep precise track of exactly
which meanings you shade and connote.}

{
 It is a necessary part of the rationalist's
art---or even the poet's art!---to focus narrowly on
unusual pebbles which possess some special quality. And look at the
details which those pebbles---and those pebbles alone!---share among
each other. This is not a sin.}

{
 It is perfectly all right for modern evolutionary biologists to
explain \textit{just} the patterns of living creatures, and not the
``evolution'' of stars or the
``evolution'' of technology. Alas,
some unfortunate souls use the same word
``evolution'' to cover the naturally
selected patterns of replicating life, \textit{and} the strictly
accidental structure of stars, \textit{and} the intelligently
configured structure of technology. And as we all know, if people use
the same word, it must all be the same thing. We should automatically
generalize anything we think we know about biological evolution to
technology. Anyone who tells us otherwise must be a mere pointless
pedant. It couldn't possibly be that our ignorance of
modern evolutionary theory is so total that we can't
tell the difference between a carburetor and a radiator.
That's unthinkable. No, the \textit{other} person---you
know, the one who's studied the math---is just too dumb
to see the connections.}

{
 And what could be more virtuous than seeing connections? Surely
the wisest of all human beings are the New Age gurus who say,
``Everything is connected to everything
else.'' If you ever say this aloud, you should pause,
so that everyone can absorb the sheer shock of this Deep Wisdom.}

{
 There is a trivial mapping between a graph and its complement. A
fully connected graph, with an edge between every two vertices, conveys
the same amount of information as a graph with no edges at all. The
important graphs are the ones where some things are \textit{not}
connected to some other things.}

{
 When the unenlightened ones try to be profound, they draw endless
verbal comparisons between this topic, and that topic, which is like
this, which is like that; until their graph is fully connected and also
totally useless. The remedy is specific knowledge and in-depth study.
When you understand things in detail, you can see how they are
\textit{not} alike, and start enthusiastically subtracting edges
\textit{off} your graph.}

{
 Likewise, the important categories are the ones that do not
contain everything in the universe. Good hypotheses can only explain
some possible outcomes, and not others.}

{
 It was perfectly all right for Isaac Newton to explain
\textit{just} gravity, \textit{just} the way things fall down---and how
planets orbit the Sun, and how the Moon generates the tides---but
\textit{not} the role of money in human society or how the heart pumps
blood. Sneering at narrowness is rather reminiscent of ancient Greeks
who thought that going out and actually \textit{looking} at things was
manual labor, and manual labor was for slaves.}

{
 As Plato put it in \textit{The Republic, Book
VII}:\textsuperscript{1}}

{
 If anyone should throw back his head and learn something by
staring at the varied patterns on a ceiling, apparently you would think
that he was contemplating with his reason, when he was only staring
with his eyes \ldots I cannot but believe that no study makes the soul
look on high except that which is concerned with real being and the
unseen. Whether he gape and stare upwards, or shut his mouth and stare
downwards, if it be things of the senses that he tries to learn
something about, I declare he never could learn, for none of these
things admit of knowledge: I say his soul is looking down, not up, even
if he is floating on his back on land or on sea!}

{
 Many today make a similar mistake, and think that narrow concepts
are as lowly and unlofty and unphilosophical as, say, going out and
looking at things---an endeavor only suited to the underclass. But
rationalists---and also poets---need narrow words to express precise
thoughts; they need categories that include only some things, and
exclude others. There's nothing wrong with focusing
your mind, narrowing your categories, excluding possibilities, and
sharpening your propositions. Really, there isn't! If
you make your words too broad, you end up with something that
isn't true and doesn't even make good
poetry.}

{
 \textit{And DON'T EVEN GET ME STARTED on people
who think Wikipedia is an ``Artificial
Intelligence,'' the invention of LSD was a
``Singularity,'' or that
corporations are
``superintelligent''!}}

\myendsectiontext


\bigskip

{
 1. Plato, \textit{Great Dialogues of Plato}, ed. Eric H.
Warmington and Philip G. Rouse (Signet Classic, 1999).}

\mysection{How to Seem (and Be) Deep}

{
 I recently attended a discussion group whose topic, at that
session, was Death. It brought out deep emotions. I think that of all
the Silicon Valley lunches I've ever attended, this one
was the most honest; people talked about the death of family, the death
of friends, what they thought about their own deaths. People really
listened to each other. I wish I knew how to reproduce those conditions
reliably. }

{
 I was the only transhumanist present, and I was extremely careful
not to be obnoxious about it. (``A fanatic is someone
who can't change his mind and won't
change the subject.'' I endeavor to at least be
capable of changing the subject.) Unsurprisingly, people talked about
the meaning that death gives to life, or how death is truly a blessing
in disguise. But I did, very cautiously, explain that transhumanists
are generally positive on life but thumbs down on death.}

{
 Afterward, several people came up to me and told me I was very
``deep.'' Well, yes, I am, but this
got me thinking about what makes people \textit{seem} deep.}

{
 At one point in the discussion, a woman said that thinking about
death led her to be nice to people because, who knows, she might not
see them again. ``When I have a nice thing to say
about someone,'' she said, ``now I
say it to them right away, instead of waiting.''}

{
 ``That is a beautiful
thought,'' I said, ``and even if
someday the threat of death is lifted from you, I hope you will keep on
doing it---''}

{
 Afterward, this woman was one of the people who told me I was
deep.}

{
 At another point in the discussion, a man spoke of some benefit X
of death, I don't recall exactly what. And I said:
``You know, given human nature, if people got hit on
the head by a baseball bat every week, pretty soon they would invent
reasons why getting hit on the head with a baseball bat was a good
thing. But if you took someone who wasn't being hit on
the head with a baseball bat, and you asked them if they wanted it,
they would say no. I think that if you took someone who was immortal,
and asked them if they wanted to die for benefit X, they would say
no.''}

{
 Afterward, this man told me I was deep.}

{
 Correlation is not causality. Maybe I was just speaking in a deep
voice that day, and so sounded wise.}

{
 But my suspicion is that I came across as
``deep'' because I coherently
violated the cached pattern for ``deep
wisdom'' in a way that made immediate sense.}

{
 There's a stereotype of Deep Wisdom. Death.
Complete the pattern: ``Death gives meaning to
life.'' Everyone knows this standard Deeply Wise
response. And so it takes on some of the characteristics of an applause
light. If you say it, people may nod along, because the brain completes
the pattern and they know they're supposed to nod. They
may even say ``What deep wisdom!,''
perhaps in the hope of being thought deep themselves. But they will not
be \textit{surprised}; they will not have heard anything outside the
box; they will not have heard anything they could not have thought of
for themselves. One might call it belief in wisdom---the thought is
labeled ``deeply wise,'' and
it's the completed standard pattern for
``deep wisdom,'' but it carries no
experience of insight.}

{
 People who \textit{try to seem} Deeply Wise often end up seeming
hollow, echoing as it were, because they're trying to
seem Deeply Wise instead of optimizing.}

{
 How much thinking did I need to do, in the course of seeming deep?
Human brains only run at 100Hz and I responded in realtime, so most of
the work must have been precomputed. The part I experienced as
effortful was picking a response understandable in one inferential step
and then phrasing it for maximum impact.}

{
 Philosophically, nearly all of my work was already done. Complete
the pattern: Existing condition X is really justified because it has
benefit Y. ``Naturalistic fallacy?''
/ ``Status quo bias?'' /
``Could we get Y without X?'' /
``If we had never even heard of X before, would we
voluntarily take it on to get Y?'' I think
it's fair to say that I execute these thought-patterns
at around the same level of automaticity as I breathe. After all, most
of human thought has to be cache lookups if the brain is to work at
all.}

{
 And I already held to the developed philosophy of transhumanism.
Transhumanism also has cached thoughts about death. Death. Complete the
pattern: ``Death is a pointless tragedy which people
rationalize.'' This was a nonstandard cache, one with
which my listeners were unfamiliar. I had several opportunities to use
nonstandard cache, and because they were all part of the developed
philosophy of transhumanism, they all visibly belonged to the same
theme. This made me seem \textit{coherent}, as well as original.}

{
 I suspect this is one reason Eastern philosophy seems deep to
Westerners---it has nonstandard but coherent cache for Deep Wisdom.
Symmetrically, in works of Japanese fiction, one sometimes finds
Christians depicted as repositories of deep wisdom and/or mystical
secrets. (And sometimes not.)}

{
 If I recall correctly, an economist once remarked that popular
audiences are so unfamiliar with standard economics that, when he was
called upon to make a television appearance, he just needed to repeat
back Econ 101 in order to sound like a brilliantly original thinker.}

{
 Also crucial was that my listeners could see \textit{immediately}
that my reply made sense. They might or might not have agreed with the
thought, but it was not a complete non-sequitur unto them. I know
transhumanists who are unable to seem deep because they are unable to
appreciate what their listener does not already know. If you want to
sound deep, you can never say anything that is more than a single step
of inferential distance away from your listener's
current mental state. That's just the way it is.}

{
 To \textit{seem} deep, study nonstandard philosophies. Seek out
discussions on topics that will give you a chance to appear deep. Do
your philosophical thinking in advance, so you can concentrate on
explaining well. Above all, practice staying within the
one-inferential-step bound.}

{
 To \textit{be} deep, think for yourself about
``wise'' or important or emotionally
fraught topics. Thinking for yourself isn't the same as
coming up with an unusual answer. It does mean seeing for yourself,
rather than letting your brain complete the pattern. If you
don't stop at the first answer, and cast out replies
that seem vaguely unsatisfactory, in time your thoughts will form a
coherent whole, flowing from the single source of yourself, rather than
being fragmentary repetitions of other people's
conclusions.}

\myendsectiontext

\mysection{We Change Our Minds Less Often Than We Think}

{
 Over the past few years, we have discreetly approached colleagues
faced with a choice between job offers, and asked them to estimate the
probability that they will choose one job over another. The average
confidence in the predicted choice was a modest 66\%, but only 1 of the
24 respondents chose the option to which he or she initially assigned a
lower probability, yielding an overall accuracy rate of 96\%.}

{\raggedleft
 {}---Dale Griffin and Amos Tversky\textsuperscript{1}
\par}


\bigskip

{
 ~}

{
 When I first read the words above---on August 1st, 2003, at around
3 o'clock in the afternoon---it changed the way I
thought. I realized that \textit{once I could guess what my answer
would be---}once I could assign a higher probability to deciding one
way than other---then I had, in all probability, already decided. We
change our minds less often than we think. And most of the time we
become able to guess what our answer will be within half a second of
hearing the question.}

{
 How swiftly that unnoticed moment passes, when we
can't yet guess what our answer will be; the tiny
window of opportunity for intelligence to act. In questions of choice,
as in questions of fact.}

{
 The principle of the bottom line is that only the actual causes of
your beliefs determine your effectiveness as a rationalist. Once your
belief is fixed, no amount of argument will alter the truth-value; once
your decision is fixed, no amount of argument will alter the
consequences.}

{
 You might think that you could arrive at a belief, or a decision,
by non-rational means, and then try to justify it, and if you found you
couldn't justify it, reject it.}

{
 But we change our minds less often---\textit{much} less
often---than we think.}

{
 I'm sure that you can think of at least one
occasion in your life when you've changed your mind. We
all can. How about all the occasions in your life when you
didn't change your mind? Are they as available, in your
heuristic estimate of your competence?}

{
 Between hindsight bias, fake causality, positive bias,
anchoring/priming, et cetera, et cetera, and above all the dreaded
confirmation bias, once an idea gets into your head,
it's probably going to stay there.}

\myendsectiontext


\bigskip

{
 1. Dale Griffin and Amos Tversky, ``The Weighing
of Evidence and the Determinants of Confidence,''
\textit{Cognitive Psychology} 24, no. 3 (1992): 411--435,
doi:10.1016/0010-0285(92)90013-R.}

\mysection{Hold Off On Proposing Solutions}

{
 From Robyn Dawes's \textit{Rational Choice in an
Uncertain World}.\textsuperscript{1} Bolding added.}

{
 Norman R. F. Maier noted that when a group faces a problem, the
natural tendency of its members is to propose possible solutions as
they begin to discuss the problem. Consequently, the group interaction
focuses on the merits and problems of the proposed solutions, people
become emotionally attached to the ones they have suggested, and
superior solutions are not suggested. Maier enacted an edict to enhance
group problem solving: \textbf{``Do not propose
solutions until the problem has been discussed as thoroughly as
possible without suggesting any.''} It is easy to
show that this edict works in contexts where there are objectively
defined good solutions to problems.}

{
 Maier devised the following ``role
playing'' experiment to demonstrate his point. Three
employees of differing ability work on an assembly line. They rotate
among three jobs that require different levels of ability, because the
most able---who is also the most dominant---is strongly motivated to
avoid boredom. In contrast, the least able worker, aware that he does
not perform the more difficult jobs as well as the other two, has
agreed to rotation because of the dominance of his able co-worker. An
``efficiency expert'' notes that if
the most able employee were given the most difficult task and the least
able the least difficult, productivity could be improved by 20\%, and
the expert recommends that the employees stop rotating. The three
employees and \ldots a fourth person designated to play the role of
foreman are asked to discuss the expert's
recommendation. Some role-playing groups are given
Maier's edict not to discuss solutions until having
discussed the problem thoroughly, while others are not. Those who are
not given the edict immediately begin to argue about the importance of
productivity versus worker autonomy and the avoidance of boredom.
Groups presented with the edict have a much higher probability of
arriving at the solution that the two more able workers rotate, while
the least able one sticks to the least demanding job---a solution that
yields a 19\% increase in productivity.}

{
 I have often used this edict with groups I have
led---\textbf{particularly when they face a very tough problem, which
is when group members are most apt to propose solutions immediately.}
While I have no objective criterion on which to judge the quality of
the problem solving of the groups, Maier's edict
appears to foster better solutions to problems.}

{
 This is so true it's not even funny. And it gets
worse and worse the tougher the problem becomes. Take Artificial
Intelligence, for example. A surprising number of people I meet seem to
know exactly how to build an Artificial General Intelligence, without,
say, knowing how to build an optical character recognizer or a
collaborative filtering system (much easier problems). And as for
building an AI with a positive impact on the world---a Friendly AI,
loosely speaking---why, \textit{that} problem is so incredibly
difficult that an actual \textit{majority} resolve the whole issue
within fifteen seconds. \textit{Give} me a \textit{break.}}

{
 This problem is by no means unique to AI. Physicists encounter
plenty of nonphysicists with their own theories of physics, economists
get to hear lots of amazing new theories of economics. If
you're an evolutionary biologist, anyone you meet can
instantly solve any open problem in your field, usually by postulating
group selection. Et cetera.}

{
 Maier's advice echoes the principle of the bottom
line, that the effectiveness of our decisions is determined only by
whatever evidence and processing we did in first arriving at our
decisions---after you write the bottom line, it is too late to write
more reasons above. If you make your decision very early on, it will,
in fact, be based on very little thought, no matter how many amazing
arguments you come up with afterward.}

{
 And consider furthermore that We Change Our Minds Less Often than
We Think: 24 people assigned an average 66\% probability to the future
choice thought more probable, but only 1 in 24 actually chose the
option thought less probable. \textbf{Once you can guess what your
answer will be, you have probably already decided.} If you can guess
your answer half a second after hearing the question, then you have
half a second in which to be intelligent. It's not a
lot of time.}

{
 Traditional Rationality emphasizes \textit{falsification---}the
ability to \textit{relinquish} an initial opinion when confronted by
clear evidence against it. But once an idea gets into your head, it
will probably require way too much evidence to get it out again. Worse,
we don't always have the luxury of overwhelming
evidence.}

{
 I suspect that a more powerful (and more difficult) method is to
\textit{hold off on thinking of an answer}. To suspend, draw out, that
tiny moment when we can't yet guess what our answer
will be; thus giving our intelligence a longer time in which to act.}

{
 Even half a minute would be an improvement over half a second.}

\myendsectiontext


\bigskip

{
 1. Dawes, \textit{Rational Choice in An Uncertain World}, 55--56.}

\mysection{The Genetic Fallacy}

{
 In lists of logical fallacies, you will find included
``the genetic fallacy''---the
fallacy of attacking a belief based on someone's causes
for believing it. }

{
 This is, at first sight, a very strange idea---if the causes of a
belief do not determine its systematic reliability, what does? If Deep
Blue advises us of a chess move, we trust it based on our understanding
of the \textit{code} that searches the game tree, being unable to
evaluate the actual game tree ourselves. What could license any
probability assignment as
``rational,'' except that it was
produced by some systematically reliable process?}

{
 Articles on the genetic fallacy will tell you that genetic
reasoning is not always a fallacy---that the origin of evidence
\textit{can} be relevant to its evaluation, as in the case of a trusted
expert. But other times, say the articles, it \textit{is} a fallacy;
the chemist Kekulé first saw the ring structure of benzene in a dream,
but this doesn't mean we can never trust this belief.}

{
 So sometimes the genetic fallacy is a fallacy, and sometimes
it's not?}

{
 The genetic fallacy is formally a fallacy, because the
\textit{original cause} of a belief is not the same as its
\textit{current justificational status}, the sum of all the support and
antisupport \textit{currently} known.}

{
 Yet we change our minds less often than we think. Genetic
accusations have a force among humans that they would not have among
ideal Bayesians.}

{
 Clearing your mind is a \textit{powerful heuristic} when
you're faced with new suspicion that many of your ideas
may have come from a flawed source.}

{
 Once an idea gets into our heads, it's not always
easy for evidence to root it out. Consider all the people out there who
grew up believing in the Bible; later came to reject (on a deliberate
level) the idea that the Bible was written by the hand of God; and who
nonetheless think that the Bible contains indispensable ethical wisdom.
They have failed to clear their minds; they could do significantly
better by doubting anything the Bible said \textit{because the Bible
said it}.}

{
 At the same time, they would have to bear firmly in mind the
principle that reversed stupidity is not intelligence; the goal is to
genuinely shake your mind loose and do independent thinking, not to
negate the Bible and let that be your algorithm.}

{
 Once an idea gets into your head, you tend to find support for it
everywhere you look---and so when the original source is suddenly cast
into suspicion, you would be very wise indeed to suspect all the leaves
that originally grew on that branch \ldots}

{
 If you can! It's not easy to clear your mind. It
takes a convulsive effort to \textit{actually reconsider}, instead of
letting your mind fall into the pattern of rehearsing cached arguments.
``It ain't a true crisis of faith
unless things could just as easily go either way,''
said Thor Shenkel.}

{
 You should be \textit{extremely suspicious} if you have many ideas
suggested by a source that you now know to be untrustworthy, but by
golly, it seems that all the ideas still ended up being right---the
Bible being the obvious archetypal example.}

{
 On the other hand \ldots there's such a thing as
sufficiently clear-cut evidence, that it no longer significantly
matters where the idea originally came from. Accumulating that kind of
clear-cut evidence is what Science is all about. It
doesn't matter any more that Kekulé first saw the ring
structure of benzene in a dream---it wouldn't matter if
we'd found the hypothesis to test by generating random
computer images, or from a spiritualist revealed as a fraud, or even
from the Bible. The ring structure of benzene is pinned down by enough
experimental evidence to make the source of the suggestion irrelevant.}

{
 In the absence of such clear-cut evidence, then you do need to pay
attention to the original sources of ideas---to give experts more
credence than layfolk, if their field has earned respect---to suspect
ideas you originally got from suspicious sources---to distrust those
whose motives are untrustworthy, \textit{if} they cannot present
arguments independent of their own authority.}

{
 The genetic fallacy is a \textit{fallacy} when there exist
justifications \textit{beyond} the genetic fact asserted, but the
genetic accusation is presented as if it settled the issue. Hal Finney
suggests that we call correctly appealing to a claim's
origins ``the genetic heuristic.''}

{
 Some good rules of thumb (for humans):}

{
 Be suspicious of genetic accusations against beliefs that you
dislike, especially if the proponent claims justifications beyond the
simple authority of a speaker. ``Flight is a religious
idea, so the Wright Brothers must be liars'' is one
of the classically given examples.}

{
 By the same token, don't think you can get good
information about a technical issue just by sagely psychoanalyzing the
personalities involved and their flawed motives. If technical arguments
exist, they get priority.}

{
 When new suspicion is cast on one of your fundamental sources, you
really \textit{should} doubt all the branches and leaves that grew from
that root. You are not licensed to reject them outright as conclusions,
because reversed stupidity is not intelligence, but \ldots}

{
 Be extremely suspicious if you find that you still believe the
early suggestions of a source you later rejected.}

\myendsectiontext

\chapter{Death Spirals}

\mysection{The Affect Heuristic}

{
 The \textit{affect heuristic} is when subjective impressions of
goodness/badness act as a heuristic---a source of fast, perceptual
judgments. Pleasant and unpleasant feelings are central to human
reasoning, and the affect heuristic comes with lovely biases---some of
my favorites. }

{
 Let's start with one of the relatively less crazy
biases. You're about to move to a new city, and you
have to ship an antique grandfather clock. In the first case, the
grandfather clock was a gift from your grandparents on your fifth
birthday. In the second case, the clock was a gift from a remote
relative and you have no special feelings for it. How much would you
pay for an insurance policy that paid out \$100 if the clock were lost
in shipping? According to Hsee and Kunreuther, subjects stated
willingness to pay more than twice as much in the first
condition.\textsuperscript{1} This may sound rational---why not pay
more to protect the more valuable object?---until you realize that the
insurance doesn't \textit{protect} the clock, it just
pays if the clock is lost, and pays exactly the same amount for either
clock. (And yes, it was stated that the insurance was with an outside
company, so it gives no special motive to the movers.)}

{
 All right, but that doesn't \textit{sound} too
insane. Maybe you could get away with claiming the subjects were
insuring affective outcomes, not financial outcomes---purchase of
consolation.}

{
 Then how about this? Yamagishi showed that subjects judged a
disease as more dangerous when it was described as killing 1,286 people
out of every 10,000, versus a disease that was 24.14\% likely to be
fatal.\textsuperscript{2} Apparently the mental image of a thousand
dead bodies is much more alarming, compared to a single person
who's more likely to survive than not.}

{
 But wait, it gets worse.}

{
 Suppose an airport must decide whether to spend money to purchase
some new equipment, while critics argue that the money should be spent
on other aspects of airport safety. Slovic et al. presented two groups
of subjects with the arguments for and against purchasing the
equipment, with a response scale ranging from 0 (would not support at
all) to 20 (very strong support).\textsuperscript{3} One group saw the
measure described as saving 150 lives. The other group saw the measure
described as saving 98\% of 150 lives. The hypothesis motivating the
experiment was that saving 150 lives sounds vaguely good---is that a
lot? a little?---while saving 98\% of something is clearly very good
because 98\% is so close to the upper bound of the percentage scale. Lo
and behold, saving 150 lives had mean support of 10.4, while saving
98\% of 150 lives had mean support of 13.6.}

{
 Or consider the report of Denes-Raj and
Epstein:\textsuperscript{4} Subjects offered an opportunity to win \$1
each time they randomly drew a red jelly bean from a bowl, often
preferred to draw from a bowl with more red beans and a smaller
proportion of red beans. E.g., 7 in 100 was preferred to 1 in 10.}

{
 According to Denes-Raj and Epstein, these subjects reported
afterward that even though they knew the probabilities were against
them, they felt they had a better chance when there were more red
beans. This may sound crazy to you, oh Statistically Sophisticated
Reader, but if you think more carefully you'll realize
that it makes perfect sense. A 7\% probability versus 10\% probability
may be bad news, but it's more than made up for by the
increased number of red beans. It's a worse
probability, yes, but you're still more likely to
\textit{win}, you see. You should meditate upon this thought until you
attain enlightenment as to how the rest of the planet thinks about
probability.}

{
 Finucane et al. found that for nuclear reactors, natural gas, and
food preservatives, presenting information about high benefits made
people perceive lower risks; presenting information about higher risks
made people perceive lower benefits; and so on across the
quadrants.\textsuperscript{5} People conflate their judgments about
particular good/bad aspects of something into an overall good or bad
feeling about that thing.}

{
 Finucane et al. also found that time pressure greatly
\textit{increased} the inverse relationship between perceived risk and
perceived benefit, consistent with the general finding that time
pressure, poor information, or distraction all increase the dominance
of perceptual heuristics over analytic deliberation.}

{
 Ganzach found the same effect in the realm of
finance.\textsuperscript{6} According to ordinary economic theory,
return and risk should correlate \textit{positively}{}---or to put it
another way, people pay a premium price for safe investments, which
lowers the return; stocks deliver higher returns than bonds, but have
correspondingly greater risk. When judging \textit{familiar} stocks,
analysts' judgments of risks and returns were
positively correlated, as conventionally predicted. But when judging
\textit{unfamiliar} stocks, analysts tended to judge the stocks as if
they were generally good or generally bad---low risk and high returns,
or high risk and low returns.}

{
 For further reading I recommend Slovic's fine
summary article, ``Rational Actors or Rational Fools:
Implications of the Affect Heuristic for Behavioral
Economics.''\textsuperscript{7}}

\myendsectiontext


\bigskip

{
 1. Christopher K. Hsee and Howard C. Kunreuther,
``The Affection Effect in Insurance
Decisions,'' \textit{Journal of Risk and Uncertainty}
20 (2 2000): 141--159, doi:10.1023/A:1007876907268.}

{
 2. Kimihiko Yamagishi, ``When a 12.86\% Mortality
Is More Dangerous than 24.14\%: Implications for Risk
Communication,'' \textit{Applied Cognitive
Psychology} 11 (6 1997): 461--554.}

{
 3. Paul Slovic et al., ``Rational Actors or
Rational Fools: Implications of the Affect Heuristic for Behavioral
Economics,'' \textit{Journal of Socio-Economics} 31,
no. 4 (2002): 329--342, doi:10.1016/S1053-5357(02)00174-9.}

{
 4. Veronika Denes-Raj and Seymour Epstein,
``Conflict between Intuitive and Rational Processing:
When People Behave against Their Better Judgment,''
\textit{Journal of Personality and Social Psychology} 66 (5 1994):
819--829, doi:10.1037/0022-3514.66.5.819.}

{
 5. Finucane et al., ``The Affect Heuristic in
Judgments of Risks and Benefits.''}

{
 6. Yoav Ganzach, ``Judging Risk and Return of
Financial Assets,'' \textit{Organizational Behavior
and Human Decision Processes} 83, no. 2 (2000): 353--370,
doi:10.1006/obhd.2000.2914.}

{
 7. Slovic et al., ``Rational Actors or Rational
Fools.''}

\mysection{Evaluability (and Cheap Holiday Shopping)}

{
 With the \textit{expensive} part of the Hallowthankmas season now
approaching, a question must be looming large in our
readers' minds:}

{
 ``Dear \textit{Overcoming Bias}, are there biases
I can exploit to be \textit{seen} as generous without \textit{actually}
spending lots of money?''}

{
 I'm glad to report the answer is yes! According to
Hsee---in a paper entitled ``Less is better: When
low-value options are valued more highly than high-value
options''---if you buy someone a \$45 scarf, you are
more likely to be seen as generous than if you buy them a \$55
coat.\textsuperscript{1}}

{
 This is a special case of a more general phenomenon. In an earlier
experiment, Hsee asked subjects how much they would be willing to pay
for a second-hand music dictionary:\textsuperscript{2}}

{
 Dictionary A, from 1993, with 10,000 entries, in like-new
condition.}

{
 Dictionary B, from 1993, with 20,000 entries, with a torn cover
and otherwise in like-new condition.}

{
 The gotcha was that some subjects saw both dictionaries
side-by-side, while other subjects only saw \textit{one} dictionary
\ldots}

{
 Subjects who saw only \textit{one} of these options were willing
to pay an average of \$24 for Dictionary A and an average of \$20 for
Dictionary B. Subjects who saw \textit{both} options, side-by-side,
were willing to pay \$27 for Dictionary B and \$19 for Dictionary A.}

{
 Of course, the number of entries in a dictionary is more important
than whether it has a torn cover, at least if you ever plan on using it
for anything. But if you're only presented with a
single dictionary, and it has 20,000 entries, the number 20,000
doesn't mean very much. Is it a little? A lot? Who
knows? It's \textit{non-evaluable}. The torn cover, on
the other hand---that stands out. That has a definite affective
valence: namely, bad.}

{
 Seen side-by-side, though, the number of entries goes from
\textit{non-evaluable} to \textit{evaluable}, because there are two
compatible quantities to be compared. And, once the number of entries
becomes evaluable, that facet swamps the importance of the torn cover.}

{
 From Slovic et al.: Which would you prefer?\textsuperscript{3}}

{
 A 29/36 chance to win \$2.}

{
 A 7/36 chance to win \$9.}

{
 While the average \textit{prices} (equivalence values) placed on
these options were \$1.25 and \$2.11 respectively, their mean
attractiveness ratings were 13.2 and 7.5. Both the prices and the
attractiveness rating were elicited in a context where subjects were
told that two gambles would be randomly selected from those rated, and
they would play the gamble with the higher price or higher
attractiveness rating. (Subjects had a motive to rate gambles as more
attractive, or price them higher, that they would actually prefer to
play.)}

{
 The gamble worth more money seemed less attractive, a classic
preference reversal. The researchers hypothesized that the dollar
values were more compatible with the pricing task, but the probability
of payoff was more compatible with attractiveness. So (the researchers
thought) why not try to make the gamble's payoff more
emotionally salient---more affectively evaluable---more attractive?}

{
 And how did they do this? By adding a very small loss to the
gamble. The old gamble had a 7/36 chance of winning \$9. The new gamble
had a 7/36 chance of winning \$9 and a 29/36 chance of losing 5 cents.
In the old gamble, you implicitly evaluate the attractiveness of \$9.
The new gamble gets you to evaluate the attractiveness of winning \$9
\textit{versus} losing 5 cents.}

{
 ``The results,'' said Slovic et
al., ``exceeded our expectations.''
In a new experiment, the simple gamble with a 7/36 chance of winning
\$9 had a mean attractiveness rating of 9.4, while the complex gamble
that included a 29/36 chance of losing 5 cents had a mean
attractiveness rating of 14.9.}

{
 A follow-up experiment tested whether subjects preferred the old
gamble to a certain gain of \$2. Only 33\% of students preferred the
old gamble. Among another group asked to choose between a certain \$2
and the new gamble (with the added possibility of a 5 cents loss),
fully 60.8\% preferred the gamble. After all, \$9 isn't
a very attractive amount of money, but \$9 / 5 cents is an
\textit{amazingly} attractive win/loss ratio.}

{
 You can make a gamble more attractive by adding a strict loss!
Isn't psychology fun? This is why no one who truly
appreciates the wondrous intricacy of human intelligence wants to
design a human-like AI.}

{
 Of course, it only works if the subjects don't see
the two gambles side-by-side.}

{
 Similarly, which of these two ice creams do you think subjects in
Hsee's 1998 study preferred?}

{
 ~}


{\centering
\mygraphics{Rationality20From20AI20to20Zombies2020Eliezer20Yudkowsky-img128.jpg}
 \newline
 From Hsee, © 1998 John Wiley \& Sons, Ltd.
\par}


\bigskip

{
 ~}

{
 Naturally, the answer depends on whether the subjects saw a single
ice cream, or the two side-by-side. Subjects who saw a single ice cream
were willing to pay \$1.66 to Vendor H and \$2.26 to Vendor L. Subjects
who saw both ice creams were willing to pay \$1.85 to Vendor H and
\$1.56 to Vendor L.}

{
 What does this suggest for your holiday shopping? That if you
spend \$400 on a 16GB iPod Touch, your recipient sees the most
expensive MP3 player. If you spend \$400 on a Nintendo Wii, your
recipient sees the least expensive game machine. Which is better value
for the money? Ah, but that question only makes sense if you see the
two side-by-side. \textit{You'll} think about them
side-by-side while you're shopping, but the recipient
will only see what they get.}

{
 If you have a fixed amount of money to spend---and your goal is to
display your friendship, rather than to actually \textit{help} the
recipient---you'll be better off deliberately not
shopping for value. Decide how much money you want to spend on
impressing the recipient, then find the most worthless object which
costs that amount. The cheaper the \textit{class} of objects, the more
expensive a \textit{particular} object will appear, given that you
spend a fixed amount. Which is more memorable, a \$25 shirt or a \$25
candle?}

{
 Gives a whole new meaning to the Japanese custom of buying \$50
melons, doesn't it? You look at that and shake your
head and say ``What \textit{is} it with the
Japanese?'' And yet they get to be perceived as
incredibly generous, spendthrift even, while spending only \$50. You
could spend \$200 on a fancy dinner and not appear as wealthy as you
can by spending \$50 on a melon. If only there was a custom of gifting
\$25 toothpicks or \$10 dust specks; they could get away with spending
even less.}

{
 PS: If you actually use this trick, I want to know what you
bought.}

\myendsectiontext


\bigskip

{
 1. Christopher K. Hsee, ``Less Is Better: When
Low-Value Options Are Valued More Highly than High-Value
Options,'' \textit{Behavioral Decision Making} 11 (2
1998): 107--121.}

{
 2. Christopher K. Hsee, ``The Evaluability
Hypothesis: An Explanation for Preference Reversals between Joint and
Separate Evaluations of Alternatives,''
\textit{Organizational Behavior and Human Decision Processes} 67 (3
1996): 247--257, doi:10.1006/obhd.1996.0077.}

{
 3. Slovic et al., ``Rational Actors or Rational
Fools.''}

\mysection{Unbounded Scales, Huge Jury Awards, and Futurism}

{
 ``Psychophysics,'' despite the
name, is the respectable field that links physical effects to sensory
effects. If you dump acoustic energy into air---make noise---then
\textit{how loud} does that sound to a person, as a function of
acoustic energy? How much more acoustic energy do you have to pump into
the air, before the noise sounds twice as loud to a human listener?
It's not twice as much; more like eight times as much.
}

{
 Acoustic energy and photons are straightforward to measure. When
you want to find out how loud an acoustic stimulus \textit{sounds}, how
bright a light source \textit{appears}, you usually ask the listener or
watcher. This can be done using a bounded scale from
``very quiet'' to
``very loud,'' or
``very dim'' to
``very bright.'' You can also use an
unbounded scale, whose zero is ``not audible at
all'' or ``not visible at
all,'' but which increases from there without limit.
When you use an unbounded scale, the observer is typically presented
with a constant stimulus, the \textit{modulus}, which is given a fixed
rating. For example, a sound that is assigned a loudness of 10. Then
the observer can indicate a sound twice as loud as the modulus by
writing 20.}

{
 And this has proven to be a fairly reliable technique. But what
happens if you give subjects an unbounded scale, but no modulus? Zero
to infinity, with no reference point for a fixed value? Then they make
up their own modulus, of course. The \textit{ratios} between stimuli
will continue to correlate reliably between subjects. Subject A says
that sound X has a loudness of 10 and sound Y has a loudness of 15. If
subject B says that sound X has a loudness of 100, then
it's a good guess that subject B will assign loudness
in the vicinity of 150 to sound Y. But if you don't
know what subject C is using as their modulus---their scaling
factor---then there's no way to guess what subject C
will say for sound X. It could be 1. It could be 1,000.}

{
 For a subject rating a \textit{single} sound, on an
\textit{unbounded} scale, \textit{without} a fixed standard of
comparison, nearly \textit{all} the variance is due to the arbitrary
choice of modulus, rather than the sound itself.}

{
 ``Hm,'' you think to yourself,
``this sounds an awful lot like juries deliberating on
punitive damages. No wonder there's so much
variance!'' An interesting analogy, but how would you
go about demonstrating it experimentally?}

{
 Kahneman et al. presented 867 jury-eligible subjects with
descriptions of legal cases (e.g., a child whose clothes caught on
fire) and asked them to either}

{
 Rate the outrageousness of the defendant's
actions, on a bounded scale,}

{
 Rate the degree to which the defendant should be punished, on a
bounded scale, or}

{
 Assign a dollar value to punitive damages.\textsuperscript{1}}

{
 And, lo and behold, while subjects correlated very well with each
other in their outrage ratings and their punishment ratings, their
punitive damages were all over the map. Yet subjects'
\textit{rank-ordering} of the punitive damages---their ordering from
lowest award to highest award---correlated well across subjects.}

{
 If you asked how much of the variance in the
``punishment'' scale could be
explained by the specific scenario---the particular legal case, as
presented to multiple subjects---then the answer, even for the raw
scores, was 0.49. For the \textit{rank orders} of the dollar responses,
the amount of variance predicted was 0.51. For the \textit{raw dollar}
amounts, the variance explained was 0.06!}

{
 Which is to say: if you knew the scenario presented---the
aforementioned child whose clothes caught on fire---you could take a
good guess at the punishment rating, and a good guess at the
\textit{rank-ordering} of the dollar award relative to other cases, but
the dollar award itself would be completely unpredictable.}

{
 Taking the median of twelve randomly selected responses
didn't help much either.}

{
 So a jury award for punitive damages isn't so much
an economic valuation as an attitude expression---a psychophysical
measure of outrage, expressed on an unbounded scale with no standard
modulus.}

{
 I observe that many \textit{futuristic predictions} are, likewise,
best considered as attitude expressions. Take the question,
``How long will it be until we have human-level
AI?'' The responses I've seen to this
are all over the map. On one memorable occasion, a mainstream AI guy
said to me, ``Five hundred years.''
(!!)}

{
 Now the reason why time-to-AI is just \textit{not very
predictable}, is a long discussion in its own right. But
it's not as if the guy who said ``Five
hundred years'' was looking into the future to find
out. And he can't have gotten the number using the
standard bogus method with Moore's Law. So what did the
number 500 \textit{mean}?}

{
 As far as I can guess, it's as if
I'd asked, ``On a scale where zero is
`not difficult at all,' how difficult
does the AI problem \textit{feel} to you?'' If this
were a bounded scale, every sane respondent would mark
``extremely hard'' at the right-hand
end. Everything \textit{feels} extremely hard when you
don't know how to do it. But instead
there's an unbounded scale with no standard modulus. So
people just make up a number to represent ``extremely
difficult,'' which may come out as 50, 100, or even
500. Then they tack ``years'' on the
end, and that's their futuristic prediction.}

{
 ``How hard does the AI problem
feel?'' isn't the only substitutable
question. Others respond as if I'd asked
``How positive do you feel about
AI?,'' except lower numbers mean more positive
feelings, and then they also tack
``years'' on the end. But if these
``time estimates'' represent
anything other than attitude expressions on an unbounded scale with no
modulus, I have been unable to determine it.}

\myendsectiontext


\bigskip

{
 1. Daniel Kahneman, David A. Schkade, and Cass R. Sunstein,
``Shared Outrage and Erratic Awards: The Psychology of
Punitive Damages,'' \textit{Journal of Risk and
Uncertainty} 16 (1 1998): 48--86, doi:10.1023/A:1007710408413; Daniel
Kahneman, Ilana Ritov, and David Schkade, ``Economic
Preferences or Attitude Expressions?: An Analysis of Dollar Responses
to Public Issues,'' \textit{Journal of Risk and
Uncertainty} 19, nos. 1--3 (1999): 203--235,
doi:10.1023/A:1007835629236.}

\mysection{The Halo Effect}

{
 The affect heuristic is how an overall feeling of goodness or
badness contributes to many other judgments, whether
it's logical or not, whether you're
aware of it or not. Subjects told about the benefits of nuclear power
are likely to rate it as having fewer risks; stock analysts rating
unfamiliar stocks judge them as generally good or generally bad---low
risk and high returns, or high risk and low returns---in defiance of
ordinary economic theory, which says that risk and return should
correlate positively. }

{
 The halo effect is the manifestation of the affect heuristic in
social psychology. Robert Cialdini, in \textit{Influence: Science and
Practice},\textsuperscript{1} summarizes:}

{
 Research has shown that we automatically assign to good-looking
individuals such favorable traits as talent, kindness, honesty, and
intelligence (for a review of this evidence, see Eagly, Ashmore,
Makhijani, and Longo, 1991).\textsuperscript{2} Furthermore, we make
these judgments without being aware that physical attractiveness plays
a role in the process. Some consequences of this unconscious assumption
that ``good-looking equals good''
scare me. For example, a study of the 1974 Canadian federal elections
found that attractive candidates received more than two and a half
times as many votes as unattractive candidates (Efran and Patterson,
1976).\textsuperscript{3} Despite such evidence of favoritism toward
handsome politicians, follow-up research demonstrated that voters did
not realize their bias. In fact, 73 percent of Canadian voters surveyed
denied in the strongest possible terms that their votes had been
influenced by physical appearance; only 14 percent even allowed for the
possibility of such influence (Efran and Patterson,
1976).\textsuperscript{4} Voters can deny the impact of attractiveness
on electability all they want, but evidence has continued to confirm
its troubling presence (Budesheim and DePaola,
1994).\textsuperscript{5}}

{
 A similar effect has been found in hiring situations. In one
study, good grooming of applicants in a simulated employment interview
accounted for more favorable hiring decisions than did job
qualifications---this, even though the interviewers claimed that
appearance played a small role in their choices (Mack and Rainey,
1990).\textsuperscript{6} The advantage given to attractive workers
extends past hiring day to payday. Economists examining US and Canadian
samples have found that attractive individuals get paid an average of
12--14 percent more than their unattractive coworkers (Hamermesh and
Biddle, 1994).\textsuperscript{7}}

{
 Equally unsettling research indicates that our judicial process is
similarly susceptible to the influences of body dimensions and bone
structure. It now appears that good-looking people are likely to
receive highly favorable treatment in the legal system (see Castellow,
Wuensch, and Moore, 1991; and Downs and Lyons, 1990, for
reviews).\textsuperscript{8} For example, in a Pennsylvania study
(Stewart, 1980),\textsuperscript{9} researchers rated the physical
attractiveness of 74 separate male defendants at the start of their
criminal trials. When, much later, the researchers checked court
records for the results of these cases, they found that the handsome
men had received significantly lighter sentences. In fact, attractive
defendants were twice as likely to avoid jail as unattractive
defendants. In another study---this one on the damages awarded in a
staged negligence trial---a defendant who was better looking than his
victim was assessed an average amount of \$5,623; but when the victim
was the more attractive of the two, the average compensation was
\$10,051. What's more, both male and female jurors
exhibited the attractiveness-based favoritism (Kulka and Kessler,
1978).\textsuperscript{10}}

{
 Other experiments have demonstrated that attractive people are
more likely to obtain help when in need (Benson, Karabenic, and Lerner,
1976)\textsuperscript{11} and are more persuasive in changing the
opinions of an audience (Chaiken, 1979) \ldots\textsuperscript{12}}

{
 The influence of attractiveness on ratings of intelligence,
honesty, or kindness is a clear example of bias---especially when you
judge these other qualities based on fixed text---because we
wouldn't expect judgments of honesty and attractiveness
to conflate for any legitimate reason. On the other hand, how much of
my perceived intelligence is due to my honesty? How much of my
perceived honesty is due to my intelligence? Finding the truth, and
saying the truth, are not as widely separated in nature as looking
pretty and looking smart \ldots}

{
 But these studies on the halo effect of attractiveness should make
us suspicious that there may be a similar halo effect for kindness, or
intelligence. Let's say that you know someone who not
only seems very intelligent, but also honest, altruistic, kindly, and
serene. You should be suspicious that some of these perceived
characteristics are influencing your perception of the others. Maybe
the person is genuinely intelligent, honest, and altruistic, but not
all that kindly or serene. You should be suspicious if the people you
know seem to separate too cleanly into devils and angels.}

{
 And---I know you don't think \textit{you} have to
do it, but maybe \textit{you} should---be just a little more skeptical
of the more attractive political candidates.}

\myendsectiontext


\bigskip

{
 1. Robert B. Cialdini, \textit{Influence: Science and Practice}
(Boston: Allyn \& Bacon, 2001).}

{
 2. Alice H. Eagly et al., ``What Is Beautiful Is
Good, But \ldots A Meta-analytic Review of Research on the Physical
Attractiveness Stereotype,'' \textit{Psychological
Bulletin} 110 (1 1991): 109--128, doi:10.1037/0033-2909.110.1.109.}

{
 3. M. G. Efran and E. W. J. Patterson, ``The
Politics of Appearance'' (Unpublished PhD thesis,
1976).}

{
 4. Ibid.}

{
 5. Thomas Lee Budesheim and Stephen DePaola,
``Beauty or the Beast?: The Effects of Appearance,
Personality, and Issue Information on Evaluations of Political
Candidates,'' \textit{Personality and Social
Psychology Bulletin} 20 (4 1994): 339--348,
doi:10.1177/0146167294204001.}

{
 6. Denise Mack and David Rainey, ``Female
Applicants' Grooming and Personnel
Selection,'' \textit{Journal of Social Behavior and
Personality} 5 (5 1990): 399--407.}

{
 7. Daniel S. Hamermesh and Jeff E. Biddle,
``Beauty and the Labor Market,''
\textit{The American Economic Review} 84 (5 1994): 1174--1194.}

{
 8. Wilbur A. Castellow, Karl L. Wuensch, and Charles H. Moore,
``Effects of Physical Attractiveness of the Plaintiff
and Defendant in Sexual Harassment Judgments,''
\textit{Journal of Social Behavior and Personality} 5 (6 1990):
547--562; A. Chris Downs and Phillip M. Lyons,
``Natural Observations of the Links Between
Attractiveness and Initial Legal Judgments,''
\textit{Personality and Social Psychology Bulletin} 17 (5 1991):
541--547, doi:10.1177/0146167291175009.}

{
 9. John E. Stewart, ``Defendants'
Attractiveness as a Factor in the Outcome of Trials: An Observational
Study,'' \textit{Journal of Applied Social
Psychology} 10 (4 1980): 348--361,
doi:10.1111/j.1559-1816.1980.tb00715.x.}

{
 10. Richard A. Kulka and Joan B. Kessler, ``Is
Justice Really Blind?: The Effect of Litigant Physical Attractiveness
on Judicial Judgment,'' \textit{Journal of Applied
Social Psychology} 8 (4 1978): 366--381,
doi:10.1111/j.1559-1816.1978.tb00790.x.}

{
 11. Peter L. Benson, Stuart A. Karabenick, and Richard M. Lerner,
``Pretty Pleases: The Effects of Physical
Attractiveness, Race, and Sex on Receiving Help,''
\textit{Journal of Experimental Social Psychology} 12 (5 1976):
409--415, doi:10.1016/0022-1031(76)90073-1.}

{
 12. Shelly Chaiken, ``Communicator Physical
Attractiveness and Persuasion,'' \textit{Journal of
Personality and Social Psychology} 37 (8 1979): 1387--1397,
doi:10.1037/0022-3514.37.8.1387.}

\mysection{Superhero Bias}

{
 Suppose there's a heavily armed sociopath, a
kidnapper with hostages, who has just rejected all requests for
negotiation and announced his intent to start killing. In real life,
the good guys don't usually kick down the door when the
bad guy has hostages. But sometimes---\textit{very} rarely, but
sometimes---life imitates Hollywood to the extent of genuine good guys
needing to smash through a door. }

{
 Imagine, in two widely separated realities, two heroes who charge
into the room, first to confront the villain.}

{
 In one reality, the hero is strong enough to throw cars, can fire
power blasts out of his nostrils, has X-ray hearing, and his skin
doesn't just \textit{deflect} bullets but annihilates
them on contact. The villain has ensconced himself in an elementary
school and taken over two hundred children hostage; their parents are
waiting outside, weeping.}

{
 In another reality, the hero is a New York police officer, and the
hostages are three prostitutes the villain collected off the street.}

{
 Consider this question very carefully: Who is the greater hero?
And who is more likely to get their own comic book?}

{
 The halo effect is that perceptions of all positive traits are
correlated. Profiles rated higher on scales of attractiveness are also
rated higher on scales of talent, kindness, honesty, and intelligence.}

{
 And so comic-book characters who seem strong and invulnerable,
both positive traits, also seem to possess more of the heroic traits of
courage and heroism. And yet:}

{
 How tough can it be to act all brave and courageous when
you're pretty much invulnerable?}

{\raggedleft
 {}---Adam Warren, \textit{Empowered}, Vol. 1\textsuperscript{1}
\par}


\bigskip

{
 I can't remember if I read the following point
somewhere, or hypothesized it myself: \textit{Fame}, in particular,
seems to combine additively with all other personality characteristics.
Consider Gandhi. Was Gandhi the \textit{most altruistic} person of the
twentieth century, or just the \textit{most famous} altruist? Gandhi
faced police with riot sticks and soldiers with guns. But Gandhi was a
celebrity, and he was protected by his celebrity. What about the others
in the march, the people who faced riot sticks and guns even though
there wouldn't be international headlines if they were
put in the hospital or gunned down?}

{
 What did Gandhi think of getting the headlines, the celebrity, the
fame, the place in history, \textit{becoming the archetype} for
non-violent resistance, when he took less risk than any of the people
marching with him? How did he feel when one of those anonymous heroes
came up to him, eyes shining, and told Gandhi how wonderful he was? Did
Gandhi ever visualize his world in those terms? I don't
know; I'm not Gandhi.}

{
 This is not in any sense a criticism of Gandhi. The point of
non-violent resistance is not to show off your courage. That can be
done much more easily by going over Niagara Falls in a barrel. Gandhi
couldn't help being somewhat-but-not-entirely protected
by his celebrity. And Gandhi's actions did take
courage---not as much courage as marching anonymously, but still a
great deal of courage.}

{
 The bias I wish to point out is that Gandhi's fame
score seems to get perceptually \textit{added} to his justly
accumulated altruism score. When you think about nonviolence, you think
of Gandhi---not an anonymous protestor in one of
Gandhi's marches who faced down riot clubs and guns,
and got beaten, and had to be taken to the hospital, and walked with a
limp for the rest of her life, \textit{and no one ever remembered her
name.}}

{
 Similarly, which is greater---to risk your life to save two
hundred children, or to risk your life to save three adults?}

{
 The answer depends on what one means by \textit{greater.} If you
ever have to \textit{choose} between saving two hundred children and
saving three adults, then choose the former. ``Whoever
saves a single life, it is as if he had saved the whole
world'' may be a fine applause light, but
it's terrible moral advice if you've
got to pick one or the other. So if you mean
``greater'' in the sense of
``Which is more important?'' or
``Which is the preferred outcome?''
or ``Which should I choose if I have to do one or the
other?'' then it is greater to save two hundred than
three.}

{
 But if you ask about greatness in the sense of revealed virtue,
then someone who would risk their life to save only three lives reveals
more courage than someone who would risk their life to save two hundred
but not three.}

{
 This doesn't mean that you can deliberately choose
to risk your life to save three adults, and let the two hundred
schoolchildren go hang, because you want to reveal more virtue. Someone
who risks their life \textit{because they want to be virtuous} has
revealed far less virtue than someone who risks their life
\textit{because they want to save others}. Someone who chooses to save
three lives rather than two hundred lives, because they think it
reveals greater virtue, is so selfishly fascinated with their own
``greatness'' as to have committed
the moral equivalent of manslaughter.}

{
 It's one of those \textit{wu wei} scenarios: You
cannot reveal virtue by trying to reveal virtue. Given a choice between
a safe method to save the world which involves no personal sacrifice or
discomfort, and a method that risks your life and requires you to
endure great privation, you cannot become a hero by deliberately
choosing the second path. There is nothing heroic about wanting to look
like a hero. It would be a lost purpose.}

{
 Truly virtuous people who are genuinely trying to save lives,
rather than trying to reveal virtue, will constantly seek to save more
lives with less effort, which means that less of their virtue will be
revealed. It may be confusing, but it's not
contradictory.}

{
 But we cannot always choose to be invulnerable to bullets. After
we've done our best to reduce risk and increase scope,
any \textit{remaining} heroism is well and truly revealed.}

{
 The police officer who puts their life on the line with no
superpowers, no X-Ray vision, no super-strength, no ability to fly, and
above all no invulnerability to bullets, reveals far greater virtue
than Superman---who is a \textit{mere superhero}.}

\myendsectiontext


\bigskip

{
 1. Adam Warren, \textit{Empowered}, vol. 1 (Dark Horse Books,
2007).}

\mysection{Mere Messiahs}

{
 I discussed how the halo effect, which causes people to see all
positive characteristics as correlated---for example, more attractive
individuals are also perceived as more kindly, honest, and
intelligent---causes us to admire heroes more if
they're super-strong and immune to bullets. Even
though, logically, it takes much more courage to be a hero if
you're \textit{not} immune to bullets. Furthermore, it
reveals more virtue to act courageously to save one life than to save
the world. (Although if you have to do one or the other, of course you
should save the world.) }

{
 But let's be more specific.}

{
 John Perry was a New York City police officer who also happened to
be an Extropian and transhumanist, which is how I come to know his
name. John Perry was due to retire shortly and start his own law
practice, when word came that a plane had slammed into the World Trade
Center. He died when the north tower fell. I didn't
know John Perry personally, so I cannot attest to this of direct
knowledge; but very few Extropians believe in God, and I expect that
Perry was likewise an atheist.}

{
 Which is to say that Perry knew he was risking his very existence,
every week on the job. And it's not, like most people
in history, that he knew he had only a choice of how to die, and chose
to make it matter---because Perry was a transhumanist; he had genuine
hope. And Perry went out there and put his life on the line anyway. Not
because he expected any divine reward. Not because he expected to
experience anything at all, if he died. But because there were other
people in danger, and they didn't have immortal souls
either, and his hope of life was worth no more than theirs.}

{
 I did not know John Perry. I do not know if he saw the world this
way. But the fact that an atheist and a transhumanist can still be a
police officer, can still run into the lobby of a burning building,
says more about the human spirit than all the martyrs who ever hoped of
heaven.}

{
 So that is one specific police officer \ldots}

{
 \ldots and now for the superhero.}

{
 As the Christians tell the story, Jesus Christ could walk on
water, calm storms, drive out demons with a word. It must have made for
a comfortable life. Starvation a problem? Xerox some bread.
Don't like a tree? Curse it. Romans a problem? Sic your
Dad on them. Eventually this charmed life ended, when Jesus voluntarily
presented himself for crucifixion. Being nailed to a cross is not a
comfortable way to die. But as the Christians tell the story, Jesus did
this knowing he would come back to life three days later, and then go
to Heaven. What was the threat that moved Jesus to face this temporary
suffering followed by eternity in Heaven? Was it the life of a single
person? Was it the corruption of the church of Judea, or the oppression
of Rome? No: as the Christians tell the story, the eternal fate of
every human went on the line before Jesus suffered himself to be
temporarily nailed to a cross.}

{
 But I do not wish to condemn a man who is not truly so guilty.
What if Jesus---no, let's pronounce his name correctly:
Yeishu---what if Yeishu of Nazareth never walked on water, and
\textit{nonetheless} defied the church of Judea established by the
powers of Rome?}

{
 Would that not deserve greater honor than that which adheres to
Jesus Christ, who was a mere messiah?}

{
 Alas, somehow it seems greater for a hero to have steel skin and
godlike powers. Somehow it seems to reveal more virtue to die
temporarily to save the whole world, than to die permanently
confronting a corrupt church. It seems so \textit{common}, as if many
other people through history had done the same.}

{
 Comfortably ensconced two thousand years in the future, we can
levy all sorts of criticisms at Yeishu, but Yeishu did what he believed
to be right, confronted a church he believed to be corrupt, and died
for it. Without benefit of hindsight, he could hardly be expected to
predict the true impact of his life upon the world. Relative to most
other prophets of his day, he was probably relatively more honest,
relatively less violent, and relatively more courageous. If you strip
away the unintended consequences, the worst that can be said of Yeishu
is that others in history did better. (Epicurus, Buddha, and Marcus
Aurelius all come to mind.) Yeishu died forever, and---from one
perspective---he did it for the sake of honesty. Fifteen hundred years
before science, religious honesty was not an oxymoron.}

{
 As Sam Harris said:\textsuperscript{1}}

{
 It is not enough that Jesus was a man who transformed himself to
such a degree that the Sermon on the Mount could be his
heart's confession. He also had to be the Son of God,
born of a virgin, and destined to return to earth trailing clouds of
glory. The effect of such dogma is to place the example of Jesus
forever out of reach. His teaching ceases to become a set of empirical
claims about the linkage between ethics and spiritual insight and
instead becomes a gratuitous, and rather gruesome, fairy tale.
According to the dogma of Christianity, becoming just like Jesus is
impossible. One can only enumerate one's sins, believe
the unbelievable, and await the end of the world.}

{
 I severely doubt that Yeishu ever spoke the Sermon on the Mount.
Nonetheless, Yeishu deserves honor. He deserves more honor than the
Christians would grant him.}

{
 But since Yeishu probably anticipated his soul would survive, he
doesn't deserve more honor than John Perry.}

\myendsectiontext


\bigskip

{
 1. Sam Harris, \textit{The End of Faith: Religion, Terror, and the
Future of Reason} (WW Norton \& Company, 2005).}

\mysection{Affective Death Spirals}

{
 Many, many, many are the flaws in human reasoning which lead us to
overestimate how well our beloved theory explains the facts. The
phlogiston theory of chemistry could explain just about anything, so
long as it didn't have to predict it in advance. And
the more phenomena you use your favored theory to explain, the truer
your favored theory seems---has it not been confirmed by these many
observations? As the theory seems truer, you will be more likely to
question evidence that conflicts with it. As the favored theory seems
more general, you will seek to use it in more explanations. }

{
 If you know anyone who believes that Belgium secretly controls the
US banking system, or that they can use an invisible blue spirit force
to detect available parking spaces, that's probably how
they got started.}

{
 (Just keep an eye out, and you'll observe much
that seems to confirm this theory \ldots)}

{
 This positive feedback cycle of credulity and confirmation is
indeed fearsome, and responsible for much error, both in science and in
everyday life.}

{
 But it's nothing compared to the death spiral that
begins with a charge of positive affect---a thought that \textit{feels
really good.}}

{
 A new political system that can save the world. A great leader,
strong and noble and wise. An amazing tonic that can cure upset
stomachs and cancer.}

{
 Heck, why not go for all three? A great cause needs a great
leader. A great leader should be able to brew up a magical tonic or
two.}

{
 The halo effect is that any perceived positive characteristic
(such as attractiveness or strength) increases perception of any other
positive characteristic (such as intelligence or courage). Even when it
makes no sense, or less than no sense.}

{
 Positive characteristics enhance perception of every other
positive characteristic? That sounds a lot like how a fissioning
uranium atom sends out neutrons that fission other uranium atoms.}

{
 Weak positive affect is subcritical; it doesn't
spiral out of control. An attractive person seems more honest, which,
perhaps, makes them seem more attractive; but the effective neutron
multiplication factor is less than one. Metaphorically speaking. The
resonance confuses things a little, but then dies out.}

{
 With intense positive affect attached to the Great Thingy, the
resonance touches everywhere. A believing Communist sees the wisdom of
Marx in every hamburger bought at McDonald's; in every
promotion they're denied that would have gone to them
in a true worker's paradise; in every election that
doesn't go to their taste; in every newspaper article
``slanted in the wrong direction.''
Every time they use the Great Idea to interpret another event, the
Great Idea is confirmed all the more. It feels better---positive
reinforcement---and of course, when something feels good, that, alas,
makes us \textit{want} to believe it all the more.}

{
 When the Great Thingy feels good enough to make you \textit{seek
out} new opportunities to feel even better about the Great Thingy,
applying it to interpret new events every day, the resonance of
positive affect is like a chamber full of mousetraps loaded with
ping-pong balls.}

{
 You could call it a ``happy
attractor,'' ``overly positive
feedback,'' a ``praise locked
loop,'' or
``funpaper.'' Personally I prefer
the term ``affective death
spiral.''}

{
 Coming up next: How to resist an affective death spiral. (Hint:
It's not by refusing to ever admire anything again, nor
by keeping the things you admire in safe little restricted
magisterium.)}

\myendsectiontext

\mysection{Resist the Happy Death Spiral}

{
 Once upon a time, there was a man who was convinced that he
possessed a Great Idea. Indeed, as the man thought upon the Great Idea
more and more, he realized that it was not just \textit{a} great idea,
but \textit{the most wonderful idea ever}. The Great Idea would unravel
the mysteries of the universe, supersede the authority of the corrupt
and error-ridden Establishment, confer nigh-magical powers upon its
wielders, feed the hungry, heal the sick, make the whole world a better
place, etc., etc., etc. }

{
 The man was Francis Bacon, his Great Idea was the scientific
method, and he was the only crackpot in all history to claim that level
of benefit to humanity and turn out to be completely right.}

{
 (Bacon didn't singlehandedly invent science, of
course, but he did contribute, and may have been the first to realize
the power.)}

{
 That's the problem with deciding that
you'll never admire anything that much: Some ideas
really \textit{are} that good. Though no one has \textit{fulfilled}
claims more audacious than Bacon's; at least, not yet.}

{
 But then how can we resist the happy death spiral with respect to
Science itself? The happy death spiral starts when you believe
something is \textit{so} wonderful that the halo effect leads you to
find \textit{more} and \textit{more} nice things to say about it,
making you see it as \textit{even more} wonderful, and so on, spiraling
up into the abyss. What if Science is \textit{in fact} so beneficial
that we cannot acknowledge its true glory and retain our sanity? Sounds
like a nice thing to say, doesn't it? \textit{Oh no
it's starting ruuunnnnn \ldots}}

{
 If you retrieve the standard cached deep wisdom for
\textit{don't go overboard on admiring science}, you
will find thoughts like ``Science gave us air
conditioning, but it also made the hydrogen bomb'' or
``Science can tell us about stars and biology, but it
can never prove or disprove the dragon in my
garage.'' But the people who \textit{originated} such
thoughts were \textit{not} trying to resist a happy death spiral. They
weren't worrying about their own admiration of science
spinning out of control. Probably they didn't like
something science had to say about their pet beliefs, and sought ways
to undermine its authority.}

{
 The \textit{standard} negative things to say about science,
aren't likely to appeal to someone who genuinely feels
the exultation of science---that's not the intended
audience. So we'll have to search for other negative
things to say instead.}

{
 But if you look selectively for something negative to say about
science---even in an attempt to resist a happy death spiral---do you
not automatically convict yourself of rationalization? Why would you
pay attention to your own thoughts, if you knew you were trying to
manipulate yourself?}

{
 I am generally skeptical of people who claim that one bias can be
used to counteract another. It sounds to me like an automobile mechanic
who says that the motor is broken on your right windshield wiper, but
instead of fixing it, they'll just break your left
windshield wiper to balance things out. This is the sort of cleverness
that leads to shooting yourself in the foot. Whatever the solution, it
ought to involve believing true things, rather than believing you
believe things that you believe are false.}

{
 Can you prevent the happy death spiral by restricting your
admiration of Science to a narrow domain? Part of the happy death
spiral is seeing the Great Idea everywhere---thinking about how
Communism could cure cancer if it was only given a chance. Probably the
single most reliable sign of a cult guru is that the guru claims
expertise, not in one area, not even in a cluster of related areas, but
in \textit{everything.} The guru knows what cult members should eat,
wear, do for a living; who they should have sex with; which art they
should look at; which music they should listen to \ldots}

{
 Unfortunately for this plan, most people fail miserably when they
try to describe the neat little box that science has to stay inside.
The usual trick, ``Hey, science won't
cure cancer'' isn't going to fly.
``Science has nothing to say about a
parent's love for their
child''---sorry, that's simply false.
If you try to sever science from e.g. parental love, you
aren't just denying cognitive science and evolutionary
psychology. You're also denying Martine
Rothblatt's founding of United Therapeutics to seek a
cure for her daughter's pulmonary hypertension.
(Successfully, I might add.) Science is legitimately related, one way
or another, to just about every important facet of human existence.}

{
 All right, so what's an example of a
\textit{false} nice claim you could make about science?}

{
 In my humble opinion, one false claim is that science is so
wonderful that scientists shouldn't even try to take
ethical responsibility for their work, it will automatically end well.
This claim, to me, seems to misunderstand the nature of the process
whereby science benefits humanity. Scientists are human, they have
prosocial concerns just like most other other people, and this is at
least \textit{part} of why science ends up doing more good than evil.}

{
 But that point is, evidently, not beyond dispute. So
here's a simpler false nice claim: ``A
cancer patient can be cured just by publishing enough journal
papers.'' Or, ``Sociopaths could
become fully normal, if they just committed themselves to never
believing anything without replicated experimental evidence with p
{\textless} 0.05.''}

{
 The way to avoid believing such statements isn't
an affective cap, deciding that science is only slightly nice. Nor
searching for reasons to believe that publishing journal papers
\textit{causes} cancer. Nor believing that science has nothing to say
about cancer one way or the other.}

{
 Rather, if you know with enough specificity how science works,
then you know that, while it may be possible for
``science to cure cancer,'' a cancer
patient writing journal papers isn't going to
experience a miraculous remission. That \textit{specific} proposed
chain of cause and effect is not going to work out.}

{
 The happy death spiral is only an emotional problem because of a
perceptual problem, the halo effect, that makes us more likely to
accept future positive claims once we've accepted an
initial positive claim. We can't get rid of this effect
just by wishing; it will probably always influence us a little. But we
can manage to slow down, stop, consider each additional nice claim as
an additional burdensome detail, and focus on the specific points of
the claim apart from its positiveness.}

{
 What if a specific nice claim
``can't be
disproven'' but there are arguments
``both for and against'' it?
Actually these are words to be wary of in general, because often this
is what people say when they're rehearsing the evidence
or avoiding the real weak points. Given the danger of the happy death
spiral, it makes sense to try to avoid being happy about
\textit{unsettled} claims---to avoid making them into a source of yet
more positive affect about something you liked already.}

{
 The happy death spiral is only a \textit{big} emotional problem
because of the overly positive feedback, the ability for the process to
go critical. You may not be able to eliminate the halo effect entirely,
but you can apply enough critical reasoning to keep the halos
subcritical---make sure that the resonance dies out rather than
exploding.}

{
 You might even say that the whole problem starts with people not
bothering to critically examine every additional burdensome
detail---demanding sufficient evidence to compensate for complexity,
searching for flaws as well as support, invoking curiosity---once
they've accepted some core premise. Without the
conjunction fallacy, there might still be a halo effect, but there
wouldn't be a happy death spiral.}

{
 Even on the nicest Nice Thingies in the known universe, a perfect
rationalist who demanded exactly the necessary evidence for every
additional (positive) claim would experience no affective resonance.
You can't do this, but you can stay close enough to
rational to keep your happiness from spiraling out of control.}

{
 The really dangerous cases are the ones where \textit{any
criticism of any positive claim about the Great Thingy feels bad or is
socially unacceptable}. Arguments are soldiers, any positive claim is a
soldier on our side, stabbing your soldiers in the back is treason.
Then the chain reaction goes \textit{super}critical. More on this
later.}

{
 Stuart Armstrong gives closely related advice:}

{
 Cut up your Great Thingy into smaller independent ideas,
\textit{and treat them as independent.}}

{
 For instance a marxist would cut up Marx's Great
Thingy into a theory of value of labour, a theory of the political
relations between classes, a theory of wages, a theory on the ultimate
political state of mankind. Then each of them should be assessed
independently, and the truth or falsity of one should not halo on the
others. If we can do that, we should be safe from the spiral, as each
theory is too narrow to start a spiral on its own.}

{
 This, metaphorically, is like keeping subcritical masses of
plutonium from coming together. Three Great Ideas are far less likely
to drive you mad than one Great Idea. Armstrong's
advice also helps promote specificity: As soon as someone says,
``Publishing enough papers can cure your
cancer,'' you ask, ``Is that a
benefit of the experimental method, and if so, at which stage of the
experimental process is the cancer cured? Or is it a benefit of science
as a social process, and if so, does it rely on individual scientists
wanting to cure cancer, or can they be
self-interested?'' Hopefully this leads you away from
the good or bad feeling, and toward noticing the confusion and lack of
support.}

{
 To summarize, you \textit{do} avoid a Happy Death Spiral by:}

{
 Splitting the Great Idea into parts;}

{
 Treating every additional detail as burdensome;}

{
 Thinking about the specifics of the causal chain instead of the
good or bad feelings;}

{
 Not rehearsing evidence; and}

{
 Not adding happiness from claims that ``you
can't \textit{prove} are wrong'';}

{
 but \textit{not} by:}

{
 Refusing to admire anything too much;}

{
 Conducting a biased search for negative points until you feel
unhappy again; or}

{
 Forcibly shoving an idea into a safe box.}

\myendsectiontext

\mysection{Uncritical Supercriticality}

{
 Every now and then, you see people arguing over whether atheism is
a ``religion.'' As I touch on
elsewhere, in Purpose and Pragmatism, arguing over the meaning of a
word nearly always means that you've lost track of the
original question. How might this argument arise to begin with? }

{
 An atheist is holding forth, blaming
``religion'' for the Inquisition,
the Crusades, and various conflicts with or within Islam. The religious
one may reply, ``But atheism is also a religion,
because you also have beliefs about God; you believe God
doesn't exist.'' Then the atheist
answers, ``If atheism is a religion, then not
collecting stamps is a hobby,'' and the argument
begins.}

{
 Or the one may reply, ``But horrors just as great
were inflicted by Stalin, who was an atheist, and who suppressed
churches in the name of atheism; therefore you are wrong to blame the
violence on religion.'' Now the atheist may be
tempted to reply ``No true
Scotsman,'' saying,
``Stalin's religion was
Communism.'' The religious one answers
``If Communism is a religion, then Star Wars fandom is
a government,'' and the argument begins.}

{
 Should a ``religious'' person
be defined as someone who has a definite opinion about the existence of
at least one God, e.g., assigning a probability lower than 10\% or
higher than 90\% to the existence of Zeus? Or should a
``religious'' person be defined as
someone who has a positive opinion, say a probability higher than 90\%,
for the existence of at least one God? In the former case, Stalin was
``religious''; in the latter case,
Stalin was ``not religious.''}

{
 But this is exactly the wrong way to look at the problem. What you
really want to know---what the argument was originally about---is why,
at certain points in human history, large groups of people were
slaughtered and tortured, ostensibly in the name of an idea. Redefining
a word won't change the facts of history one way or the
other.}

{
 Communism was a complex catastrophe, and there may be no single
\textit{why}, no single critical link in the chain of causality. But if
I had to suggest an ur-mistake, it would be \ldots well,
I'll let God say it for me:}

{
 If your brother, the son of your father or of your mother, or your
son or daughter, or the spouse whom you embrace, or your most intimate
friend, tries to secretly seduce you, saying, ``Let us
go and serve other gods,'' unknown to you or your
ancestors before you, gods of the peoples surrounding you, whether near
you or far away, anywhere throughout the world, you must not consent,
\textbf{you must not listen to him}; you must show him no pity, you
must not spare him or conceal his guilt. No, \textbf{you must kill
him}, your hand must strike the first blow in putting him to death and
the hands of the rest of the people following. You must stone him to
death, since he has tried to divert you from Yahweh your God.}

{\raggedleft
 {}---Deuteronomy 13:7--11, emphasis added
\par}


\bigskip

{
 This was likewise the rule which Stalin set for Communism, and
Hitler for Nazism: if your brother tries to tell you why Marx is wrong,
if your son tries to tell you the Jews are not planning world conquest,
then do not debate him or set forth your own evidence; do not perform
replicable experiments or examine history; but turn him in at once to
the secret police.}

{
 I suggested that one key to resisting an affective death spiral is
the principle of ``burdensome
details''---just \textit{remembering} to question the
specific details of each additional nice claim about the Great Idea.
(It's not trivial advice. People often
don't remember to do this when they're
listening to a futurist sketching amazingly detailed projections about
the wonders of tomorrow, let alone when they're
thinking about their favorite idea ever.) This wouldn't
get rid of the halo effect, but it would hopefully reduce the resonance
to below criticality, so that one nice-sounding claim triggers less
than 1.0 additional nice-sounding claims, on average.}

{
 The diametric opposite of this advice, which sends the halo effect
\textit{super}critical, is when it feels wrong to argue against
\textit{any} positive claim about the Great Idea. Politics is the
mind-killer. Arguments are soldiers. Once you know which side
you're on, you must support all favorable claims, and
argue against all unfavorable claims. Otherwise it's
like giving aid and comfort to the enemy, or stabbing your friends in
the back.}

{
 If \ldots}

{
 \ldots you feel that contradicting someone else who makes a flawed
nice claim in favor of evolution would be giving aid and comfort to the
creationists;}

{
 \ldots you feel like you get spiritual credit for each nice thing
you say about God, and arguing about it would interfere with your
relationship with God;}

{
 \ldots you have the distinct sense that the other people in the
room will dislike you for ``not supporting our
troops'' if you argue against the latest war;}

{
 \ldots saying anything against Communism gets you stoned to death
shot;}

{
 \ldots then the affective death spiral has gone supercritical. It
is now a Super Happy Death Spiral.}

{
 It's not religion, as such, that is the key
categorization, relative to our original question:
``What makes the slaughter?'' The
best distinction I've heard between
``supernatural'' and
``naturalistic'' worldviews is that
a supernatural worldview asserts the existence of ontologically basic
mental substances, like spirits, while a naturalistic worldview reduces
mental phenomena to nonmental parts. Focusing on this as the source of
the problem buys into religious exceptionalism. Supernaturalist claims
are worth distinguishing, because they always turn out to be wrong for
fairly fundamental reasons. But it's still just one
kind of mistake.}

{
 An affective death spiral can nucleate around supernatural
beliefs; especially monotheisms whose pinnacle is a Super Happy Agent,
defined primarily by agreeing with any nice statement about it;
especially meme complexes grown sophisticated enough to assert
supernatural punishments for disbelief. But the death spiral can also
start around a political innovation, a charismatic leader, belief in
racial destiny, or an economic hypothesis. The lesson of history is
that affective death spirals are dangerous whether or not they happen
to involve supernaturalism. Religion isn't special
enough, as a class of mistake, to be the key problem.}

{
 Sam Harris came closer when he put the accusing finger on
\textit{faith.} If you don't place an appropriate
burden of proof on each and every additional nice claim, the affective
resonance gets started \textit{very} easily. Look at the poor New
Agers. Christianity developed defenses against criticism, arguing for
the wonders of faith; New Agers culturally inherit the cached thought
that faith is positive, but lack Christianity's
exclusionary scripture to keep out competing memes. New Agers end up in
happy death spirals around stars, trees, magnets, diets, spells,
unicorns \ldots}

{
 But the affective death spiral turns much deadlier after criticism
becomes a sin, or a gaffe, or a crime. There are things in this world
that are worth praising greatly, and you can't
\textit{flatly} say that praise beyond a certain point is forbidden.
But there is \textit{never} an Idea so true that it's
wrong to criticize any argument that supports it. Never. Never ever
never for ever. \textit{That} is flat. The vast majority of possible
beliefs in a nontrivial answer space are false, and likewise, the vast
majority of possible \textit{supporting arguments} for a true belief
are also false, and not even the happiest idea can change that.}

{
 And it is triple ultra forbidden to respond to criticism with
violence. There are a very few injunctions in the human art of
rationality that have no ifs, ands, buts, or escape clauses. This is
one of them. Bad argument gets counterargument. Does not get bullet.
Never. Never ever never for ever.}

\myendsectiontext

\mysection{Evaporative Cooling of Group Beliefs}

{
 Early studiers of cults were surprised to discover than when cults
receive a major shock---a prophecy fails to come true, a moral flaw of
the founder is revealed---they often come back stronger than before,
with increased belief and fanaticism. The Jehovah's
Witnesses placed Armageddon in 1975, based on Biblical calculations;
1975 has come and passed. The Unarian cult, still going strong today,
survived the nonappearance of an intergalactic spacefleet on September
27, 1975. }

{
 Why would a group belief become \textit{stronger} after
encountering crushing counterevidence?}

{
 The conventional interpretation of this phenomenon is based on
cognitive dissonance. When people have taken
``irrevocable'' actions in the
service of a belief---given away all their property in anticipation of
the saucers landing---they cannot possibly admit they were mistaken.
The challenge to their belief presents an immense cognitive dissonance;
they must find reinforcing thoughts to counter the shock, and so become
more fanatical. In this interpretation, the increased group fanaticism
is the result of increased individual fanaticism.}

{
 I was looking at a Java applet which demonstrates the use of
evaporative cooling to form a Bose-Einstein condensate, when it
occurred to me that another force entirely might operate to increase
fanaticism. Evaporative cooling sets up a potential energy barrier
around a collection of hot atoms. Thermal energy is essentially
statistical in nature---not all atoms are moving at the exact same
speed. The kinetic energy of any given atom varies as the atoms collide
with each other. If you set up a potential energy barrier
that's just a little higher than the average thermal
energy, the workings of chance will give an occasional atom a kinetic
energy high enough to escape the trap. When an unusually fast atom
escapes, it takes with it an unusually large amount of kinetic energy,
and the average energy decreases. The group becomes substantially
cooler than the potential energy barrier around it. Playing with the
Java applet may make this clearer.}

{
 In Festinger, Riecken, and Schachter's classic
\textit{When Prophecy Fails}, one of the cult members walked out the
door immediately after the flying saucer failed to
land.\textsuperscript{1} Who gets fed up and leaves \textit{first}? An
\textit{average} cult member? Or a relatively more skeptical member,
who previously might have been acting as a voice of moderation, a brake
on the more fanatic members?}

{
 After the members with the highest kinetic energy escape, the
remaining discussions will be between the extreme fanatics on one end
and the slightly less extreme fanatics on the other end, with the group
consensus somewhere in the
``middle.''}

{
 And what would be the analogy to collapsing to form a
Bose-Einstein condensate? Well, there's no real need to
stretch the analogy that far. But you may recall that I used a fission
chain reaction analogy for the affective death spiral; when a group
ejects all its voices of moderation, then all the people encouraging
each other, and suppressing dissents, may internally increase in
average fanaticism. (No thermodynamic analogy here, unless someone
develops a nuclear weapon that explodes when it gets cold.)}

{
 When Ayn Rand's long-running affair with Nathaniel
Branden was revealed to the Objectivist membership, a substantial
fraction of the Objectivist membership broke off and followed Branden
into espousing an ``open system'' of
Objectivism not bound so tightly to Ayn Rand. Who stayed with Ayn Rand
even after the scandal broke? The ones who \textit{really, really}
believed in her---and perhaps some of the undecideds, who, after the
voices of moderation left, heard arguments from only one side. This may
account for how the Ayn Rand Institute is (reportedly) more fanatic
after the breakup, than the original core group of Objectivists under
Branden and Rand.}

{
 A few years back, I was on a transhumanist mailing list where a
small group espousing ``social democratic
transhumanism'' vitriolically insulted every
libertarian on the list. Most libertarians left the mailing list, most
of the others gave up on posting. As a result, the remaining group
shifted substantially to the left. Was this deliberate? Probably not,
because I don't think the perpetrators knew that much
psychology. (For that matter, I can't recall seeing the
evaporative cooling analogy elsewhere, though that
doesn't mean it hasn't been noted
before.) At most, they might have thought to make themselves
``bigger fish in a smaller pond.''}

{
 This is one reason why it's important to be
prejudiced in favor of tolerating dissent. Wait until substantially
\textit{after} it seems to you justified in ejecting a member from the
group, before actually ejecting. If you get rid of the old outliers,
the group position will shift, and someone else will become the
oddball. If you eject them too, you're well on the way
to becoming a Bose-Einstein condensate and, er, exploding.}

{
 The flip side: Thomas Kuhn believed that a science has to become a
``paradigm,'' with a shared
technical language that excludes outsiders, before it can get any real
work done. In the formative stages of a science, according to Kuhn, the
adherents go to great pains to make their work comprehensible to
outside academics. But (according to Kuhn) a science can only make real
progress as a technical discipline once it abandons the requirement of
outside accessibility, and scientists working in the paradigm assume
familiarity with large cores of technical material in their
communications. This sounds cynical, relative to what is usually said
about public understanding of science, but I can definitely see a core
of truth here.}

{
 My own theory of Internet moderation is that you have to be
willing to exclude trolls and spam to get a conversation going. You
must even be willing to exclude kindly but technically uninformed folks
from technical mailing lists if you want to get any work done. A
genuinely open conversation on the Internet degenerates fast.
It's the \textit{articulate} trolls that you should be
wary of ejecting, on this theory---they serve the hidden function of
legitimizing less extreme disagreements. But you should not have so
many articulate trolls that they begin arguing with each other, or
begin to dominate conversations. If you have one person around who is
the famous Guy Who Disagrees With Everything, anyone with a more
reasonable, more moderate disagreement won't look like
the sole nail sticking out. This theory of Internet moderation may not
have served me too well in practice, so take it with a grain of salt.}

\myendsectiontext


\bigskip

{
 1. Leon Festinger, Henry W. Riecken, and Stanley Schachter,
\textit{When Prophecy Fails: A Social and Psychological Study of a
Modern Group That Predicted the Destruction of the World}
(Harper-Torchbooks, 1956).}

\mysection{When None Dare Urge Restraint}

{
 One morning, I got out of bed, turned on my computer, and my
Netscape email client automatically downloaded that
day's news pane. On that particular day, the news was
that two hijacked planes had been flown into the World Trade Center. }

{
 These were my first three thoughts, in order:}

{
 \textit{I guess I really am living in the Future.}}

{
 \textit{Thank goodness it wasn't nuclear.}}

{
 and then}

{
 \textit{The overreaction to this will be ten times worse than the
original event.}}

{
 A mere factor of ``ten times
worse'' turned out to be a vast understatement. Even
I didn't guess how badly things would go.
That's the challenge of pessimism; it's
\textit{really hard} to aim low enough that you're
pleasantly surprised around as often and as much as
you're unpleasantly surprised.}

{
 Nonetheless, I did realize immediately that everyone everywhere
would be saying how awful, how terrible this event was; and that no one
would dare to be the voice of restraint, of proportionate response.
Initially, on 9/11, it was thought that six thousand people had died.
Any politician who'd said ``6,000
deaths is 1/8 the annual US casualties from automobile
accidents,'' would have been asked to resign the same
hour.}

{
 No, 9/11 wasn't a good day. But if
\textit{everyone} gets brownie points for emphasizing how much it
hurts, and \textit{no one} dares urge restraint in how hard to hit
back, then the reaction will be greater than the appropriate level,
whatever the appropriate level may be.}

{
 This is the even darker mirror of the happy death spiral---the
spiral of hate. Anyone who attacks the Enemy is a patriot; and whoever
tries to dissect even a single negative claim about the Enemy is a
traitor. But just as the vast majority of all complex statements are
untrue, the vast majority of negative things you can say about anyone,
even the worst person in the world, are untrue.}

{
 I think the best illustration was ``the suicide
hijackers were cowards.'' Some common sense, please?
It takes a little courage to voluntarily fly your plane into a
building. Of all their sins, cowardice was not on the list. But I guess
anything bad you say about a terrorist, no matter how silly, must be
true. Would I get even more brownie points if I accused al-Qaeda of
having assassinated John F. Kennedy? Maybe if I accused them of being
Stalinists? Really, \textit{cowardice?}}

{
 \textit{Yes,} it matters that the 9/11 hijackers
weren't cowards. Not just for understanding the
enemy's realistic psychology. There is simply too much
damage done by spirals of hate. It is just too dangerous for there to
be any target in the world, whether it be the Jews or Adolf Hitler,
about whom \textit{saying negative things} trumps \textit{saying
accurate things}.}

{
 When the defense force contains thousands of aircraft and hundreds
of thousands of heavily armed soldiers, one ought to consider that the
immune system itself is capable of wreaking more damage than nineteen
guys and four nonmilitary airplanes. The US spent billions of dollars
and thousands of soldiers' lives shooting off its own
foot more effectively than any terrorist group could dream.}

{
 If the USA had completely ignored the 9/11 attack---just shrugged
and rebuilt the building---it would have been better than the real
course of history. But that wasn't a political option.
Even if anyone privately guessed that the immune response would be more
damaging than the disease, American politicians had no
career-preserving choice but to walk straight into
al-Qaeda's trap. Whoever argues for a greater response
is a patriot. Whoever dissects a patriotic claim is a traitor.}

{
 Initially, there were smarter responses to 9/11 than I had
guessed. I saw a Congressperson---I forget who---say in front of the
cameras, ``We have forgotten that the first purpose of
government is not the economy, it is not health care, it is defending
the country from attack.'' That widened my eyes, that
a politician could say something that wasn't an
applause light. The emotional shock must have been very great for a
Congressperson to say something that \ldots real.}

{
 But within two days, the genuine shock faded, and
concern-for-image regained total control of the political discourse.
Then the spiral of escalation took over completely. Once restraint
becomes unspeakable, no matter where the discourse starts out, the
level of fury and folly can only rise with time.}

\myendsectiontext

\mysection{The Robbers Cave Experiment}

{
 Did you ever wonder, when you were a kid, whether your inane
``summer camp'' actually had some
kind of elaborate hidden purpose---say, it was all a science experiment
and the ``camp counselors'' were
really researchers observing your behavior? }

{
 Me neither.}

{
 But we'd have been more paranoid if
we'd read ``Intergroup Conflict and
Cooperation: The Robbers Cave Experiment'' by Sherif,
Harvey, White, Hood, and Sherif.\textsuperscript{1} In this study, the
experimental subjects---excuse me,
``campers''---were 22 boys between
fifth and sixth grade, selected from 22 different schools in Oklahoma
City, of stable middle-class Protestant families, doing well in school,
median IQ 112. They were as well-adjusted and as similar to each other
as the researchers could manage.}

{
 The experiment, conducted in the bewildered aftermath of World War
II, was meant to investigate the causes---and possible remedies---of
intergroup conflict. How would they spark an intergroup conflict to
investigate? Well, the 22 boys were divided into two groups of 11
campers, and---}

{
 {}---and that turned out to be quite sufficient.}

{
 The researchers' original plans called for the
experiment to be conducted in three stages. In Stage 1, each group of
campers would settle in, unaware of the other group's
existence. Toward the end of Stage 1, the groups would gradually be
made aware of each other. In Stage 2, a set of contests and prize
competitions would set the two groups at odds.}

{
 They needn't have bothered with Stage 2. There was
hostility almost from the moment each group became aware of the other
group's existence: They were using \textit{our}
campground, \textit{our} baseball diamond. On their first meeting, the
two groups began hurling insults. They named themselves the Rattlers
and the Eagles (they hadn't needed names when they were
the only group on the campground).}

{
 When the contests and prizes were announced, in accordance with
pre-established experimental procedure, the intergroup rivalry rose to
a fever pitch. Good sportsmanship in the contests was evident for the
first two days but rapidly disintegrated.}

{
 The Eagles stole the Rattlers' flag and burned it.
Rattlers raided the Eagles' cabin and stole the blue
jeans of the group leader, which they painted orange and carried as a
flag the next day, inscribed with the legend ``The
Last of the Eagles.'' The Eagles launched a
retaliatory raid on the Rattlers, turning over beds, scattering dirt.
Then they returned to their cabin where they entrenched and prepared
weapons (socks filled with rocks) in case of a return raid. After the
Eagles won the last contest planned for Stage 2, the Rattlers raided
their cabin and stole the prizes. This developed into a fistfight that
the staff had to shut down for fear of injury. The Eagles, retelling
the tale among themselves, turned the whole affair into a magnificent
victory---they'd chased the Rattlers
``over halfway back to their cabin''
(they hadn't).}

{
 Each group developed a negative stereotype of Them and a
contrasting positive stereotype of Us. The Rattlers swore heavily. The
Eagles, after winning one game, concluded that the Eagles had won
because of their prayers and the Rattlers had lost because they used
cuss-words all the time. The Eagles decided to stop using cuss-words
themselves. They also concluded that since the Rattlers swore all the
time, it would be wiser not to talk to them. The Eagles developed an
image of themselves as proper-and-moral; the Rattlers developed an
image of themselves as rough-and-tough.}

{
 Group members held their noses when members of the other group
passed.}

{
 In Stage 3, the researchers tried to reduce friction between the
two groups.}

{
 Mere contact (being present without contesting) did not reduce
friction between the two groups. Attending pleasant events
together---for example, shooting off Fourth of July fireworks---did not
reduce friction; instead it developed into a food fight.}

{
 Would you care to guess what \textit{did} work?}

{
 (Spoiler space \ldots)}

{
 The boys were informed that there might be a water shortage in the
whole camp, due to mysterious trouble with the water system---possibly
due to vandals. (The Outside Enemy, one of the oldest tricks in the
book.)}

{
 The area between the camp and the reservoir would have to be
inspected by four search details. (Initially, these search details were
composed uniformly of members from each group.) All details would meet
up at the water tank if nothing was found. As nothing was found, the
groups met at the water tank and observed for themselves that no water
was coming from the faucet. The two groups of boys discussed where the
problem might lie, pounded the sides of the water tank, discovered a
ladder to the top, verified that the water tank was full, and finally
found the sack stuffed in the water faucet. All the boys gathered
around the faucet to clear it. Suggestions from members of both groups
were thrown at the problem and boys from both sides tried to implement
them.}

{
 When the faucet was finally cleared, the Rattlers, who had
canteens, did not object to the Eagles taking a first turn at the
faucets (the Eagles didn't have canteens with them). No
insults were hurled, not even the customary ``Ladies
first.''}

{
 It wasn't the end of the rivalry. There was
another food fight, with insults, the next morning. But a few more
common tasks, requiring cooperation from both groups---e.g. restarting
a stalled truck---did the job. At the end of the trip, the Rattlers
used \$5 won in a bean-toss contest to buy malts for all the boys in
both groups.}

{
 The Robbers Cave Experiment illustrates the psychology of
hunter-gatherer bands, echoed through time, as perfectly as any
experiment ever devised by social science.}

{
 Any resemblance to modern politics is just your imagination.}

{
 (Sometimes I think humanity's second-greatest need
is a supervillain. Maybe I'll go into that line of work
after I finish my current job.)}

\myendsectiontext


\bigskip

{
 1. Muzafer Sherif et al., ``Study of Positive and
Negative Intergroup Attitudes Between Experimentally Produced Groups:
Robbers Cave Study,'' Unpublished manuscript (1954).}

\mysection{Every Cause Wants to Be a Cult}

{
 Cade Metz at \textit{The Register} recently alleged that a secret
mailing list of Wikipedia's top administrators has
become obsessed with banning all critics and possible critics of
Wikipedia. Including banning a productive user when one
administrator---solely \textit{because} of the productivity---became
convinced that the user was a spy sent by \textit{Wikipedia Review.}
And that the top people at Wikipedia closed ranks to defend their own.
(I have not investigated these allegations myself, as yet. Hat tip to
Eugen Leitl.) }

{
 Is there some deep moral flaw in seeking to systematize the
world's knowledge, which would lead pursuers of that
Cause into madness? Perhaps only people with innately totalitarian
tendencies would try to become the world's authority on
everything---}

{
 Correspondence bias alert! (Correspondence bias: making inferences
about someone's unique disposition from behavior that
can be entirely explained by the situation in which it occurs. When we
see someone else kick a vending machine, we think they are
``an angry person,'' but when we
kick the vending machine, it's because the bus was
late, the train was early, and the machine ate our money.) If the
allegations about Wikipedia are true, they're explained
by \textit{ordinary} human nature, not by \textit{extraordinary} human
nature.}

{
 The ingroup-outgroup dichotomy is part of ordinary human nature.
So are happy death spirals and spirals of hate. A Noble Cause
doesn't need a deep hidden flaw for its adherents to
form a cultish in-group. It is sufficient that the adherents be human.
Everything else follows naturally, decay by default, like food spoiling
in a refrigerator after the electricity goes off.}

{
 In the same sense that every thermal differential wants to
equalize itself, and every computer program wants to become a
collection of ad-hoc patches, every Cause \textit{wants} to be a cult.
It's a high-entropy state into which the system trends,
an attractor in human psychology. It may have nothing to do with
whether the Cause is truly Noble. You might think that a Good Cause
would rub off its goodness on every aspect of the people associated
with it---that the Cause's followers would also be less
susceptible to status games, ingroup-outgroup bias, affective spirals,
leader-gods. But believing one true idea won't switch
off the halo effect. A noble cause won't make its
adherents something other than human. There are plenty of bad ideas
that can do plenty of damage---but that's not
necessarily what's going on.}

{
 Every group of people with an unusual goal---good, bad, or
silly---will trend toward the cult attractor unless they make a
constant effort to resist it. You can keep your house cooler than the
outdoors, but you have to run the air conditioner constantly, and as
soon as you turn off the electricity---give up the fight against
entropy---things will go back to
``normal.''}

{
 On one notable occasion there was a group that went semicultish
whose rallying cry was ``Rationality! Reason!
Objective reality!'' (More on this later.) Labeling
the Great Idea ``rationality''
won't protect you any more than putting up a sign over
your house that says ``Cold!'' You
still have to run the air conditioner---expend the required energy per
unit time to reverse the natural slide into cultishness. Worshipping
rationality won't make you sane any more than
worshipping gravity enables you to fly. You can't talk
to thermodynamics and you can't pray to probability
theory. You can \textit{use} it, but not join it as an in-group.}

{
 Cultishness is quantitative, not qualitative. The question is not
``Cultish, yes or no?'' but
``How much cultishness and where?''
Even in Science, which is the archetypal Genuinely Truly Noble Cause,
we can readily point to the current frontiers of the war against
cult-entropy, where the current battle line creeps forward and back.
Are journals more likely to accept articles with a well-known authorial
byline, or from an unknown author from a well-known institution,
compared to an unknown author from an unknown institution? How much
belief is due to authority and how much is from the experiment? Which
journals are using blinded reviewers, and how effective is blinded
reviewing?}

{
 I cite this example, rather than the standard vague accusations of
``Scientists aren't open to new
ideas,'' because it shows a \textit{battle
line}{}---a place where human psychology is being actively driven back,
where accumulated cult-entropy is being pumped out. (Of course this
requires emitting some waste heat.)}

{
 This essay is not a catalog of techniques for actively pumping
against cultishness. Some such techniques I have said before, and some
I will say later. \textit{Here} I just want to point out that the
worthiness of the Cause does not mean you can spend any \textit{less}
effort in resisting the cult attractor. And that if you can point to
current battle lines, it does not mean you confess your Noble Cause
unworthy. You might think that if the question were
``Cultish, yes or no?'' that you
were obliged to answer ``No,'' or
else betray your beloved Cause. But that is like thinking that you
should divide engines into ``perfectly
efficient'' and
``inefficient,'' instead of
measuring waste.}

{
 Contrariwise, if you believe that it was the Inherent Impurity of
those Foolish Other Causes that made them go wrong, if you laugh at the
folly of ``cult victims,'' if you
think that cults are led and populated by mutants, then you will not
expend the necessary effort to pump against entropy---to resist being
human.}

\myendsectiontext

\mysection{Guardians of the Truth}

{
 The criticism is sometimes leveled against rationalists:
``The Inquisition thought \textit{they} had the truth!
Clearly this `truth' business is
dangerous.'' }

{
 There are many obvious responses, such as ``If
you think that possessing the truth \textit{would} license you to
torture and kill, you're making a mistake that has
nothing to do with epistemology.'' Or,
``So that historical statement you just made about the
Inquisition---is it true?''}

{
 Reversed stupidity is not intelligence: ``If your
current computer stops working, you can't conclude that
everything about the current system is wrong and that you need a new
system without an AMD processor, an ATI video card \ldots even though
your current system has all these things and it doesn't
work. Maybe you just need a new power cord.'' To
arrive at a poor conclusion requires only one wrong step, not every
step wrong. The Inquisitors believed that 2 + 2 = 4, but that
wasn't the source of their madness. Maybe
epistemological realism wasn't the problem either?}

{
 It does seem plausible that if the Inquisition had been made up of
relativists, professing that nothing was true and nothing mattered,
they would have mustered less enthusiasm for their torture. They would
also have been less enthusiastic if lobotomized. I think
that's a fair analogy.}

{
 And yet \ldots I think the Inquisition's attitude
toward truth played a role. The Inquisition believed that there was
such a thing as truth, and that it was important; well, likewise
Richard Feynman. But the Inquisitors were not Truth-Seekers. They were
Truth-\textit{Guardians.}}

{
 I once read an argument (I can't find the source)
that a key component of a \textit{zeitgeist} is whether it locates its
ideals in its future or its past. Nearly all cultures before the
Enlightenment believed in a Fall from Grace---that things had once been
perfect in the distant past, but then catastrophe had struck, and
everything had slowly run downhill since then:}

{
 In the age when life on Earth was full \ldots They loved each other
and did not know that this was ``love of
neighbor.'' They deceived no one yet they did not
know that they were ``men to be
trusted.'' They were reliable and did not know that
this was ``good faith.'' They lived
freely together giving and taking, and did not know that they were
generous. For this reason their deeds have not been narrated. They made
no history.}

{\raggedleft
 {}---\textit{The Way of Chuang Tzu}, trans. Thomas
Merton\textsuperscript{1}
\par}


\bigskip

{
 The perfect age of the past, according to our best anthropological
evidence, never existed. But a culture that sees life running
inexorably downward is very different from a culture in which you can
reach unprecedented heights.}

{
 (I say ``culture,'' and not
``society,'' because you can have
more than one subculture in a society.)}

{
 You could say that the difference between e.g. Richard Feynman and
the Inquisition was that the Inquisition believed they \textit{had}
truth, while Richard Feynman \textit{sought} truth. This
isn't quite defensible, though, because there were
undoubtedly some truths that Richard Feynman thought he \textit{had} as
well. ``The sky is blue,'' for
example, or ``2 + 2 = 4.''}

{
 Yes, there are effectively certain truths of science. General
Relativity may be overturned by some future physics---albeit not in any
way that predicts the Sun will orbit Jupiter; the new theory must steal
the successful predictions of the old theory, not contradict them. But
evolutionary theory takes place on a higher level of organization than
atoms, and nothing we discover about quarks is going to throw out
Darwinism, or the cell theory of biology, or the atomic theory of
chemistry, or a hundred other brilliant innovations whose truth is now
established beyond \textit{reasonable} doubt.}

{
 Are these ``absolute truths''?
Not in the sense of possessing a probability of literally 1.0. But they
are cases where science basically thinks it's got the
truth.}

{
 And yet scientists don't torture people who
question the atomic theory of chemistry. Why not? Because they
don't believe that certainty licenses torture? Well,
yes, that's the \textit{surface} difference; but why
\textit{don't} scientists believe this?}

{
 Because chemistry asserts no supernatural penalty of eternal
torture for disbelieving in the atomic theory of chemistry? But again
we recurse and ask the question,
``Why?'' Why
\textit{don't} chemists believe that you go to hell if
you disbelieve in the atomic theory?}

{
 Because journals won't publish your paper until
you get a solid experimental observation of Hell? But all too many
scientists can suppress their skeptical reflex at will. Why
don't chemists have a private cult which argues that
nonchemists go to hell, given that many are Christians anyway?}

{
 Questions like that don't have neat single-factor
answers. But I would argue that \textit{one} of the factors has to do
with assuming a \textit{productive} posture toward the truth, versus a
\textit{defensive} posture toward the truth.}

{
 When you are the Guardian of the Truth, you've got
nothing useful to contribute to the Truth \textit{but} your
guardianship of it. When you're trying to win the Nobel
Prize in chemistry by discovering the next benzene or buckyball,
someone who challenges the atomic theory isn't so much
a threat to your worldview as a waste of your time.}

{
 When you are a Guardian of the Truth, all you can do is try to
stave off the inevitable slide into entropy by zapping anything that
departs from the Truth. If there's some way to pump
against entropy, generate new true beliefs along with a little waste
heat, that same pump can keep the truth alive without secret police. In
chemistry you can replicate experiments and see for yourself---and that
keeps the precious truth alive without need of violence.}

{
 And it's not such a terrible threat if we make one
mistake somewhere---end up believing a little untruth for a little
while---because \textit{tomorrow} we can recover the lost ground.}

{
 But this whole trick only works because the experimental method is
a ``criterion of goodness'' which is
not a mere ``criterion of
comparison.'' Because experiments can recover the
truth without need of authority, they can also \textit{override}
authority and create new true beliefs where none existed before.}

{
 Where there are criteria of goodness that are not criteria of
comparison, there can exist \textit{changes} which are
\textit{improvements}, rather than \textit{threats.} Where there are
\textit{only} criteria of comparison, where there's no
way to move past authority, there's also no way to
resolve a disagreement between authorities. Except extermination. The
bigger guns win.}

{
 I don't mean to provide a grand overarching
single-factor view of history. I do mean to point out a deep
psychological difference between seeing your grand cause in life as
\textit{protecting, guarding, preserving}, versus \textit{discovering,
creating, improving}. Does the
``up'' direction of time point to
the past or the future? It's a distinction that shades
everything, casts tendrils everywhere.}

{
 This is why I've always insisted, for example,
that if you're going to start talking about
``AI ethics,'' you had better be
talking about how you are going to \textit{improve} on the current
situation using AI, rather than just keeping various things from going
wrong. Once you adopt criteria of mere comparison, you start losing
track of your ideals---lose sight of wrong and right, and start seeing
simply ``different'' and
``same.''}

{
 I would also argue that this basic psychological difference is one
of the reasons why an academic field that stops making active progress
tends to turn \textit{mean.} (At least by the refined standards of
science. \textit{Reputational} assassination is tame by historical
standards; most defensive-posture belief systems went for the real
thing.) If major shakeups don't arrive often enough to
regularly promote young scientists based on merit rather than
conformity, the field stops resisting the standard degeneration into
authority. When there's not many discoveries being
made, there's nothing left to do all day but witch-hunt
the heretics.}

{
 To get the best mental health benefits of the
discover/create/improve posture, you've got to
\textit{actually be making progress}, not just hoping for it.}

\myendsectiontext


\bigskip

{
 1. Zhuangzi and Thomas Merton, \textit{The Way of Chuang Tzu} (New
Directions Publishing, 1965).}

\mysection{Guardians of the Gene Pool}

{
 Like any educated denizen of the twenty-first century, you may
have heard of World War II. You may remember that Hitler and the Nazis
planned to carry forward a romanticized process of evolution, to breed
a new master race, supermen, stronger and smarter than anything that
had existed before. }

{
 Actually this is a common misconception. Hitler believed that the
Aryan superman \textit{had previously existed}{}---the Nordic
stereotype, the blond blue-eyed beast of prey---but had been
\textit{polluted} by mingling with impure races. There had been a
racial Fall from Grace.}

{
 It says something about the degree to which the concept of
\textit{progress} permeates Western civilization, that the one is told
about Nazi eugenics and hears ``They tried to breed a
superhuman.'' \textit{You,} dear reader---if
\textit{you} failed so hard that you endorsed coercive eugenics,
\textit{you} would try to create a superhuman. Because you locate your
ideals in your future, not in your past. Because you are
\textit{creative.} The thought of breeding back to some Nordic
archetype from a thousand years earlier would not even occur to you as
a possibility---what, just the \textit{Vikings?} That's
\textit{all?} If you failed hard enough to kill, you would damn well
try to reach heights never before reached, or what a waste it would all
be, eh? Well, that's one reason you're
not a Nazi, dear reader.}

{
 It says something about how difficult it is for the relatively
healthy to envision themselves in the shoes of the relatively sick,
that we are told of the Nazis, and distort the tale to make them
defective transhumanists.}

{
 It's the \textit{Communists} who were the
defective transhumanists. ``New Soviet
Man'' and all that. The Nazis were quite definitely
the bioconservatives of the tale.}

\myendsectiontext

\mysection{Guardians of Ayn Rand}

{
 For skeptics, the idea that reason can lead to a cult is absurd.
The characteristics of a cult are 180 degrees out of phase with reason.
But as I will demonstrate, not only can it happen, it has happened, and
to a group that would have to be considered the unlikeliest cult in
history. It is a lesson in what happens when the truth becomes more
important than the search for truth \ldots}

{\raggedleft
 {}---Michael Shermer, ``The Unlikeliest Cult in
History''\textsuperscript{1}
\par}


\bigskip

{
 ~}

{
 I think Michael Shermer is over-explaining Objectivism.
I'll get around to amplifying on that.}

{
 Ayn Rand's novels glorify technology, capitalism,
individual defiance of the System, limited government, private
property, selfishness. Her ultimate fictional hero, John Galt, was
{\textless}SPOILER{\textgreater} a scientist who invented a new form of
cheap renewable energy; but then refuses to give it to the world since
the profits will only be stolen to prop up corrupt
governments.{\textless}/SPOILER{\textgreater}}

{
 And then---somehow---it all turned into a moral and philosophical
``closed system'' with Ayn Rand at
the center. The term ``closed
system'' is not my own accusation;
it's the term the Ayn Rand Institute uses to describe
Objectivism. Objectivism is defined by the works of Ayn Rand. Now that
Rand is dead, Objectivism is closed. If you disagree with
Rand's works in any respect, you cannot be an
Objectivist.}

{
 Max Gluckman once said: ``A science is any
discipline in which the fool of this generation can go beyond the point
reached by the genius of the last generation.''
Science moves forward by slaying its heroes, as Newton fell to
Einstein. Every young physicist dreams of being the new champion that
future physicists will dream of dethroning.}

{
 Ayn Rand's philosophical idol was Aristotle. Now
maybe Aristotle was a hot young math talent 2,350 years ago, but math
has made noticeable progress since his day. Bayesian probability theory
is the quantitative logic of which Aristotle's
qualitative logic is a special case; but there's no
sign that Ayn Rand knew about Bayesian probability theory when she
wrote her magnum opus, \textit{Atlas Shrugged.} Rand wrote about
``rationality,'' yet failed to
familiarize herself with the modern research in heuristics and biases.
How can anyone claim to be a master rationalist, yet know nothing of
such elementary subjects?}

{
 ``Wait a minute,'' objects the
reader, ``that's not quite fair!
\textit{Atlas Shrugged} was published in 1957! Practically nobody knew
about Bayes back then.'' Bah. Next
you'll tell me that Ayn Rand died in 1982, and had no
chance to read \textit{Judgment Under Uncertainty: Heuristics and
Biases}, which was published that same year.}

{
 Science isn't fair. That's sorta
the point. An aspiring rationalist in 2007 starts with a huge advantage
over an aspiring rationalist in 1957. It's how we know
that progress has occurred.}

{
 To me the thought of voluntarily embracing a system explicitly
tied to the beliefs of one human being, who's
\textit{dead}, falls somewhere between the silly and the suicidal. A
computer isn't five years old before
it's obsolete.}

{
 The vibrance that Rand admired in science, in commerce, in every
railroad that replaced a horse-and-buggy route, in every skyscraper
built with \textit{new} architecture---it all comes from the principle
of \textit{surpassing the ancient masters.} How can there be science,
if the most knowledgeable scientist there will ever be, has already
lived? Who would raise the New York skyline that Rand admired so, if
the tallest building that would ever exist, had already been built?}

{
 And yet Ayn Rand acknowledged no superior, in the past, or in the
future yet to come. Rand, who began in admiring reason and
individuality, ended by ostracizing anyone who dared contradict her.
Shermer:}

{
 [Barbara] Branden recalled an evening when a friend of
Rand's remarked that he enjoyed the music of Richard
Strauss. ``When he left at the end of the evening, Ayn
said, in a reaction becoming increasingly typical, `Now
I understand why he and I can never be real soulmates. The distance in
our sense of life is too great.'''
Often she did not wait until a friend had left to make such remarks.}

{
 Ayn Rand changed over time, one suspects.}

{
 Rand grew up in Russia, and witnessed the Bolshevik revolution
firsthand. She was granted a visa to visit American relatives at the
age of 21, and she never returned. It's easy to hate
authoritarianism when you're the victim.
It's easy to champion the freedom of the individual,
when you are yourself the oppressed.}

{
 It takes a much stronger constitution to fear authority when
\textit{you} have the power. When people are looking to \textit{you}
for answers, it's harder to say ``What
the hell do I know about music? I'm a writer, not a
composer,'' or
``It's hard to see how liking a piece
of music can be untrue.''}

{
 When \textit{you're} the one crushing those who
dare offend you, the exercise of power somehow seems much more
\textit{justifiable} than when you're the one being
crushed. All sorts of excellent justifications somehow leap to mind.}

{
 Michael Shermer goes into detail on how he thinks that
Rand's philosophy ended up descending into cultishness.
In particular, Shermer says (it seems) that Objectivism failed because
Rand thought that certainty was possible, while science is never
certain. I can't back Shermer on that one. The atomic
theory of chemistry is pretty damned certain. But chemists
haven't become a cult.}

{
 Actually, I think Shermer's falling prey to
correspondence bias by supposing that there's any
particular correlation between Rand's philosophy and
the way her followers formed a cult. Every cause wants to be a cult.}

{
 Ayn Rand fled the Soviet Union, wrote a book about individualism
that a lot of people liked, got plenty of compliments, and formed a
coterie of admirers. Her admirers found nicer and nicer things to say
about her (happy death spiral), and she enjoyed it too much to tell
them to shut up. She found herself with the power to crush those of
whom she disapproved, and she didn't resist the
temptation of power.}

{
 Ayn Rand and Nathaniel Branden carried on a secret extramarital
affair. (With permission from both their spouses, which counts for a
lot in my view. If you want to turn that into a
``problem,'' you have to specify
that the spouses were \textit{unhappy}{}---and then
it's still not a matter for outsiders.) When Branden
was revealed to have ``cheated'' on
Rand with yet another woman, Rand flew into a fury and excommunicated
him. Many Objectivists broke away when news of the affair became
public.}

{
 Who stayed with Rand, rather than following Branden, or leaving
Objectivism altogether? Her \textit{strongest} supporters. Who
departed? The previous voices of moderation. (Evaporative cooling of
group beliefs.) Ever after, Rand's grip over her
remaining coterie was absolute, and no questioning was allowed.}

{
 The only extraordinary thing about the whole business, is how
ordinary it was.}

{
 You might think that a belief system which praised
``reason'' and
``rationality'' and
``individualism'' would have gained
some kind of special immunity, somehow \ldots ?}

{
 Well, it didn't.}

{
 It worked around as well as putting a sign saying
``Cold'' on a refrigerator that
wasn't plugged in.}

{
 The active effort required to resist the slide into entropy
wasn't there, and decay inevitably followed.}

{
 And if you call that the ``unlikeliest cult in
history,'' you're just calling
reality nasty names.}

{
 Let that be a lesson to all of us: Praising
``rationality'' counts for nothing.
Even saying ``You must justify your beliefs through
Reason, not by agreeing with the Great Leader'' just
runs a little automatic program that takes whatever the Great Leader
says and generates a justification that your fellow followers will view
as Reason-able.}

{
 So where is the true art of rationality to be found? Studying up
on the math of probability theory and decision theory. Absorbing the
cognitive sciences like evolutionary psychology, or heuristics and
biases. Reading history books \ldots}

{
 ``Study science, not just me!''
is probably the most important piece of advice Ayn Rand
should've given her followers and
didn't. There's no one human being who
ever lived, whose shoulders were broad enough to bear \textit{all} the
weight of a true science with many contributors.}

{
 It's noteworthy, I think, that Ayn
Rand's fictional heroes were architects and engineers;
John Galt, her ultimate, was a physicist; and yet Ayn Rand herself
wasn't a great scientist. As far as I know, she
wasn't particularly good at math. She could not aspire
to rival her own heroes. Maybe that's why she began to
lose track of the will to keep improving herself.}

{
 Now me, y'know, I admire Francis
Bacon's audacity, but I retain my ability to bashfully
confess, ``If I could go back in time, and somehow
make Francis Bacon understand the problem I'm currently
working on, his eyeballs would pop out of their sockets like champagne
corks and explode.''}

{
 I admire Newton's accomplishments. But my attitude
toward a woman's right to vote bars me from accepting
Newton as a moral paragon. Just as my knowledge of Bayesian probability
bars me from viewing Newton as the ultimate unbeatable source of
mathematical knowledge. And my knowledge of Special Relativity, paltry
and little-used though it may be, bars me from viewing Newton as the
ultimate authority on physics.}

{
 Newton couldn't realistically have discovered any
of the ideas I'm lording over him---\textit{but
progress isn't fair! That's the
point!}}

{
 Science has heroes, but no gods. The great Names are not our
superiors, or even our rivals; they are passed milestones on our road.
And the most important milestone is the hero yet to come.}

{
 To be one more milestone in humanity's road is the
best that can be said of anyone; but this seemed too lowly to please
Ayn Rand. And that is how she became a mere Ultimate Prophet.}

\myendsectiontext


\bigskip

{
 1. Michael Shermer, ``The Unlikeliest Cult in
History,'' \textit{Skeptic} 2, no. 2 (1993): 74--81,
http://www.2think.org/02\_2\_she.shtml.}

\mysection{Two Cult Koans}

{
 A novice rationalist studying under the master Ougi was rebuked by
a friend who said, ``You spend all this time listening
to your master, and talking of
`rational' this and
`rational' that---you have fallen into a
cult!'' }

{
 The novice was deeply disturbed; he heard the words,
``You have fallen into a cult!''
resounding in his ears as he lay in bed that night, and even in his
dreams.}

{
 The next day, the novice approached Ougi and related the events,
and said, ``Master, I am constantly consumed by worry
that this is all really a cult, and that your teachings are only
dogma.''}

{
 Ougi replied, ``If you find a hammer lying in the
road and sell it, you may ask a low price or a high one. But if you
keep the hammer and use it to drive nails, who can doubt its
worth?''}

{
 The novice said, ``See, now
that's just the sort of thing I worry about---your
mysterious Zen replies.''}

{
 Ougi said, ``Fine, then, I will speak more
plainly, and lay out perfectly reasonable arguments which demonstrate
that you have not fallen into a cult. But first you have to wear this
silly hat.''}

{
 Ougi gave the novice a huge brown ten-gallon cowboy hat.}

{
 ``Er, master \ldots'' said the
novice.}

{
 ``When I have explained everything to
you,'' said Ougi, ``you will see why
this was necessary. Or otherwise, you can continue to lie awake nights,
wondering whether this is a cult.''}

{
 The novice put on the cowboy hat.}

{
 Ougi said, ``How long will you repeat my words
and ignore the meaning? Disordered thoughts begin as feelings of
attachment to preferred conclusions. You are too anxious about your
self-image as a rationalist. You came to me to seek reassurance. If you
had been truly curious, not knowing one way or the other, you would
have thought of ways to resolve your doubts. Because you needed to
resolve your cognitive dissonance, you were willing to put on a silly
hat. If I had been an evil man, I could have made you pay a hundred
silver coins. When you concentrate on a real-world question, the worth
or worthlessness of your understanding will soon become apparent. You
are like a swordsman who keeps glancing away to see if anyone might be
laughing at him---''}

{
 ``All \textit{right},'' said
the novice.}

{
 ``You asked for the long
version,'' said Ougi.}

{
 This novice later succeeded Ougi and became known as Ni no Tachi.
Ever after, he would not allow his students to cite his words in their
debates, saying, ``Use the techniques and do not
mention them.''}

{
 A novice rationalist approached the master Ougi and said,
``Master, I worry that our rationality dojo is \ldots
well \ldots a little cultish.''}

{
 ``That is a grave concern,''
said Ougi.}

{
 The novice waited a time, but Ougi said nothing more.}

{
 So the novice spoke up again: ``I mean,
I'm sorry, but having to wear these robes, and the
hood---it just seems like we're the bloody Freemasons
or something.''}

{
 ``Ah,'' said Ougi,
``the robes and trappings.''}

{
 ``Well, \textit{yes} the robes and
trappings,'' said the novice. ``It
just seems terribly irrational.''}

{
 ``I will address all your
concerns,'' said the master, ``but
first you must put on this silly hat.'' And Ougi drew
out a wizard's hat, embroidered with crescents and
stars.}

{
 The novice took the hat, looked at it, and then burst out in
frustration: ``\textit{How can this possibly
help?}''}

{
 ``Since you are so concerned about the
interactions of clothing with probability theory,''
Ougi said, ``it should not surprise you that you must
wear a special hat to understand.''}

{
 When the novice attained the rank of grad student, he took the
name Bouzo and would only discuss rationality while wearing a clown
suit.}

\myendsectiontext

\mysection{Asch's Conformity Experiment}

\mygraphics{Rationality20From20AI20to20Zombies2020Eliezer20Yudkowsky-img145.jpg}


{
 ~}

{
 Solomon Asch, with experiments originally carried out in the 1950s
and well-replicated since, highlighted a phenomenon now known as
``conformity.'' In the classic
experiment, a subject sees a puzzle like the one in the nearby diagram:
Which of the lines A, B, and C is the same size as the line X? Take a
moment to determine your own answer \ldots}

{
 The gotcha is that the subject is seated alongside a number of
other people looking at the diagram---seemingly other subjects,
actually confederates of the experimenter. The other
``subjects'' in the experiment, one
after the other, say that line C seems to be the same size as X. The
real subject is seated next-to-last. How many people, placed in this
situation, would say ``C''---giving
an obviously incorrect answer that agrees with the unanimous answer of
the other subjects? What do you think the percentage would be?}

{
 Three-quarters of the subjects in Asch's
experiment gave a ``conforming''
answer at least once. A third of the subjects conformed more than half
the time.}

{
 Interviews after the experiment showed that while most subjects
claimed to have not really believed their conforming answers, some said
they'd really thought that the conforming option was
the correct one.}

{
 Asch was disturbed by these results:}

{
 That we have found the tendency to conformity in our society so
strong \ldots is a matter of concern. It raises questions about our ways
of education and about the values that guide our
conduct.\textsuperscript{1}}

{
 It is not a trivial question whether the subjects of
Asch's experiments behaved \textit{irrationally.}
Robert Aumann's Agreement Theorem shows that honest
Bayesians cannot agree to disagree---if they have common knowledge of
their probability estimates, they have the same probability estimate.
Aumann's Agreement Theorem was proved more than twenty
years after Asch's experiments, but it only formalizes
and strengthens an intuitively obvious point---other
people's beliefs are often legitimate evidence.}

{
 If you were looking at a diagram like the one above, but you knew
\textit{for a fact} that the other people in the experiment were honest
and seeing the same diagram as you, and three other people said that C
was the same size as X, then what are the odds that \textit{only you}
are the one who's right? I lay claim to no advantage of
\textit{visual} reasoning---I don't think
I'm better than an average human at judging whether two
lines are the same size. In terms of individual rationality, I hope I
would notice my own severe confusion and then assign {\textgreater}50\%
probability to the majority vote.}

{
 In terms of group rationality, seems to me that the proper thing
for an honest rationalist to say is, ``How surprising,
it \textit{looks} to me like B is the same size as X. But if
we're all looking at the same diagram and reporting
honestly, I have no reason to believe that my assessment is better than
yours.'' The last sentence is
important---it's a much weaker claim of disagreement
than, ``Oh, \textit{I} see the optical illusion---I
understand why you think it's C, of course, but the
real answer is B.''}

{
 So the conforming subjects in these experiments are not
\textit{automatically} convicted of irrationality, based on what
I've described so far. But as you might expect, the
devil is in the details of the experimental results. According to a
meta-analysis of over a hundred replications by Smith and
Bond:\textsuperscript{2}}

{
 Conformity increases strongly up to 3 confederates, but
doesn't increase further up to 10--15 confederates. If
people are conforming rationally, then the opinion of 15 other subjects
should be substantially stronger evidence than the opinion of 3 other
subjects.}

{
 Adding a single dissenter---just one other person who gives the
correct answer, or even an incorrect answer that's
different from the group's incorrect answer---reduces
conformity \textit{very} sharply, down to 5--10\% of subjects. If
you're applying some intuitive version of
Aumann's Agreement to think that when 1 person
disagrees with 3 people, the 3 are probably right, then in most cases
you should be equally willing to think that 2 people will disagree with
6 people. (Not automatically true, but true \textit{ceteris paribus}.)
On the other hand, if you've got people who are
emotionally nervous about being the odd one out, then
it's easy to see how a single other person who agrees
with you, or even a single other person who disagrees with the group,
would make you much less nervous.}

{
 Unsurprisingly, subjects in the one-dissenter condition did not
think their nonconformity had been influenced or enabled by the
dissenter. Like the 90\% of drivers who think they're
above-average in the top 50\%, some of them may be right about this,
but not all. People are not self-aware of the causes of their
conformity or dissent, which weighs against trying to argue them as
manifestations of rationality. For example, in the hypothesis that
people are socially-rationally choosing to lie in order to not stick
out, it appears that (at least some) subjects in the one-dissenter
condition do not consciously anticipate the
``conscious strategy'' they would
employ when faced with unanimous opposition.}

{
 When the single dissenter suddenly switched to \textit{conforming
to the group}, subjects' conformity rates went back up
to just as high as in the no-dissenter condition. Being the first
dissenter is a valuable (and costly!) social service, but
you've got to keep it up.}

{
 Consistently within and across experiments, all-female groups (a
female subject alongside female confederates) conform significantly
more often than all-male groups. Around one-half the women conform more
than half the time, versus a third of the men. If you argue that the
average subject is rational, then apparently women are too agreeable
and men are too disagreeable, so neither group is actually
\textit{rational} \ldots}

{
 Ingroup-outgroup manipulations (e.g., a handicapped subject
alongside other handicapped subjects) similarly show that conformity is
significantly higher among members of an ingroup.}

{
 Conformity is lower in the case of blatant diagrams, like the one
at the beginning of this essay, versus diagrams where the errors are
more subtle. This is hard to explain if (all) the subjects are making a
socially rational decision to avoid sticking out.}

{
 Paul Crowley reminds me to note that when subjects can respond in
a way that will not be seen by the group, conformity also drops, which
also argues against an Aumann interpretation.}

\myendsectiontext


\bigskip

{
 1. Solomon E. Asch, ``Studies of Independence and
Conformity: A Minority of One Against a Unanimous
Majority,'' \textit{Psychological Monographs} 70
(1956).}

{
 2. Rod Bond and Peter B. Smith, ``Culture and
Conformity: A Meta-Analysis of Studies Using Asch's
(1952b, 1956) Line Judgment Task,''
\textit{Psychological Bulletin} 119 (1996): 111--137.}

\mysection{On Expressing Your Concerns}

{
 The scary thing about Asch's conformity
experiments is that you can get many people to say black is white, if
you put them in a room full of other people saying the same thing. The
hopeful thing about Asch's conformity experiments is
that a single dissenter tremendously drove down the rate of conformity,
even if the dissenter was only giving a different wrong answer. And the
\textit{wearisome} thing is that dissent was not \textit{learned} over
the course of the experiment---when the single dissenter started siding
with the group, rates of conformity rose back up. }

{
 Being a voice of dissent can bring real benefits to the group. But
it also (famously) has a cost. And then you have to keep it up. Plus
you could be wrong.}

{
 I recently had an interesting experience wherein I began
discussing a project with two people who had previously done some
planning on their own. I thought they were being too optimistic and
made a number of safety-margin-type suggestions for the project. Soon a
fourth guy wandered by, who was providing one of the other two with a
ride home, and began making suggestions. At this point I had a sudden
insight about how groups become overconfident, because whenever I
raised a possible problem, the fourth guy would say,
``Don't worry, I'm
sure we can handle it!'' or something similarly
reassuring.}

{
 An individual, working alone, will have natural doubts. They will
think to themselves ``Can I really do
XYZ?,'' because there's nothing
impolite about doubting your \textit{own} competence. But when two
unconfident people form a group, it is polite to say nice and
reassuring things, and impolite to question the other
person's competence. Together they become more
optimistic than either would be on their own, each
one's doubts quelled by the other's
seemingly confident reassurance, not realizing that the other person
initially had the same inner doubts.}

{
 The most fearsome possibility raised by Asch's
experiments on conformity is the specter of everyone agreeing with the
group, swayed by the confident voices of others, careful not to let
their own doubts show---not realizing that others are suppressing
similar worries. This is known as ``pluralistic
ignorance.''}

{
 Robin Hanson and I have a long-running debate over when, exactly,
aspiring rationalists should dare to disagree. I tend toward the widely
held position that you have no real choice but to form your own
opinions. Robin Hanson advocates a more iconoclastic position, that
\textit{you}{}---not just other people---should consider that others
may be wiser. Regardless of our various disputes, we both agree that
Aumann's Agreement Theorem extends to imply that common
knowledge of a factual disagreement shows \textit{someone} must be
irrational. Despite the funny looks we've gotten,
we're sticking to our guns about modesty: Forget what
everyone tells you about individualism, you \textit{should} pay
attention to what other people think.}

{
 Ahem. The point is that, for rationalists, disagreeing with the
group is serious business. You can't wave it off with
``Everyone is entitled to their own
opinion.''}

{
 I think the most important lesson to take away from
Asch's experiments is to distinguish
``expressing concern'' from
``disagreement.'' Raising a point
that others haven't voiced is not a promise to disagree
with the group at the end of its discussion.}

{
 The ideal Bayesian's process of convergence
involves sharing evidence that is unpredictable to the listener. The
Aumann agreement result holds only for \textit{common knowledge}, where
you know, I know, you know I know, etc. Hanson's post
or paper on ``We Can't Foresee to
Disagree'' provides a picture of how strange it would
look to watch ideal rationalists converging on a probability estimate;
it doesn't look anything like two bargainers in a
marketplace converging on a price.}

{
 Unfortunately, there's not much difference
\textit{socially} between ``expressing
concerns'' and
``disagreement.'' A group of
rationalists might agree to pretend there's a
difference, but it's not how human beings are really
wired. Once you speak out, you've committed a socially
irrevocable act; you've become the nail sticking up,
the discord in the comfortable group harmony, and you
can't undo that. Anyone insulted by a concern you
expressed about their competence to successfully complete task XYZ,
will probably hold just as much of a grudge afterward if you say
``No problem, I'll go along with the
group'' at the end.}

{
 Asch's experiment shows that the power of dissent
to inspire others is real. Asch's experiment shows that
the power of conformity is real. If everyone refrains from voicing
their private doubts, that will indeed lead groups into madness. But
history abounds with lessons on the price of being the first, or even
the second, to say that the Emperor has no clothes. Nor are people
hardwired to distinguish ``expressing a
concern'' from ``disagreement even
with common knowledge''; this distinction is a
rationalist's artifice. If you read the more cynical
brand of self-help books (e.g., Machiavelli's
\textit{The Prince}) they will advise you to mask your nonconformity
entirely, \textit{not} voice your concerns first and then agree at the
end. If you perform the group service of being the one who gives voice
to the obvious problems, don't expect the group to
thank you for it.}

{
 These are the costs and the benefits of dissenting---whether you
``disagree'' or just
``express concern''---and the
decision is up to you.}

\myendsectiontext

\mysection{Lonely Dissent}

{
 Asch's conformity experiment showed that the
presence of a single dissenter tremendously reduced the incidence of
``conforming'' wrong answers.
Individualism is easy, experiment shows, when you have company in your
defiance. Every other subject in the room, except one, says that black
is white. You become the second person to say that black is black. And
it feels glorious: the two of you, lonely and defiant rebels, against
the world! (Followup interviews showed that subjects in the
one-dissenter condition expressed strong feelings of camaraderie with
the dissenter---though, of course, they didn't think
the presence of the dissenter had influenced their own nonconformity.)
}

{
 But you can only \textit{join} the rebellion, after someone,
somewhere, becomes the \textit{first} to rebel. Someone has to say that
black is black after hearing \textit{everyone} else, one after the
other, say that black is white. And that---experiment shows---is a
\textit{lot harder.}}

{
 Lonely dissent doesn't feel like going to school
dressed in black. It feels like going to school wearing a clown suit.}

{
 That's the difference between \textit{joining the
rebellion} and \textit{leaving the pack}.}

{
 If there's one thing I can't
stand, it's fakeness---you may have noticed this. Well,
lonely dissent has got to be one of the most commonly, most
ostentatiously faked characteristics around. Everyone wants to be an
iconoclast.}

{
 I don't mean to degrade the act of joining a
rebellion. There are rebellions worth joining. It does take courage to
brave the disapproval of your peer group, or perhaps even worse, their
shrugs. Needless to say, going to a rock concert is not rebellion. But,
for example, vegetarianism is. I'm not a vegetarian
myself, but I respect people who are, because I expect it takes a
noticeable amount of quiet courage to tell people that hamburgers
won't work for dinner. (Albeit that in the Bay Area,
people ask as a matter of routine.)}

{
 Still, if you tell people that you're a
vegetarian, they'll think they understand your motives
(even if they don't). They may disagree. They may be
offended if you manage to announce it proudly enough, or for that
matter, they may be offended just because they're
easily offended. But they know how to relate to you.}

{
 When someone wears black to school, the teachers and the other
children understand the role thereby being assumed in their society.
It's Outside the System---in a very standard way that
everyone recognizes and understands. Not, y'know,
\textit{actually} outside the system. It's a Challenge
to Standard Thinking, of a standard sort, so that people indignantly
say ``I can't understand why
you---'' but don't have to actually
think any thoughts they had not thought before. As the saying goes,
``Has any of the `subversive
literature' you've read caused you to
modify any of your political views?''}

{
 What takes \textit{real} courage is braving the outright
\textit{incomprehension} of the people around you, when you do
something that \textit{isn't} Standard Rebellion \#37,
something for which they lack a ready-made script. They
don't hate you for a rebel, they just think
you're, like, weird, and turn away. This prospect
generates a much deeper fear. It's the difference
between explaining vegetarianism and explaining cryonics. There are
other cryonicists in the world, somewhere, but they
aren't there next to you. You have to explain it,
alone, to people who just think it's \textit{weird}.
Not forbidden, but outside bounds that people don't
even think about. You're going to get your head frozen?
You think that's going to stop you from dying? What do
you mean, brain information? Huh? What? Are you \textit{crazy?}}

{
 I'm tempted to essay a post facto explanation in
evolutionary psychology: You could get together with a small group of
friends and walk away from your hunter-gatherer band, but having to go
it \textit{alone} in the forests was probably a death sentence---at
least reproductively. We don't reason this out
explicitly, but that is not the nature of evolutionary psychology.
Joining a rebellion that everyone knows about is scary, but nowhere
near as scary as doing something really differently. Something that in
ancestral times might have ended up, not with the band splitting, but
with you being driven out alone.}

{
 As the case of cryonics testifies, the fear of thinking
\textit{really} different is stronger than the fear of death.
Hunter-gatherers had to be ready to face death on a routine basis,
hunting large mammals, or just walking around in a world that contained
predators. They needed that courage in order to live. Courage to defy
the tribe's standard ways of thinking, to entertain
thoughts that seem truly weird---well, that probably
didn't serve its bearers as well. We
don't reason this out explicitly;
that's not how evolutionary psychology works. We human
beings are just built in such fashion that many more of us go skydiving
than sign up for cryonics.}

{
 And that's not even the highest courage.
There's more than one cryonicist in the world. Only
Robert Ettinger had to say it \textit{first.}}

{
 To be a \textit{scientific} revolutionary, you've
got to be the first person to contradict what everyone else you know is
thinking. This is not the only route to scientific greatness; it is
rare even among the great. No one can become a scientific revolutionary
by trying to imitate revolutionariness. You can only get there by
pursuing the correct answer in all things, whether the correct answer
is revolutionary or not. But if, in the due course of time---if, having
absorbed all the power and wisdom of the knowledge that has already
accumulated---if, after all that and a dose of sheer luck, you find
your pursuit of mere correctness taking you into new territory \ldots
\textit{then} you have an opportunity for your courage to fail.}

{
 This is the true courage of lonely dissent, which every damn rock
band out there tries to fake.}

{
 Of course not everything that takes courage is a good idea. It
would take courage to walk off a cliff, but then you would just go
splat.}

{
 The \textit{fear} of lonely dissent is a hindrance to good ideas,
but not every dissenting idea is good. See also Robin
Hanson's Against Free Thinkers. Most of the difficulty
in having a new true scientific thought is in the
``true'' part.}

{
 It really isn't \textit{necessary} to be different
for the sake of being different. If you do things differently only when
you see an overwhelmingly good reason, you will have more than enough
trouble to last you the rest of your life.}

{
 There are a few genuine packs of iconoclasts around. The Church of
the SubGenius, for example, seems to genuinely aim at
\textit{confusing} the mundanes, not merely offending them. And there
are islands of genuine tolerance in the world, such as science fiction
conventions. There \textit{are} certain people who have no fear of
departing the pack. Many fewer such people really exist, than imagine
themselves rebels; but they do exist. And yet scientific
revolutionaries are tremendously rarer. Ponder that.}

{
 Now \textit{me}, you know, I \textit{really am} an iconoclast.
Everyone thinks they are, but with me it's
\textit{true}, you see. I would \textit{totally} have worn a clown suit
to school. My serious conversations were with books, not with other
children.}

{
 But if you think you would \textit{totally} wear that clown suit,
then don't be too proud of that either! It just means
that you need to make an effort in the \textit{opposite direction} to
avoid dissenting too easily. That's what I have to do,
to correct for my own nature. Other people do have reasons for thinking
what they do, and ignoring that completely is as bad as being afraid to
contradict them. You wouldn't want to end up as a free
thinker. It's not a \textit{virtue}, you see---just a
bias either way.}

\myendsectiontext

\mysection{Cultish Countercultishness}

{
 In the modern world, joining a cult is probably one of the worse
things that can happen to you. The best-case scenario is that
you'll end up in a group of sincere but deluded people,
making an honest mistake but otherwise well-behaved, and
you'll spend a lot of time and money but end up with
nothing to show. Actually, that could describe any failed Silicon
Valley startup. Which is supposed to be a hell of a harrowing
experience, come to think. So yes, very scary. }

{
 Real cults are vastly worse. ``Love
bombing'' as a recruitment technique, targeted at
people going through a personal crisis. Sleep deprivation. Induced
fatigue from hard labor. Distant communes to isolate the recruit from
friends and family. Daily meetings to confess impure thoughts.
It's not unusual for cults to take \textit{all} the
recruit's money---life savings plus weekly
paycheck---forcing them to depend on the cult for food and clothing.
Starvation as a punishment for disobedience. Serious brainwashing and
serious harm.}

{
 With all that taken into account, I should probably sympathize
more with people who are terribly nervous, embarking on some
odd-seeming endeavor, that \textit{they might be joining a cult.} It
should not grate on my nerves. Which it does.}

{
 Point one: ``Cults'' and
``non-cults'' aren't
separated natural kinds like dogs and cats. If you look at any list of
cult characteristics, you'll see items that could
easily describe political parties and
corporations---``group members encouraged to distrust
outside criticism as having hidden motives,''
``hierarchical authoritative
structure.'' I've written on group
failure modes like group polarization, happy death spirals,
uncriticality, and evaporative cooling, all of which seem to feed on
each other. When these failures swirl together and meet, they combine
to form a Super-Failure stupider than any of the parts, like Voltron.
But this is not a cult \textit{essence}; it is a cult
\textit{attractor.}}

{
 Dogs are born with dog DNA, and cats are born with cat DNA. In the
current world, there is no in-between. (Even with genetic manipulation,
it wouldn't be as simple as creating an organism with
half dog genes and half cat genes.) It's not like
there's a mutually reinforcing set of
dog-characteristics, which an individual cat can wander halfway into
and become a semidog.}

{
 The human mind, as it thinks about categories, seems to prefer
essences to attractors. The one wishes to say ``It is
a cult'' or ``It is not a
cult,'' and then the task of classification is over
and done. If you observe that Socrates has ten fingers, wears clothes,
and speaks fluent Greek, then you can say ``Socrates
is human'' and from there deduce
``Socrates is vulnerable to
hemlock'' without doing specific blood tests to
confirm his mortality. You have decided Socrates's
humanness once and for all.}

{
 But if you observe that a certain group of people seems to exhibit
ingroup-outgroup polarization and see a positive halo effect around
their Favorite Thing Ever---which could be Objectivism, or
vegetarianism, or neural networks{}---you cannot, \textit{from the
evidence gathered so far,} deduce whether they have achieved
uncriticality. You cannot deduce whether their main idea is true, or
false, or genuinely useful but not quite as useful as they think.
\textit{From the information gathered so far}, you cannot deduce
whether they are otherwise polite, or if they will lure you into
isolation and deprive you of sleep and food. The characteristics of
cultness are not all present or all absent.}

{
 If you look at online arguments over ``X is a
cult,'' ``X is not a
cult,'' then one side goes through an online list of
cult characteristics and finds one that applies and says
``Therefore it is a cult!'' And the
defender finds a characteristic that does not apply and says
``Therefore it is not a cult!''}

{
 You cannot build up an accurate picture of a
group's reasoning dynamic using this kind of
essentialism. You've got to pay attention to individual
characteristics individually.}

{
 Furthermore, reversed stupidity is not intelligence. If
you're interested in the central \textit{idea}, not
just the implementation group, then smart ideas can have stupid
followers. Lots of New Agers talk about ``quantum
physics'' but this is no strike against quantum
physics. Of course stupid ideas can also have stupid followers. Along
with binary essentialism goes the idea that if you infer that a group
is a ``cult,'' therefore their
beliefs must be false, because false beliefs are characteristic of
cults, just like cats have fur. If you're interested in
the idea, then look at the idea, not the people. Cultishness is a
characteristic of \textit{groups} more than \textit{hypotheses.}}

{
 The second error is that when people nervously ask,
``This isn't a cult, is
it?,'' it sounds to me like they're
seeking \textit{reassurance of rationality.} The notion of a
rationalist not getting too attached to their self-image as a
rationalist deserves its own essay (though see Twelve Virtues, Why
Truth? And \ldots, and Two Cult Koans). But even without going into
detail, surely one can see that \textit{nervously seeking reassurance}
is not the best frame of mind in which to evaluate questions of
rationality. You will not be genuinely curious or think of ways to
fulfill your doubts. Instead, you'll find some online
source which says that cults use sleep deprivation to control people,
you'll notice that Your-Favorite-Group
doesn't use sleep deprivation, and
you'll conclude ``It's
not a cult. Whew!'' If it doesn't
have fur, it must not be a cat. Very reassuring.}

{
 But Every Cause Wants To Be A Cult, whether the cause itself is
wise or foolish. The ingroup-outgroup dichotomy etc. are part of human
nature, not a special curse of mutants. Rationality is the exception,
not the rule. You have to put forth a constant effort to maintain
rationality against the natural slide into entropy. If you decide
``It's not a cult!''
and sigh with relief, then you will not put forth a continuing effort
to push back \textit{ordinary} tendencies toward cultishness.
You'll decide the cult-essence is absent, and stop
pumping against the entropy of the cult-attractor.}

{
 If you are terribly nervous about cultishness, then you will want
to deny any hint of any characteristic that resembles a cult. But
\textit{any} group with a goal seen in a positive light is at risk for
the halo effect, and will have to pump against entropy to avoid an
affective death spiral. This is true even for ordinary institutions
like political parties---people who think that
``liberal values'' or
``conservative values'' can cure
cancer, etc. It is true for Silicon Valley startups, both failed and
successful. It is true of Mac users and of Linux users. The halo effect
doesn't become okay just because everyone does it; if
everyone walks off a cliff, you wouldn't too. The error
in reasoning is to be fought, not tolerated. But if
you're too nervous about ``Are you
\textit{sure} this isn't a cult?''
then you will be reluctant to see \textit{any} sign of cultishness,
because that would imply you're in a cult, and
\textit{It's not a cult!!} So you won't
see the current battlefields where the \textit{ordinary} tendencies
toward cultishness are creeping forward, or being pushed back.}

{
 The third mistake in nervously asking ``This
isn't a cult, is it?'' is that, I
strongly suspect, the \textit{nervousness} is there for entirely the
wrong reasons.}

{
 Why is it that groups which praise their Happy Thing to the stars,
encourage members to donate all their money and work in voluntary
servitude, and run private compounds in which members are kept tightly
secluded, are called ``religions''
rather than ``cults'' once
they've been around for a few hundred years?}

{
 Why is it that most of the people who nervously ask of cryonics,
``This isn't a cult, is
it?'' would not be equally nervous about attending a
Republican or Democrat political rally? Ingroup-outgroup dichotomies
and happy death spirals can happen in political discussion, in
mainstream religions, in sports fandom. If the \textit{nervousness}
came from fear of \textit{rationality errors}, people would ask
``This isn't an ingroup-outgroup
dichotomy, is it?'' about Democrat or Republican
political rallies, in just the same fearful tones.}

{
 There's a legitimate reason to be less fearful of
Libertarianism than of a flying-saucer cult, because Libertarians
don't have a reputation for employing sleep deprivation
to convert people. But cryonicists don't have a
reputation for using sleep deprivation, either. So why be any more
worried about having your head frozen after you stop breathing?}

{
 I suspect that the \textit{nervousness} is not the fear of
believing falsely, or the fear of physical harm. It is the fear of
lonely dissent. The nervous feeling that subjects get in
Asch's conformity experiment, when all the other
subjects (actually confederates) say one after another that line C is
the same size as line X, and it looks to the subject like line B is the
same size as line X. The fear of leaving the pack.}

{
 That's why groups whose beliefs have been around
long enough to seem ``normal''
don't inspire the same nervousness as
``cults,'' though some mainstream
religions may also take all your money and send you to a monastery.
It's why groups like political parties, that are
strongly liable for rationality errors, don't inspire
the same nervousness as ``cults.''
The word ``cult''
isn't being used to symbolize rationality errors,
it's being used as a label for something that
\textit{seems weird.}}

{
 Not every change is an improvement, but every improvement is
necessarily a change. That which you want to do better, you have no
choice but to do differently. Common wisdom does embody a fair amount
of, well, actual wisdom; yes, it makes sense to require an extra burden
of proof for weirdness. But the \textit{nervousness}
isn't that kind of deliberate, rational consideration.
It's the fear of believing something that will make
your friends look at you really oddly. And so people ask
``This isn't a \textit{cult}, is
it?'' in a tone that they would never use for
attending a political rally, or for putting up a gigantic Christmas
display.}

{
 \textit{That's} the part that bugs me.}

{
 It's as if, as soon as you believe anything that
your ancestors did not believe, the Cult Fairy comes down from the sky
and infuses you with the Essence of Cultness, and the next thing you
know, you're all wearing robes and chanting. As if
``weird'' beliefs are the
\textit{direct cause} of the problems, never mind the sleep deprivation
and beatings. The harm done by cults---the Heaven's
Gate suicide and so on---just goes to show that everyone with an odd
belief is crazy; the first and foremost characteristic of
``cult members'' is that they are
Outsiders with Peculiar Ways.}

{
 Yes, socially unusual belief puts a group at risk for
ingroup-outgroup thinking and evaporative cooling and other problems.
But the unusualness is a risk factor, not a disease in itself. Same
thing with having a goal that you think is worth accomplishing. Whether
or not the belief is true, having a nice goal always puts you at risk
of the happy death spiral. But that makes lofty goals a risk factor,
not a disease. Some goals are genuinely worth pursuing.}

{
 On the other hand, I see no legitimate reason for sleep
deprivation or threatening dissenters with beating, full stop. When a
group does this, then whether you call it
``cult'' or
``not-cult,'' you have directly
answered the pragmatic question of whether to join.}

{
 Problem four: The fear of lonely dissent is something that
\textit{cults themselves} exploit. Being afraid of your friends looking
at you disapprovingly is \textit{exactly the effect that real cults use
to convert and keep members---}surrounding converts with wall-to-wall
agreement among cult believers.}

{
 The fear of strange ideas, the impulse to conformity, has no doubt
warned many potential victims away from flying-saucer cults. When
you're out, it keeps you out. But when
you're \textit{in}, it keeps you \textit{in.}
Conformity just glues you to wherever you are, whether
that's a good place or a bad place.}

{
 The one wishes there was some way they could be \textit{sure} that
they weren't in a
``cult.'' Some definite, crushing
rejoinder to people who looked at them funny. Some way they could know
once and for all that they were doing the right thing, without these
constant doubts. I believe that's called
``need for closure.'' And---of
course---cults exploit that, too.}

{
 Hence the phrase, ``Cultish
countercultishness.''}

{
 Living with doubt is not a virtue---the purpose of every doubt is
to annihilate itself in success or failure, and a doubt that just hangs
around accomplishes nothing. But sometimes a doubt does take a while to
annihilate itself. Living with a stack of currently unresolved doubts
is an unavoidable fact of life for rationalists. Doubt
shouldn't be scary. Otherwise you're
going to have to choose between living one heck of a hunted life, or
one heck of a stupid one.}

{
 If you really, genuinely can't figure out whether
a group is a ``cult,'' then
you'll just have to choose under conditions of
uncertainty. That's what decision theory is all about.}

{
 Problem five: Lack of strategic thinking.}

{
 I know people who are cautious around Singularitarianism, and
they're \textit{also} cautious around political parties
and mainstream religions. \textit{Cautious,} not nervous or defensive.
These people can see at a glance that Singularitarianism is obviously
not a full-blown cult with sleep deprivation etc. But they worry that
Singularitarianism will \textit{become} a cult, because of risk factors
like turning the concept of a powerful AI into a Super Happy Agent (an
agent defined primarily by agreeing with any nice thing said about it).
Just because something isn't a cult now,
doesn't mean it won't become a cult in
the future. Cultishness is an attractor, not an essence.}

{
 Does \textit{this} kind of caution annoy me? Hell no. I spend a
lot of time worrying about that scenario myself. I try to place my Go
stones in advance to block movement in that direction. Hence, for
example, the series of essays on cultish failures of reasoning.}

{
 People who talk about
``rationality'' also have an added
risk factor. Giving people advice about how to think is an inherently
dangerous business. But it is a \textit{risk factor}, not a
\textit{disease.}}

{
 Both of my favorite Causes are at-risk for cultishness. Yet
somehow, I get asked ``Are you sure this
isn't a cult?'' a lot more often when
I talk about powerful AIs, than when I talk about probability theory
and cognitive science. I don't know if one risk factor
is higher than the other, but I know which one \textit{sounds weirder}
\ldots}

{
 Problem \#6 with asking ``This
isn't a cult, is it?'' \ldots}

{
 Just the question itself places me in a very annoying sort of
Catch-22. An actual Evil Guru would surely use the
one's nervousness against them, and design a plausible
elaborate argument explaining Why This Is Not A Cult, and the one would
be eager to accept it. Sometimes I get the impression that this is what
people \textit{want} me to do! Whenever I try to write about
cultishness and how to avoid it, I keep feeling like
I'm giving in to that flawed desire---that I am, in the
end, providing people with \textit{reassurance.} Even when I tell
people that a constant fight against entropy is required.}

{
 It feels like I'm making myself a first dissenter
in Asch's conformity experiment, telling people,
``Yes, line X really is the same as line B,
it's okay for you to say so too.''
They shouldn't need to ask! Or, even worse, it feels
like I'm presenting an elaborate argument for Why This
Is Not A Cult. It's a \textit{wrong question.}}

{
 Just look at the group's reasoning processes for
yourself, and decide for yourself whether it's
something you want to be part of, once you get rid of the fear of
weirdness. It is your own responsibility to stop yourself from thinking
cultishly, no matter which group you currently happen to be operating
in.}

{
 Once someone asks ``This isn't a
cult, is it?'' then no matter how I answer, I always
feel like I'm defending something. I do not like this
feeling. It is not the function of a Bayesian Master to give
reassurance, nor of rationalists to defend.}

{
 Cults feed on groupthink, nervousness, desire for reassurance. You
cannot make nervousness go away by wishing, and false self-confidence
is even worse. But so long as someone needs reassurance---even
reassurance about being a rationalist---that will always be a flaw in
their armor. A skillful swordsman focuses on the target, rather than
glancing away to see if anyone might be laughing. When you know what
you're trying to do and why, you'll
know whether you're getting it done or not, and whether
a group is helping you or hindering you.}

{
 (PS: If the one comes to you and says, ``Are you
\textit{sure} this isn't a cult?,''
don't try to explain all these concepts in one breath.
You're underestimating inferential distances. The one
will say, ``Aha, so you're
\textit{admitting} you're a cult!''
or ``Wait, you're saying I
shouldn't worry about joining
cults?'' or ``So \ldots the fear of
cults is cultish? That sounds awfully cultish to
me.'' So the last annoyance factor---\#7 if
you're keeping count---is that all of this is such a
long story to explain.)}

\myendsectiontext


\chapter{Letting Go}

\mysection{The Importance of Saying ``Oops''}

{
 I just finished reading a history of Enron's
downfall, \textit{The Smartest Guys in the Room}, which hereby wins my
award for ``Least Appropriate Book
Title.'' }

{
 An unsurprising feature of Enron's slow rot and
abrupt collapse was that the executive players never admitted to having
made a \textit{large} mistake. When catastrophe \#247 grew to such an
extent that it required an actual policy change, they would say,
``Too bad that didn't work out---it
was such a good idea---how are we going to hide the problem on our
balance sheet?'' As opposed to, ``It
now seems obvious in retrospect that it was a mistake from the
beginning.'' As opposed to,
``I've been
stupid.'' There was never a watershed moment, a
moment of humbling realization, of acknowledging a \textit{fundamental}
problem. After the bankruptcy, Jeff Skilling, the former COO and brief
CEO of Enron, declined his own lawyers' advice to take
the Fifth Amendment; he testified before Congress that Enron had been a
\textit{great} company.}

{
 Not every change is an improvement, but every improvement is
necessarily a change. If we only admit small local errors, we will only
make small local changes. The motivation for a \textit{big} change
comes from acknowledging a \textit{big} mistake.}

{
 As a child I was raised on equal parts science and science
fiction, and from Heinlein to Feynman I learned the tropes of
Traditional Rationality: theories must be bold and expose themselves to
falsification; be willing to commit the heroic sacrifice of giving up
your own ideas when confronted with contrary evidence; play nice in
your arguments; try not to deceive yourself; and other fuzzy
verbalisms.}

{
 A traditional rationalist upbringing tries to produce arguers who
will concede to contrary evidence \textit{eventually}{}---there should
be \textit{some} mountain of evidence sufficient to move you. This is
not trivial; it distinguishes science from religion. But there is less
focus on \textit{speed}, on giving up the fight \textit{as quickly as
possible}, integrating evidence \textit{efficiently} so that it only
takes a \textit{minimum} of contrary evidence to destroy your cherished
belief.}

{
 I was raised in Traditional Rationality, and thought myself quite
the rationalist. I switched to Bayescraft (Laplace / Jaynes / Tversky /
Kahneman) in the aftermath of \ldots well, it's a long
story. Roughly, I switched because I realized that Traditional
Rationality's fuzzy verbal tropes had been insufficient
to prevent me from making a large mistake.}

{
 After I had finally and fully admitted my mistake, I looked back
upon the path that had led me to my Awful Realization. And I saw that I
had made a series of small concessions, minimal concessions, grudgingly
conceding each millimeter of ground, realizing as little as possible of
my mistake on each occasion, admitting failure only in small tolerable
nibbles. I could have moved so much faster, I realized, if I had simply
screamed \textit{``OOPS!''}}

{
 And I thought: \textit{I must raise the level of my game.}}

{
 There is a \textit{powerful advantage} to admitting you have made
a \textit{large} mistake. It's painful. It can also
change your whole life.}

{
 It is \textit{important} to have the watershed moment, the moment
of humbling realization. To acknowledge a \textit{fundamental} problem,
not divide it into palatable bite-size mistakes.}

{
 Do not indulge in drama and become proud of admitting errors. It
is surely superior to get it right the first time. But if you do make
an error, better by far to see it all at once. Even hedonically, it is
better to take one large loss than many small ones. The alternative is
stretching out the battle with yourself over years. The alternative is
Enron.}

{
 Since then I have watched others making their own series of
minimal concessions, grudgingly conceding each millimeter of ground;
never confessing a global mistake where a local one will do; always
learning as little as possible from each error. What they could fix in
one fell swoop voluntarily, they transform into tiny local patches they
must be argued into. Never do they say, after confessing one mistake,
\textit{I've been a fool.} They do their best to
minimize their embarrassment by saying \textit{I was right in
principle}, or \textit{It could have worked}, or \textit{I still want
to embrace the true essence of
whatever-I'm-attached-to.} Defending their pride in
this passing moment, they ensure they will again make the same mistake,
and again need to defend their pride.}

{
 Better to swallow the entire bitter pill in one terrible gulp.}

\myendsectiontext

\mysection{The Crackpot Offer}

{
 When I was very young---I think thirteen or maybe fourteen---I
thought I had found a disproof of Cantor's Diagonal
Argument, a famous theorem which demonstrates that the real numbers
outnumber the rational numbers. Ah, the dreams of fame and glory that
danced in my head! }

{
 My idea was that since each whole number can be decomposed into a
bag of powers of 2, it was possible to map the whole numbers onto the
set of subsets of whole numbers simply by writing out the binary
expansion. The number 13, for example, 1101, would map onto
{\textbackslash}{\textquotesingle}7b0, 2,
3{\textbackslash}{\textquotesingle}7d. It took a whole week before it
occurred to me that perhaps I should \textit{apply}
Cantor's Diagonal Argument to my clever construction,
and of course it found a counterexample---the binary number (\ldots
1111), which does not correspond to any finite whole number.}

{
 So I found this counterexample, and saw that my attempted disproof
was false, along with my dreams of fame and glory.}

{
 I was initially a bit disappointed.}

{
 The thought went through my mind:
``I'll get that theorem eventually!
\textit{Someday} I'll disprove Cantor's
Diagonal Argument, even though my first try failed!''
I resented the theorem for being obstinately true, for depriving me of
my fame and fortune, and I began to look for other disproofs.}

{
 And then I realized something. I realized that I had made a
mistake, and that, now that I'd spotted my mistake,
there was absolutely no reason to suspect the strength of
Cantor's Diagonal Argument any more than other major
theorems of mathematics.}

{
 I saw then very clearly that I was being offered the opportunity
to become a math crank, and to spend the rest of my life writing angry
letters in green ink to math professors. (I'd read a
book once about math cranks.)}

{
 I did not wish this to be my future, so I gave a small laugh, and
let it go. I waved Cantor's Diagonal Argument on with
all good wishes, and I did not question it again.}

{
 And I don't remember, now, if I thought this at
the time, or if I thought it afterward \ldots but what a terribly unfair
test to visit upon a child of thirteen. That I had to be that rational,
already, at that age, or fail.}

{
 The smarter you are, the younger you may be, the first time you
have what looks to you like a really revolutionary idea. I was lucky in
that I saw the mistake myself; that it did not take another
mathematician to point it out to me, and perhaps give me an outside
source to blame. I was lucky in that the disproof was simple enough for
me to understand. Maybe I would have recovered eventually, otherwise.
I've recovered from much worse, as an adult. But if I
had gone wrong that early, would I ever have developed that skill?}

{
 I wonder how many people writing angry letters in green ink were
thirteen when they made that first fatal misstep. I wonder how many
were promising minds before then.}

{
 I made a mistake. That was all. I was not \textit{really right,
deep down}; I did not win a moral victory; I was not displaying
ambition or skepticism or any other wondrous virtue; it was not a
reasonable error; I was not half right or even the tiniest fraction
right. I thought a thought I would never have thought if I had been
wiser, and that was all there ever was to it.}

{
 If I had been unable to admit this to myself, if I had
reinterpreted my mistake as virtuous, if I had insisted on being at
least a \textit{little} right for the sake of pride, then I would not
have let go. I would have gone on looking for a flaw in the Diagonal
Argument. And, sooner or later, I might have found one.}

{
 Until you admit you were wrong, you cannot get on with your life;
your self-image will still be bound to the old mistake.}

{
 Whenever you are tempted to hold on to a thought you would never
have thought if you had been wiser, you are being offered the
opportunity to become a crackpot---even if you never write any angry
letters in green ink. If no one bothers to argue with you, or if you
never tell anyone your idea, you may still be a crackpot.
It's the \textit{clinging} that defines it.}

{
 It's not true. It's not true deep
down. It's not half-true or even a little true.
It's nothing but a thought you should never have
thought. Not every cloud has a silver lining. Human beings make
mistakes, and not all of them are disguised successes. Human beings
make mistakes; it happens, that's all. Say
``oops,'' and get on with your
life.}

\myendsectiontext

\mysection{Just Lose Hope Already}

{
 Casey Serin, a 24-year-old web programmer with no prior experience
in real estate, owes banks 2.2 million dollars after lying on mortgage
applications in order to simultaneously buy eight different houses in
different states. He took cash out of the mortgage (applied for larger
amounts than the price of the house) and spent the money on living
expenses and real-estate seminars. He was expecting the market to go
up, it seems. }

{
 That's not even the sad part. The sad part is that
\textit{he still hasn't given up.} Casey Serin does not
accept defeat. He refuses to declare bankruptcy, or get a job; he still
thinks he can make it big in real estate. He went on spending money on
seminars. He tried to take out a mortgage on a ninth house. He
hasn't \textit{failed}, you see, he's
just had a \textit{learning experience.}}

{
 That's what happens when you refuse to lose hope.}

{
 While this behavior may seem to be merely stupid, it also puts me
in mind of two Nobel-Prize-winning economists \ldots}

{
 \ldots namely Merton and Scholes of Long-Term Capital Management.}

{
 While LTCM raked in giant profits over its first three years, in
1998 the inefficiences that LTCM were exploiting had started to
vanish---other people knew about the trick, so it stopped working.}

{
 LTCM refused to lose hope. Addicted to 40\% annual returns, they
borrowed more and more leverage to exploit tinier and tinier margins.
When everything started to go wrong for LTCM, they had equity of \$4.72
billion, leverage of \$124.5 billion, and derivative positions of
\$1.25 trillion.}

{
 Every profession has a different way to be smart---different
skills to learn and rules to follow. You might therefore think that the
study of ``rationality,'' as a
general discipline, wouldn't have much to contribute to
real-life success. And yet it seems to me that \textit{how to not be
stupid} has a great deal in common across professions. If you set out
to teach someone \textit{how to not turn little mistakes into big
mistakes}, it's nearly the same art whether in hedge
funds or romance, and one of the keys is this: Be ready to admit you
lost.}

\myendsectiontext

\mysection{The Proper Use of Doubt}

{
 Once, when I was holding forth upon the Way, I remarked upon how
most organized belief systems exist to \textit{flee from doubt.} A
listener replied to me that the Jesuits must be immune from this
criticism, because they practice organized doubt: their novices, he
said, are told to doubt Christianity; doubt the existence of God; doubt
if their calling is real; doubt that they are suitable for perpetual
vows of chastity and poverty. And I said: \textit{Ah, but
they're supposed to overcome these doubts, right?} He
said: \textit{No, they are to doubt that perhaps their doubts may grow
and become stronger.} }

{
 Googling failed to confirm or refute these allegations. (If anyone
in the audience can help, I'd be much obliged.) But I
find this scenario fascinating, worthy of discussion, regardless of
whether it is true or false of Jesuits. \textit{If} the Jesuits
practiced deliberate doubt, as described above, would they
\textit{therefore} be virtuous as rationalists?}

{
 I think I have to concede that the Jesuits, in the (possibly
hypothetical) scenario above, would not properly be described as
``fleeing from doubt.'' But the
(possibly hypothetical) conduct still strikes me as highly suspicious.
To a truly virtuous rationalist, doubt should not be scary. The conduct
described above sounds to me like a program of desensitization for
something \textit{very} scary, like exposing an arachnophobe to spiders
under carefully controlled conditions.}

{
 But even so, they are encouraging their novices to doubt---right?
Does it matter if their reasons are flawed? Is this not still a worthy
deed unto a rationalist?}

{
 All curiosity seeks to annihilate itself; there is no curiosity
that does not \textit{want} an answer. But if you obtain an answer, if
you satisfy your curiosity, then the glorious mystery will no longer be
mysterious.}

{
 In the same way, every doubt exists in order to annihilate some
particular belief. If a doubt fails to destroy its target, the doubt
has died unfulfilled---but that is still a resolution, an ending,
albeit a sadder one. A doubt that neither destroys itself nor destroys
its target might as well have never existed at all. It is the
\textit{resolution} of doubts, not the mere act of doubting, which
drives the ratchet of rationality forward.}

{
 Every improvement is a change, but not every change is an
improvement. Every rationalist doubts, but not all doubts are rational.
Wearing doubts doesn't make you a rationalist any more
than wearing a white medical lab coat makes you a doctor.}

{
 A rational doubt comes into existence for a specific reason---you
have some specific justification to suspect the belief is wrong. This
reason in turn, implies an avenue of investigation which will either
destroy the targeted belief, or destroy the doubt. This holds even for
highly abstract doubts, like ``I wonder if there might
be a simpler hypothesis which also explains this
data.'' In this case you investigate by trying to
think of simpler hypotheses. As this search continues longer and longer
without fruit, you will think it less and less likely that the next
increment of computation will be the one to succeed. Eventually the
cost of searching will exceed the expected benefit, and
you'll stop searching. At which point you can no longer
claim to be \textit{usefully doubting.} A doubt that is not
investigated might as well not exist. Every doubt exists to destroy
itself, one way or the other. An unresolved doubt is a null-op; it does
not turn the wheel, neither forward nor back.}

{
 If you really believe a religion (not just believe in it), then
why would you tell your novices to consider doubts that must die
unfulfilled? It would be like telling physics students to painstakingly
doubt that the twentieth-century revolution might have been a mistake,
and that Newtonian mechanics was correct all along. If you
don't \textit{really} doubt something, why would you
\textit{pretend} that you do?}

{
 Because we all want to be seen as rational---and doubting is
\textit{widely believed} to be a virtue of a rationalist. But it is not
widely understood that you need a particular reason to doubt, or that
an unresolved doubt is a null-op. Instead people think
it's about \textit{modesty}, a submissive demeanor,
maintaining the tribal status hierarchy---almost exactly the same
problem as with humility, on which I have previously written. Making a
great public display of doubt to convince yourself that you are a
rationalist will do around as much good as wearing a lab coat.}

{
 To avoid professing doubts, remember:}

{
 A rational doubt exists to destroy its target belief, and if it
does not destroy its target it dies unfulfilled.}

{
 A rational doubt arises from some specific reason the belief might
be wrong.}

{
 An unresolved doubt is a null-op.}

{
 An uninvestigated doubt might as well not exist.}

{
 You should not be proud of mere doubting, although you can justly
be proud when you have just \textit{finished} tearing a cherished
belief to shreds.}

{
 Though it may take courage to face your doubts, never forget that
\textit{to an ideal mind} doubt would not be scary in the first place.}

\myendsectiontext

\mysection{You Can Face Reality}

{
 What is true is already so.}

{
 Owning up to it doesn't make it worse.}

{
 Not being open about it doesn't make it go away.}

{
 And because it's true, it is what is there to be
interacted with.}

{
 Anything untrue isn't there to be lived.}

{
 People can stand what is true,}

{
 for they are already enduring it.}

{\raggedleft
 {}---\textit{Eugene Gendlin}
\par}


\bigskip

{
 ~}

\myendsectiontext

\mysection{The Meditation on Curiosity}

{
 The first virtue is curiosity.}

{\raggedleft
 {}---The Twelve Virtues of Rationality
\par}


\bigskip

{
 ~}

{
 As rationalists, we are obligated to criticize ourselves and
question our beliefs \ldots are we not?}

{
 Consider what happens to you, on a psychological level, if you
begin by saying: ``It is my duty to criticize my own
beliefs.'' Roger Zelazny once distinguished between
``wanting to be an author'' versus
``wanting to write.'' Mark Twain
said: ``A classic is something that everyone wants to
have read and no one wants to read.'' Criticizing
yourself from a sense of duty leaves you \textit{wanting to have
investigated}, so that you'll be able to say afterward
that your faith is not blind. This is not the same as \textit{wanting
to investigate.}}

{
 This can lead to motivated stopping of your investigation. You
consider an objection, then a counterargument to that objection, then
you \textit{stop there.} You repeat this with several objections, until
you feel that you have done your duty to investigate, and then you
\textit{stop there.} You have achieved your underlying psychological
objective: to get rid of the cognitive dissonance that would result
from thinking of yourself as a rationalist and yet knowing that you had
not tried to criticize your belief. You might call it purchase of
rationalist satisfaction---trying to create a ``warm
glow'' of discharged duty.}

{
 Afterward, your stated probability level will be high enough to
justify your keeping the plans and beliefs you started with, but not so
high as to evoke incredulity from yourself or other rationalists.}

{
 When you're really curious, you'll
gravitate to inquiries that seem most promising of producing shifts in
belief, or inquiries that are least like the ones
you've tried before. Afterward, your probability
distribution likely should \textit{not} look like it did when you
started out---shifts should have occurred, whether up or down; and
either direction is equally fine to you, if you're
genuinely curious.}

{
 Contrast this to the subconscious motive of keeping your inquiry
on familiar ground, so that you can get your investigation over with
quickly, so that you can \textit{have investigated}, and restore the
familiar balance on which your familiar old plans and beliefs are
based.}

{
 As for what I think true curiosity should look like, and the power
that it holds, I refer you to A Fable of Science and Politics. Each of
the characters is intended to illustrate different lessons. Ferris, the
last character, embodies the power of innocent curiosity: which is
lightness, and an eager reaching forth for evidence.}

{
 Ursula K. LeGuin wrote: ``In innocence there is
no strength against evil. But there is strength in it for
good.''\textsuperscript{1} Innocent curiosity may
turn innocently awry; and so the training of a rationalist, and its
accompanying sophistication, must be dared as a danger if we want to
become stronger. Nonetheless we can try to keep the lightness and the
eager reaching of innocence.}

{
 As it is written in the Twelve Virtues:}

{
 If in your heart you believe you already know, or if in your heart
you do not wish to know, then your questioning will be purposeless and
your skills without direction. Curiosity seeks to annihilate itself;
there is no curiosity that does not want an answer.}

{
 There just isn't any good substitute for genuine
curiosity. ``A burning itch to know is higher than a
solemn vow to pursue truth.'' But you
can't produce curiosity just by willing it, any more
than you can will your foot to feel warm when it feels cold. Sometimes,
all we have is our mere solemn vows.}

{
 So what can you do with duty? For a start, we can try to take an
interest in our dutiful investigations---keep a close eye out for
sparks of genuine intrigue, or even genuine ignorance and a desire to
resolve it. This goes right along with keeping a special eye out for
possibilities that are painful, that you are flinching away
from---it's not all negative thinking.}

{
 It should also help to meditate on Conservation of Expected
Evidence. For every \textit{new} point of inquiry, for every piece of
\textit{unseen} evidence that you suddenly look at, the expected
posterior probability should equal your prior probability. In the
microprocess of inquiry, your belief should always be evenly poised to
shift in either direction. Not every point may suffice to blow the
issue wide open---to shift belief from 70\% to 30\% probability---but
if your current belief is 70\%, you should be as ready to drop it to
69\% as raising it to 71\%. You should not think that you know which
direction it will go in (on average), because by the laws of
probability theory, if you know your destination, you are already
there. If you can investigate honestly, so that each \textit{new} point
really does have equal potential to shift belief upward or downward,
this may help to keep you interested or even curious about the
microprocess of inquiry.}

{
 If the argument you are considering is \textit{not} new, then why
is your attention going here? Is this where you would look if you were
genuinely curious? Are you subconsciously criticizing your belief at
its strong points, rather than its weak points? Are you rehearsing the
evidence?}

{
 If you can manage not to rehearse already known support, and you
can manage to drop down your belief by one tiny bite at a time from the
new evidence, you may even be able to relinquish the belief
entirely---to realize from which quarter the winds of evidence are
blowing against you.}

{
 Another restorative for curiosity is what I have taken to calling
the Litany of Tarski, which is really a meta-litany that specializes
for each instance (this is only appropriate). For example, if I am
tensely wondering whether a locked box contains a diamond, then, rather
than thinking about all the wonderful consequences if the box does
contain a diamond, I can repeat the Litany of Tarski:}

{
 \textit{If the box contains a diamond,}\newline
\textit{ I desire to believe that the box contains a diamond;}\newline
\textit{ If the box does not contain a diamond,}\newline
\textit{ I desire to believe that the box does not contain a
diamond;}\newline
\textit{ Let me not become attached to beliefs I may not want.}}

{
 Then you should meditate upon the possibility that there is no
diamond, and the subsequent advantage that will come to you if you
believe there is no diamond, and the subsequent disadvantage if you
believe there is a diamond. See also the Litany of Gendlin.}

{
 If you can find within yourself the slightest shred of true
uncertainty, then guard it like a forester nursing a campfire. If you
can make it blaze up into a flame of curiosity, it will make you light
and eager, and give purpose to your questioning and direction to your
skills.}

\myendsectiontext


\bigskip

{
 1. Ursula K. Le Guin, \textit{The Farthest Shore} (Saga Press,
2001).}

\mysection{No One Can Exempt You From Rationality's Laws}

{
 Traditional Rationality is phrased in terms of \textit{social
rules}, with violations interpretable as cheating---as defections from
cooperative norms. If you want me to accept a belief from you, you are
obligated to provide me with a certain amount of evidence. If you try
to get out of it, we all know you're cheating on your
obligation. A theory is obligated to make bold predictions for itself,
not just steal predictions that other theories have labored to make. A
theory is obligated to expose itself to falsification---if it tries to
duck out, that's like trying to duck out of a fearsome
initiation ritual; you must pay your dues. }

{
 Traditional Rationality is phrased similarly to the customs that
govern human societies, which makes it easy to pass on by word of
mouth. Humans detect social cheating with much greater reliability than
isomorphic violations of abstract logical rules. But viewing
rationality as a social obligation gives rise to some strange ideas.}

{
 For example, one finds religious people defending their beliefs by
saying, ``Well, \textit{you} can't
justify your belief in science!'' In other words,
``How dare you criticize me for having unjustified
beliefs, you hypocrite! You're doing it
too!''}

{
 To Bayesians, the brain is an engine of accuracy: it processes and
concentrates entangled evidence into a map that reflects the territory.
The principles of rationality are laws in the same sense as the Second
Law of Thermodynamics: obtaining a reliable belief requires a
calculable amount of entangled evidence, just as reliably cooling the
contents of a refrigerator requires a calculable minimum of free
energy.}

{
 In principle, the laws of physics are time-reversible, so
there's an infinitesimally tiny
probability---indistinguishable from zero to all but
mathematicians---that a refrigerator will spontaneously cool itself
down while generating electricity. There's a slightly
larger infinitesimal chance that you could accurately draw a detailed
street map of New York without ever visiting, sitting in your living
room with your blinds closed and no Internet connection. But I
wouldn't hold your breath.}

{
 Before you try mapping an unseen territory, pour some water into a
cup at room temperature and wait until it spontaneously freezes before
proceeding. That way you can be sure the general trick---ignoring
infinitesimally tiny probabilities of success---is working properly.
You might not realize directly that your map is wrong, especially if
you never visit New York; but you can see that water
doesn't freeze itself.}

{
 If the rules of rationality are social customs, then it may seem
to excuse behavior X if you point out that others are doing the same
thing. It wouldn't be \textit{fair} to demand evidence
from you, if we can't provide it ourselves. We will
realize that none of us are better than the rest, and we will relent
and mercifully excuse you from your social obligation to provide
evidence for your belief. And we'll all live happily
ever afterward in liberty, fraternity, and equality.}

{
 If the rules of rationality are mathematical laws, then trying to
justify evidence-free belief by pointing to someone else doing the same
thing, will be around as effective as listing thirty reasons why you
shouldn't fall off a cliff. Even if we all vote that
it's unfair for your refrigerator to need electricity,
it still won't run (with probability \~{}1). Even if we
all vote that you shouldn't have to visit New York, the
map will still be wrong. Lady Nature is famously indifferent to such
pleading, and so is Lady Math.}

{
 So---to shift back to the social language of Traditional
Rationality---don't think you can \textit{get away
with} claiming that it's okay to have arbitrary beliefs
about XYZ, because other people have arbitrary beliefs too. If two
parties to a contract both behave equally poorly, a human judge may
decide to impose penalties on neither. But if two engineers design
their engines equally poorly, neither engine will work. One design
error cannot excuse another. Even if \textit{I'm} doing
XYZ wrong, it doesn't help you, or exempt you from the
rules; it just means we're both screwed.}

{
 As a matter of human law in liberal democracies, everyone is
entitled to their own beliefs. As a matter of Nature's
law, you are not entitled to accuracy. We don't arrest
people for believing weird things, at least not in the wiser countries.
But no one can revoke the law that you need evidence to generate
\textit{accurate} beliefs. Not even a vote of the whole human species
can obtain mercy in the court of Nature.}

{
 Physicists don't decide the laws of physics, they
just guess what they are. Rationalists don't decide the
laws of rationality, we just guess what they are. You cannot
``rationalize'' anything that is not
rational to begin with. If by dint of extraordinary persuasiveness you
convince all the physicists in the world that you are exempt from the
law of gravity, and you walk off a cliff, you'll fall.
Even saying ``\textit{We} don't
decide'' is too anthropomorphic. There is no higher
authority that could exempt you. There is only cause and effect.}

{
 Remember this, when you plead to be excused just this once. We
\textit{can't} excuse you. It isn't up
to us.}

\myendsectiontext

\mysection{Leave a Line of Retreat}

{
 When you surround the enemy}

{
 Always allow them an escape route.}

{
 They must see that there is}

{
 An alternative to death.}

{\raggedleft
 {}---Sun Tzu, \textit{The Art of War}\textsuperscript{1}
\par}


\bigskip

{
 ~}

{
 Don't raise the pressure, lower the wall.}

{\raggedleft
 {}---Lois McMaster Bujold, \textit{Komarr}\textsuperscript{2}
\par}


\bigskip

{
 ~}

{
 Once I happened to be conversing with a nonrationalist who had
somehow wandered into a local rationalists' gathering.
She had just declared (a) her belief in souls and (b) that she
didn't believe in cryonics because she believed the
soul wouldn't stay with the frozen body. I asked,
``But how do you know that?'' From
the confusion that flashed on her face, it was pretty clear that this
question had never occurred to her. I don't say this in
a bad way---she seemed like a nice person with absolutely no training
in rationality, just like most of the rest of the human species. I
really need to write that book.}

{
 Most of the ensuing conversation was on items already covered on
\textit{Overcoming Bias}{}---if you're \textit{really}
curious about something, you probably \textit{can} figure out a good
way to test it; try to attain accurate beliefs first and then let your
emotions flow from that---that sort of thing. But the conversation
reminded me of one notion I haven't covered here yet:}

{
 ``Make sure,'' I suggested to
her, ``that you visualize what the world would be like
if there are no souls, and what you would do about that.
Don't think about all the reasons that it
can't be that way, just accept it as a premise and then
visualize the consequences. So that you'll think,
`Well, if there are no souls, I can just sign up for
cryonics,' or `If there is no God, I can
just go on being moral anyway,' rather than it being
too horrifying to face. As a matter of self-respect you should try to
believe the truth no matter how uncomfortable it is, like I said
before; but as a matter of human nature, it helps to make a belief less
uncomfortable, \textit{before} you try to evaluate the evidence for
it.''}

{
 The principle behind the technique is simple: as Sun Tzu advises
you to do with your enemies, you must do with yourself---leave yourself
a line of retreat, so that you will have less trouble retreating. The
prospect of losing your job, say, may seem a lot more scary when you
can't even bear to think about it, than after you have
calculated exactly how long your savings will last, and checked the job
market in your area, and otherwise planned out exactly what to do next.
Only then will you be ready to \textit{fairly} assess the probability
of keeping your job in the planned layoffs next month. Be a true
coward, and plan out your retreat in detail---visualize every
step---preferably before you first come to the battlefield.}

{
 The hope is that it takes less courage to visualize an
uncomfortable state of affairs \textit{as a thought experiment}, than
to consider \textit{how likely} it is to be true. But then after you do
the former, it becomes easier to do the latter.}

{
 Remember that Bayesianism is precise---even if a scary proposition
really should seem unlikely, it's still important to
count up all the evidence, for and against, exactly fairly, to arrive
at the rational quantitative probability. Visualizing a scary belief
does \textit{not} mean admitting that you think, deep down,
it's probably true. You can visualize a scary belief on
general principles of good mental housekeeping. ``The
thought you cannot think controls you more than thoughts you speak
aloud''---this happens even if the unthinkable
thought is false!}

{
 The leave-a-line-of-retreat technique does require a certain
minimum of self-honesty to use correctly.}

{
 For a start: You must at least be able to admit to yourself
\textit{which} ideas scare you, and which ideas you are attached to.
But this is a substantially less difficult test than fairly counting
the evidence for an idea that scares you. Does it help if I say that I
have occasion to use this technique myself? A rationalist does not
reject all emotion, after all. There are ideas which scare me, yet I
still believe to be false. There are ideas to which I know I am
attached, yet I still believe to be true. But I still plan my retreats,
not because I'm planning \textit{to} retreat, but
because planning my retreat in advance helps me think about the problem
without attachment.}

{
 But the greater test of self-honesty is to \textit{really} accept
the uncomfortable proposition as a premise, and figure out how you
would \textit{really} deal with it. When we're faced
with an uncomfortable idea, our first impulse is naturally to think of
all the reasons why it \textit{can't possibly} be so.
And so you will encounter a certain amount of psychological resistance
in yourself, if you try to visualize exactly how the world would be,
and what you would do about it, if My-Most-Precious-Belief were false,
or My-Most-Feared-Belief were true.}

{
 Think of all the people who say that, without God, morality was
impossible. (And yes, this topic did come up in the conversation; so I
am not offering a strawman.) If theists could visualize their
\textit{real} reaction to believing as a fact that God did not exist,
they could realize that, no, they wouldn't go around
slaughtering babies. They could realize that atheists are reacting to
the nonexistence of God in pretty much the way they themselves would,
if they came to believe that. I say this, to show that it \textit{is} a
considerable challenge to visualize the way you \textit{really would}
react, to believing the opposite of a tightly held belief.}

{
 Plus it's always counterintuitive to realize that,
yes, people do get over things. Newly minted quadriplegics are not as
sad, six months later, as they expect to be, etc. It can be equally
counterintuitive to realize that if the scary belief turned out to be
true, you \textit{would} come to terms with it somehow. Quadriplegics
deal, and so would you.}

{
 See also the Litany of Gendlin and the Litany of Tarski. What is
true is already so; owning up to it doesn't make it
worse. You shouldn't be afraid to just
\textit{visualize} a world you fear. If that world is already actual,
visualizing it won't make it worse; and if it is
\textit{not} actual, visualizing it will do no harm. And remember, as
you visualize, that if the scary things you're
imagining really are true---which they may not be!---then you would,
indeed, want to believe it, and you should visualize that too; not
believing wouldn't help you.}

{
 How many religious people would retain their belief in God, if
they could \textit{accurately} visualize that hypothetical world in
which there was no God and they themselves have become atheists?}

{
 Leaving a line of retreat is a powerful technique, but
it's not easy. \textit{Honest} visualization
doesn't take as much effort as admitting
\textit{outright} that God doesn't exist, but it does
take an effort.}

\myendsectiontext


\bigskip

{
 1. Sun Tzu, \textit{The Art of War} (Cloud Hands, Inc., 2004).}

{
 2. Lois McMaster Bujold, \textit{Komarr}, Miles Vorkosigan
Adventures (Baen, 1999).}

\mysection{Crisis of Faith}

{
 It ain't a true crisis of faith unless things
could just as easily go either way.}

{\raggedleft
 {}---Thor Shenkel
\par}


\bigskip

{
 ~}

{
 Many in this world retain beliefs whose flaws a ten-year-old could
point out, \textit{if} that ten-year-old were hearing the beliefs for
the first time. These are not subtle errors we are talking about. They
would be child's play for an unattached mind to
relinquish, if the skepticism of a ten-year-old were applied without
evasion. As Premise Checker put it, ``Had the idea of
god not come along until the scientific age, only an exceptionally
weird person would invent such an idea and pretend that it explained
anything.''}

{
 And yet skillful scientific specialists, even the major innovators
of a field, even in this very day and age, do not apply that skepticism
successfully. Nobel laureate Robert Aumann, of Aumann's
Agreement Theorem, is an Orthodox Jew: I feel reasonably confident in
venturing that Aumann must, at one point or another, have questioned
his faith. And yet he did not doubt successfully. We change our minds
less often than we think.}

{
 This should scare you down to the marrow of your bones. It means
you can be a world-class scientist \textit{and} conversant with
Bayesian mathematics \textit{and} still fail to reject a belief whose
absurdity a fresh-eyed ten-year-old could see. It shows the invincible
defensive position which a belief can create for itself, if it has long
festered in your mind.}

{
 What does it take to defeat an error that has built itself a
fortress?}

{
 But by the time you \textit{know} it is an error, it is already
defeated. The dilemma is not ``How can I reject
long-held false belief X?'' but
``How do I know if long-held belief X is
false?'' Self-honesty is at its most fragile when
we're not \textit{sure} which path is the righteous
one. And so the question becomes:}

{
 How can we create in ourselves a true crisis of faith, that could
just as easily go either way?}

{
 Religion is the trial case we can all imagine. (Readers born to
atheist parents have missed out on a fundamental life trial, and must
make do with the poor substitute of thinking of their religious
friends.) But if you have cut off all sympathy and now think of theists
as evil mutants, then you won't be able to imagine the
real internal trials they face. You won't be able to
ask the question:}

{
 ``What general strategy would a religious person
have to follow in order to escape their religion?''}

{
 I'm sure that some, looking at this challenge, are
already rattling off a list of standard atheist talking
points---``They would have to admit that there
wasn't any Bayesian evidence for God's
existence,'' ``They would have to
see the moral evasions they were carrying out to excuse
God's behavior in the Bible,''
``They need to learn how to use
Occam's Razor---''}

{
 WRONG! WRONG WRONG WRONG! This kind of rehearsal, where you just
cough up points \textit{you already thought of long before}, is
\textit{exactly} the style of thinking that keeps people within their
current religions. If you stay with your cached thoughts, if your brain
fills in the obvious answer so fast that you can't see
originally, you surely will not be able to conduct a crisis of faith.}

{
 Maybe it's just a question of not enough people
reading \textit{Gödel, Escher, Bach} at a sufficiently young age, but
I've noticed that a large fraction of the
population---even technical folk---have trouble following arguments
that go this meta. On my more pessimistic days I wonder if the camel
has two humps.}

{
 Even when it's explicitly pointed out, some people
seemingly \textit{cannot follow the leap} from the object-level
``Use Occam's Razor! You have to see
that your God is an unnecessary belief!'' to the
meta-level ``Try to stop your mind from completing the
pattern the usual way!'' Because in the same way that
all your rationalist friends talk about Occam's Razor
like it's a good thing, and in the same way that
Occam's Razor leaps right up into your mind, so too,
the obvious friend-approved religious response is
``God's ways are mysterious and it is
presumptuous to suppose that we can understand
them.'' So for you to think that the \textit{general}
strategy to follow is ``Use Occam's
Razor,'' would be like a theist saying that the
general strategy is to have faith.}

{
 ``But---but Occam's Razor really
is better than faith! That's not like preferring a
different flavor of ice cream! Anyone can see, looking at history, that
Occamian reasoning has been far more productive than
faith---''}

{
 Which is all true. But beside the point. The point is that you,
saying this, are rattling off a standard justification
that's already in your mind. The challenge of a crisis
of faith is to handle the case where, possibly, our standard
conclusions are \textit{wrong} and our standard justifications are
\textit{wrong.} So if the standard justification for X is
``Occam's Razor!,''
and you want to hold a crisis of faith around X, you should be
questioning if Occam's Razor really endorses X, if your
understanding of Occam's Razor is correct, and---if you
want to have sufficiently deep doubts---whether simplicity \textit{is}
the sort of criterion that has worked well historically in this case,
or could reasonably be \textit{expected} to work, et cetera. If you
would advise a religionist to question their belief that
``faith'' is a good justification
for X, then you should advise yourself to put forth an equally strong
effort to question your belief that
``Occam's Razor'' is
a good justification for X.}

{
 (Think of all the people out there who don't
understand the Minimum Description Length or Solomonoff induction
formulations of Occam's Razor, who think that
Occam's Razor outlaws many-worlds or the Simulation
Hypothesis. They would need to question their formulations of
Occam's Razor and their notions of why simplicity is a
good thing. Whatever X in contention you just justified by saying
``Occam's Razor!,''
I bet it's not the same level of Occamian slam dunk as
gravity.)}

{
 If ``Occam's
Razor!'' is your usual reply, your standard reply,
the reply that all your friends give---then you'd
better block your brain from instantly completing that pattern, if
you're trying to instigate a true crisis of faith.}

{
 Better to think of such rules as, ``Imagine what
a skeptic would say---and then imagine what they would say to your
response---and then imagine what else they might say, that would be
harder to answer.''}

{
 Or, ``Try to think the thought that hurts the
most.''}

{
 And above all, the rule:}

{
 ``Put forth the same level of desperate effort
that it would take for a theist to reject their
religion.''}

{
 Because, if you \textit{aren't} trying that hard,
then---for all \textit{you} know---your head could be stuffed full of
nonsense as ridiculous as religion.}

{
 Without a convulsive, wrenching effort to be rational, the kind of
effort it would take to throw off a religion---then how dare you
believe anything, when Robert Aumann believes in God?}

{
 Someone (I forget who) once observed that people had only until a
certain age to reject their religious faith. Afterward they would have
answers to all the objections, and it would be too late. That is the
kind of existence you must surpass. This is a test of your strength as
a rationalist, and it is very severe; but if you cannot pass it, you
will be weaker than a ten-year-old.}

{
 But again, by the time you know a belief is an error, it is
already defeated. So we're not talking about a
desperate, convulsive effort to undo the effects of a religious
upbringing, \textit{after} you've come to the
conclusion that your religion is wrong. We're talking
about a desperate effort to \textit{figure out} if you should be
throwing off the chains, or keeping them. Self-honesty is at its most
fragile when we don't \textit{know} which path
we're supposed to take---that's when
rationalizations are not \textit{obviously} sins.}

{
 Not every doubt calls for staging an all-out Crisis of Faith. But
you should consider it when:}

{
 A belief has long remained in your mind;}

{
 It is surrounded by a cloud of known arguments and refutations;}

{
 You have sunk costs in it (time, money, public declarations);}

{
 The belief has emotional consequences (note this does not make it
wrong);}

{
 It has gotten mixed up in your personality generally.}

{
 None of these warning signs are immediate disproofs. These
attributes place a belief at risk for all sorts of dangers, and make it
very hard to reject when it \textit{is} wrong. But they also hold for
Richard Dawkins's belief in evolutionary biology as
well as the Pope's Catholicism. This does not say that
we are only talking about different flavors of ice cream. Only the
unenlightened think that all deeply-held beliefs are on the same level
regardless of the evidence supporting them, just because they are
deeply held. The point is not to have shallow beliefs, but to have a
map which reflects the territory.}

{
 I emphasize this, of course, so that you can admit to yourself,
``My belief has these warning
signs,'' without having to say to yourself,
``My belief is false.''}

{
 But what these warning signs \textit{do} mark, is a belief that
will take \textit{more than an ordinary effort to doubt effectively.}
So that if it were in fact false, you would in fact reject it. And
where you cannot doubt effectively, you are blind, because your brain
will hold the belief unconditionally. When a retina sends the same
signal regardless of the photons entering it, we call that eye blind.}

{
 When should you stage a Crisis of Faith?}

{
 Again, think of the advice you would give to a theist: If you find
yourself feeling a little unstable inwardly, but trying to rationalize
reasons the belief is still solid, then you should probably stage a
Crisis of Faith. If the belief is as solidly supported as gravity, you
needn't bother---but think of all the theists who would
desperately want to conclude that God is as solid as gravity. So try to
imagine what the skeptics out there would say to your
``solid as gravity'' argument.
Certainly, one reason you might fail at a crisis of faith is that you
never really sit down and question in the first place---that you never
say, ``Here is something I need to put effort into
doubting properly.''}

{
 If your thoughts get that complicated, you should go ahead and
stage a Crisis of Faith. Don't try to do it
haphazardly, don't try it in an ad-hoc spare moment.
Don't rush to get it done with quickly, so that you can
say ``I have doubted as I was obliged to
do.'' That wouldn't work for a theist
and it won't work for you either. Rest up the previous
day, so you're in good mental condition. Allocate some
uninterrupted hours. Find somewhere quiet to sit down. Clear your mind
of all standard arguments, try to see from scratch. And make a
desperate effort to put forth a true doubt that would destroy a false,
and \textit{only} a false, deeply held belief.}

{
 Elements of the Crisis of Faith technique have been scattered over
many essays:}

{
 Avoiding Your Belief's Real Weak Points---One of
the first temptations in a crisis of faith is to doubt the strongest
points of your belief, so that you can rehearse your good answers. You
need to seek out the most painful spots, not the arguments that are
most reassuring to consider.}

{
 The Meditation on Curiosity---Roger Zelazny once distinguished
between ``wanting to be an author''
versus ``wanting to write,'' and
there is likewise a distinction between wanting to have investigated
and wanting to investigate. It is not enough to say
``It is my duty to criticize my own
beliefs''; you must be curious, and only uncertainty
can create curiosity. Keeping in mind Conservation of Expected Evidence
may help you Update Yourself Incrementally: for every \textit{single}
point that you consider, and each element of new argument and new
evidence, you should not expect your beliefs to shift more (on average)
in one direction than another---thus you can be truly curious each time
about how it will go.}

{
 Original Seeing---Use Pirsig's technique to
prevent standard cached thoughts from rushing in and completing the
pattern.}

{
 The Litany of Gendlin and the Litany of Tarski---People can stand
what is true, for they are already enduring it. If a belief is true you
will be better off believing it, and if it is false you will be better
off rejecting it. You would advise a religious person to try to
visualize fully and deeply the world in which there is no God, and to,
without excuses, come to the full understanding that \textit{if} there
is no God \textit{then} they will be better off believing there is no
God. If one cannot come to accept this on a deep emotional level, one
will not be able to have a crisis of faith. So you should put in a
sincere effort to visualize the \textit{alternative} to your belief,
the way that the best and highest skeptic would want you to visualize
it. Think of the effort a religionist would have to put forth to
imagine, without corrupting it for their own comfort, an
atheist's view of the universe.}

{
 Make an Extraordinary Effort---See the concept of
\textit{isshokenmei}, the desperate convulsive effort to be rational,
the effort that it would take to surpass the level of Robert Aumann and
all the great scientists throughout history who never let go of their
religions.}

{
 The Genetic Heuristic---You should be extremely suspicious if you
have many ideas suggested by a source that you now know to be
untrustworthy, but by golly, it seems that all the ideas still ended up
being right. (E.g., the one concedes that the Bible was written by
human hands, but still clings to the idea that it contains
indispensable ethical wisdom.)}

{
 The Importance of Saying
``Oops''---It really is less painful
to swallow the entire bitter pill in one terrible gulp.}

{
 Singlethink---The opposite of doublethink. See the thoughts you
flinch away from, that appear in the corner of your mind for just a
moment before you refuse to think them. If you become aware of what you
are not thinking, you can think it.}

{
 Affective Death Spirals and Resist the Happy Death
Spiral---Affective death spirals are prime generators of false beliefs
that it will take a Crisis of Faith to shake loose. But since affective
death spirals can also get started around real things that are
genuinely nice, you don't have to admit that your
belief is a lie, to try and resist the halo effect at every
point---refuse false praise even of genuinely nice things. Policy
debates should not appear one-sided.}

{
 Hold Off On Proposing Solutions---Don't propose
any solutions until the problem has been discussed as thoroughly as
possible. Make your mind wait on knowing what its answer will be; and
try for five minutes before giving up, both generally, and especially
when pursuing the devil's point of view.}

{
 And these standard techniques are particularly relevant:}

{
 The sequence on The Bottom Line and Rationalization, which
explains why it is always wrong to selectively argue one side of a
debate.}

{
 Positive Bias and motivated skepticism and motivated stopping,
lest you selectively look for support, selectively look for
counter-counterarguments, and selectively stop the argument before it
gets dangerous. Missing alternatives are a special case of stopping. A
special case of motivated skepticism is fake humility, where you
bashfully confess that no one can know something you would rather not
know. Don't selectively demand too much authority of
counterarguments.}

{
 Beware of Semantic Stopsigns, Applause Lights, and your choice to
Explain/Worship/Ignore.}

{
 Feel the weight of Burdensome Details; each detail a separate
burden, a point of crisis.}

{
 But really there's rather a lot of relevant
material, here and on \textit{Overcoming Bias}. The Crisis of Faith is
only the critical point and sudden clash of the longer
\textit{isshoukenmei}{}---the lifelong uncompromising effort to be so
incredibly rational that you rise above the level of stupid damn
mistakes. It's when you get a chance to use the skills
that you've been practicing for so long, all-out
against yourself.}

{
 I wish you the best of luck against your opponent. Have a
wonderful crisis!}

\myendsectiontext

\mysection{The Ritual}

{
 The room in which Jeffreyssai received his
non-\textit{beisutsukai} visitors was quietly formal, impeccably
appointed in only the most conservative tastes. Sunlight and outside
air streamed through a grillwork of polished silver, a few sharp edges
making it clear that this wall was not to be opened. The floor and
walls were glass, thick enough to distort, to a depth sufficient that
it didn't matter what might be underneath. Upon the
surfaces of the glass were subtly scratched patterns of no particular
meaning, scribed as if by the hand of an artistically inclined child
(and this was in fact the case). }

{
 Elsewhere in Jeffreyssai's home there were rooms
of other style; but this, he had found, was what most outsiders
expected of a Bayesian Master, and he chose not to enlighten them
otherwise. That quiet amusement was one of life's
little joys, after all.}

{
 The guest sat across from him, knees on the pillow and heels
behind. She was here solely upon the business of her Conspiracy, and
her attire showed it: a form-fitting jumpsuit of pink leather with even
her hands gloved---all the way to the hood covering her head and hair,
though her face lay plain and unconcealed beneath.}

{
 And so Jeffreyssai had chosen to receive her in this room.}

{
 Jeffreyssai let out a long breath, exhaling.
``Are you \textit{sure}?''}

{
 ``Oh,'' she said,
``and do I have to be \textit{absolutely certain}
before my advice can shift your opinions? Does it not suffice that I am
a domain expert, and you are not?''}

{
 Jeffreyssai's mouth twisted up at the corner in a
half-smile. ``How do \textit{you} know so much about
the rules, anyway? You've never had so much as a Planck
length of formal training.''}

{
 ``Do you even need to ask?''
she said dryly. ``If there's one thing
that you \textit{beisutsukai} do love to go on about,
it's the reasons why you do
things.''}

{
 Jeffreyssai inwardly winced at the thought of trying to pick up
rationality by watching other people talk about it---}

{
 ``And don't inwardly wince at me
like that,'' she said.
``I'm not trying to be a rationalist
myself, just trying to win an argument with a rationalist.
There's a difference, as I'm sure you
tell your students.''}

{
 \textit{Can she really read me that well?} Jeffreyssai looked out
through the silver grillwork, at the sunlight reflected from the
faceted mountainside. Always, always the golden sunlight fell each day,
in this place far above the clouds. An unchanging thing, that light.
The distant Sun, which that light represented, was in five billion
years burned out; but now, in \textit{this} moment, the Sun still
shone. And that could never alter. Why wish for things to stay the same
way forever, when that wish was already granted as absolutely as any
wish could be? The paradox of permanence and impermanence: only in the
latter perspective was there any such thing as progress, or loss.}

{
 ``You have always given me good
counsel,'' Jeffreyssai said.
``Unchanging, that has been. Through all the time
we've known each other.''}

{
 She inclined her head, acknowledging. This was true, and there was
no need to spell out the implications.}

{
 ``So,'' Jeffreyssai said.
``Not for the sake of arguing. Only because I want to
know the answer. \textit{Are} you sure?'' He
didn't even see how she could \textit{guess.}}

{
 ``Pretty sure,'' she said,
``we've been collecting statistics for
a long time, and in nine hundred and eighty-five out of a thousand
cases like yours---''}

{
 Then she laughed at the look on his face. ``No,
I'm joking. Of course I'm not sure.
This thing only you can decide. But I \textit{am} sure that you should
go off and do whatever it is you people do---I'm quite
sure you have a ritual for it, even if you won't
discuss it with outsiders---when you \textit{very seriously consider}
abandoning a long-held premise of your existence.''}

{
 It was hard to argue with that, Jeffreyssai reflected, the more so
when a domain expert had told you that you were, in fact, probably
wrong.}

{
 ``I concede,'' Jeffreyssai
said. Coming from his lips, the phrase was spoken with a commanding
finality. \textit{There is no need to argue with me any further: you
have won.}}

{
 ``Oh, stop it,'' she said. She
rose from her pillow in a single fluid shift without the slightest
wasted motion. She didn't flaunt her age, but she
didn't conceal it either. She took his outstretched
hand, and raised it to her lips for a formal kiss.
``Farewell, sensei.''}

{
 ``Farewell?'' repeated
Jeffreyssai. That signified a higher order of departure than
\textit{goodbye}. ``I do intend to visit you again,
milady; and you are always welcome here.''}

{
 She walked toward the door without answering. At the doorway she
paused, without turning around. ``It
won't be the same,'' she said. And
then, without the movements seeming the least rushed, she walked away
so swiftly it was almost like vanishing.}

{
 Jeffreyssai sighed. But at least, from here until the challenge
proper, all his actions were prescribed, known quantities.}

{
 Leaving that formal reception area, he passed to his arena, and
caused to be sent out messengers to his students, telling them that the
next day's classes must be improvised in his absence,
and that there would be a test later.}

{
 And then he did nothing in particular. He read another hundred
pages of the textbook he had borrowed; it wasn't very
good, but then the book he had loaned out in exchange
wasn't very good either. He wandered from room to room
of his house, idly checking various storages to see if anything had
been stolen (a deck of cards was missing, but that was all). From time
to time his thoughts turned to tomorrow's challenge,
and he let them drift. Not directing his thoughts at all, only blocking
out every thought that had ever \textit{previously} occurred to him;
and disallowing any kind of conclusion, or even any thought as to where
his thoughts might be trending.}

{
 The sun set, and he watched it for a while, mind carefully put in
idle. It was a fantastic balancing act to set your mind in idle without
having to obsess about it, or exert energy to keep it that way; and
years ago he would have sweated over it, but practice had long since
made perfect.}

{
 The next morning he awoke with the chaos of the
night's dreaming fresh in his mind, and, doing his best
to preserve the feeling of the chaos as well as its memory, he
descended a flight of stairs, then another flight of stairs, then a
flight of stairs after that, and finally came to the least fashionable
room in his whole house.}

{
 It was white. That was pretty much it as far as the color scheme
went.}

{
 All along a single wall were plaques, which, following the classic
and suggested method, a younger Jeffreyssai had very carefully scribed
himself, burning the \textit{concepts} into his mind with each touch of
the brush that wrote the words. \textit{That which can be destroyed by
the truth should be. People can stand what is true, for they are
already enduring it. Curiosity seeks to annihilate itself.} Even one
small plaque that showed nothing except a red horizontal slash. Symbols
could be made to stand for \textit{anything}; a flexibility of visual
power that even the Bardic Conspiracy would balk at admitting
outright.}

{
 Beneath the plaques, two sets of tally marks scratched into the
wall. Under the plus column, two marks. Under the minus column, five
marks. Seven times he had entered this room; five times he had decided
not to change his mind; twice he had exited something of a different
person. There was no set ratio prescribed, or set range---that would
have been a mockery indeed. But if there were no marks in the plus
column after a while, you might as well admit that there was no point
in having the room, since you didn't have the ability
it stood for. Either that, or you'd been born knowing
the truth and right of everything.}

{
 Jeffreyssai seated himself, not facing the plaques, but facing
away from them, at the featureless white wall. It was better to have no
visual distractions.}

{
 In his mind, he rehearsed first the meta-mnemonic, and then the
various sub-mnemonics referenced, for the seven major principles and
sixty-two specific techniques that were most likely to prove needful in
the Ritual Of Changing One's Mind. To this, Jeffreyssai
added another mnemonic, reminding himself of his own fourteen most
embarrassing oversights.}

{
 He did not take a deep breath. Regular breathing was best.}

{
 And then he asked himself the question.}

\myendsectiontext


